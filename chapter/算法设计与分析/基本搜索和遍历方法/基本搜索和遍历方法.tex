\section{基本搜索和遍历方法}
搜索和遍历是计算机问题求解最常用的技术之一。本章讨论基本搜索和遍历方法,分析它们的性能。
\subsection*{基本概念}

\begin{enumerate}
	\item 搜索:是一种通过系统地检查给定数据对象的每个结点,寻找一条从开始结点到答案结点的路径,最终输出问题的求解方法。
	\item 遍历:要求系统地检查数据对象的每个结点。根据被遍历的数据对象的结构不同,可分为树遍历和图遍历。
	\item 状态空间:用于描述所求问题的各种可能的情况,每一种情况对应于状态空间中的一个状态。
	\item 无知搜索:按事先约定的某种次序,系统地在状态空间中搜索目标状态,而无须对状态空间有较多了解。
	\item 有知搜索:具有某些关于问题和问题解的知识,那么,便可运用这些知识,克服无知搜索的盲目性。采用经验法则的搜索方法称为启发式搜索(heuristic search)。
\end{enumerate}

深度优先搜索(depth first search)和广度优先搜索(breadth first search)是两种基本的盲目搜索方法,介于两者之间的有D-搜索(depth search)。
\subsection*{图的搜索和遍历} 
遵循某种次序,系统地访问一个数据结构的全部元素,并且每个元素仅访问一次,这种运算称为遍历。很显然,实现遍历运算的关键是规定结点被访问的次序。

在树形结构中,一个结点的直接后继结点是它的孩子结点;在图形结构中,一个结点的后继结点是邻接于该结点的所有邻接点。为深入认识搜索算法的特点,不妨将被搜索的数据结构中的结点按其状态分成4类:
\begin{enumerate}
	\item 未访问。$x$尚未访问
	\item 未检测。$x$自身已访问,但$x$的后继结点尚未全部访问。
	\item 正扩展。检测一个结点$x$,从$x$出发,访问$x$的某个后继结点$y$,$x$被称为扩展结点也叫E-结点。在算法执行的任何时刻,最多只有一个结点为E-结点。
	\item 已检测。$x$自身已访问,且$x$的后继结点也已全部访问。
\end{enumerate}

根据如何选择E-结点的规则不同,得到两种不同的搜索算法:深度优先搜索和广度优先搜索。
\subsection*{广度优先搜索}
对于广度优先搜索,一个结点$x$一旦成为E-结点,算法将依次访问完它的全部未访问的后继结点。每访问一个结点,就将它加入活结点表。直到$x$检测完毕,算法才从活结点表另选一个活结点作为E-结点。广度优先搜索以队列作为活结点表。

\textbf{时间分析}:$\mathrm{O}(n+e)$,如果用邻接矩阵表示图,则所需时间为$\mathrm{O}(n^2)$
\subsection*{深度优先搜索}
如果一个遍历算法在访问了E-结点$x$的某个后继结点$y$后,立即使$y$成为新的E-结点,去访问$y$的后继结点,直到完全检测结点$y$后,$x$才能再次成为E-结点,继续访问$x$的其他未访问的后继结点,这种遍历称为深度优先搜索。深度优先搜索使用堆栈为活结点表。

\textbf{时间分析}:$\mathrm{O}(n+e)$,如果用邻接矩阵表示图,则所需时间为$\mathrm{O}(n^2)$
\subsection*{双连通分量}
无向图的双连通性在网络应用中非常有价值。如果一个无向图的任意两个结点之间至少有两条不同的路径相通,则称该无向图是双连通的。

在一个无向连通图$G=(V,E)$中,可能存在某个(或多个)结点$a$,使得一旦删除$a$及其相关联的边,图$G$不再是连通图,则结点$a$称为图$G$的关节点。如果删除图$G$的某条边$b$,该图分离成两个非空子图,则称边$b$是图$G$的桥。
如果无向连通图G中不包含任何关节点,则称图G为双连通图(biconnected graph)。一个远射连通图$G$的双连通分量是图$G$的极大双连通子图。

两个双连通分量至多有一个公共结点,且此结点必为关节点。两个双连通分量不可能共有同一条边。同时还可以看到,每个双连通分量至少包含两个结点(除非无向图只有一个结点)。

\textbf{发现关节点}
在网络应用中,通常不希望网络中存在关节点,因为这意味着一旦在这些位置出现故障,势必导致大面积的通信中断。一个无向连通图不是双连通图的充要条件是图中存在关节点。在无向图中识别关节点的最简单的做法是:从图$G$中删除一个结点$a$和该结点的关联边,再检查图$G$的连通性。如果图$G$因此而不再是连通图,则结点$a$是关节点。这一方法显然太费时,

采用深度优先搜索识别无向图的关节点的方法有很好的时间性能。无向图的深度优先树中只包含树边和反向边两类边;一个双连通图中不包含关节点,即要求图中任意一对结点之间存在简单回路。因此具有下列性质 

给定无向连通图$G=(V,E),S=(V,T)$是图$G$的一棵深度优先树,图中结点$a$是一个关节点,当且仅当
\begin{enumerate}
	\item $a$是根,且$a$至少有两个孩子;
	\item 或者$a$不是根,且$a$的某棵子树上没有指向$a$的祖先的反向边。
\end{enumerate}

\textbf{构造双连通图}
\subsection*{与或图}
很多复杂问题很难或无法直接求解,但可以将它们分解成一系列(类型可以不同)子问题。这些子问题又可进一步分解成一些更小的子问题,这种问题分解过程可以一直进行下去,直到所生成的子问题已经足够简单,可用一些已知的普通求解方法求解为止。然后由这些子问题的解再逐步导出原始问题的解。这种将一个问题分解成若干个子问题,继而分别求解子问题,最后又从子问题的解导出原始问题的解的方法称为问题归约。

\subsection*{本章小结}
搜索一个数据结构就是以一种系统的方式访问该数据结构中的每一个结点。对树和图的搜索和遍历是许多算法的核心。许多人工智能问题的求解过程就是搜索状态空间树。通过系统地检查问题的状态空间树中的状态,寻找一条从起始状态到答案状态的路径作为搜索算法的解。搜索和遍历也是许多重要图算法基础。通过对图的搜索获取图的结构信息来求解问题。另一些图算法实际上是由基本的图搜索算法经过简单的扩充而成的。本章讨论图的搜索与遍历。回溯法和分枝限界法将介绍问题求解的状态空间树搜索方法。