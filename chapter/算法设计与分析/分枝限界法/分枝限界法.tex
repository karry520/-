通过搜索状态空间树的方法求问题答案的方法可分为两类:深度优先搜索(DFS)和广度优先搜索(BFS、D-检索)。如果在运用搜索算法时使用剪枝函数,便成为回溯法和分枝限界法。一般而言,回溯法的求解目标是在状态空间树上找出满足约束条件的所有解,而分枝限界法的求解目标则是满足约束条件的一个解,或是在满足约束条件的解中找出最优解。

\section{一般方法}
\subsection*{分析限界法概述}
采用广度优先产生状态空间树的结点,并使用剪枝函数的方法称为分枝限界法。按照广度优先的原则,一个活结点一旦成为扩展结点(E-结点)R后,算法将依次生成它的全部孩子结点,将将它们一一加入活结点表,此时R自身成为死结点。算法从活结点表中另选一个活结点作为E-结点。

不同的活结点表形成不同的分枝限界法。分为
\begin{enumerate}
	\item FIFO分枝限界法
	\item LIFO分枝限界法
	\item LC分枝限界法
\end{enumerate}
回溯法、FIFO和LIFO分枝限界法从活结点表中选择一个活结点,作为新E-结点的做法是盲目的,它们只是机械地按照FIFO或LIFO原则选取下一个活结点。使用LC分枝限界法可根据每个活结点的优先权进行选择。

\subsection*{LC分枝限界法}
利用LC分枝限界法时,为了尽快搜索到一个答案结点,需要对活结点使用一个“有智力的”评价函数作为优先权来选择下一个E-结点。该评价函数通过衡量一个活结点的搜索代价,来确定哪个活结点能够引导尽快到达一个答案结点。
\begin{enumerate}
	\item 代价函数c($\cdot$)
	\item 相对代价函数g($\cdot$)
	\item 相对代价估计函数$\hat{g}(\cdot)$
	\item 代价估计函数$\hat{c}(\cdot)$
\end{enumerate}
\subsection*{15谜问题}
在一个$4\times 4$的方形棋盘上放置了15块编了号的牌,还剩下一个空格,考察如何从初始排列到目标排列。
\begin{theorem}{}{}
	对给定的初始状态,当且仅当$$\sum_{k=1}^{16}\mathrm{Less}(k)+i+j$$为偶数时,可以由此初始状态到达目标状态,其中i,j分别是空格在棋盘上的行和列下标。Less(i)被定义为满足下列情况的呈牌的数目:这些号牌的牌号小于i,但当前被放置在号牌i的位置之后。
\end{theorem}
\section{求最优解的分枝限界法}
\subsection*{上下界函数}
当分枝限界法用于求最优解时,需要使用上下界函数作为限界函数。下面再次使用代价函数的概念,但此外的代价函数不再是上一节的搜索代价,而是一个与最优化问题的目标函数有关的量。

定义$u(\cdot)$和$\hat{c}(\cdot)$分别是代价函数$c(\cdot)$的上界和下界函数。对所有结点X,总有$\hat{c}(X)\leq c(X)\leq u(X)$

上下界函数的作用是一种限界作用,也是一种剪枝函数。假定目标函数取最小值时为最优解,那么算法需要一个上界变量,设为U,它在算法执行中,记录迄今为止已知的关于最小代价的上界值。换句话说,最小代价答案结点的代价值不会超过U。

值得注意的是,U的值是不断修改的,它根据在搜索中获取的越来越多的关于最小代价的上界信息,使U的值逐渐逼近该最小代价值,直至最终找到最小代价答案结点。显然,U的值越接近最优解值,以$\hat{c}(X)\geq U$作为条件的剪枝操作也越有效。

基于上下界函数的分枝限界法的限界方法可描述如下:
\begin{enumerate}
	\item 如果X是答案结点,cost(X)是X所代表的可行解的目标函数值,u(X)是该子树上最小代价答案结点代价的上界值则$U=\min\{\mathrm{cost}(X),u(X)+\epsilon,U\}$
	\item 如果X代表部分向量,则$U=\min\{u(X)+\epsilon,U \}$
\end{enumerate}
\subsection*{FIFO分枝限界法}
\subsection*{LC分枝限界法}
\section{带时限的作业排序}
\subsection*{问题描述}
\subsection*{分枝限界法求解}
\subsection*{带时限作业排序算法}
\section{旅行商问题}
\subsection*{问题描述}
\subsection*{分枝限界法求解}
\section{批处理作业调度}
\subsection*{问题描述}
\subsection*{分枝限界法求解}
\subsection*{批处理作业调度算法}
