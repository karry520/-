问题分解是求解复杂问题时很自然的做法。求解一个复杂问题可以将其分解成若干个子问题,子问题还可以进一步分解成更小的问题,直到分解所得的小问题是一些基本问题,并且其求解方法是已知的,可以直接求解为止。分治法作为一种算法设计策略,要求分解所得的子问题是同类问题,并要求原问题的解可以通过组合子问题的解来获取。
本章首先介绍分治法的一般方法,它的算法框架,算法分析的递推关系。本章以后各小节将通过若干常见的分治算法问题,如二分搜索、选择问题和矩阵相乘问题等,加深对分治法所能求解的问题特征的理解,并学会运用公法策略来求解问题的方法。分析递归算法的时间复杂度是本章的另一项任务。
\section{一般方法}
分治法顾名思义就是分而治之。一个问题能够用分治法求解的要素是:第一,问题能够按照某种方式分解成若干个规模较小、相互独立且与原问题类型相同的子问题;第二,子问题足够小时可以直接求解;第三,能够将子问题的解组合成原问题的解。因此,分治法求解很自然的导致一个递归算法。
\lstinputlisting[language=c++]{./chapter/算法设计与分析/code/Divide1-1.cpp}
\subsection*{算法分析}
采用分治法求解问题通常得到一个递归算法。往往可得到如下的递推关系式:
\begin{equation}
	T(n)=aT(n/b)+cn^k,\ T(1)=c
\end{equation}
使用主方法得到下面的定理
\begin{theorem}{}{}
	\begin{equation}
		T(n)=\begin{cases}
			\Theta(n^{\log_b^a})\quad &\text{如果}a>b^k\\
			\Theta(n^k\log n)\quad &\text{如果}a=b^k\\
			\Theta(n^k)\quad &\text{如果}a<b^k\\
		\end{cases}
	\end{equation}
\end{theorem}
\section{求最大最小元}
讨论用分治法在一个元素集合中寻找最大元素和最小元素的问题。
\lstinputlisting[language=c++]{./chapter/算法设计与分析/code/MaxMin.cpp}
\subsection*{时间分析}
\begin{equation}
	T(n)=3n/2-2
\end{equation}
\section{二分搜索}
搜索运算是数据处理中经常使用的一种重要运算。在一个表中搜索确定一个关键字值为给定值的元素是一种常见的运算。若表中存在这样的元素,则称搜索成功,搜索结果可以返回整个数据元素,也可指示该元素在表中的位置;若表中不存在关键字值等于给定值的元素,则称搜索不成功。本节讨论采用分治法求解在有序表中搜索给定元素的问题。

二分搜索框架
\lstinputlisting[language=c++]{./chapter/算法设计与分析/code/BSearch.cpp}
\subsection*{搜索算法的时间下界}
在一个有$n$个元素的集合中,通过关键字值之间的比较,搜索指定关键字值的元素,任意这样的算法在最坏情况下至少需要进行$\lfloor \log n\rfloor +1$次比较。
\section{排序问题}
\subsection*{归并排序}
归并排序的基本算法是把两个或多个有序序列合并成一个有序序列。使用分治法的两路归并排序算法可描述为:将待排序的元素序列一分为二,得到两个长度基本相等的子序列,类似对半搜索的做法;然后对两个子序列分别排序,如果子序列较长,还可继续细分,直到子序列的长度不超过1为止;当分解所得的子序列已排列有序时,将两个有序子序列合并成一个有序子序列,实现将子问题的解组合成原问题解。
\lstinputlisting[language=c++]{./chapter/算法设计与分析/code/MergeSort.cpp}
\subsection*{快速排序}
快速排序又称分划交换排序。当使用分治法设计排序算法时,可以采取与两路合并排序完全不同的方式对问题进行分解。分划操作是快速排序的核心操作。
\lstinputlisting[language=c++]{./chapter/算法设计与分析/code/QuickSort.cpp}

合并排序和快速排序虽然都运用分治策略,但两者的角度不同,得到的排序算法也不同。可见,一种算法设计策略提供了一种设计算法的启示。对于同一问题,基于同一算法设计策略,算法设计者可以根据各自对问题的理解和分析,提出不同的具体解决方法设计出不同的算法。
\subsection*{排序算法的时间下界}
任何一个通过关键字值比较对$n$个元素进行排序的算法,在最坏情况下,至少需要做$(n/4)\log n$次比较。
\section{选择问题}
选择问题是指在$n$个元素的集合中,选出某个元素值大小在集合中处于第k位的元素,即所谓的求第k小元素的问题。

\subsection*{分治法求解}
如果使用快速排序中所采用的分划方法,以主元为基准,将一个表划分成左右两个子表,左子表中的所有元素均小于或等于主元,而右子表中的元素均大于或等于主元。设其左子表长度为p,那么k=p,则主元就是第k小元素;否则若k<p,第k小元素必定在左子表中,否则就在右子表中。
\section{斯特拉森矩阵乘法}
普通的矩阵相乘算法的时间复杂度为$\Theta (n^3)$。斯特拉森分治法是一种尝试,他的巧妙设计使得矩阵乘法在计算时间的数量级上得到突破,成为$\mathrm{O}(n^{2.81})$。
\section{本章小结}
分治法是一种非常有用的算法设计技术,它可用于求解许多算法问题。分治法设计的算法一般是递归的。分析递归算法的时间得到一个递推方程。递推方程可使用替换方法、迭代方法和主方法求解。

本章通过对最大最小元、二分搜索、排序、选择及斯特拉森矩阵相乘等典型示例的讨论,详细介绍了如何运用法设计算法的方法,以及分析算法的时间和空间效率的方法。在按照分治法的要素分析一个问题时,如果分析问题的角度不同,则可以得到完全不同的算法。快速排序和合并排序算法说明了这一点。

本章还讨论了基于元素间比较的搜索问题和排序问题的时间下界。问题求解的时间下界对算法设计有指导意义。