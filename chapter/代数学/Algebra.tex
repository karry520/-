代数学思想发展的一个重要前提条件是由对于数的运算过渡到使用表示不定量的字母。数学中的这个革命是法国数学家F.韦达于16世纪后半叶完成的。

代数学的现代结构理论起源于20世纪20年代E.诺特在哥根廷以及E.阿廷在汉堡讲授的课程。1930年出版了B.L.范德瓦尔登的书**Modern Algebra**《近世代数》,这个理论第一次以专著的形式呈现于世人。事实上,该书出版了多种版本,并且至今仍然是近世代数的标准参考读物。

然而,这个工作的基础是在19世纪奠定的。高斯(分圆域)、阿贝尔(代数函数)、伽罗瓦(群论和代数方程)、黎曼(代数函数的亏格和除子)、库默尔和戴德金(理想论)、克罗内克(数域)、若尔当(群论)及希尔伯特(数域和不变量理论)对此起了重要的推动作用。
\chapter{初等代数}
\section{组合学}

\section{行列式}

\section{矩阵}

\section{线性方程组}

\section{多项式的计算}

\section{代数学基本定理(根据高斯的观点)}

\section{部分分式分解}

\chapter{矩阵}
\section{矩阵的谱}

\section{矩阵的正规形式}

\section{矩阵函数}

\chapter{线性代数}
\section{基本思想}

\section{线性空间}

\section{线性算子}

\section{线性空间的计算}

\section{对偶性}

\chapter{多线性代数}
\section{代数}

\section{多线性型的计算}

\section{泛积}

\section{李代数}

\section{超代数}

\chapter{代数结构}
\section{群}

\section{环}

\section{域}

\chapter{伽罗瓦理论和代数方程}
\section{三个著名古代问题}

\section{伽罗瓦理论的主要定理}

\section{广义代数学基本定理}

\section{域扩张的分类}

\section{根式可解方程的主定理}

\section{尺规作图}

\chapter{数论}
\section{基本思想}

\section{欧几里德算法}

\section{素数分布}

\section{加性分解}

\section{用有理数及连分数逼近无理数}

\section{超越数}

\section{对数$\pi$的应用}

\section{高斯同余式}

\section{闵可夫斯基数的几何}

\section{数论中局部-整体基本原理}

\section{理想和因子理论}

\section{对二次数域的应用}

\section{解析类数公式}

\section{一般数域的希尔伯特类域论}

