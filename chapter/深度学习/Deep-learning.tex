%\chapter{引言}
%\section{深度学习的历史趋势}

\chapter{深度前馈网络}
\section{实例:学习XOR}

\section{基于梯度的学习}

\section{隐藏单元}

\section{架构设计}

\section{反向传播和其他的微分算法}

\chapter{深度学习中的正则化}
\section{参数范数惩罚}

\section{作为约束的范数惩罚}

\section{正则化和欠约束问题}

\section{数据集增强}

\section{噪声鲁棒性}

\section{半监督学习}

\section{多任务学习}

\section{提前终止}

\section{参数绑定和参数共享}

\section{稀疏表示}

\section{Bagging和其他集成方法}

\section{Dropout}

\section{切面距离、正切传播和流形正切分类器}

\chapter{深度模型中的优化}
\section{学习和纯优化有什么不同}

\section{神经网络优化中的挑战}

\section{基本算法}

\section{参数初始化策略}

\section{自适应学习率算法}

\section{二阶近似方法}

\section{优化策略和元算法}

\chapter{卷积网络}
\section{卷积运算}

\section{动机}

\section{池化}

\section{卷积与池化作为一种无限强的先验}

\section{基本卷积函数的变体}

\section{结构化输出}

\section{数据类型}

\section{高效的卷积算法}

\section{随机或无监督的特征}

\section{卷积网络的神经科学基础}

\section{卷积网络与深度学习的历史}

\chapter{序列建模:循环和递归网络}
\section{展开计算图}

\section{循环神经网络}

\section{双向RNN}

\section{基于编码-解码的序列到序列架构}

\section{深度循环网络}

\section{递归神经网络}

\section{长期依赖的挑战}

\section{回声状态网络}

\section{渗漏单元和其他多时间尺度的策略}

\section{长短期记忆和其他门控RNN}

\section{优化长期依赖}

\section{外显记忆}

\chapter{实践方法论}
\section{性能度量}

\section{默认的基准模型}

\section{决定是否收集更多数据}

\section{选择超参数}

\section{调试策略}

\section{示例:多位数字识别}

\chapter{应用}
\section{大规模深度学习}

\section{计算机视觉}

\section{语音识别}

\section{自然语言处理}

\section{其他应用}

\chapter{线性因子模型}
\section{概率PCA和因子分析}

\section{独立成分分析}

\section{慢特征分析}

\section{稀疏编码}

\section{PCA的流形解释}

\chapter{自编码器}
\section{欠完备自编码器}

\section{正则自编码器}

\section{表示能力、层的大小和深度}

\section{随机编码器和解码器}

\section{去噪自编码器详解}

\section{使用自编码器学习流形}

\section{收缩自编码器}

\section{预测稀疏分解}

\section{自编译器的应用}

\chapter{表示学习}
\section{贪心逐层无监督预训练}

\section{迁移学习和领域自适应}

\section{半监督解释因果关系}

\section{分布式表示}

\section{得益于深度的指数增益}

\section{提供发现潜在原因的线索}

\chapter{深度学习中的结构化概率模型}
\section{非结构化建模的挑战}

\section{使用图描述模型结构}

\section{从图模型中采样}

\section{结构化建模的优势}

\section{学习依赖关系}

\section{推断和近似推断}

\section{结构化概率模型的深度学习方法}

\chapter{深度生成模型}
\section{玻尔兹曼机}

\section{受限玻尔兹曼机}

\section{深度信念网络}

\section{深度玻尔兹曼机}

\section{实值数据上的玻尔兹曼机}

\section{卷积玻尔兹曼机}

\section{用于结构化或序列输出的玻尔兹曼机}

\section{其他玻尔兹曼机}

\section{通过随机操作的反向传播}

\section{有向生成网络}

\section{从自编码器采样}

\section{生成随机网络}

\section{其他生成方案}

\section{评估生成模型}

\section{结论}

