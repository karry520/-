本章研究非线性规划的最优解所满足的必要条件和充分条件。这些条件很重要,它们将为各种算法的推导和分析提供必不可少的理论基础。
\section{无约束问题的极值条件}
考虑非线性规划问题
\begin{equation}
	\mathrm{min}\ f(\boldsymbol{x}),\quad  \boldsymbol{x}\in \mathbb{R}^n
\end{equation}
其中$f(\boldsymbol{x})$是定义在$\mathbb{R}^n$上的实函数。称为无约束极值问题。

\begin{theorem}{}{}
	\begin{enumerate}
		\item (必要条件)
		\begin{enumerate}
			\item 设函数$f(\boldsymbol{x})$在点$\bar{\boldsymbol{x}}$可微,若$\bar{\boldsymbol{x}}$是局部极小点,则梯度$\bigtriangledown f(\boldsymbol{x})=0$
			\item 设函数$f(\boldsymbol{x})$在点$\bar{\boldsymbol{x}}$二次可微,若$\bar{\boldsymbol{x}}$是局部极小点,则梯度$\bigtriangledown f(\boldsymbol{x})=0$,并且Hesse矩阵$\bigtriangledown^2f(\bar{\boldsymbol{x}})$半正定。
		\end{enumerate}
		 
		\item (充分条件) 
		设函数$f(\boldsymbol{x})$在点$\bar{\boldsymbol{x}}$二次可微,若梯度$\bigtriangledown f(\boldsymbol{x})=0$,且Hesse矩阵$\bigtriangledown^2f(\bar{\boldsymbol{x}})$正定,则$\bar{\boldsymbol{x}}$是局部极小点
		\item (充要条件)
		设$f(\boldsymbol{x})$是定义在$\mathbb{R}^n$上的可微凸函数,$\bar{\boldsymbol{x}}\in \mathbb{R}^n$,则$\bar{\boldsymbol{x}}$为全局极小点的充分必要条件是梯度$\bigtriangledown f(\boldsymbol{x})=0$
	\end{enumerate}
	
\end{theorem}
\section{约束极值问题的最优性条件}