\section{单纯形方法} 
\subsection*{单纯形方法原理}
若线性规划(标准形式)有最优解,则必存在最优基本可行解。因此求解线性规划问题归结为找最优基本可行解。单纯形方法的基本思想,就是从一个基本可行解出发,求一个使目标函数值有所改善的基本可行解;通过不断改进基本可行解,力图达到最优基本可行解。

\textbf{考虑问题}:
\begin{equation}
	\begin{aligned}
	min \quad & f \overset{def}{=} cx \\
	s.t. \quad & Ax = b, \\
	\quad & x \geq 0,
	\end{aligned}
\end{equation}
其中$A$是$m\times n$矩阵,秩为$m,c$是$n$维行向量,$x$是$n$维列向量,$b\geq 0$是$m$维列向量。
记
\begin{equation}
	A = (p_1,p_2,\dots ,p_n)
\end{equation}
现将$A$分解成$(B,N)$,使得其中$B$是基矩阵,$N$是非基矩阵,设
\begin{equation}
	x^{(0)} = 
	\begin{bmatrix}
	B^{-1}b \\ 0
	\end{bmatrix}
\end{equation}
是基本可行解,在$x^{(0)}$处的目标函数值
\begin{equation}
	\begin{aligned}
	f_0 &= cx^{(0)} = (c_B,c_N) \begin{bmatrix} B^{-1}b \\ 0 \end{bmatrix} \\
	&= c_BB^{-1}b,
	\end{aligned} 
\end{equation}
现在分析怎样从基本可行解$x^{(0)}$出发,求一个改进的基本可行解,设
\begin{equation}
	x = \begin{bmatrix} x_B \\ x_N \end{bmatrix}
\end{equation}
是任一个可行解,则由$Ax = b$得到 
\begin{equation}
	\begin{aligned}
	AX &= (B, N)\begin{pmatrix}x_B \\ x_N \end{pmatrix} = Bx_B + Nx_N = b \\ 
	\implies x_B &= B^{-1}b - B^{-1}Nx_N 
	\end{aligned}
\end{equation}
在点$x$处的目标函数值
\begin{equation}
	\begin{aligned}
	f &= cx = (c_B,c_N) \begin{bmatrix} B^{-1}b \\ 0 \end{bmatrix} \\
	&= c_Bx_B + c_Nx_N \\
	&= c_B(B^{-1}b - B^{-1}Nx_N) + c_Nx_N \\
	&= c_BB^{-1}b - (c_BB^{-1}N - c_N)x_N \\ 
	&= f_0 - \sum_{j\in R}(c_BB^{-1}p_j - c_j)x_j \\
	&= f_0 - \sum_{j\in R}(z_j - c_j)x_j 
	\end{aligned}
\end{equation}
适当选取自由未知量$x_j(j\in R)$的数值就有可能使得
\begin{equation}
	\sum_{j \in R}(z_j - c_j)x_j > 0
\end{equation}
这里假设$z_k - c_k > 0,x_k$由零变为正数后,得到方程组$Ax = b$的解
\begin{equation}
	x_B = B^{-1}b - B^{-1}p_kx_k = \bar{b} - y_kx_k
\end{equation}
其中$\bar{b}$和$y_k$是$m$维列向量,$\bar{b} = B^{-1}b,y_k = B^{-1}p_k$。
从而得到使目标函数值减少的新的基本可行解。因此选择$x_k$,使
\begin{equation}
	z_k - c_k = max_{j\in R}\{ z_j - c_j \}
\end{equation}
在新得到的点,目标函数值是
\begin{equation}
	f = f_0 - (z_k - c_k)x_k
\end{equation}
再来分析怎样确定$x_k$的取值。一方面,$x_k$取值越大函数值下降越多;另一方面,$x_k$的取值受到可行性的限制,它不能无限增大。
\begin{equation}
	x_k \leq \frac{\bar{b_i}}{y_{ik}}
\end{equation}
因此,为使$x_B \geq 0$应令
\begin{equation}
	x_k = min \left\{ \frac{\bar{b_i}}{y_{ik}} \middle | y_{ik} > 0 \right\} = \frac{\bar{b_r}}{y_{rk}}
\end{equation}
$x_k$取值$\frac{\bar{b_r}}{y_{rk}}$后,原来的基变量$x_{B} = 0$,得到新的可行解
\begin{equation}
	x = (\overbrace{x_{B_1}, \dots ,x_{B_{r-1}} ,\color{red}{0},x_{b_{r+1}}, \dots}^{x_B},\underbrace{ 0, \dots, \color{red}{x_k},0,\dots ,0}_{x_N})^T
\end{equation}
这个解一定是基本可行解。
经上述转换,$x_k$由原来的非基变量变成基变量,而原来的基变量$x_{B_r}$变成非基变量。在新的基本可行解处,目标函数值比原来减少了$(z_k - c_k)x_k$。重复以上过程,可以进一步改进基本可行解,直到所有$z_j - c_j$均非正数,以致任何一个非基变量取正值都不能使目标函数值减少时为止。

在线性规划中,通常称$z_j - c_j$为\textbf{判别数}或\textbf{检验数}。

\subsection*{单纯性方法计算步骤}
\begin{enumerate}
	\item 解$Bx_{B} = b$,求得$x_B = B^{-1}b = \bar{b}$,令$x_N = 0$,计算目标函数值$f = c_Bx_B$;
	\item 求单纯性乘子$w$,解$wB = c_B$,得到$w = c_BB^{-1}$。对于所有非基变量,计算判别数$z_j - c_j = wp_j - c_j$令
	$$z_k - c_k = max_{j \in R}\{ z_j - c_j \}$$
	若$z_k - c_k \leq 0$,则对于所有非基变量$z_j - c_j \leq 0$,对应基变量的判别数总是零,因此停止计算,现行基本可行解是最优解。否则,进行下一步;
	\item 解$By_k = p_k$,得到$y_k = B^{-1}p_k$,若$y_k \leq 0$,即$y_k$的每个分量均非正数,则停止计算,问题不存在有限最优解。否则,进行步骤(4);
	\item 确定下标$r$,使
	\begin{equation}
		\frac{\bar{b_r}}{y_{rk}} = min \left\{ \frac{\bar{b_i}}{y_{ik}} \middle | y_{ik} > 0 \right\}
	\end{equation}
	$x_{B_r}$为离基变量,$x_k$为进基变量,用$p_k$替换$p_{B_r}$,得到新的基矩阵$B$,返回步骤(1)。
\end{enumerate}
\subsection*{收敛性}
对于非退化问题,单纯性方法经有限次迭代或达到最优基本可行解,或得出无界的结论。