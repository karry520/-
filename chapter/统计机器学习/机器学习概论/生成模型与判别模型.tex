\section{生成模型与判别模型}
监督学习方法又可以分为生成方法(generative approach)和判别方法(discriminative approach)。所学到的模型分别称为生成模型和判别模型。

\textbf{生成方法}由数据学习联合分布$P(X,Y)$,然后求出条件概率分布$P(Y|X)$作为预测的模型,即生成模型:
\begin{equation}
	P(Y|X) = \frac{P(X,Y)}{P(X)}
\end{equation}
这样的方法之所以称为生成方法,是因为模型表示了给定输入X产生输出Y的生成关系。

\textbf{判别方法}由数据直接学习决策函数$f(X)$或者条件概率分布$P(Y|X)$作为预测的模型,即判模型。判别方法关心的是对给定的输入X,应该预测什么样的输出Y。

生成方法的特点:生成方法可以还原出联合概率分布P(X|Y),而判别方法则不能;生成方法的学习收敛速度更快,即当样本容量增加的时候,学到的模型可以更快地收敛于真实模型;当存在隐变量时,仍可以用生成方法学习,此时判别方法就不能用。

判别方法的特点:判别方法直接学习的是条件概率P(Y|X)或决策函数$f(X)$,直接面对预测,往往学习的准确率更高;由于直接学习P(Y|X)或$f(X)$,可以对数据进行各种程度上的抽象、定义特征并使用特征,因此可以简化学习问题。