\section{条件混合模型}
标准的决策树被限制为对输入空间的硬的、与坐标轴对齐的划分。这些限制可以通过引入软的、概率形式的划分的方式得到缓解,这些划分是所有输入变量的函数,而不仅仅是某个输入变量的函数。这样做的代价是它的直观意义的消失。如果我们也给叶结点的模型赋予一个概率的形式,那么我们就得到了一个纯粹的概率形式的基于树的模型,被称为专家层次混合(hierarchical mixture of experts)。

另一种得到专家层次混合模型的方法是从标准的非条件密度模型(例如高斯分布)的概率混合开始,将分量概率密度替换为条件概率分布。这里,我们考虑线性回归模型的混合以及Logistic回归模型的混合。在最简单的情况下,混合系数与输入变量无关。如果我们进行进一步的泛化,使得混合系数同样依赖于输入,那么我们就得到了专家混合(mixture of experts)模型。最后,如果我们使得混合模型的每个分量本身都是一个专家混合模型,那么我们就得到了专家层次混合模型。