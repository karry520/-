在之前的章节中,我们研究了一系列不同的模型用于解决分类问题和回归问题。经常发现的一件事情是,我们可以通过以某种方式将多个模型结合到一起的方法来提升性能,而不是独立地使用一个单独的模型。例如,我们可以训练L个不同的模型,然后使用每个模型给出的预测的平均值进行预测。这样的模型组合有时被称为委员会(committee)。

委员会方法的一个重要的变体,被称为提升方法(boosting)。这种方法按顺序训练多个模型,其中用来训练一个特定模型的误差函数依赖于前一个模型的表现。与单一模型相比,这个模型可以对性能产生显著的提升。

与对一组模型的预测求平均的方法不同,另一种形式的模型组合是选择一个模型进行预测,其中模型的选择是输入变量的一个函数。因此不同的模型用于对输入空间的不同的区域进行预测。这种方法的一种广泛使用的框架被称为决策树(decision tree),其中选择的过程可以被描述为一个二值选择的序列,对应于对树结构的遍历。这种情况下,各个单独的模型通常被选得非常简单,整体的模型灵活性产生于与输入相关的选择过程。决策树既可以应用于分类问题也可应用于回归问题。