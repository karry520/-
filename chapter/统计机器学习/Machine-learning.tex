\chapter{机器学习概论}
\section{统计学习}

\subsection*{统计学习的特点}
\textbf{统计学习(statistical learning)}是关于计算机基于数据构建概率统计模型并运用模型对数据进行预测与分析的一门学科。
\begin{itemize}
	\item 统计学习以计算机及网络为平台;
	\item 统计学习以数据为研究对象;
	\item 统计学习的目的是对数据进行预测与分析;
	\item 统计学习以方法为中心;
	\item 统计学习是概率论、统计学、信息论、计算理论、最优化理论及计算机科学等多个领域的交叉学科 。
\end{itemize}

\subsection*{统计学习的对象}
统计学习的对象是数据(data)。它从数据出发,提取数据的特征,抽象出数据的模型,发现数据中的知识,又回到对数据的分析与预测中去。\textbf{统计学习关于数据的基本假设是同类数据具有一定的统计规律性,这是统计学习的前提}。
\subsection*{统计学习的目的}
统计学习用于对数据进行预测与分析,特别是对未知新数据进行预测与分析。对数据的预测与分析是通过构建概率统计模型实现的。统计学习总的目标就是考虑学习什么样的模型和如何学习模型,以使模型能对数据进行准确的预测与分析,同时也要考虑尽可能地提高学习效率。
\subsection*{统计学习的方法}
统计学习的方法是基于数据构建统计模型从而对数据进行预测与分析。
\begin{itemize}
	\item 监督学习(supervised learning)
	\item 非监督学习(unsupervised learning)
	\item 半监督学习(semi-supervised learning)
	\item 强化学习(reinforcement learning)
\end{itemize}
\textbf{实现统计学习方法的步骤如下:}
\begin{enumerate}[(1)]
	\item 得到一个有限的训练数据集合
	\item 确定包含所有可能的模型的假设空间,即学习模型的集合
	\item 确定模型选择的准则,即学习的策略
	\item 实现求解最优化模型的算法,即学习的算法 
	\item 通过学习方法选择最优模型
	\item 利用学习的最优模型对新数据进行预测或分析
\end{enumerate}

\subsection*{统计学习的研究}
统计学习方法(statistical learning method),旨在开发新的学习方法。

统计学习理论(statistical learning theory),旨在探求统计学习方法的有效性与效率。

统计学习应用(application of statistical learning),旨在将统计学习方法应用到实际问题中去,解决实际问题。
\subsection*{统计学习的重要性}
统计学习是处理海量数据的有效方法;
统计学习是计算机智能化的有效手段;
统计学习是计算机科学习发展的一个重要组成部分。





\section{统计学习三要素}
\begin{enumerate}[(1)]
	\item 模型:统计学习首先要考虑的问题是学习什么样的模型
	\item 策略:有了模型的假设空间,统计学习接着需要考虑的是按照什么样的准则学习或选择最优的模型
	\item 算法 :算法是指学习模型的具体计算方法
\end{enumerate}
\section{模型评估与模型选择}

\section{正则化与交叉验证}
模型选择的典型方法是正则化(regularization)。正则化是结构风险最小化策略的实现,是在经验风险上加一个正则化项(regularizer)和罚项(penalty term)
\begin{equation}
	\mathop{min}\limits_{f\in F}\quad \frac{1}{2}\sum_{i=1}^NL(y_i, f(x_i)) + \lambda J(f)
\end{equation}
另一种常用的模型选择方法是交叉验证(cross validation)
训练集用来训练模型
验证集用来选择模型 
测试集用来对模型进行评估
\begin{enumerate}
	\item 简单交叉验证
	\item S折交叉验证
	\item 留一交叉验证
\end{enumerate}

\section{生成模型与判别模型}

\chapter{回归的线性模型}
\section{线性基函数模型}
回归问题的最简单模型是输入变量的线性组合
$$
y(\boldsymbol{x,w})=w_0+w_1x_1 + \dots w_Dx_D
$$
其中$x=(x_1,\dots,x_D)^T$。这通常被简单地称为\textbf{线性回归(linear regression)}。这个模型的关键性质是它是参数$w_0,\dots,w_D$的一个线性函数。但是,它也是输入变量$x_i$的一个线性函数,这给模型带来了极大的局限性。因此,我们扩展模型的类别:将输入变量的固定的非线性函数进行线性组合,形式为
$$
y(\boldsymbol{x,w})=w_0+\sum_{y=1}^{M-1}w_j\phi_j(\boldsymbol{x})=\sum_{j=0}^{M-1}w_j\phi_j(\boldsymbol{x})=\boldsymbol{w}^T\phi(\boldsymbol{x})
$$
其中$\phi_j(\boldsymbol{x})$被称为基函数(basis function)。
\begin{enumerate}
	\item 高斯基函数 
	$$
	\phi_j(x)=exp\left\{-\frac{(x-\mu_j)^2}{2s^2}\right\}
	$$
	\item sigmoid基函数 
	$$
	\phi_j(x)=\frac{1}{1+exp\left(\frac{x-\mu_j}{s}\right)}
	$$
	\item 傅里叶基函数
\end{enumerate}
\section{偏置-方差分解}
目前为止,我们对于回归的线性模型的讨论中,我们假定了基函数的形式和数量都是固定的。如果使用有限规模的数据集来训练复杂的模型,那么使用最大似然法,或者等价地使用最小平方法,会导致严重的过拟合问题。正如前面所说,过拟合现象确实是最大似然方法的一个不好的性质。但是当我们在使用贝叶斯方法对参数进行求和或者积分时,过拟合现象不会出现。从贝叶斯观点讨论模型的复杂度之前,从频率学家的观点考虑一下模型的复杂度的问题——偏置-方差折中(bias-variance trade-off)。

最优的预测由条件期望(记作$h(x)$)给出,即
\begin{equation}
	h(x)=\mathbb{E}[t|x]=\int tp(t|x)dt
\end{equation}
平方损失函数的期望可以写成
\begin{equation}
\label{llk}
\begin{aligned}
	\mathbb{E}[L]&=\iint \{y(x)-h(x)\}^2p(x,t)dxdt\\
	&=\int \{y(x)-h(x) \}^2p(x)dx + \iint \{h(x)-t\}^2p(x,t)dxdt\\
\end{aligned}
\end{equation}
考虑第一项的被积函数,对于一个特定的数据集$D$,它的形式为
\begin{equation}
	\{y(x;D)-h(x)\}^2
\end{equation}
由于这个量与特定的数据集$D$相关,因此我们对所有的数据集取平均。如果我们在括号内加上然后减去$\mathbb{E}_D[y(x;D)]$,然后展开,我们有
\begin{equation}
	\begin{aligned}
		\{y(x;D)&-\mathbb{E}_D[y(x;D)]+\mathbb{E}_D[y(x;D)] -h(x) \}^2\\
		&=\{y(x;D)-\mathbb{E}_D[y(x;D)]\}^2\\
		&+\{\mathbb{E}_D[y(x;D)]-h(x) \}^2\\
		&+ 2\{y(x;D)-\mathbb{E}_D[y(x;D)]\}\{\mathbb{E}_D[y(x;D)]-h(x) \}
	\end{aligned}
\end{equation}
现在关于$D$求期望,然后注意到\textbf{最后一项等于零},可得
\begin{equation}
\label{kk}
\begin{aligned}
	\mathbb{E}_D&[\{y(x;D)-h(x) \}^2]\\
	&=\underbrace{\mathbb{E}_D[y(x;D)]-h(x)\}^2}_{(\text{偏置})^2}+\underbrace{\mathbb{E}_D[\{y(x;D)-\mathbb{E}_D[y(x;D)] \}^2]}_{\text{方差}}
\end{aligned}
\end{equation}
将式$\ref{kk}$代入式$\ref{llk}$中,就得到了对于期望平方损失的分解
\begin{equation}
	\text{期望损失}=\text{偏置}^2+\text{方差}+\text{噪声}
\end{equation}
其中
\begin{flalign}
	\text{偏置}^2&=\int \{\mathbb{E}_D[y(x;D)]-h(x)\}^2p(x)dx\\
	\text{方差}&=\int \mathbb{E}_D[\{y(x;D)-\mathbb{E}_D[y(x;D)] \}^2]p(x)dx\\
	\text{噪声}&=\iint \{h(x)-t\}^2p(x,t)dxdt
\end{flalign}

我们的目标是最小化期望损失,它可以分解为(平方)偏置、方差和一个常数噪声项的和。对于非常灵活的模型来说,偏置较小,方差较大。对于相对固定的模型来说,偏置较大,方差较小。有着最优预测能力的模型是在偏置和方差之前取得最优的平衡的模型。
\section{贝叶斯线性回归}
使用最大似然方法设置线性回归模型的参数时,由基函数的数量控制的模型的复杂度需要根据数据集的规模进行调整。为对数似然函数增加一个正则化项意味着模型的复杂度可以通过正则化系数的值进行控制,虽然基函数的数量和形式的选择仍然对于确定模型的整体行为十分重要。

这就产生了对于特定的应用确定合适的模型复杂度的问题。这个问题不能简单地通过最大化似然函数来确定,因为这总会产生过于复杂的模型和过拟合现象。独立的额外数据能够用来确定模型的复杂度,但这需要较大的计算量,并且浪费了有价值的数据。因此我们转而考虑线性回归的贝叶斯方法,这会避免最大似然的过拟合问题,也会引出使用训练数据本身确定模型复杂度的自动化方法。
\subsection*{参数分布}
关于线性拟合的贝叶斯方法的讨论,我们首先引入模型参数$w$的先验概率分布。把噪声精度参数$\beta$当做已知常数。对应的共轭先验是高斯分布。
\begin{equation}
	p(w)\equiv \mathcal{N}(w|\boldsymbol{m}_0,\boldsymbol{S}_0)
\end{equation}
均值为$\boldsymbol{m}_0$协方差为$\boldsymbol{S}_0$
后验分布
\begin{equation}
	p(w|\boldsymbol{t})=\mathcal{N}(w|\boldsymbol{m}_N,\boldsymbol{S}_N)
\end{equation}
其中,
\begin{flalign}
\label{fafa}
	\boldsymbol{m}_N&=\boldsymbol{S}_N(\boldsymbol{S}_0^{-1}+\beta\Phi^T\boldsymbol{t})\\
	\boldsymbol{S}_N^{-1}&=\boldsymbol{S}_0^{-1}+\beta\Phi^T\Phi
\end{flalign}
为了简化起见,考虑高斯先验的一个特定的形式——零均值各向同性高斯分布。这个分布由一个精度参数$\alpha$控制,即
\begin{equation}
	p(\boldsymbol{w}|\alpha)=\mathcal{N}(\boldsymbol{w}|0,\alpha^{-1}I)
\end{equation}
后验概率分布的对数由对数似然函数与先验的对数求和的方式得到。它是$w$的函数,形式为
\begin{equation}
	\ln p(\boldsymbol{w}|\boldsymbol{t})-\frac{\beta}{2}\sum_{n=1}^{N}\{t_n-\boldsymbol{w}^T\phi(x_n) \}^2-\frac{\alpha}{2}\boldsymbol{w}^T\boldsymbol{w}+\text{常数}
\end{equation}
于是,后验分布关于$\boldsymbol{w}$的最大化等价于对平方和误差函数加上一个二次正则项进行最小化。
\subsection*{预测分布}
在实际应用中,我们通常感兴趣的不是$\boldsymbol{w}$本身的值,而是对于新的$\boldsymbol{x}$值预测出$t$的值。这需要计算出预测分布(predictive distribution),定义为
\begin{equation}
	p(t|\boldsymbol{t},\alpha,\beta)=\int p(t|\boldsymbol{w},\beta)p(\boldsymbol{w}|\boldsymbol{t},\alpha,\beta)d\boldsymbol{w}
\end{equation}
其中$\boldsymbol{t}$是训练数据的目标变量的值组成的向量。并且,为了简化记号,我们在右侧省略了条件概率中出现的输入向量。预测分布的形式为
\begin{equation}
	p(t|\boldsymbol{x},\boldsymbol{t},\alpha,\beta)=\mathcal{N}(t|\boldsymbol{m}^T_N\phi(\boldsymbol{x}),\sigma^2_N(\boldsymbol{x}))
\end{equation}
其中预测分布的方差$\sigma_N^2(\boldsymbol{x})$为
\begin{equation}
\label{aaa}
	\sigma_N^2(\boldsymbol{x})=\frac{1}{\beta}+\phi(\boldsymbol{x})^T\boldsymbol{S}_N\phi(\boldsymbol{x})
\end{equation}
公式$\ref{aaa}$的第一项表示数据中的噪声,而第二项反映了与参数$\boldsymbol{w}$关联的不确定性。由于噪声和$\boldsymbol{w}$的分布是相互独立的高斯分布,因此它们的值是可以相加的。注意,当额外的数据点被观测到的时候,后验概率分布会变窄。从而可以证明出$\sigma^2_{N+1}(\boldsymbol{x})\leqslant \sigma_N^2(\boldsymbol{x})$在极限$N\to \infty$的情况下,公式$\ref{aaa}$的第二项趋于零,从而预测分布的方差只与参数$\beta$控制的具有可加性的噪声有关。
\subsection*{等价核}
把公式$\ref{fafa}$代入线性基函数模型中,可以写成下面的形式
\begin{equation}
\begin{aligned}
	y(\boldsymbol{x},\boldsymbol{m}_N)&=\boldsymbol{m}_N^T\phi(\boldsymbol{x})\\
	&=\beta\phi(\boldsymbol{x}^T\boldsymbol{S}_N\Phi^T\boldsymbol{t})\\
	&=\sum_{n=1}^{N}\underline{\beta\phi(\boldsymbol{x})^T\boldsymbol{S}_N\phi(\boldsymbol{x}_n)}t_n\\
	&=\sum_{n=1}^{N}k(\boldsymbol{x},\boldsymbol{x}_n)t_n
\end{aligned}
\end{equation}
可以看成在点$\boldsymbol{x}$处的预测均值由训练集目标变量$t_n$的线性组合给出。函数$k(x,x_n)$被称为平滑矩阵(smoother matrix)或者等价核(equivalent kernel)。像这样的回归函数,通过对训练集里目标值进行线性组合做预测,被称为线性平滑(linear smoother)。核函数$k(x,x^{'})$给出了$x$与$x^{'}$的函数关系。可以看到,$x$处的预测分布的均值$y(x,\boldsymbol{m}_N)$可以通过对目标值加权组合的方式获得。距离$x$较近的数据点可以赋一个较高的权值,而距离$x$较远的数据点可以赋一个较低的权值。这种局部性不仅对于局部的高斯基函数成立,对于非局部的多项式基函数和sigmoid基函数也成立。

考虑$y(\boldsymbol{x})$和$y(\boldsymbol{x^{'}})$的协方差
\begin{equation}
	\begin{aligned}
	cov[y(\boldsymbol{x}),y(\boldsymbol{x^{'}})]&=cov[\phi(\boldsymbol{x})^T\boldsymbol{w},\boldsymbol{w}^T\phi(\boldsymbol{x^{'}})]\\
	&=\phi(\boldsymbol{x})^T\boldsymbol{S}_N\phi(\boldsymbol{x^{'}})\\
	&=\beta^{-1}k(\boldsymbol{x},\boldsymbol{x^{'}})
	\end{aligned}
\end{equation}
根据等价核的形式,我们可以看到在附近的点处的预测均值相关性较高,而对于距离较远的点处,相关性就较低。

用核函数表示线性回归给出了解决回归问题的另一种方法。我们不引入一组基函数(它隐式地定义了一个等价的核),而是直接定义一个局部的核函数,然后在给定观测数据集的条件下,使用这个核函数对新的输入变量$\boldsymbol{x}$做预测。这就引入了用于回归问题(以及分类问题)的一个很实用的框架,被称为高斯过程(Gaussian process)。
\section{贝叶斯模型比较}

\section{证据近似}
在处理线性基函数模型的纯粹的贝叶斯方法中,我们会引入超参数$\alpha$和$\beta$的先验分布,然后通过对超参数以及参数$w$求积分的方式做预测。但是,虽然我们可以解析地求出对$w$的积分或者求出对超参数的积分,但是对所有这些变量完整地求积分是没有解析解的。这里讨论一种近似方法。这种方法中,首先对参数$w$求积分,得到边缘似然函数(marginal likelihood function),然后通过最大化边缘似然函数,确定超参数的值。这个框架在统计学的文献中被称为经验贝叶斯,或者被称为第二类最大似然,或者被称为推广的最大似然。在机器学习的文献中,这种方法也被称为证据近似(evidence approximation)。

如果引入$\alpha$和$\beta$上的超先验分布,那么预测分布可以通过对$w,\alpha,\beta$求积分的方法得到 
\begin{equation}
	p(t|\boldsymbol{t})=\iiint p(t|\boldsymbol{w},\beta)p(\boldsymbol{w}|\boldsymbol{t},\alpha,\beta)p(\alpha,\beta|\boldsymbol{t})d\boldsymbol{w}d\alpha d\beta
\end{equation}
如果后验分布$p(\alpha,\beta|\boldsymbol{t})$在$\hat{\alpha}$和$\hat{\beta}$附近有尖峰,那么预测分布可以通过对$w$积分的方式简单地得到,其中$\alpha$和$\beta$被固定为$\hat{\alpha}$和$\hat{\beta}$
\begin{equation}
	p(t|\boldsymbol{t})\simeq p(t|\boldsymbol{t},\hat{\alpha},\hat{\beta})=\int p(t|\boldsymbol{w},\hat{\beta})p(\boldsymbol{w}|\boldsymbol{t},\hat{\alpha},\hat{\beta})d\boldsymbol{w}
\end{equation}
根据贝叶斯定理,$\alpha$和$\beta$的后验分布为
\begin{equation}
	p(\alpha,\beta|\boldsymbol{t})\propto p(\boldsymbol{t}|\alpha,\beta)p(\alpha,\beta)
\end{equation}
如果先验分布相对比较平,那么在证据框架中,$\hat{\alpha}$和$\hat{\beta}$可以通过最大化边缘似然函数$p(\boldsymbol{t}|\alpha,\beta)$来获得。
\section{固定基函数的局限性}

\chapter{分类的线性模型}
\section{判别函数}
判别函数是一个以向量$x$为输入,把它分配到K个类别中的某一个类别(记作$C_k$)的函数。
\subsection*{二分类}
线性判别函数的最简单的形式是输入向量的线性函数,即
\begin{equation}
	y(\boldsymbol{x})=\boldsymbol{w}^T\boldsymbol{x}+w_0
\end{equation}
其中$\boldsymbol{w}$被称为权向量(weight vector),$w_0$被称为偏置(bias)。对应的决策边界由$y(\boldsymbol{x})=0$确定。对于一个输入向量$\boldsymbol{x}$,如果$y(\boldsymbol{x})\geqslant 0$,那么它被分到$C_1$中,否则被分到$C_2$中。向量$\boldsymbol{w}$与决策面内的任何向量都正交,从而$w$确定了决策面的方向。任意一点$x$到决策面的距离为
\begin{equation}
	r=\frac{y(\boldsymbol{x})}{\Vert \boldsymbol{w} \Vert}
\end{equation}
\subsection*{多分类}
考虑把线性判别函数推广到$K>2$个类别。
\begin{enumerate}
	\item 1对其他。使用K-1个分类器,每个分类器用来解决一个二分类问题,把属于类别$C_k$和不属于那个类别的点分开。
	\item 1对1。引入$\frac{K(K-1)}{2}$个二元判别函数,对一对类别都设置一个判别函数。
\end{enumerate}
上述两种方法都会产生输入空间无法分类的区域。通过引入K类判别函数可以避免这些问题。K类判别函数由K个线性函数组成,形式为
\begin{equation}
	y_k(\boldsymbol{x})=\boldsymbol{w}_k^T\boldsymbol{x}+w_{k0}
\end{equation}
对于点$\boldsymbol{x}$,如果所有的$j\ne k$都有$y_k(\boldsymbol{x})>y_j(\boldsymbol{x})$,那么就把它分到$C_k$。于是类别$C_k$和$C_j$之间的决策面为$y_k(\boldsymbol{x})=y_j(\boldsymbol{x})$,并且对应于一个$(D-1)$维超平面,形式为
\begin{equation}
	(\boldsymbol{w}_k-\boldsymbol{w}_j)^T\boldsymbol{x}+(w_{k0}-w_{j0})=0
\end{equation}
这样的判别函数的决策区域总是单连通的,并且是凸的。现在介绍三种学习线性判别函数的参数的方法,即基于最小平方的方法、Fisher线性判别函数,以及感知器算法。
\subsection*{用于分类的最小平方方法}
最小平方误差函数的最小化产生了参数值的简单的解析解。使用最小平方方法的一个理由是它在给定输入向量的情况下,近似了目标值的条件期望$\mathbb{E}[t|\boldsymbol{x}]$。对于二元表示方法,条件期望由后验类概率向量给出。但不幸的是,这些概率通常很难近似。事实上,近似的过程有可能产生位于区间$(0,1)$之外的值,这是因为线性模型的灵活性很受限。

使用向量记号,我们可以很容易地把这量聚焦在一起表示,即
\begin{equation}
	\boldsymbol{y}(\boldsymbol{x})=\tilde{W}^T\tilde{\boldsymbol{x}}
	=
	\begin{pmatrix}
	w_{1,0}& w_{2,0}& \dots& w_{K,0}\\
	w_{1,1}& w_{2,1}& \dots& w_{K,1}\\
	\vdots& \vdots & \ddots& \vdots\\
	w_{1,D}& w_{2,D}& \dots& w_{K,D}
	\end{pmatrix}^T
	\begin{pmatrix}
	1\\x_1\\\vdots\\x_{D}
	\end{pmatrix}
	=
	\begin{pmatrix}
	y_0\\y_1\\\vdots \\y_{K}
	\end{pmatrix}
\end{equation}
其中$\tilde{W}$是一个矩阵,第$k$列由$D+1$维向量$\tilde{\boldsymbol{w}}_k=(w_{k0},\boldsymbol{w}_k^T)^T$组成,$\tilde{\boldsymbol{x}}$是对应的增广输入向量$(1,\boldsymbol{x}^T)^T$,它带有一个虚输入$x_0=1$。

通过最小化平方和误差函数来确定参数矩阵$\tilde{W}$,考虑一个训练数据集$\boldsymbol{x}_n,\boldsymbol{t}_n$,其中$n=1,\dots,N$,然后定义一个矩阵$\boldsymbol{T}$,它的第$n$行是向量$\boldsymbol{t}_n^T$。还定义了一个矩阵$\tilde{\boldsymbol{x}}_n^T$。这样,平方和误差函数可以写成
\begin{equation}
	E_D(\tilde{W})=\frac{1}{2}Tr\{(\tilde{X}\tilde{W}-\tilde{\boldsymbol{T}})^T(\tilde{X}\tilde{W}-\boldsymbol{T}) \}
\end{equation}
令上式关于$\tilde{W}$的导数等于零,整理,可以得到$\tilde{W}$的解,形式为
\begin{equation}
	\tilde{W}=(\tilde{X}^T\tilde{X})^{-1}\tilde{X}^T\boldsymbol{T}=\tilde{X}^{\dag}\boldsymbol{T}
\end{equation}
这样,我们得到了判别函数,形式为
\begin{equation}
	y(\boldsymbol{x})=\tilde{W}^T\tilde{\boldsymbol{x}}=\boldsymbol{T}^T(\tilde{X}^{\dag})^T\tilde{\boldsymbol{x}}
\end{equation}

最小平方方法对于判别函数的参数给出了精确的解析解。但是,最小平方解对于离群点缺少鲁棒性。最小平方方法对应于高斯条件分布假设下的最大似然法,而二值目标向量的概率分布显然不是高斯分布。通过使用更恰当的概率模型,我们会得到性质比最小平方方法更好的分类方法。
\subsection*{Fisher线性判别函数}
Fisher提出的思想是最大化一个函数,这个函数能够让类均值的投影分开得较大,同时让每个类别内部的方差较小,从而最小人化了类别的重叠。

首先考虑一个二分类问题,这个问题中有$C_1$类的$N_1$个点,以及$C_2$类的$N_2$个点。因此两类的均值向量为
\begin{equation}
\label{kka}
	\boldsymbol{m}_1=\frac{1}{N_1}\sum_{n\in C_1}\boldsymbol{x}_n,\qquad \boldsymbol{m}_2=\frac{1}{N_2}\sum_{n\in C_2}\boldsymbol{x}_n
\end{equation}
如果投影到$\boldsymbol{w}$上,那么最简单的度量类别之间分开程度的方式就是类别均值投影之后的距离。这说明,我们可以选择$\boldsymbol{w}$使得下式取得最大值
\begin{equation}
	m_2-m_1=\boldsymbol{w}^T(\boldsymbol{m}_2-\boldsymbol{m}_1)
\end{equation}
其中
\begin{equation}
\label{kkb}
	m_k=\boldsymbol{w}^T\boldsymbol{m}_k
\end{equation}
是来自类别$C_k$的投影数据的均值。但是,通过增大$\boldsymbol{w}$,这个表达式可以任意大。为了解决这个问题,我们可以将$\boldsymbol{w}$限制为单位长度,即$\sum_iw_i^2=1$。使用拉格朗日乘数法来进行有限制条件的最大化问题的求解,我们可以发现$\boldsymbol{w}\propto (\boldsymbol{m}_2-\boldsymbol{m}_1)$。

来自类别$C_k$的数据经过变换后的类内方差为
\begin{equation}
	s_k^2=\sum_{n\in C_k}(\boldsymbol{w}^T\boldsymbol{x}_n-m_k)^2
\end{equation}
可以把整个数据集的总的方差定义为$s_1^2+s_2^2$。Fisher准则根据类间方差和类内方差的比值定义,即
\begin{equation}
	J(\boldsymbol{w})=\frac{(m_2-m_1)^2}{s_1^2+s_2^2}
\end{equation}
利用公式$\ref{kka},\ref{kkb}$
\begin{equation}
	\begin{aligned}
	(m_2-m_1)^2&=\left(\frac{1}{N_2}\sum_{n\in C_2}\boldsymbol{w}^T\boldsymbol{x}_n-\frac{1}{N_1}\sum_{n\in C_1}\boldsymbol{w}^T\boldsymbol{x}_n\right)^2\\
	&=\left(\boldsymbol{w}^T(\bar{x}_{C_1}-\bar{x}_{C_2}) \right)^2\\
	&=\boldsymbol{w}^T(\bar{x}_{C_1}-\bar{x}_{C_2})(\bar{x}_{C_1}-\bar{x}_{C_2})^T\boldsymbol{w} \\
	&=\boldsymbol{w}^TS_B\boldsymbol{w}\\
	s_1^2&=\frac{1}{N}\sum_{n\in C_1}(\boldsymbol{w}^T\boldsymbol{x}_n - \frac{1}{N}\sum_{n\in C_1}\boldsymbol{w}^T\boldsymbol{x}_n)(\boldsymbol{w}^T\boldsymbol{x}_n - \frac{1}{N}\sum_{n\in C_1}\boldsymbol{w}^T\boldsymbol{x}_n)^T\\
	&=\frac{1}{N}\sum_{n\in C_1}\boldsymbol{w}^T(\boldsymbol{x}_n-\bar{x}_{C_1})(\boldsymbol{x}_n-\bar{x}_{C_1})^T\boldsymbol{w}\\
	&=\boldsymbol{w}^TS_1\boldsymbol{w}\\
	s_1^2+s_2^2&=\boldsymbol{w}^T(S_1+S_2)\boldsymbol{w}\\
	&=\boldsymbol{w}^TS_W\boldsymbol{w}
	\end{aligned}
\end{equation}
显式地表达出$J(\boldsymbol{w})$对$\boldsymbol{w}$的依赖
\begin{equation}
	J(\boldsymbol{w})=\frac{\boldsymbol{w}^TS_B\boldsymbol{w}}{\boldsymbol{w}^TS_W\boldsymbol{w}}
\end{equation}
其中$S_B$是类间(between-class)协方差矩阵,$S_W$被称为类内(within-class)协方差矩阵。对该式关于$\boldsymbol{w}$求导,并令其为零,有
\begin{equation}
	\begin{aligned}
	\frac{\partial J(\boldsymbol{w})}{\partial \boldsymbol{w}}&=\frac{\partial [\boldsymbol{w}^TS_B\boldsymbol{w}\cdot (\boldsymbol{w}^TS_W\boldsymbol{w})^{-1}]}{\partial\boldsymbol{w}}\\
		&=2S_B\boldsymbol{w}(\boldsymbol{w}^TS_W\boldsymbol{w})^{-1}-2\boldsymbol{w}^TS_B\boldsymbol{w}\cdot (\boldsymbol{w}^TS_W\boldsymbol{w})^{-2}\cdot S_W\boldsymbol{w}\\
		&=0\\
		&\Rightarrow S_B\boldsymbol{w}\cdot \boldsymbol{w}^TS_W\boldsymbol{w}=\boldsymbol{w}^TS_B\boldsymbol{w}\cdot  S_W\boldsymbol{w}\\
		&\Rightarrow S_W\boldsymbol{w}=\frac{\boldsymbol{w}^TS_W\boldsymbol{w}}{\boldsymbol{w}^TS_B\boldsymbol{w}}\cdot S_B\boldsymbol{w}\\
		&\Rightarrow \boldsymbol{w} \propto S_W^{-1}S_B\boldsymbol{w}=S_W^{-1}(\boldsymbol{m}_2-\boldsymbol{m}_1)(\boldsymbol{m}_2-\boldsymbol{m}_1)^T\boldsymbol{w}
	\end{aligned}
\end{equation}
我们看到$S_B\boldsymbol{w}$总是在$\boldsymbol{m}_2-\boldsymbol{m}_1$的方向上。更重要的是,我们不关心$\boldsymbol{w}$的大小,只关心它的方向,因此
\begin{equation}
\label{fi}
	\boldsymbol{w}\propto S_W^{-1}(\boldsymbol{m}_2-\boldsymbol{m}_1)
\end{equation}
公式$\ref{fi}$的结果被称为Fisher线性判别函数(Fisher linear discriminant)。如果类内协方差是各向同性的,从而$S_W$正比于单位矩阵,那么我们看到$\boldsymbol{w}$正比于类均值的差。
\subsection*{与最小平方的关系}
最小平方方法确定线性判别函数的目标是使模型的预测尽可能地与目标值接近。相反,Fisher判别准则的目标是使输出空间的类别有最大的区分度。对于二分类问题,Fisher准则可以看成最小平方的一个特例。
\subsection*{多分类的Fisher判别函数}
\subsection*{感知器算法}
线性判别模型的另一个例子是Rosenblatt提出的感知器算法。它对应于一个二分类的模型,这个模型中,输入向量$\boldsymbol{x}$首先使用一个固定的非线性变换得到一个特征向量$\phi(\boldsymbol{x})$,这个特征向量然后被用于构造一个一般的线性模型,形式为
\begin{equation}
	y(\boldsymbol{x})=f(\boldsymbol{w}^T\phi(\boldsymbol{x}))
\end{equation}
其中非线性激活函数$f(\cdot)$是一个阶梯函数,形式为
\begin{equation}
	f(a)=\begin{cases}
		+1,\quad a\geqslant 0\\
		-1,\quad a<0
	\end{cases}
\end{equation}
向量$\phi(\boldsymbol{x})$通常包含一个偏置分量$\phi_0(\boldsymbol{x})=1$。用来确定感知器的参数$\boldsymbol{w}$的算法可以很容易地从误差函数最小化的思想中得到。一个自然的选择是误分类的模型的总数。但是,这样做会使得学习算法不会很简单,因为这样做会使误差函数变为$\boldsymbol{w}$的分段常函数,从而当$\boldsymbol{w}$的变化使得决策边界移过某个数据点时,这个函数会不连续变化。这样会不连续变化。这样做还使得使用误差函数改变$\boldsymbol{w}$的方法无法使用,因为在几乎所有的地方梯度都等于零。

因此考虑一个另外的误差函数,被称为感知器准则(perceptron criterion)。
\begin{equation}
	E_P(\boldsymbol{w})=-\sum_{n\in \mathcal{M}}\boldsymbol{w}^T\phi_nt_n
\end{equation}
其中$\phi_n=\phi(\boldsymbol{x}_n)$和$\mathcal{M}$表示所有误分类模型的集合。某个特定的误分类模型对于误差函数的贡献是$\boldsymbol{w}$空间中模式被误分类的区域中$\boldsymbol{w}$的线性函数,而在正确分类的区域,误差函数等于零。总的误差函数因此是分段线性的。我们现在对这个误差函数使用随机梯度下降算法。这样,权向量$\boldsymbol{w}$的变化为
\begin{equation}
	\boldsymbol{w}^{(\tau+1)}=\boldsymbol{w}^{(\tau)}-\eta \triangledown E_P(\boldsymbol{w})=\boldsymbol{w}^{(\tau)}+\eta\phi_nt_n
\end{equation}
感知器收敛定理(perceptron convergence theorem)表明,如果存在一个精确的解(即,如果训练数据线性可分),那么感知器算法可以保证在有限步骤内找到一个精确解。

\begin{tikzpicture}
\node {$\text{这里要画一个图}$};
\end{tikzpicture}
\section{概率生成式模型}
接下来用概率的观点考察分类问题,并且说明具有线性决策边界的模型如何通过对数据分布的简单假设得到。

首先考虑二分类的情形。类别$C_1$的后验概率可以写成
\begin{equation}
	\begin{aligned}
	p(C_1|\boldsymbol{x})&=\frac{p(\boldsymbol{x}|C_1)p(C_1)}{p(\boldsymbol{x}|C_1)+p(\boldsymbol{x}|C_2)p(C_2)}\\
	&=\frac{1}{1+exp(-a)}=\sigma(a)
	\end{aligned}
\end{equation}
其中我们定义了
\begin{equation}
	a=\ln \frac{p(\boldsymbol{x}|C_1)p(C_1)}{p(\boldsymbol{x}|C_2)p(C_2)}
\end{equation}
$\sigma(a)$是logistic sigmoid函数。“sigmoid”的意思是“S形”。这种函数有时被称为“挤压函数”,这个函数在许多分类算法中都有着重要的作用。它满足下面的对称性
\begin{equation}
	\sigma(-a)=1-\sigma(a)
\end{equation}
\begin{center}
	\begin{tikzpicture}[xscale=0.5,yscale=2]
	\draw [->] (-5,0) -- (5,0) node [right] {$x$};
	\draw [->] (0,0) -- (0,1.5) node [left] {$y$};
	\draw [dashed] (-5,1) -- (5,1) node [right]{$y=1$};
	\draw [domain=-5:5] plot (\x,{1/(1+exp(-\x))});
	\node at (0,0) [below left] {$o$};
	\node at (0,0.5) [right] {$0.5$};
	\end{tikzpicture}
\end{center}
logistic sigmoid的反函数为
\begin{equation}
	a=\ln \left(\frac{\sigma}{1-\sigma} \right)
\end{equation}
被称为logit函数。它表示两类的概率比值的对数$\ln \left[\frac{p(C_1|\boldsymbol{x})}{p(C_2|\boldsymbol{x})} \right]$,也被称为log odds函数。

对于K>2个类别的情形,我们有
\begin{equation}
\label{keke}
\begin{aligned}
	p(C_k|\boldsymbol{x})&=\frac{p(\boldsymbol{x}|C_k)p(C_k)}{\sum_jp(\boldsymbol{x}|C_j)p(C_j)}\\
	&=\frac{exp(a_k)}{\sum_jexp(a_j)}
\end{aligned}
\end{equation}
它被称为归一化指数(normalized exponential),可以被当做logistic sigmoid函数对于多类情况的推广。这里$a_k$被定义为
\begin{equation}
	a_k=\ln p((\boldsymbol{x}|C_k)p(C_k))
\end{equation}
归一化指数也被称为softmax函数,因为它表示“max”函数的一个平滑版本。
\subsection*{连续输入}
假设类条件概率密度是高斯分布,然后求解后验概率的形式。首先,假设所有的类别的协方差矩阵相同。这样类别$C_k$的类条件概率为
\begin{equation}
	p(\boldsymbol{x}|C_k)=\frac{1}{(2\pi)^{\frac{D}{2}}}\frac{1}{|\Sigma|^{\frac{1}{2}}}exp\left\{-\frac{1}{2}(\boldsymbol{x}-\boldsymbol{\mu}_k)^T\Sigma^{-1}(\boldsymbol{x}-\boldsymbol{\mu}_k) \right\}
\end{equation}
首先考虑两类的情形。
\begin{equation}
\begin{aligned}
	p(C_1|\boldsymbol{x})&=\sigma(a)\\
	\Rightarrow\ a&=\ln \frac{p(\boldsymbol{x}|C_1)p(C_1)}{p(\boldsymbol{x}|C_2)p(C_2)}\\
	&=-\frac{1}{2}(\boldsymbol{x}-\boldsymbol{\mu}_1)^T\Sigma^{-1}(\boldsymbol{x}-\boldsymbol{\mu}_1)+\frac{1}{2}(\boldsymbol{x}-\boldsymbol{\mu}_2)^T\Sigma^{-1}(\boldsymbol{x}-\boldsymbol{\mu}_2)+\ln\frac{p(C_1)}{p(C_2)} \\
	&=-\frac{1}{2}\boldsymbol{x}^T\Sigma^{-1}\boldsymbol{x}+\frac{1}{2}\boldsymbol{\mu}_1^T\Sigma^{-1}\boldsymbol{x}+\frac{1}{2}\boldsymbol{x}^T\Sigma^{-1}\boldsymbol{\mu}_1-\frac{1}{2}\boldsymbol{\mu}_1^T\Sigma^{-1}\boldsymbol{\mu}_1\\
	&\quad +\frac{1}{2}\boldsymbol{x}^T\Sigma^{-1}\boldsymbol{x}-\frac{1}{2}\boldsymbol{\mu}_2^T\Sigma^{-1}\boldsymbol{x}-\frac{1}{2}\boldsymbol{x}^T\Sigma^{-1}\boldsymbol{\mu}_2+\frac{1}{2}\boldsymbol{\mu}_2^T\Sigma^{-1}\boldsymbol{\mu}_2\\
	&\quad +\ln\frac{p(C_1)}{p(C_2)}\\
	&=\Sigma^{-1}(\boldsymbol{\mu}_1-\boldsymbol{\mu}_2)\boldsymbol{x} - \frac{1}{2}\boldsymbol{\mu}_1^T\Sigma^{-1}\boldsymbol{\mu}_1+\frac{1}{2}\boldsymbol{\mu}_2^T\Sigma^{-1}\boldsymbol{\mu}_2+\ln\frac{p(C_1)}{p(C_2)}\\
	&=\boldsymbol{w}^T\boldsymbol{x}+w_0\\
	\Rightarrow \ \sigma(a)&=\sigma(\boldsymbol{w}^T\boldsymbol{x}+w_0)
\end{aligned}	
\end{equation}
高斯概率密度的指数项中$\boldsymbol{x}$的二次型消失了(这是因为我们假设类概率的协方差矩阵相同),从而得到了参数为$\boldsymbol{x}$的线性函数的logistic sigmoid函数。最终求得决策边界$a=0$为常数的决策面。先验概率密度$p(C_k)$只出现在偏置参数$w_0$中,因此先验的改变的效果是平移决策边界,即平移后验概率中的常数轮廓线。

对于K个类别的一般情形,根据公式$\ref{keke}$,有
\begin{flalign}
	a_k(\boldsymbol{x})&=\boldsymbol{w}_k^T\boldsymbol{x}+w_{k0}\\
	w_{k0}&=-\frac{1}{2}\boldsymbol{\mu}_k^T\Sigma^{-1}\boldsymbol{\mu}_k+\ln p(C_k)
\end{flalign}
其中定义了
\begin{equation}
	\boldsymbol{w}_k=\Sigma^{-1}\boldsymbol{\mu}_k
\end{equation}
如果我们不假设各个类别的协方差矩阵相同,允许每个类条件概率密度$p(\boldsymbol{x}|C_k)$有自己的协方差矩阵$\Sigma_k$,那么之前二次项消去的现象不会出现,从而我们会得到$\boldsymbol{x}$的二次函数,这就引出了二次判别函数(quadratic discriminant)。
\subsection*{最大似然解}
一旦具体化了类条件概率密度$p(\boldsymbol{x}|C_k)$的参数化的函数形式,我们就能够使用最大似然法确定参数的值,以及先验类概率$p(C_k)$。

首先考虑两类的情形,每个类别都有一个高斯类条件概率密度,且协方差矩阵相同。假设有一个数据集$\{x_n,t_n\}$,其中$n=1,\dots,N$。先验概率记作$p(C_1)=\pi$,从而$p(C_2)=1-\pi$。对于一个来自类别$C_1$的数据点$\boldsymbol{x}_n$,我们有$t_n=1$,因此
\begin{flalign}
	p(\boldsymbol{x}_n,C_1)&=p(C_1)p(\boldsymbol{x}_n|C_1)=\pi\mathcal{N}(\boldsymbol{x}_n|\boldsymbol{\mu}_1,\Sigma)\\
	p(\boldsymbol{x}_n,C_2)&=p(C_2)p(\boldsymbol{x}_n|C_2)=(1-\pi)\mathcal{N}(\boldsymbol{x}_n|\boldsymbol{\mu}_2,\Sigma)
\end{flalign}
于是,似然函数为
\begin{equation}
	p(\boldsymbol{t},\boldsymbol{X}|\pi,\boldsymbol{\mu}_1,\boldsymbol{\mu}_2,\Sigma)=\prod_{n=1}^{N}[\pi\mathcal{N}(\boldsymbol{x}_n|\boldsymbol{\mu}_1,\Sigma)]^{t_n}[(1-\pi)\mathcal{N}(\boldsymbol{x}_n|\boldsymbol{\mu}_2,\Sigma)]^{1-t_n}
\end{equation}
其中$\boldsymbol{t}=(t_1,\dots,t_N)^T$。对数似然函数为
\begin{equation}
\begin{aligned}
	\log p(\boldsymbol{t},\boldsymbol{X}|\pi,\boldsymbol{\mu}_1,\boldsymbol{\mu}_2,\Sigma)&=\sum_{n=1}^{N}\log \left\{\pi\mathcal{N}(\boldsymbol{x}_n|\boldsymbol{\mu}_1,\Sigma)]^{t_n}[(1-\pi)\mathcal{N}(\boldsymbol{x}_n|\boldsymbol{\mu}_2,\Sigma)]^{1-t_n}\right\}\\
	&=\sum_{n=1}^{N}\left\{ \log \pi^{t_n}(1-\pi)^{1-t_n}+\log \mathcal{N}(\boldsymbol{x}_n|\boldsymbol{\mu}_1,\Sigma)^{t_n}+\log\mathcal{N}(\boldsymbol{x}_n|\boldsymbol{\mu}_2,\Sigma)^{1-t_n} \right\}
\end{aligned}
\end{equation}
最大化对数似然函数
\begin{enumerate}
	\item 求$\pi$
	\begin{equation}
	\begin{aligned}
	\frac{\partial }{\partial\pi}\left[\sum\limits_{n=1}^{N}\log \pi^{t_n}(1-\pi)^{1-t_n} \right]&=\sum_{n=1}^{N}\frac{t_n}{\pi}-\frac{(1-t_n)}{1-\pi}=0\\
	&=\sum_{n=1}^{N}t_n-\pi t_n -\pi +\pi t_n\\
	&=\sum_{n=1}^{N}t_n-\pi\\
	&=\sum_{n=1}^{N}t_n-N\pi=0\\
	\Rightarrow \pi &= \frac{1}{N}\sum_{n=1}^{N}t_n=\frac{N_1}{N}=\frac{N_1}{N_1+N_2}\\
	\end{aligned}
	\end{equation}
	
	\item 求$\boldsymbol{\mu}$
	\begin{equation}
		\begin{aligned}
		\frac{\partial }{\partial \boldsymbol{\mu}_1}\left[\log \mathcal{N}(\boldsymbol{x}_n|\boldsymbol{\mu}_1,\Sigma)^{t_n}\right]&\Rightarrow \frac{\partial }{\partial \boldsymbol{\mu}_1}\left[\sum_{n=1}^{N}-\frac{1}{2}t_n(\boldsymbol{x}_n-\boldsymbol{\mu}_1)^T\Sigma^{-1}(\boldsymbol{x}_n-\boldsymbol{\mu}_1) \right]\\
		&=\frac{\partial }{\partial \boldsymbol{\mu}_1}\left[-\frac{1}{2}\sum_{n=1}^{N}t_n(\boldsymbol{x}_n^T\Sigma^{-1}-\boldsymbol{\mu}_1^T\Sigma^{-1})(\boldsymbol{x}_n-\boldsymbol{\mu}_1) \right]\\
		&=\frac{\partial }{\partial \boldsymbol{\mu}_1}\left[-\frac{1}{2}\sum_{n=1}^{N}t_n(\underbrace{\boldsymbol{x}_n^T\Sigma^{-1}\boldsymbol{x}_n}_{\text{常数}} - 2\boldsymbol{\mu}_1^T\Sigma^{-1}\boldsymbol{x}_n+\boldsymbol{\mu}_1^T\Sigma^{-1}\boldsymbol{\mu}_1) \right]\\
		&=\sum_{n=1}^{N}t_n(\Sigma^{-1}\boldsymbol{x}_n+\Sigma^{-1}\boldsymbol{\mu}_1)=0\\
		&\Rightarrow\sum_{n=1}^{N}t_n(\boldsymbol{\mu}_1-\boldsymbol{x}_n)=0\\
		\Rightarrow \boldsymbol{\mu}_1&=\frac{\sum\limits_{n=1}^{N}t_n\boldsymbol{x}_n}{\sum\limits_{n=1}^{N}t_n}\\
		&=\frac{1}{N_1}\sum\limits_{n=1}^{N}t_n\boldsymbol{x}_n
		\end{aligned}
	\end{equation}
	同理
	\begin{equation}
		\boldsymbol{\mu}_2=\frac{1}{N_2}\sum_{n=1}^{N}(1-t_n)\boldsymbol{x}_n
	\end{equation}
	\item 求$\Sigma$
	\begin{theorem}{}{}
		\begin{flalign}
			Tr(AB)&=Tr(BA)\\
			\frac{\partial Tr(AB)}{\partial A}&=B^T\\
			\frac{\partial |A|}{\partial A}&=|A|A^{-1}\\
			\frac{\partial \ln |A|}{\partial A}&=A^{-1}
		\end{flalign}
	\end{theorem}
	考察部分分式
	\begin{equation}
		\begin{aligned}
			\sum_{n=1}^{N}\log \mathcal{N}(\boldsymbol{\mu},\Sigma)			&=\sum_{n=1}^{N}\log |\Sigma|^{-\frac{1}{2}}+(-\frac{1}{2}(\boldsymbol{x}_n-\boldsymbol{\mu})^T\Sigma^{-1}(\boldsymbol{x}_n-\boldsymbol{\mu}))+C\\
			&=-\frac{1}{2}N\log |\Sigma| - \frac{1}{2}\sum_{n=1}^{N}(\boldsymbol{x}_n-\boldsymbol{\mu})^T\Sigma^{-1}(\boldsymbol{x}_n-\boldsymbol{\mu})+C\\
			&=-\frac{1}{2}N\log |\Sigma|-\frac{1}{2}N Tr(S\cdot \Sigma^{-1})+C
		\end{aligned}
	\end{equation}
	其中
	\begin{equation}
		S=\frac{1}{N}\sum_{n=1}^{N}(\boldsymbol{x}_n-\boldsymbol{\mu})(\boldsymbol{x}_n-\boldsymbol{\mu})^T
	\end{equation}
	将与$\Sigma$相关的式子组合到一起,有,求关于$\Sigma$的偏导,并令其为零
	\begin{equation}
		\begin{aligned}
		\triangle &= \sum_{n=1}^{N}\log \mathcal{N}(\boldsymbol{x}_n|\boldsymbol{\mu}_1,\Sigma)^{t_n}+\log\mathcal{N}(\boldsymbol{x}_n|\boldsymbol{\mu}_2,\Sigma)^{1-t_n}\\
		&=-\frac{1}{2}\log |\Sigma|-\frac{1}{2}N_1Tr(S_1\cdot \Sigma^{-1})-\frac{1}{2}N_2Tr(S_2\cdot \Sigma^{-1})+C\\
		\frac{\partial \triangle}{\partial \Sigma}&=-\frac{1}{2}(N\Sigma^{-1}-N_1S_1\Sigma^{-2}-N_2S_2\Sigma^{-2})=0\\
		&\Rightarrow \Sigma=\frac{N_1}{N}S_1+\frac{N_2}{N}S_2
		\end{aligned}
	\end{equation}
\end{enumerate}

这个结果很容易推广到K类问题,得到参数的对应的最大似然解。其中我们假定每个类条件概率密度都是高斯分布,协方差矩阵相同。注意,拟合类高斯分布的方法对于离群点并不鲁棒,因为高斯的最大似然估计是不鲁棒的。
\subsection*{离散特征}
考虑离散特征值$x_i$的情形。为了简化起见,我们首先考虑二元特征值$x_i\in \{0,1\}$,稍后会讨论如何推广到更一般的离散特征。如果有D个输入,那么一般的概率分布会对应于一个大小为$2^D$的表格,包含$2^D-1$个独立变量 。由于这会随着特征的数量指数增长,因此我们想寻找一个更加严格的表示方法。这里,我们做出朴素贝叶斯(naive Bayes)假设,这个假设中,特征值被看成相互独立的,以类别$C_k$为条件,因此。我们得到类条件分布,形式为
\begin{equation}
\begin{aligned}
	p(\boldsymbol{X}=\boldsymbol{x}|C_k)&=p(X^{(1)}=x^{(1)},\dots,X^{(D)}=x^{(D)}|C_k)\\
	&=\prod_{i=1}^{D}\mu_{k_i}^{x_i}(1-\mu_{k_i})^{1-x_i}
\end{aligned}
\end{equation}
其中对于每个类别,都有D个独立的参数。代入softmax函数中,有
\begin{equation}
	\begin{aligned}
	a_k(\boldsymbol{x})&=\ln p(\boldsymbol{x}|C_k)p(C_k)\\
	&=\ln \prod_{i=1}^{D}\mu_{k_i}^{x_i}(1-\mu_{k_i})^{1-x_i} + \ln p(C_k)\\
	&=\sum_{i=1}^{D}\{x_i\ln \mu_{k_i}+(1-x_i)\ln (1-\mu_{k_i}) \}+\ln p(C_k)
	\end{aligned}
\end{equation}
与之前一样,这是输入变量$x_i$的线性函数。对于$K=2$个类别的情形,我们可以考虑另一种方法——logistic sigmoid函数。离散变量也有类似的结果,其中,每个离散变量有$M>2$种状态。

朴素贝叶斯法实际上学习到生成数据的机制,所以属于生成模型,条件独立性假设等于是说用于分类的特征在类确定的条件下都是条件独立的。这一假设使得朴素贝叶斯变得简单,但有时会牺牲一定的分类准确率。

朴素贝叶斯法分类时,对给定的输入$x$,通过学习到的模型计算后验概率分布$p(C_k|X=x)$,将后验概率最大的类作为$x$类的输出。后验概率计算根据贝叶斯定理进行。
\begin{equation}
	y=f(\boldsymbol{x})=arg \mathop{max}\limits_{c_k} \frac{p(C_k)\prod_jp(X^{(j)}|C_k)}{\sum_kp(C_k)\prod_jp(X^{(j)}|C_k)}
\end{equation}
注意到分母对所有$C_k$都是相同的,所以 
\begin{equation}
	y=arg \mathop{max}\limits_{c_k} p(C_k)\prod_jp(X^{(j)}|C_k)
\end{equation}
朴素贝叶斯法将实例分到后验概率最大的类中。这等价于期望风险最小化。假设选择0-1损失函数:
\begin{equation}
	L(Y,f(X))=\begin{cases}
	1,\quad Y\ne f(X)\\
	0,\quad Y=f(X)
	\end{cases}
\end{equation}
式中$f(X)$是分类决策函数。这时,期望风险函数为
\begin{equation}
	\begin{aligned}
	E_{exp}(f)&=E[L(Y,f(X))]\\
	&=\int_{x\mathcal{X}y}L(y,f(x))p(x,y)dxdy\\
	&=\int_x[L(y,f(x))p(y|x)dy]p(x)dx\\
	&=E_X\sum_{k=1}^{K}[L(C_k,f(X))]p(C_k|X)
	\end{aligned}
\end{equation}
为了使期望风险最小化,只需对$X=x$逐个极小化,由此得到:
\begin{equation}
	\begin{aligned}
	f(x)&=arg \mathop{min}_{y\in \mathcal{y}}\sum_{k=1}^{K}L(C_k,y)p(C_k|X=x)\\
	&=arg \mathop{min}_{y\in \mathcal{y}}\sum_{k=1}^{K}p(y\ne C_k|X=x)\\
	&=arg \mathop{min}_{y\in \mathcal{y}}(1-p(y=C_k|X=x))\\
	&=arg \mathop{max}_{y\in \mathcal{y}}p(y=C_k|X=x)
	\end{aligned}
\end{equation}
下面给出朴素贝叶斯法的学习与分类算法

输入:训练数据$T=\{(x_1,y_1),(x_2,y_2),\dots, (x_N,y_N)\}$,其中$x_i=(x_i^{(1)},x_i^{(2)},\dots,x_i^{(n)})^T$,$x_i^{(j)}$是第i个样本的第j个特征,$x_i^{(j)}\in \{a_{j1},a_{j2},\dots, a_{jS_j}\}$,$a_{jl}$是第j个特征可能取的第l个值,$j=1,2,\dots, n,l=1,2,\dots, S_j, y_i\in \{c_1,c_2,\dots,c_k\}$;实例x;

输出:实例x的分类。
\begin{enumerate}
	\item 计算先验概率及条件概率
	
	\begin{flalign}
		P(Y=C_k)&=\frac{\sum_{i=1}^NI(y_i=C_k)}{N},k=1,2,\dots,K \\
		P(X^{(j)}=a_{jl}|Y=C_k)&=\frac{\sum_{i=1}^NI(x_i^{(j)}=a_{jl},y_i=C_k)}{\sum_{i=1}^NI(y_i=C_k)} \\
		j&=1,2,\dots, N,l=1,2,\dots, S_j, k=1,2,\dots,K
	\end{flalign}
	\item 对于给定的实例$x=(x^{(1)},x^{(2)},\dots,x^{(n)})^T$,计算
	\begin{equation}
		P(Y=C_k)\prod_{j=1}^nP(X^{(j)}=x^{(j)}|Y=C_k),k=1,2,\dots,K
	\end{equation}
	\item 确定实例x的类
	
	\begin{equation}
		y=arg\mathop{max}\limits _{C_k}P(Y=C_k)\prod_{j=1}^nP(x^{(j)}=x^{(j)}|Y=Cc_k)
	\end{equation}
\end{enumerate}

\subsection*{指数族分布}
正如我们已经看到的,无论是服从高斯分布的输入,还是离散的输入,后验类概率密度都是由一般的线性模型和logistic sigmoid($K=2$个类别)或者softmax($K\geqslant 2$个类别)激活函数给出。通过假设类条件概率密度$p(\boldsymbol{x}|C_k)$是指数族分布的成员,我们可以看到上述结果都是更一般的结果的特例。
\section{概率判别式模型}
对于二分类问题,我们已经看到,对于一大类的类条件概率密度$p(\boldsymbol{x}|C_k)$的选择,类别$C_1$后验概率分布可以写成作用于$\boldsymbol{x}$的线性函数上的logistic sigmoid函数的形式。类似地,对于多分类的情形,类别$C_k$的后验概率由$\boldsymbol{x}$的线性函数的softmax变换给出。对于类条件概率密度$p(\boldsymbol{x}|C_k)$的具体的选择,我们已经使用了最大似然方法估计了概率密度的参数以及类别先验$p(C_k)$,然后使用贝叶斯定理就可以求出后验类概率。

另一种方法是显示地使用一般的线性模型的函数形式,然后使用最大似然法直接确定它的参数。

寻找一般的线性模型参数的间接方法是,分别寻找类条件概率密度和类别先验,然后使用贝叶斯定理。这是生成式建模的一个例子。在直接方法中,我们最大化由条件概率分布$p(C_k|\boldsymbol{x})$定义的似然函数。这种方法代表了判别式训练的一种形式。判别式方法的一个优点是通常有更少的可调节的参数需要确定,并且预测表现会提升,尤其是当类条件概率密度的假设没有很好地近似真实的分布的时候更是如此。
\subsection*{固定基函数}
如果首先使用一个基函数向量$\phi(\boldsymbol{x})$对输入变量进行一个固定的非线性变换,所有的这些算法仍然同样适用。
\subsection*{logistic回归}
首先通过二分类问题开始对于一般线性模型的讨论。类别$C_1$的后验概率可以写成作用在特征向量$\phi$的线性函数上的logistic sigmoid函数的形式,即
\begin{flalign}
	p(C_1|\phi)&=y(\phi)=\sigma(\boldsymbol{w}^T\phi)\\
	p(C_2|\phi)&=1-p(C_1|\phi)
\end{flalign}
这个模型被称为logistic回归,这是一个分类模型而不是回归模型。

对于一个M维特征空间$\phi$,这个模型有M个可调节参数。相反,如果我们使用最大似然方法调节高斯类条件概率密度,那么我们有2M个参数来描述均值,以及$\frac{M(M+1)}{2}$个参数来描述协方差矩阵。算上类先验$p(C_1)$,参数的总数为$\frac{M(M+5)}{2}+1$,这随着M的增长而以二次的方式增长。这和logstic回归方法中对于参数数据M的线性依赖不同。对于大的M值,直接使用logistic回归模型有着很明显的优势。

现在使用最大似然方法来确定logistic回归模型的参数。对于一个数据集$\phi_n,t_n$,其中$t_n\in \{0,1\},\phi_n=\phi(\boldsymbol{x}_n),n=1,\dots,N$,似然函数可以写成
\begin{equation}
	p(\boldsymbol{t}|\boldsymbol{w})=\prod_{n=1}^{N}y_n^{t_n}\{1-y_n \}^{1-t_n}
\end{equation}
其中$\boldsymbol{t}=(t_1,\dots,t_N)^T$且$y_n=p(C_1|\phi_n)$。通过取似然函数的负对数的方式,定义一个误差函数。这种方式产生了交叉熵(cross-entropy)误差函数,形式为
\begin{equation}
	\begin{aligned}
	E(\boldsymbol{w})&=-\ln p(\boldsymbol{t}|\boldsymbol{w})\\
	&=-\sum_{n=1}^{N}\{t_n\ln y_n+(1-t_n)\ln (1-y_n) \}
	\end{aligned}
\end{equation}
其中$y_n=\sigma(a_n)$且$a_n=\boldsymbol{w}^T\phi_n$。两侧关于$\boldsymbol{w}$取误差函数的梯度,我们有
\begin{equation}
\begin{aligned}
	\triangledown E(\boldsymbol{w})&=-\sum_{n=1}^{N}\{\frac{t_n}{y_n}y_n^{'}+\frac{(1-t_n)}{1-y_n}(1-y_n)^{'} \}\\
	&=-\sum_{n=1}^{N}\frac{t_n}{\sigma}\sigma(1-\sigma)\phi_n - \frac{1-t_n}{1-\sigma}\sigma(1-\sigma)\phi_n\\
	&=\sum_{n=1}^{N}(y_n-t_n)\phi_n
\end{aligned}
\end{equation}
推导时用到了
\begin{equation}
	\frac{d\sigma}{da}=\sigma(1-\sigma)
\end{equation}
我们看到,涉及到logistic sigmoid的导数的因子已经被消去,使得对数似然函数的梯度的形式十分简单。特别地,数据点$n$对梯度的贡献为目标值和模型预测值之间的“误差”与基函数向量$\phi_n$相乘。此外它的函数形式与线性回归模型中的平方和误差函数的梯度的函数形式完全相同。

问题变成了以对数似然函数为目标函数的最优化问题。logistic回归学习中通常采用的方法是梯度下降法及拟牛顿法。

值得注意的一点是,最大似然方法对于线性可分的数据集会产生严重的过拟合现象。最大似然方法无法区分某个解优于另一个解,并且在实际应用中哪个解被找到将会依赖于优化算法的选择和参数的初始化。只要数据是线性可分的,这个问题就会出现。通过引入先验概率,然后寻找$\boldsymbol{w}$的MAP解,或者等价地,通过给误差函数增加一个正则化项,这种奇异性就可以被避免。
\subsection*{迭代重加权最小平方}
\subsection*{多类logistic回归}
\subsection*{probit回归}
\subsection*{标准链接函数}
\section{拉普拉斯近似}
特别地,我们不能够精确地关于参数向量$\boldsymbol{x}$求积分,因为后验概率分布不再是高斯分布。因此,有必要介绍某种形式的近似。后面章节中,我们会介绍一系列分析估计和数值采样的技术。

拉普拉斯近似的目标是找到定义在一组连续变量上的概率密度的高斯近似。首先考虑单一连续变量$z$的情形,假设分布$p(z)$的定义为
\begin{equation}
	p(z)=\frac{1}{Z}f(z)
\end{equation}
其中$Z=\int f(z)dz$是归一化系数。我们假定Z的值是未知的。在拉普拉斯方法中,目标是寻找一个高斯近似$q(z)$,它的中心位于$p(z)$的众数的位置。第一步是寻找$p(z)$的众数,即寻找一个点$z_0$使得$p^{'}(z_0)=0$,或者等价地
\begin{equation}
	\frac{df(z)}{dz}\Bigg|_{z=z_0}=0
\end{equation}
高斯分布有一个性质,即它的对数是变量的二次函数。于是我们考虑$\ln f(z)$以众数$z_0$为中心的泰勒展开,即
\begin{equation}
	\ln f(z)\simeq \ln f(z_0) -\frac{1}{2}A(z-z_0)^2
\end{equation}
其中
\begin{equation}
	A=-\frac{d^2}{dz^2}\ln f(z)\Bigg|_{z=z_0}
\end{equation}
注意,泰勒展开式中的一阶项没有出现,因为$z_0$是概率分布的局部最大值。两侧同时取指数,我们有
\begin{equation}
\label{4135}
	f(z)\simeq f(z_0)\exp\left\{-\frac{A}{2}(z-z_0)^2 \right\}
\end{equation}
这样,使用归一化的高斯分布的标准形式,我们就可以得到归一化的概率分布$q(z)$,即
\begin{equation}
	q(z)=\left(\frac{A}{2\pi} \right)^{\frac{1}{2}}\exp\left\{-\frac{A}{2}(z-z_0)^2 \right\}
\end{equation}
注意,高斯近似只在精度A>0时有良好的定义,换句话说,驻点$z_0$一定是一个局部最大值,使得$f(z)$在驻点$z_0$处的二阶导数为负。

我们可以将拉普拉斯方法推广,使其近似定义在M维空间$\boldsymbol{z}$上的概率分布$p(\boldsymbol{z})=\frac{f(\boldsymbol{z})}{Z}$。在驻点$\boldsymbol{z}_0$处,梯度$\triangledown f(\boldsymbol{z})$将会消失。在驻点处展开,我们有
\begin{equation}
	\ln f(\boldsymbol{z})\simeq \ln f(\boldsymbol{z}_0)-\frac{1}{2}(\boldsymbol{z}-\boldsymbol{z}_0)^T\boldsymbol{A}(\boldsymbol{z}-\boldsymbol{z}_0)
\end{equation}
其中$M\times M$的Hessian矩阵$\boldsymbol{A}$的定义为
\begin{equation}
	\boldsymbol{A}=-\triangledown\triangledown \ln f(\boldsymbol{z})|_{\boldsymbol{z}=\boldsymbol{z}_0}
\end{equation}
两边同时取指数,我们有
\begin{equation}
	f(\boldsymbol{z})\simeq f(\boldsymbol{z}_0)\exp\left\{-\frac{1}{2}(\boldsymbol{z}-\boldsymbol{z}_0)^T\boldsymbol{A}(\boldsymbol{z}-\boldsymbol{z}_0) \right\}
\end{equation}
分布$q(\boldsymbol{z})$正比于$f(\boldsymbol{z})$,归一化系数可以通过观察归一化的多元高斯分布的标准形式得到。因此
\begin{equation}
	q(\boldsymbol{z})=\frac{|\boldsymbol{A}|^{\frac{1}{2}}}{(2\pi)^{\frac{M}{2}}}\exp\left\{-\frac{1}{2}(\boldsymbol{z}-\boldsymbol{z}_0)^T\boldsymbol{A}(\boldsymbol{z}-\boldsymbol{z}_0) \right\}=\mathcal{N}(\boldsymbol{z}|\boldsymbol{z}_0,\boldsymbol{A}^{-1})
\end{equation}

在应用拉普拉斯方法时,真实概率分布的归一化常数$Z$不必事先知道。根据中心极限定理,我们可以预见模型的后验概率会随着观测数据点的增多而越来越近似于高斯分布,因此我们可以预见在数据点相对较多的情况下,拉普拉斯近似会更有用。拉普拉斯近似的一个主要缺点是,由于它是以高斯分布为基础的,因此它只能直接应用于实值变量。在其他情况下,可以将拉普拉斯近似应用于变换之后的变量上。但是,拉普拉斯框架的最严重的局限性是,它完全依赖于真实概率分布在变量的某个具体位置上的性质,因此会无法描述一些重要的全局属性。
\subsection*{模型比较和BIC}
除了近似概率分布$p(z)$,我们也可以获得对归一化常数Z的一个近似。
\begin{equation}
\label{dada}
	\begin{aligned}
	   Z&=\int f(z)dz\\
		&\simeq f(z_0)\int \exp\left\{-\frac{1}{2}(z-z_0)^TA(z-z_0) \right\}dz\\
		&=f(z_0)\frac{(2\pi)^{\frac{M}{2}}}{|A|^{\frac{1}{2}}}
	\end{aligned}
\end{equation}

我们可以使用公式$\ref{dada}$的结果来获得对于模型证据的一个近似。考虑一个数据集D以及一组模型$\{M_i\}$,模型参数为$\{\boldsymbol{\theta}_i\}$。对于每个模型,我们定义一个似然函数$p(D|\boldsymbol{\theta}_i,M_i)$。如果我们引入一个参数的先验概率$p(\boldsymbol{\theta}_i|M_i)$,那么我们感兴趣的是计算不同模型的模型证据$p(D|M_i)$。从现在开始,为了简化记号,我们省略对于$M_i$的条件依赖。根据贝叶斯定理,模型证据为
\begin{equation}
	p(D)=\int p(D|\boldsymbol{\theta})p(\boldsymbol{\theta})\mathrm{d}\boldsymbol{\theta}
\end{equation}
令$f(\boldsymbol{\theta})=p(D|\boldsymbol{\theta})p(\boldsymbol{\theta})$以及$Z=p(D)$,然后使用公式$\ref{dada}$,我们有
\begin{equation}
\label{4137}
	\ln p(D)\simeq \ln p(D|\boldsymbol{\theta}_{MAP})+\underbrace{\ln p(\boldsymbol{\theta}_{MAP})+\frac{M}{2}\ln (2\pi) -\frac{1}{2}\ln |\boldsymbol{A}|}_{Occam\text{因子}}	
\end{equation}
其中$\boldsymbol{\theta}_{MAP}$是在后验概率分布众数位置的$\boldsymbol{\theta}$的值,$\boldsymbol{A}$是负对数后验概率的二阶导数组成的Hessian矩阵。
\begin{equation}
	\boldsymbol{A}=-\triangledown\triangledown \ln p(\boldsymbol{\theta}_{MAP}|D)
\end{equation}
公式$\ref{4137}$表明使用最优参数计算的对数似然值,而余下的三项由“Occam因子”组成,它对模型的复杂度进行惩罚。

如果我们假设参数的高斯先验分布比较宽,且Hessian矩阵是満秩的,那么我们可以使用下式来非常粗略地近似公式$\ref{4137}$。
\begin{equation}
	\ln p(D)\simeq \ln p(D|\boldsymbol{\theta}_{MAP})-\frac{1}{2}M\ln N
\end{equation}
其中N是数据点的总数,M是$\boldsymbol{\theta}$中参数的数量,并且我们省略了一些额外的常数。这被称为贝叶斯信息准则(Bayesian Information Criterion)(BIC),或者称为Schwarz准则。注意,这与AIC相比,这个信息准则对模型复杂度的惩罚更严重。
\section{贝叶斯logistic回归}
我们现在考虑logistic回归的贝叶斯观点。对于logistic回归,精确的贝叶斯推断是无法处理的。特别地,计算后验概率分布需要对先验概率分布于似然函数的乘积进行归一化,而似然函数本身由一系列logistic sigmoid函数的乘积组成,每个数据点都有一个logistic sigmoid函数。对于预测分布的计算类似地也是无法处理的。这里我们考虑使用拉普拉斯近似来处理贝叶斯logistic回归的问题。
\subsection*{拉普拉斯近似}
为了获得后验概率的高斯近似,我们首先最大化后验概率分布,得到MAP(最大后验)解$\boldsymbol{w}_{MAP}$,它定义了高斯分布的均值。这样协方差就是负对数似然函数的二阶导数矩阵的逆矩阵,形式为
\begin{equation}
	\boldsymbol{S}_N^{-1}=-\triangledown\triangledown \ln p(\boldsymbol{w}|\boldsymbol{t})=\boldsymbol{S}_0^{-1}+\sum_{n=1}^{N}y_n(1-y_n)\phi_n\phi^T
\end{equation}
于是后验概率分布的高斯近似的形式为
\begin{equation}
\label{lala}
	q(\boldsymbol{w})=\mathcal{N}(\boldsymbol{w}|\boldsymbol{w}_{MAP},\boldsymbol{S}_N)
\end{equation}
\subsection*{预测分布}
给定一个新的特征向量$\phi(\boldsymbol{x})$,类别$C_1$的预测分布可以通过对后验概率$p(\boldsymbol{w}|\boldsymbol{t})$积分,后验概率本身由高斯分布$q(\boldsymbol{w})$近似,即
\begin{equation}
	p(C_1|\phi,\boldsymbol{t})=\int p(C_1|\phi,\boldsymbol{w})p(\boldsymbol{w}|\boldsymbol{t})d\boldsymbol{w}\simeq \int \sigma(\boldsymbol{w}^T\phi)q(\boldsymbol{w})d\boldsymbol{w}
\end{equation}
类别$C_2$的对应的概率为$p(C_2|\phi,\boldsymbol{t})=1-p(C_1|\phi,\boldsymbol{t})$。

为了计算预测分布,我们首先注意到函数$\sigma(\boldsymbol{w}^T\phi)$对于$\boldsymbol{w}$的依赖只通过它在$\phi$上的投影而实现。记$a=\boldsymbol{w}^T\phi$,我们有
\begin{equation}
	\sigma(\boldsymbol{w}^T\phi)=\int \delta(a-\boldsymbol{w}^T\phi)\sigma(a)da
\end{equation}
其中$\delta(\cdot)$是狄拉克Delta函数。由此我们有
\begin{equation}
	\int \sigma(\boldsymbol{w}^T\phi)q(\boldsymbol{w})d\boldsymbol{w}=\int \sigma(a)p(a)da
\end{equation}
其中 
\begin{equation}
	p(a)=\int \delta(a-\boldsymbol{w}^T\phi)q(\boldsymbol{w})d\boldsymbol{w}
\end{equation}
我们可以这样计算$p(a)$:注意到Delta函数给$\boldsymbol{w}$施加了一个线性限制,因此在所有与$\phi$正交的方向上积分,就得到了联合概率分布$q(\boldsymbol{w})$的边缘分布。由于$q(\boldsymbol{w})$是高斯分布,因此我们知道边缘概率分布也是高斯分布。我们可以通过计算各阶矩然后交换$a$和$\boldsymbol{w}$的积分顺序的方式计算均值和协方差,即
\begin{equation}
	\mu_a=\mathbb{E}[a]=\int p(a)ada=\int q(\boldsymbol{w})\boldsymbol{w}^T\phi d\boldsymbol{w}=\boldsymbol{w}^T_{MAP}\phi
\end{equation}
推导时使用了$\ref{lala}$给出的后验概率分布$q(\boldsymbol{w})$的结果。类似地
\begin{equation}
\begin{aligned}
	\sigma_a^2&=var[a]=\int p(a)\{a^2-\mathbb{E}[a]^2 \}da\\
	&=\int q(\boldsymbol{w})\{(\boldsymbol{w}^T\phi)^2-(\boldsymbol{m}_N^T\phi)^2\}d\boldsymbol{w}=\phi^T\boldsymbol{S}_N\phi
\end{aligned}
\end{equation}
因此我们对于预测分布的近似变成了
\begin{equation}
	p(C_1|\boldsymbol{t})=\int \sigma(a)p(a)da=\int \sigma(a)\mathcal{N}(a|\mu_a,\sigma_a^2)da
\end{equation}
关于$a$的积分表示一个高斯分布和一个logistic sigmoid函数的卷积,不能够解析地求值。然而,我们可以利用logistic sigmoid函数$\sigma(a)$和逆probit函数$\Phi(a)$的高度相似性来获得一个较好的近似。

使用逆Probit函数的一个优势是它与高斯的卷积可以用另一个逆probit解析地表示出来。特别地,我们可以证明 
\begin{equation}
	\int \Phi(\lambda a)\mathcal{N}(a|\mu,\sigma^2)da=\Phi\left(\frac{\mu}{(\lambda^{-2}+\sigma^2)^{\frac{1}{2}}} \right)
\end{equation}
我们现在将逆probit函数的近似$\sigma(a)\simeq \Phi(\lambda a)$应用于这个方程的两侧,得到下面的对于loigstic sigmoid函数与高斯的卷积的近似
\begin{equation}
	\int \sigma(a)\mathcal{N}(a|\mu,\sigma^2)da\simeq \sigma(k(\sigma^2)\mu)
\end{equation}
其中我们定义了
\begin{equation}
	k(\sigma^2)=(1+\frac{\pi\sigma^2}{8})^{-\frac{1}{2}}
\end{equation}

把这个结果应用于公式中,我们得到了近似的预测分布,形式为
\begin{equation}
\label{jiasup}
	p(C_1|\phi,\boldsymbol{t})=\sigma(k(\sigma_a^2)\mu_a)
\end{equation}
\chapter{神经网络}
在上两章,我们考虑了由固定基函数的线性组合构成的回归模型和分类模型。我们看到,这些模型具有一些有用的分析性质和计算性质,但是它们的实际应用被维数灾难问题限制了。为了将这些模型应用于大规模的问题,有必要根据数据调节基函数。

支持向量机是这样解决这个问题的:首先定义以训练数据点为中心的基函数,然后在训练过程中选择一个子集。支持向量机的一个优点是,虽然训练阶段涉及到非线性优化,但是目标函数是凸函数,因此最优化问的解相对很直接,并且通常随着数据规模的增加而增多。相关向量机也选择固定基函数集合的一个子集,通常会生成一个相当稀疏的模型。与支持向量机不同,相关向量机也产生概率形式的输出,嘎然这种输出的产生会以训练阶段的非凸优化为代价。

另一种方法是事先固定基函数的数量,但是允许基函数可调节。换名话,就是使用参数形式的基函数,这些参数可以在训练阶段调节 。在模式识别中,这种类型的最成功的模型是有前馈神经网络,也被称为多层感知器(multilayer perceptron)。与具有同样泛化能力的支持向量机相比,最终的模型会相当简洁,因此计算的速度更快。这种简洁性带来的代价就是,与相关向量机一样,构成了网络训练根基的似然函数不再是模型参数的凸函数。然而,在实际应用中,考察模型在训练阶段消耗的计算资源是很有价值的,这样做会得到一个简洁的模型,它可以快速地处理新数据。

首先,我们考虑神经网络的函数形式,包括基函数的具体参数,然后我们讨论使用最大似然框架确定神经网络参数的问题,这涉及到非线性最优化问题的解。这种方法需要计算对数似然函数关于神经网络参数的导数,我们会看到这些导数可以使用误差反向传播(error backpropagation)的方法高效地获得。我们还会说明误差反向传播的框架如何推广到计算其他的导数,例如Jacobian矩阵和Hessian矩阵。接下来,我们讨论神经网络训练的正则化和各种方法,以及方法之间的关系。我们还会考虑神经网络模型的一些扩展。特别地,我们会描述一个通用的框架,用来对条件概率密度建模。这个框架被称为混合密度网络(mixture density network)。最后,我们讨论神经网络的贝叶斯观点。
\section{前馈神经网络}

\section{网络训练}
根据解决的问题的类型,关于输出单元激活函数和对应的误差函数,都存在一个自然的选择。
\begin{enumerate}
	\item 对于回归问题,我们使用线性输出和平方和误差函数
	\item 对于(多类独立的)二元分类问题,我们使用logistic sigmoid输出以及交叉熵误差函数
	\item 对于多类分类问题,我们使用softmax输出以及对应的多分类交叉熵错误函数。
\end{enumerate}
对于涉及到两类的分类问题,我们可以使用单一的logistic sigmoid输出,也可以使用神经网络,这个神经网络有两个输出,且输出激活函数为softmax函数。
\subsection*{参数最优化}
下面考虑寻找能够使得选定的误差函数$E(\boldsymbol{w})$达到最小值的权向量$\boldsymbol{w}$。由于误差$E(\boldsymbol{w})$是$\boldsymbol{w}$的光滑连续函数,因此它的最小值出现在权空间中误差函数梯度等于零的位置上,即
\begin{equation}
	\triangledown E(\boldsymbol{w})=0
\end{equation}
然而,误差函数通常与权值和偏置参数的关系是高度非线性的,因此权值空间中会有很多梯度为零(或者梯度非常小)的点。由于显然无法找到方程$\triangledown E(\boldsymbol{w})=0$的解析解,因此我们使用迭代的数值方法。大多数方法涉及到为权向量选择某个初始值$\boldsymbol{w}_0$,然后在权空间中进行一系列移动,形式为
\begin{equation}
	\boldsymbol{w}^{(\tau +1)}=\boldsymbol{w}^{(\tau)}+\triangle w^{(\tau)}
\end{equation}
其中$\tau$表示迭代次数。不同的算法涉及到权向量更新$\triangle w^{(\tau)}$的不同选择。许多算法使用梯度信息,为了理解梯度信息的重要性,有必要考虑误差函数基于泰勒展开的局部近似。
\subsection*{局部二次近似}
通过讨论误差函数的局部二次近似,我们可以更深刻地认识最优化问题,以及各种解决最优化问题的方法。考虑$E(\boldsymbol{w})$在权空间某点$\hat{\boldsymbol{w}}$处的泰勒展开 
\begin{equation}
	E(\boldsymbol{w})\simeq E(\hat{\boldsymbol{w}})+(\boldsymbol{w}-\hat{\boldsymbol{w}})^T\boldsymbol{b}+\frac{1}{2}(\boldsymbol{w}-\hat{\boldsymbol{w}})^T\boldsymbol{H}(\boldsymbol{w}-\hat{\boldsymbol{w}})
\end{equation}
$\boldsymbol{b}$被定义为E的梯度在$\hat{\boldsymbol{w}}$处的值。
\begin{equation}
	\boldsymbol{b}\equiv \triangledown E|_{\boldsymbol{w}=\hat{\boldsymbol{w}}}
\end{equation}
Hessian矩阵$\boldsymbol{H}=\triangledown\triangledown E$
梯度的局部近似为
\begin{equation}
	\triangledown E\simeq \boldsymbol{b}+\boldsymbol{H}(\boldsymbol{w}-\hat{\boldsymbol{w}})
\end{equation}
考虑一个特殊情况:在误差函数最小值点$\boldsymbol{w}^*$附近的局部二次近似。因为$\triangledown E=0$
\begin{equation}
E(\boldsymbol{w})\simeq E(\boldsymbol{w}^*)+\frac{1}{2}(\boldsymbol{w}-\boldsymbol{w}^*)^T\boldsymbol{H}(\boldsymbol{w}-\boldsymbol{w}^*)
\end{equation}
为了用几何的形式表示这个结果,考虑Hessian矩阵的特征值方程
\begin{equation}
	\boldsymbol{H} \boldsymbol{u}_i=\lambda_i\boldsymbol{u}_i
\end{equation}
其中特征向量$\boldsymbol{u}_i$构成了完备的单位正交集合,即
\begin{equation}
	\boldsymbol{u}_i^T\boldsymbol{u}_j=\delta_{ij}
\end{equation}
现在把$(\boldsymbol{w}-\boldsymbol{w}^*)$展开成特征值的线性组合的形式
\begin{equation}
	\boldsymbol{w}-\boldsymbol{w}^*=\sum_i \alpha_i \boldsymbol{u}_i
\end{equation}
误差函数可以写成
\begin{equation}
	E(\boldsymbol{w})=E(\boldsymbol{w}^*)+\frac{1}{2}\sum_i \lambda_i\alpha_i^2
\end{equation}
在最小值$\boldsymbol{w}^*$的领域中,误差函数可以用二次函数近似。这样,常数误差函数的轮廓线为椭圆,它的轴与Hessian矩阵的特征向量$\boldsymbol{u}_i$给出,长度与对应的特征值$\lambda_i$的平方根成反比。在新的坐标第中,基向量是特征向量$\{\boldsymbol{u}_i\}$,E为常数的轮廓线是以原点为中心的椭圆。对应的D维的结论是,在$\boldsymbol{w}^*$处的Hessian矩阵是正定矩阵。
\subsection*{使用梯度信息}
使用误差反向传播的方法可以高效地计算误差函数的梯度。这个梯度信息的使用可以大幅度加快找到极小值点的速度。使用梯度信息构成了训练神经网络的实际算法的基础。
\subsection*{梯度下降最优化}
最简单的使用梯度信息的方法是,每次权值更新都是在负梯度方向上的一次小的移动,即
\begin{equation}
	\boldsymbol{w}^{(\tau +1)}=\boldsymbol{w}^{(\tau)}-\eta \triangledown E(\boldsymbol{w}^{(\tau)})
\end{equation}
这种方法被称为梯度下降法(gradient descent)或者最陡峭下降法(steepest descent)。虽然这种方法在直觉上看比较合理,但是实际上可以证明它是一个很差的算法。对于批量最优化方法,存在更高效的方法,例如共轭梯度法(conjugate gradient)或者拟牛顿法(quasi-Newton)。这些方法具有这样的性质:误差函数在每次迭代时总是减小的,除非权向量到达了局部的或者全局的最小值。

为了找到一个足够好的极小值,可能有必要多次运行基于梯度的算法,每次都使用一个不同的随机选择额起始点,然后在一个独立的验证集上对比最终的表现。

梯度下降法有一个在线的版本,这个版本被证明在实际应用中对于使用大规模数据集来训练神经网络的情形很有用。基于一组独立观测的最大似然函数的误差函数由一个求和式构成,求和式的每一项都对应着一个数据点
\begin{equation}
	E(\boldsymbol{w})=\sum_{n=1}^{N}E_n(\boldsymbol{w})
\end{equation}
在线梯度下降,也被称为顺序梯度下降(sequential gradient descent)或者随机梯度下降(stochasitc gradient descent),使得权向量的更新每次只依赖于一个数据点,即
\begin{equation}
		\boldsymbol{w}^{(\tau +1)}=\boldsymbol{w}^{(\tau)}-\eta \triangledown E_n(\boldsymbol{w}^{(\tau)})
\end{equation}
这个更新在数据集上循环重复进行,并且即可以顺序地处理数据,也可以随机地有重复地选择数据点。当然,也有折中的方法,即每次更新依赖于数据点的一小部分。

与批处理相比,在线方法的一个优点是可以更加高效地处理数据中的冗余性。在线梯度下隆方法的另一个性质是,可以逃离局部极小值点,因为整个数据集的关于误差函数的驻点通常不会是每个数据点各自的驻点。
\section{误差反向传播}
本节中,我们的目标是寻找一种计算前馈神经网络的误差函数$E(\boldsymbol{w})$的梯度的一种高效的方法。在局部信息传递的思想中,信息在神经网络中交替地向前、向后传播。这种方法被称为误差反向传播(error backpropagation),有时简称“反传”(backprop)。

为了不让概念发生混淆,仔细研究一下训练过程的本质是很有用的。大部分训练算法涉及到一个迭代的步骤用于误差函数的最小化,以及通过一系列的步骤进行的权值调节。在每一个迭代过程中,我们可以区分这两个不同的阶段。
\begin{enumerate}
	\item 第一个阶段,误差函数关于权值的导数必须被计算出来。
	\item 第二个阶段,导数用于计算权值的调整量。
\end{enumerate}
反向传播方法的一个重要的贡献是提供了计算这些导数的一个高效的方法。由于正是这个阶段,误差通过网络进行反向传播,因此我们将专门使用反向传播这个术语来描述计算导数的过程。
\subsection*{误差函数导数的计算}
我们现在推导适用于一般神经网络的反向传播算法。许多实际应用中使用的误差函数,例如针对一组独立同分布的数据的最大似然方法定义的误差函数,由若干的求和式组成,每一项对应于训练集的一个数据点,即
\begin{equation}
	E(\boldsymbol{w})=\sum_{n=1}^{N}E_n(\boldsymbol{w})
\end{equation}
这里,我们要考虑的是计算$\triangledown E_n(\boldsymbol{w})$的问题。这可以使用顺序优化的方法计算,或者使用批处理方法在训练集上进行累加。

首先考虑一个简单的线性模型,其中输出$y_k$是输入变量$x_i$的线性组合,即,
\begin{equation}
	y_k=\sum_i w_{ki}x_i
\end{equation}
对于一个特定的输入模式n,误差函数的形式为
\begin{equation}
	E_n=\frac{1}{2}\sum_k(y_{nk}-t_{nk})^2
\end{equation}
其中$y_{nk}=y_k(\boldsymbol{x}_n,\boldsymbol{w})$。这个误差函数关于一个权值$w_{ji}$的梯度为
\begin{equation}
	\frac{\partial E_n}{\partial w_{ji}}=(y_{nj}-t_{nj})x_{ni}
\end{equation}
它可以表示为链接$w_{ji}$的输出端相关联的“误差信号”$y_{nj}-t_{nj}$和与链接的输入端相关联的变量$x_{nj}$的乘积。
反向传播算法可以总结如下 
\begin{enumerate}
	\item 对于网络的一个输入向量$\boldsymbol{x}_n$,使用下列进行正向传播,找到所有隐含单元和输出单元的激活。
	\begin{flalign}
		a_j&=\sum_iw_{ji}z_i\\
		z_j&=h(a_j)
	\end{flalign}
	其中$z_i$是一个单元的激活,或者是输入。它向单元j发送一个链接,$w_{ji}$是与这个链接关联的权值。
	\item 计算所有输出单元的$\delta_k$
	\begin{equation}
		\delta_k=y_k-t_k
	\end{equation}
	\item 获得网络中所有隐含单元的$\delta_j$
	\begin{equation}
	\begin{aligned}
			\delta_j\equiv \frac{\partial E_n}{\partial a_j}&=\sum_k \frac{\partial E_n}{\partial a_k}\frac{\partial a_k}{\partial a_j}\\
			&=h^{'}(a_j)\sum_k w_{kj}\delta_k
	\end{aligned}
	\end{equation}
	其中求和式的作用对象是所有向单元j发送链接的单元k。注意,单元k可以包含其他的隐含单元和输出单元。
	\item 计算导数
	\begin{equation}
		\frac{\partial E_n}{\partial w_{ji}}=\delta_jz_i
	\end{equation}
\end{enumerate}
对于批处理方法,总误差函数E的导数可以通过下面的方式得到:对于训练集里的每个模式,重复上面的步骤,然后对所有的模式求和,即
\begin{equation}
	\frac{\partial E}{\partial w_{ji}}=\sum_{n}\frac{\partial E_n}{\partial w_{ji}}
\end{equation}
上面的推导中,我们隐式地假设网络中的每个隐含单元或输入单元相同的激活函数$h(\cdot)$。
\subsection*{一个简单的例子}
上面对于反向传播算法的推导适用于一般形式的误差函数、激活函数、以及网络拓扑结构。为了说明这个算法的应用,我们考虑一个具体的例子。具体地,我们考虑两层神经网络,误差函数为平方和误差函数,输出单元的激活函数为线性激活函数,即$y_k=a_k$,而隐含单元的激活函数为S型函数,形式为
\begin{equation}
	h(a)\equiv \tanh(a)=\frac{e^a-e^{-a}}{e^a+e^{-a}}
\end{equation}
这个函数的一个有用的特征是,它的导数可以表示成一个相当简单的形式
\begin{equation}
	h^{'}(a)=1-h(a)^2
\end{equation}
也考虑一个标准的平方和误差函数,即对于模式n,误差为
\begin{equation}
	E_n=\frac{1}{2}\sum_{k=1}^{K}(y_k-t_k)^2
\end{equation}
其中,对于一个特定的输入模式$\boldsymbol{x}_n,y_k$是输出单元k的激活,$t_k$是对应的目标值。

对于训练集里的每个模式,我们首先使用下面的公式进行前向传播。
\begin{flalign}
	a_j&=\sum_{i=0}^{D}w_{ji}^{(1)}x_i\\
	z_j&=\tanh(a_j)\\
	y_k&=\sum_{j=0}^{M}w_{kj}^{(2)}z_j
\end{flalign}
接下来,使用下面的公式计算每个输出单元的$\delta$值
\begin{flalign}
	\delta_k=y_k-t_k
\end{flalign}
然后,将这些$\delta$值反向传播,得到隐含单元的$\delta$值
\begin{equation}
	\delta_j=(1-Z_j^2)\sum_{k=1}^{K}w_{kj}\delta_k
\end{equation}
最后,关于第一层权值和第二层权值的导数为
\begin{equation}
	\frac{\partial E_n}{\partial w_{ji}^{(1)}}=\delta_jx_i,\quad \frac{\partial E_n}{\partial w_{ki}^{(2)}}=\delta_kz_i
\end{equation}
\subsection*{反向传播的效率}
反向传播的一个重要的方面是它的计算效率。考察误差函数导数的计算次数与网络中权值和偏置总数W的关系。
\subsection*{Jacobian矩阵}
误差反向传播技术也可以用来计算其他类型的导数。考虑Jacobian矩阵的计算,它的元素的值是网络的输出关于输入的导数
\begin{equation}
	J_{ki}\equiv \frac{\partial y_k}{\partial x_i}
\end{equation}
其中,计算每个这样的导数时,其他的输入都固定。Jacobian矩阵在由许多不同模块构建的系统中很有用。Jacobian矩阵度量了输出对于每个输入变量的改变的敏感性,因此它也允许与输入关联的任意已知的误差$\triangle x_i$在训练过的网络中传播,从而估计他们对于输出误差$\triangle y_k$的贡献。

%可以将计算Jacobian矩阵的方法总结如下。将输入空间中要寻找Jacobian矩阵的点映射成一个输入向量,将这个输入向量作为网络的输入,使用通常的正向传播方法,得到网络的所有隐含单元和输出单元的激活。接下来,对于Jacobian矩阵的每一行k(对应于输出单元k),使用递归关系
\section{Hessian矩阵}
反向传播也可以用来计算误差函数的二阶导数,形式为
\begin{equation}
	\frac{\partial^2 E}{\partial w_{ji}\partial w_{lk}}
\end{equation}
有时将所有的权值和偏置参数看成一个向量(记作$\boldsymbol{w}$)的元素$w_i$更方便,此时二阶导数组成了Hessian矩阵$\boldsymbol{H}$的元素$H_{ij}$,其中$i,j\in \{1,\dots,W\}$,且W是权值和偏置的总数。Hessian矩阵在神经网络计算的许多方面都有着重要的作用,包括
\begin{enumerate}
	\item 一些用来训练神经网络的非线性最优化算法是基于误差曲面的二阶性质的,这些性质由Hessian矩阵控制。
	\item 对于训练数据的微小改变,Hessian矩阵构成了快速重新训练前馈网络的算法的基础。
	\item Hessian矩阵的逆矩阵用来鉴别神经网络中最不重要的权值,这是网络“剪枝”算法的一部分。
	\item Hessian矩阵是贝叶斯神经网络的拉普拉斯近似的核心。它的逆矩阵用来确定训练过的神经网络的预测分布,它的特征值确定了超参数的值,它的行列式用来计算模型证据。
\end{enumerate}

计算神经网络的Hessian矩阵有很多近似方法,然而,使用反向传播方法的一个扩展,Hessian矩阵可以精确地被计算出来。
\subsection*{对角近似}
Hessian矩阵的一些应用需要求出Hessian矩阵的逆矩阵,而不是Hessian矩阵本身。因此,我们对Hessian矩阵的对角化近似比较感兴趣。换句话说,就是把非对角线上的元素置为零,因此这样做之后,矩阵的逆矩阵很容易计算。Hessian矩阵的对角线元素可以写成
\begin{equation}
\begin{aligned}
	\frac{\partial^2 E_n}{\partial w_{ji}^2}&=\frac{\partial^2 E_n}{\partial a_j^2}\frac{\partial a_j}{\partial w_{ji}}z_i+0\\
	&=\frac{\partial^2 E_n}{\partial a_j^2}z_i^2
\end{aligned}
\end{equation}
公式右侧的二阶导数可以通过递归地使用微分的链式法则的方式求出。这样,可以得到反向传播方程的形式为
\begin{equation}
\begin{aligned}
	\frac{\partial^2 E_n}{\partial a_j^2}&=\frac{\partial}{\partial a_j}\left[\sum_{k}\frac{\partial E_n}{\partial a_k}\frac{\partial a_k}{\partial a_j} \right]\\
	&=\frac{\partial}{\partial a_j}\left[h^{'}(a_j)\sum_{k}w_{kj}\frac{\partial E_n}{\partial a_k} \right]\\
	&=h^{'}(a_j)^2\sum_{k}\sum_{k^{'}}w_{kj}w_{k^{'}j}\frac{\partial^2 E_n}{\partial a_k\partial a_{k^{'}}}+h^{''}(a_j)\sum_{k}w_{kj}\frac{\partial E_n}{\partial a_k}
\end{aligned}
\end{equation}
如果忽略二阶导数中非对角线无素,那么我们有
\begin{equation}
	\frac{\partial^2 E_n}{\partial a_j^2}=h^{'}(a_j)^2\sum_{k}w_{kj}^2\frac{\partial^2 E_n}{\partial a_k^2}+h^{''}(a_j)\sum_{k}w_{kj}\frac{\partial E_n}{\partial a_k}
\end{equation}
对角近似的主要问题是,在实际应用中Hessian矩阵通常是强烈非对角化的,因此为了计算方便而采取的这些近似手段必须谨慎使用。
\subsection*{外积近似}
当神经网络应用于回归问题时,通常使用平方和误差函数。
我们可以把Hessian矩阵写成下面的形式
\begin{equation}
	\begin{aligned}
		\boldsymbol{H}&=\triangledown\triangledown E=\frac{\partial^2}{\partial y_n^2}\left[\frac{1}{2}\sum_{n=1}^{N}(y_n-t_n)^2 \right]\\
		&=\frac{\partial}{\partial y_n}\left[\triangledown y_n\sum_{n=1}^{N}(y_n-t_n) \right]\\
		&=\sum_{n=1}^{N}\triangledown y_n(\triangledown y_n)^T+\sum_{n=1}^{N}(y_n-t_n)\triangledown\triangledown y_n
	\end{aligned}
\end{equation}
如果网络已经在数据集上训练过,输出$y_n$恰好非常接近$t_n$,那么公式的第二项会很小,可以被忽略。我们就得到了Levenberg-Marquardt近似,或者称为外积近似(outer product approximation)。形式为
\begin{equation}
	\boldsymbol{H}\simeq \sum_{n=1}^{N}\boldsymbol{b}_n\boldsymbol{b}_n^T
\end{equation}
其中$\boldsymbol{b}_n\equiv \triangledown a_n=\triangledown y_n$,因为输出单元的激活函数就是恒等函数。Hessian矩阵近似的计算是很容易的,因为它只涉及到误差函数的一阶导数,这可能通过使用标准的反向传播算法在$O(W)$个步骤内高效地求出。需要强调的是,这种近似只在网络被恰当地训练时才成立,对于一个一般的网络映射,公式右侧的二阶导数项通常不能忽略。

在误差函数为交叉熵误差函数,输出单元激活函数为logistic sigmoid函数的神经网络中,对应的近似为
\begin{equation}
	\boldsymbol{H}\simeq \sum_{n=1}^{N}y_n(1-y_n)\boldsymbol{b}_n\boldsymbol{b}_n^T
\end{equation}
对于输出函数为softmax函数的多类神经网络,可以得到类似的结果。
\subsection*{Hessian矩阵的逆矩阵}
使用外积近似可以提出一个计算Hessian矩阵的逆矩阵的高效方法。首先,我们用矩阵的记号写出外积近似,即
\begin{equation}
\boldsymbol{H}= \sum_{n=1}^{N}\boldsymbol{b}_n\boldsymbol{b}_n^T
\end{equation}
其中,$\boldsymbol{b}_n\equiv \triangledown_{\boldsymbol{w}}a_n$是数据点n产生的输出单元激活对梯度的贡献。

现上推导一个建立Hessian矩阵的顺序步骤,每次处理一个数据点。假设我们已经使用前L个数据点得到了Hessian矩阵的逆矩阵。通过将第L+1个数据点的贡献单独写出来,我们有
\begin{equation}
	\boldsymbol{H}_{L+1}=\boldsymbol{H}_L+\boldsymbol{b}_{L+1}\boldsymbol{b}_{L+1}^T
\end{equation}
为了计算Hessian矩阵的逆矩阵,我们考虑Sherman-Morrison-Woodbury公式
\begin{equation}
\label{booy}
	(A+UV^T)^{-1}=A^{-1}-A^{-1}U(I_k+V^TA^{-1}U)^{-1}V^TA^{-1}
\end{equation}
其中,$A\in \mathbb{R}^{n\times n}$非奇异,即$A^{-1}$存在,$U,V\in \mathbb{R}^{n\times k}$。当$k=1$时的特殊形式
\begin{equation}
\begin{aligned}\label{pa}
		(A+uv^T)^{-1}&=A^{-1}-A^{-1}u(1+v^TA^{-1}u)^{-1}v^TA^{-1}\\
		&=A^{-1}-\frac{A^{-1}uv^TA^{-1}}{1+v^TA^{-1}u}
\end{aligned}
\end{equation}
SM公式看似复杂,但可以通过求解以下线性方程组来推导出来:
\begin{equation}
\label{1}
	(A+uv^T)x=b
\end{equation}
式$\ref{1}$两边同乘以$A^{-1}$,令$A^{-1}u=z,A^{-1}b=y$,则有
\begin{equation}
\label{2}
	x+zv^Tx=y
\end{equation}
注意到$V^Tx$是标量,令$a=v^Tx$。式$\ref{2}$两边同时乘以$v^T$,得
\begin{equation}
\label{3}
	a+v^Tza=v^Ty
\end{equation}
由于式$\ref{3}$中$v^Tz$和$v^Ty$都是标量,从而由式$\ref{3}$可解得
\begin{equation}
	a=\frac{v^Ty}{1+v^Tz}
\end{equation}
由式$\ref{2}$和$y,z,a$的定义可得
\begin{equation}
\label{4}
	\begin{aligned}
	x&=y-az\\
	&=A^{-1}b-A^{-1}u(1+v^TA^{-1}u)^{-1}v^TA^{-1}b\\
	&=\left[A^{-1}-A^{-1}u(1+v^TA^{-1}u)^{-1}v^TA^{-1} \right]b
	\end{aligned}
\end{equation}
由$\ref{1}$和$\ref{4}$即可得Sherman-Morrison公式,即$\ref{pa}$。
\begin{proof}
	公式$\ref{booy}$两边同乘$(A+UV^T)$
	\begin{equation}
		\begin{aligned}
		I_n&=(A+UV^T)\left[A^{-1}-A^{-1}U(I_k+V^TA^{-1}U)^{-1}V^TA^{-1}\right]\\
		&=I_n+UV^TA^{-1}-U(I_k+V^TA^{-1}U)^{-1}V^TA^{-1}-UV^TA^{-1}U(I_k+V^TA^{-1}U)^{-1}V^TA^{-1}\\
		&=I_n+UV^TA^{-1}-U(I_k+V^TA^{-1}U)(I_k+V^TA^{-1}U)^{-1}V^TA^{-1}\\
		&=I_n
		\end{aligned}
	\end{equation}
\end{proof}
Woodbury恒等式的主要用途是当$n>k$时把一个相对较大的n阶矩阵求逆的问题转化为求一个相对较小的k阶矩阵的逆矩阵。我们可以把公式看成在已知$A^{-1}$的情况下对$A+\triangle A$求逆的工具。其中$\triangle A$是一个低秩扰动。

如果我们令$\boldsymbol{H}_L=A$,且$\boldsymbol{b}_{L+1}=\boldsymbol{v}$,我们有
\begin{equation}
	\boldsymbol{H}_{L+1}^{-1}=\boldsymbol{H}_{L}^{-1}-\frac{\boldsymbol{H}_{L}^{-1}\boldsymbol{b}_{L+1}\boldsymbol{b}_{L+1}^T\boldsymbol{H}_L^{-1} }{1+\boldsymbol{b}_{L+1}^T\boldsymbol{H}_L^{-1}\boldsymbol{b}_{L+1} }
\end{equation}
使用这种方式,数据点可以依次使用,直到$L+1=N$,整个数据集被处理完毕。于是,这个结果表示一个计算Hessian矩阵的逆矩阵的算法,这个算法只需对数据集扫描一次。最开始的矩阵$\boldsymbol{H}_0$被选为$\alpha \boldsymbol{I}$,其中$\alpha$是一个较小的量,从而算法实际找的是$\boldsymbol{H}+\aleph \boldsymbol{I}$的逆矩阵。结果对于$\alpha$的精确值不是特别敏感。
\subsection*{有限差}
与误差函数的一阶导数的形式相同,我们可以使用有限差的方法求二阶导数,精度受数值计算的精度限制。
\subsection*{Hessian矩阵的精确计算}
对于一个任意的前馈拓扑结构的网络,Hessian矩阵也可以精确地计算。计算的方法是使用反向传播算法计算一阶导数的推广,同时也保留了计算一阶导数的方法的许多良好的性质,包括计算效率。这种方法可以应用于任何可微的可以表示成网络输出的函数形式的误差函数,以及任何具有可微的激活函数的神经网络。

这里我们考虑一个具体的情况,即具有两层权值的网络。这种网络中待求的方程很容易推导。我们将使用下标$i$和$i^{'}$表示输入,用下标$j$和$j^{'}$表示隐含单元,用下标$k$和$k^{'}$表示输出。首先我们定义 
\begin{equation}
	\delta_k=\frac{\partial E_n}{\partial a_k},\quad M_{kk^{'}}\equiv \frac{\partial^2 E_n}{\partial a_k \partial a_{k^{'}}}
\end{equation}
其中$E_n$是数据点n对误差函数的贡献。于是,这个网络的Hessian矩阵可以被看成三个独立的模块,即
\begin{enumerate}
	\item 两个权值都在第二层
	\begin{equation}
		\frac{\partial ^2E_n}{\partial w_{kj}^{(2)}\partial w_{k^{'}j^{'}}^{(2)}}=z_jz_{j^{'}}M_{kk^{'}}
	\end{equation}
	\item 两个权值都在第一层
	\begin{equation}
		\begin{aligned}
			\frac{\partial ^2E_n}{\partial w_{ji}^{(1)}\partial w_{j^{'}i^{'}}^{(1)}}&=x_ix_{i^{'}}h^{''}(a_{j^{'}})I_{jj^{'}}\sum_{k}w_{kj^{'}}^{(2)}\delta_k\\
			&\ +x_ix_{i^{'}}h^{'}(a_{j^{'}})\sum_{k}\sum_{k^{'}}w_{k^{'}j^{'}}^{(2)}w_{kj}^{(2)}M_{kk^{'}}
		\end{aligned}
	\end{equation}
	\item 每一层有一个权值
	\begin{equation}
		\frac{\partial ^2E_n}{\partial w_{ji}^{(1)}\partial w_{kj^{'}}^{(2)}}=x_ih^{'}(a_j)\left\{\delta_kI_{j^{'}j}+z^{j^{'}}\sum_{k^{'}}w_{k^{'}j}^{(2)}M_{kk^{'}} \right\}
	\end{equation}
\end{enumerate}
这里$I_{jj^{'}}$是单位矩阵的第$j,j^{'}$个元素。如果权值中的一个或者两个偏置项,那么只需将激活设为1即可得到对应的表达式。很容易将这个结果推广到允许网络包含跨层链接的情形。
\subsection*{Hessian矩阵的快速乘法}
\section{神经网络的正则化}
神经网络的输入单元和输出单元的数量通常由数据集的维度确定,而隐含单元的数量M是一个自由的参数,可以通过调节来给出最好的预测性能。M控制了网络中参数(权值和偏置)的数量。然后,泛化误差与M的关系不是一个简单的函数关系,因为误差函数中存在局部极小值。有其他的方式控制神经网络的模型复杂度来避免过拟合。一种方法是选择一个相对大的M值,然后通过给误差函数增加一个正则化项,来控制模型的复杂度。最简单的正则化项是二次的,给出了正则化的误差函数,形式为
\begin{equation}
\label{na}
	\tilde{E}(\boldsymbol{w})=E(\boldsymbol{w})+\frac{\lambda}{2}\boldsymbol{w}^T\boldsymbol{w}
\end{equation}
这个正则化项也被称为权值衰减(weight decay)。这样模型复杂度可以通过选择正则化系数$\lambda$来确定。正则化项可以表示为权值$\boldsymbol{w}$上的零均值高斯先验分布的负对数。
\subsection*{相容的高斯先验}
公式$\ref{na}$给出的简单权值衰减的一个局限性是,它与网络映射的确定缩放性质不相容。为了说明这一点,考虑一个多层感知器网络,这个网络有两层权值和线性输出单元,它给出了从输入变量集合$\{x_i\}$到输出变量集合$\{y_k\}$的映射。第一个隐含层的隐含单元的激活的形式为
\begin{equation}
	z_j=h\left(\sum_iw_{ji}x_i+w_{j0} \right)
\end{equation}
输出单元的激活为
\begin{equation}
	y_k=\sum_jw_{kj}z_j+w_{k0}
\end{equation}
假设我们对输入变量进行一个线性变换,形式为
\begin{equation}
	x_i\rightarrow \tilde{x}_i=ax_i+b
\end{equation}
然后我们可以根据这个映射对网络进行调整,使得网络给出的映射不变。调整的方法为,对从输入单元到隐含层单元的权值和偏置也进行一个对应的线性变换,形式为
\begin{flalign}
	w_{ji}\rightarrow \tilde{w}_{ji}&=\frac{1}{a}w_{ji}\\
	w_{j0}\rightarrow \tilde{w}_{j0}&=w_{j0}-\frac{b}{a}\sum_iw_{ji}
\end{flalign}
类似地,网络的输出变量的线性变换
\begin{equation}
	y_k\rightarrow \tilde{y}_k = cy_k + d
\end{equation}
可以通过对第二层的权值和偏置进行线性变换的方式实现。变换的形式为
\begin{flalign}
	w_{kj}\rightarrow \tilde{w}_{kj}&=cw_{kj}\\
	w_{k0}\rightarrow \tilde{w}_{k0}&=cw_{k0}+d
\end{flalign}
如果我们使用原始数据训练一个网络,还使用输入和(或)目标变换进行了上面的线性变换的数据训练一个网络,那么相容性要求这两个网络应该是等价的,差别仅在于上面给出的权值的线性变换。任何正则化项都应该与这个性质相容,否则模型就会倾向于选择某个解,而忽视某个等价的解。显然,简单的权值衰减由于把所有的权值和偏置同等对待,因此不满足这个性质。

于是我们要寻找一个正则化项,它在线性变换下具有不变性。这需要正则化项应该对于权值的重新缩放不变,对于偏置的平移不变,这样的正则化项为
\begin{equation}
\label{zz}
	\frac{\lambda_1}{2}\sum_{w\in \mathcal{W}_1}w^2+\frac{\lambda_2}{2}\sum_{w\in \mathcal{W}_2}w^2
\end{equation}
其中$\mathcal{W}_1$表示第一层的权值集合,$\mathcal{W}_2$表示第二层的权值集合,偏置未出现在求和式中。这个正则化项在权值的变换下不会发生变化,只要正则化参数进行下面的重新放缩即可:$\lambda_1 \rightarrow a^{\frac{1}{2}}\lambda_1,\lambda_2\to c^{-\frac{1}{2}}\lambda_2$

正则化项$\ref{zz}$对应于下面形式的先验概率分布
\begin{equation}
	p(\boldsymbol{w}|\alpha_1,\alpha_2)\propto \exp\left(-\frac{\alpha_1}{2}\sum_{w\in \mathcal{W}_1}w^2-\frac{\alpha_2}{2}\sum_{w\in \mathcal{W}_2}w^2 \right)
\end{equation}
注意,这种形式的先验是反常的(improper)(不能被归一化),因为偏置参数没有限制。使用反常先验会给正则化系数的选择造成很大的困难,也会给贝叶斯框架下的模型选择造成很大的困难,因为对应的模型证据等于零。因此,通常的做法是单独包含一个有着自己单独的一套超参数的偏置的先验(这就破坏了平移不变性)。

更一般地,我们可以考虑权值被分为任意数量的组$W_k$的情况下的先验,即
\begin{equation}
	p(\boldsymbol{w})\propto \exp \left(-\frac{1}{2}\sum_k \alpha_k\lVert \boldsymbol{w}\rVert_k^2 \right)
\end{equation}
其中
\begin{equation}
	\lVert \boldsymbol{w}\rVert_k^2=\sum_{j\in W_k}w_j^2
\end{equation}
作为这种形式的先验的一个特殊情况,如果我们将每个输入单元关联的权值设为一个分组,并且关于对应的参数$\alpha_k$最优化边缘似然函数,那么我们就得到了自动相关性确定(automatic relevance determination)的方法。
\subsection*{早停止}
另一种控制网络的复杂度的正则化方法是早停止(early stopping)。非线性网络模型的训练对应于误差函数的迭代减小,其中误差函数是关于训练数据集定义的。对于许多用于网络训练的最优化算法,误差函数是一个关于迭代次数的不增函数。然而,在独立数据(通常被称为验证集)上测量的误差,通常首先减小,接下来由于模型开始过拟合而逐渐增大。于是训练过程可以在关于验证集误差最小的点停止,这样可以得到一个有着较好泛化性能的网络。
\subsection*{不变性}
在许多模型识别的应用中,在对于输入变量进行了一个或者多个变换之后,预测不应该发生变化,或者说应用具有不变性(invariant)。如果可以得到足够多的训练模式,那么可调节的模型(例如神经网络)可以学习到不变性,至少可以近似地学习到。这涉及到训练集里包含足够多的表示各种变换的效果的样本。如果训练样本数受限,或者有多个不变性(变换的组合的数量随着变换的数量指数增长),那么这种方法就很不实用。于是,我们要寻找另外的方法来让可调节的模型能够表述所需的不变性。这此方法大致可以分为四类。
\begin{enumerate}
	\item 通过复制训练模式,同时根据要求的不变性进行变换,对训练集进行扩展。
	\item 为误差函数加上一个正则化项,用来惩罚当输入进行变换时,输出发生的改变。这引出了切线传播方法。
	\item 通过抽取在要求的变换下不发生改变的特征,不变性被整合到预处理过程中。任何后续的使用这些特征作为输入的回归或者分类系统就会具有不变性。
	\item 把不变性的性质整合到神经网络的构建过程中,或者对于相关向量机的方法,整合到核函数中。一种方法是通过使用局部接收场和共享权值。如卷积神经网络。
\end{enumerate}
\subsection*{切线传播}
通过切线传播(tangent propagation)的方法,我们可以使用正则化来让模型对于输入的变换具有不变性。对于一个特定的输入向量$\boldsymbol{x}_n$,考虑变换产生的效果。假设变换是连续的(例如平移或者旋转,而不是镜像翻转),那么变换的模式会扫过D维输入空间的一个流形$M$。考虑$D=2$的情形,假设变换由单一参数$\xi$控制(例如,$\xi$可能是旋转的一个角度)。那么被$\boldsymbol{x}_n$扫过的子空间$M$是一维的,并且以$\xi$为参数。令这个变换作用于$\boldsymbol{x}_n$上产生的向量为$\boldsymbol{s}(\boldsymbol{x}_n,\xi)$,且$\boldsymbol{s}(\boldsymbol{x},0)=\boldsymbol{x}$。这样曲线$M$的切线就由方向导数$\boldsymbol{\tau}=\frac{\partial \boldsymbol{x}}{\partial \xi}$给出,且点$\boldsymbol{x}_n$处的切线向量为
\begin{equation}
	\boldsymbol{\tau}_n=\frac{\partial \boldsymbol{s}(\boldsymbol{x}_n,\xi)}{\partial \xi}\Bigg|_{\xi=0}
\end{equation}
对于输入向量进行变换之后,网络的输出通常会发生变化。输出$k$关于$\xi$的导数为
\begin{equation}
\label{5126}
	\frac{\partial y_k}{\partial \xi}\Bigg|_{\xi=0}=\sum_{i=1}^{D}\frac{\partial y_k}{\partial x_i}\frac{\partial x_i}{\partial \xi}\Bigg|_{\xi=0}=\sum_{i=1}^{D}J_{ki}\tau_i
\end{equation}
其中$J_{ki}$为Jacobian矩阵$J$的第$(k,i)$个元素,公式$\ref{5126}$给出的结果可以用于修改标准的误差函数,使得在数据点的邻域之内具有不变性。修改的方法为:给原始的误差函数$E$增加一个正则化函数$\Omega$,得到下面形式的误差函数
\begin{equation}
	\tilde{E}=E+\lambda\Omega
\end{equation}
其中$\lambda$是正则化系数,且
\begin{equation}
\label{5128}
	\Omega=\frac{1}{2}\sum_n\sum_k\left(\frac{\partial y_{nk}}{\partial \xi}\Bigg|_{\xi=0} \right)^2=\frac{1}{2}\sum_n\sum_k\left(\sum_{i=1}^{D}J_{nk}\tau_{ni}\right)^2
\end{equation}
当网络映射函数在每个模式向量的邻域内具有变换不变性时,正则化函数等于零。$\lambda$的值确定了训练数据和学习不变性之间的平衡。

如果变换由L个参数控制(例如,对于二维图像的平移变换与面内旋转变换项结合),那么流形$M$的维度为L,对应的正则化项由形如$\ref{5128}$的项求和得到,每个变换都对应求和式中的一项。如果同时考虑若干个变换,并且让网络映射对于每个变换分别具有不变性,那么对于变换的组合来说就会具有(局部)不变性。

一个相关的技术,被称为切线距离(tangent distance),可以用来构造基于距离的方法(例如最近邻分类器)的不变性。
\subsection*{用变换后的数据训练}
我们已经看到,让模型对于一组变换具有不变性的一种方法是使用原始输入模式的变换后的模式来扩展训练集。这里,我们会说明,这种方法与切线传播的方法密切相关。

与上一节一样,我们要考虑由单一参数$\xi$控制的变换,且这个变换由函数$\boldsymbol{s}(\boldsymbol{x}_n,\xi)$描述,基中$\boldsymbol{s}(\boldsymbol{x}_n,0)=\boldsymbol{x}$。我们也会考虑平方和误差函数。对于未经过变换的输入,误差函数可以写成(在无限数据集的极限情况下)
\begin{equation}
	E=\frac{1}{2}\iint \{y(\boldsymbol{x})-t \}^2p(t|\boldsymbol{x})p(\boldsymbol{x})\mathrm{d}\boldsymbol{x}\mathrm{d}t
\end{equation}
这里,为了保持记号的简洁,我们考虑一个输出单元的网络。如果我们现在考虑每个数据点的无穷多个副本,每个副本都由一个变换施加了扰动,这个变换的参数为$\xi$,且$\xi$服从概率分布$p(\xi)$,那么在这个扩展的误差函数上定义的误差函数可以写成
\begin{equation}
\label{5130}
	\tilde{E}=\frac{1}{2}\iiint \{y(\boldsymbol{s}(\boldsymbol{x},\xi))-t \}^2p(t|\boldsymbol{x})p(\boldsymbol{x})p(\xi)\mathrm{d}\boldsymbol{x}\mathrm{d}t\mathrm{d}\xi
\end{equation}
我们现在假设分布$p(\xi)$的均值为零,方差很小,即我们只考虑对原始输入向量的小的变换。我们可以对变换函数进行关于$\xi$的展开,可得
\begin{equation}
	\begin{aligned}
		\boldsymbol{s}(\boldsymbol{x},\xi)&=\boldsymbol{s}(\boldsymbol{x},0)+\xi \frac{\partial }{\partial \xi}\boldsymbol{s}(\boldsymbol{x},\xi)\Bigg|_{\xi=0}+\frac{\xi^2}{2}\frac{\partial^2}{\partial \xi^2}\Bigg|_{\xi=0}+O(\xi^3)\\
		&=\boldsymbol{x}+\xi\boldsymbol{\tau}+\frac{1}{2}\xi^2\boldsymbol{\tau}^{'}+O(\xi^3)
	\end{aligned}
\end{equation}
其中$\boldsymbol{\tau}^{'}$表示$\boldsymbol{s}(\boldsymbol{x},\xi)$关于$\xi$的二阶导数在$\xi=0$处的值。这使得我们可以展开模型函数,可得 
\begin{equation}
	y(\boldsymbol{s}(\boldsymbol{x},\xi))=y(\boldsymbol{x})+\xi\boldsymbol{\tau}^T\triangledown y(\boldsymbol{x})+\frac{\xi^2}{2}\left[(\boldsymbol{\tau}^{'})^T\triangledown y(\boldsymbol{x})+\boldsymbol{\tau}^T\triangledown\triangledown y(\boldsymbol{x})\boldsymbol{\tau} \right]+O(\xi^3)
\end{equation}
代入平均误差函数$\ref{5130}$,我们有
\begin{equation}
	\begin{aligned}
		\tilde{E}&=\frac{1}{2}\iint \{y(\boldsymbol{x})-t \}^2p(t|\boldsymbol{x})p(\boldsymbol{x})\mathrm{d}\boldsymbol{x}\mathrm{d}t\\
		&+\mathbb{E}[\xi]\iint \{y(\boldsymbol{x})-t \}\boldsymbol{\tau}^T\triangledown y(\boldsymbol{x})p(t|\boldsymbol{x})p(\boldsymbol{x})\mathrm{d}\boldsymbol{x}\mathrm{d}t\\
		&+\mathbb{E}[\xi^2]\frac{1}{2}\iint \left[\{y(\boldsymbol{x})-t \}\{(\boldsymbol{\tau}^{'})^T\triangledown y(\boldsymbol{x})+\boldsymbol{\tau}^T\triangledown\triangledown y(\boldsymbol{x})\boldsymbol{\tau} \} \right.\\
		&\left.+(\boldsymbol{\tau}^T\triangledown y(\boldsymbol{x}))^2\right]p(t|\boldsymbol{x})p(\boldsymbol{x})\mathrm{d}\boldsymbol{x}\mathrm{d}t +O(\xi^3)
	\end{aligned}
\end{equation}
由于变换的分布的均值为零,因此我们有$\mathbb{E}[\xi]=0$。并且,我们把$\mathbb{E}[\xi^2]$记作$\lambda$。省略$O(\xi^3)$项,这样平均误差函数就变成了
\begin{equation}
\label{5131}
	\tilde{E}=E+\lambda\Omega
\end{equation}
其中$E$是原始的平方和误差,正则化项$\Omega$的形式为
\begin{equation}
	\begin{aligned}
		\Omega&=\frac{1}{2}\int \left[\{y(\boldsymbol{x})-\mathbb{E}[t|\boldsymbol{x}] \}\{(\boldsymbol{\tau}^{'})^T\triangledown y(\boldsymbol{x})+\boldsymbol{\tau}^T\triangledown\triangledown y(\boldsymbol{x})\boldsymbol{\tau} \}+(\boldsymbol{\tau}^T\triangledown y(\boldsymbol{x}))^2 \right]p(\boldsymbol{x})\mathrm{d}\boldsymbol{x}
	\end{aligned}
\end{equation}

我们可以进一步简化这个正则化项,如下所述,我们已经看到,使平方和误差函数达到最小值的函数为目标值$t$的条件均值$\mathbb{E}[t|\boldsymbol{x}]$。根据公式$\ref{5131}$,我们看到正则化的误差函数等于非正则化的误差函数加上一个$O(\xi^2)$的项,因此最小化总误差函数的网络函数的形式为
\begin{equation}
	y(\boldsymbol{x})=\mathbb{E}[t|\boldsymbol{x}]+O(\xi^2)
\end{equation}
从而,正则化项中的第一项消失,剩下的项为
\begin{equation}
	\Omega=\frac{1}{2}\int (\boldsymbol{\tau}^T\triangledown y(\boldsymbol{x}))^2p(\boldsymbol{x})\mathrm{d}\boldsymbol{x}
\end{equation}
这等价于切线传播的正则化项。

如果我们考虑一个特殊情况,即输入变量的变换只是简单地添加随机噪声,从而$\boldsymbol{x}\to \boldsymbol{x}+\boldsymbol{\xi}$,那么正则化项的形式为
\begin{equation}
	\Omega=\frac{1}{2}\int \lVert \triangledown y(\boldsymbol{x})\rVert^2 p(\boldsymbol{x})\mathrm{d} \boldsymbol{x}
\end{equation}
这被称为Tikhonov正则化。这个正则化项关于网络的权值的导数可以使用反向传播算法求出。我们看到,对于小的噪声,Tikhonov正则化与输入添加随机噪声有关系。可以证明,在恰当的情况下,这种做法会提升模型的泛化能力。
\subsection*{卷积神经网络}
另一种构造对输入变量的变换具有不变性的模型的方法是将不变性的性质融入到神经网络结构的构建中。这就是卷积神经网络(convolutional neural network)的基础,它被广泛地应用于图像处理领域。

考虑手写数字识别这个具体的任务。我们知道,数字的各类对于平移、缩放以及旋转具有不变性。一种简单的方法是把图像作为一个完全链接的神经网络的输入。假如数据集充分大,那么这样的网络原则上可以产生这个问题的一个较好的解,从而可以从样本中学习到恰当的不变性。然而,这种方法忽略了图像的一个关键性质,即距离较近的像素的相关性要远大于距离较远的像素的相关性。这些想法被整合到了卷积神经网络中,通过下面的三种方式:
\begin{enumerate}[(1)]
	\item 局部接收声场
	\item 权值共享
	\item 下采样
\end{enumerate}

实际构造中,可能有若干对卷积层和下采样层。在每个阶段,与前一层相比,都会有一个更高层次的关于输入变换的不变性。

整个网络可以使用误差函数最小化的方法计算。误差函数梯度的计算可以使用反向传播算法。这需要对通常的反向传播算法进行微小的修改,确保共享权值的限制能够满足。由于使用局部接收场,网络中权值的数量要小于完全连接的网络的权值数量。此外,由于权值的本质数量的限制,需要从训练数据中学习到的独立参数的数量仍然相当小。
\subsection*{软权值共享}
降低具有大量权值参数的网络复杂度的一种方法是将权值分组,然后令分组内的权值相等。这里,我们考虑软权值共享(soft weight sharing)。这种方法中,权值相等的硬限制被替换为一种形式的正则化,其中权值的分组倾向于取近似的值。此外,权值的分组、每组权值的均值,以及分组内的取值范围全都作为学习过程的一部分被确定。

简单的权值衰减正则化项可以被看成权值上的高斯分布的负对数。我们可以将权值分成若干组,而不是将所有权值分为一个组。分组的方法是使用高斯混合概率分布。混合分布中,每个高斯分量的均值、方差,以及混合系数,都会作为可调节的参数在学习过程中被确定。于是,我们有下面形式的概率密度
\begin{equation}
	p(\boldsymbol{w})=\prod_ip(w_i)
\end{equation}
其中
\begin{equation}
	p(w_i)=\sum_{j=1}^{M}\pi_j\mathcal{N}(w_i|\mu_j,\sigma_j^2)
\end{equation}
$\pi_j$为混合系数。取负对数,即可得到正则化函数,形式为
\begin{equation}
	\Omega(\boldsymbol{w})=-\sum_i\ln\left(\sum_{j=1}^{M}\pi_j\mathcal{N}(w_i|\mu_j,\sigma_j^2) \right)
\end{equation}
从而,总的误差函数为
\begin{equation}
	\tilde{E}(\boldsymbol{w})=E(\boldsymbol{w})+\lambda \Omega(\boldsymbol{w})
\end{equation}
其中,$\lambda$是正则化系数。这个误差函数同时关于权值$w_i$和混合模型参数$\{\pi_j,\mu_j,\sigma_j \}$进行最小化。如果权值是常数,那么混合模型的参数可以由EM算法确定。然而,权值分布本身在学习过程中是不断变化的,因此为了避免数值的不稳定性,我们同时关于权值和混合模型参数进行最优化。可以使用标准的最优化算法来完成这件事情。

为了最小化总的误差函数,难免计算出它关于各个可调节参数的参数是很有必要的。为了完成这一点,比较方便的做法是把$\{\pi_j \}$当成先验概率,然后引入对应的后验概率。后验概率由贝叶斯定理给出,形式为
\begin{equation}
	\gamma_j(w)=\frac{\pi_j\mathcal{N}(w_i|\mu_j,\sigma_j^2)}{\sum_k\pi_k\mathcal{N}(w_i|\mu_k,\sigma_k^2)}
\end{equation}
这样,总的误差函数关于权值的导数为
\begin{equation}
	\begin{aligned}
	\frac{\partial \tilde{E}}{\partial w_i}&=\frac{\partial E}{\partial w_i}+\lambda\frac{\partial \Omega}{\partial w_i}\\
	&=\frac{\partial E}{\partial w_i}+\lambda\sum_{j}\gamma_j(w_i)\frac{(w_i-\mu_j)}{\sigma_j^2}
	\end{aligned}
\end{equation}
正则化项的效果是把每个权值拉向第j个高斯分布的中心,拉力正比于给定权值的高斯分布的后验概率。

误差函数关于高斯分布的中心的导数
\begin{equation}
	\frac{\partial \tilde{E}}{\partial \mu_j}=\lambda\sum_i\gamma_j(w_i)\frac{(\mu_j-w_i)}{\sigma_j^2}
\end{equation}
效果是把$\mu_j$拉向了权值的平均值,拉力为第j个高斯分量产生的权值参数的后验概率。

关于方差的导数为
\begin{equation}
	\frac{\partial \tilde{E}}{\partial \sigma_j}=\lambda\sum_i\gamma_j(w_i)\left(\frac{1}{\sigma_j}-\frac{(w_i-\mu_j)^2}{\sigma_j^3} \right)
\end{equation}
效果是将$\sigma_j$拉向权值在对应的中心$\mu_j$附近的偏差的平方的加权平均,加权平均的权系数与之前一样,等于由第j个高斯分量产生的权值参数的后验概率。

关于混合系数$\pi_j$的导数,我们需要考虑下面的限制条件 
\begin{equation}
	\sum_{j}\pi_j=1,\quad 0\leqslant \pi_i \leqslant 1
\end{equation}
这个限制的产生,是因为我们把$\pi_i$看成了先验概率。可以这样做:将混合系数通过一组辅助变量$\{\eta_j \}$用softmax函数表示,即
\begin{equation}
	\pi_j=\frac{exp(\eta_j)}{\sum_{k=1}^{M}exp(\eta_k)}
\end{equation}
这样,正则化的误差函数关于$\{\eta_j \}$的导数的形式为
\begin{equation}
	\frac{\partial \tilde{E}}{\partial \eta_j}=\sum_i\{\pi_j-\gamma_j(w_i) \}
\end{equation}
效果是$\pi_j$被拉向第j个高斯分量的平均后验概率。
\section{混合密度网络}
有监督学习的目标是对条件概率分布$p(\boldsymbol{t}|\boldsymbol{x})$建模。对于许多简单的回归问题来说,这个分布都被选为高斯分布。然而,实际的机器学习问题中,经常会遇到与高斯分布差别相当大的概率分布。例如,在逆问题(inverse problem)中,概率分布可以是多峰的,这种情况下,高斯分布的假设就会产生相当差的预测结果。

正向问题通常对应于物理系统的因果关系,通常有唯一解。然而在模式识别中,我们通常不得不求解逆问题,例如在给定症状的情况下,推断疾病的种类。如果正向问题涉及到多对一映射,那么逆问题就会有多个解。例如,多种不同的疾病可能会导致相同的症状。

考虑一个相当简单的问题,这个问题中我们可以很容易地看出多峰性质。这个问题的数据的生成方式为:对服从区间$(0,1)$的均匀分布的变量$x$进行取样,得到一组值$\{x_n\}$,对应的目标值$t_n$通过下面的方式得到:计算函数$x_n+0.3\sim(2\pi x_n)$,然后添加一个服从$(-0.1,0.1)$上的均匀分布的噪声。这样,逆问题就可以这样得到:使用相同的数据点,但是交换$x$和$t$的角色。在高斯分布的假设下,最小平方方法对应于最大似然方法。我们看到,对于不服从高斯分布的逆问题,这种解法产生的模型非常差。

于是,我们寻找一个对条件概率密度建模的一般的框架。可以这样做:为$p(\boldsymbol{t}|\boldsymbol{x})$使用一个混合模型,模型的混合系数和每个分量的概率分布都是输入向量$\boldsymbol{x}$的一个比较灵活的函数,这就构成了混合密度网络(mixture density network)。对于任意给定的$\boldsymbol{x}$值。混合模型提供了一个通用的形式,用来对任意条件概率密度函数$p(\boldsymbol{t}|\boldsymbol{x})$进行建模。假设我们考虑一个足够灵活的网络,那么我们就有了一个近似任意条件概率分布的框架。

这里,我们显式地令模型的分量为高斯分布,即
\begin{equation}
	p(\boldsymbol{t}|\boldsymbol{x})=\sum_{k=1}^{K}\pi_k(\boldsymbol{x})\mathcal{N}(\boldsymbol{t}|\boldsymbol{\mu}_k(\boldsymbol{x}),\sigma_k^2(\boldsymbol{x})\boldsymbol{I})
\end{equation}

我们现在为混合模型取各种不同的参数,这些参数包括混合系数$\pi_k(\boldsymbol{x})$、均值$\boldsymbol{\mu}_k(\boldsymbol{x})$以及方差$\sigma_k^2(\boldsymbol{x})$,这些参数控制了以$\boldsymbol{x}$作为输入的神经网络的输出。混合密度网络使用相同的函数来预测所有分量概率分布的参数以及混合参数,因此非线性隐含单元被依赖于输入的函数所共享。

如果混合模型中有K个分量,且$\boldsymbol{t}$有L个分量 ,那么网络就会有K个输出单元激活(记作$a_k^{\pi}$)确定混合系数$\pi_k(\boldsymbol{x})$,有$K$个输出(记作$a_k^{\sigma}$)确定核宽度$\sigma_k(\boldsymbol{x})$,有$K\times L$个输出(记作$a_{kj}^{\mu}$)确定核中心$\boldsymbol{\mu}_k(\boldsymbol{x})$的分量$\mu_{kj}(\boldsymbol{x})$。网络输出的总数为$(L+2)K$,这与通常的网络的L个输出不同。通常的网络只是简单地预测目标变量的条件均值。

混合系数必须满足下面的限制。
\begin{equation}
	\sum_{k=1}^{K}\pi_k(\boldsymbol{x})=1,\quad 0\leqslant \pi_k(\boldsymbol{x}) \leqslant 1
\end{equation}
可以通过使用一组softmax输出来实现。
\begin{equation}
	\pi_k(\boldsymbol{x})=\frac{\exp (a_k^{\pi})}{\sum_{l=1}^{K}\exp (a_l^{\pi})}
\end{equation}
类似地,方差必须满足$\sigma_k^2(\boldsymbol{x})\geqslant 0$,因此可以使用对应的网络激活的指数形式来表示,即
\begin{equation}
	\sigma_k(\boldsymbol{x})=\exp (a_k^{\sigma})
\end{equation}
最后,由于均值$\boldsymbol{\mu}_k(\boldsymbol{x})$有实数分量,因此它们可以直接用网络的输出激活表示
\begin{equation}
	\mu_{kj}(\boldsymbol{x})=a_{kj}^{\mu}
\end{equation}

混合密度网络的可调节参数由权向量$\boldsymbol{w}$和偏置组成。这些参数可以通过最大似然法确定,或者等价地,使用最小化误差函数(负对数似然函数)的方法确定。对于独立的数据,误差函数的形式为
\begin{equation}
\label{5153}
	E(\boldsymbol{w})=-\sum_{n=1}^{N}\ln \left\{\sum_{k=1}^{K}\pi_k(\boldsymbol{x}_n,\boldsymbol{w})\mathcal{N}(\boldsymbol{t}_n|\boldsymbol{\mu}_k(\boldsymbol{x}_n,\boldsymbol{w}),\sigma_k^2(\boldsymbol{x}_n,\boldsymbol{w})\boldsymbol{I}) \right\}
\end{equation}

为了最小化误差函数,我们需要计算误差函数$E(\boldsymbol{w})$关于$\boldsymbol{w}$的分量的导数。如果我们得到了误差函数关于输出单元激活的层数的表达式,那么我们就可以通过标准的反向传播方法来计算误差函数关于$\boldsymbol{w}$的分量的导数。误差函数关于输出单元激活的导数代表了每个模式和每个输出单元的误差信号$\sigma$,并且可以反向传播到隐含单元,从而误差函数的导数可以按照通常的方式进行计算。由于误差函数$\ref{5153}$由一组项的求和式构成,每一项都对应一个训练数据点,因此我们可以考虑对于特定的模式$n$的导数,然后通过求和的方式找到$E$的导数。

由于我们处理的是混合概率分布,因此比较方便的做法是把混合系数$\pi_k(\boldsymbol{x})$看成与$\boldsymbol{x}$相关的先验概率分布,从而就引入了对应的后验概率,形式为
\begin{equation}
	\gamma_{nk}=\gamma_{k}(\boldsymbol{t}_n|\boldsymbol{x}_n)=\frac{\pi_k\mathcal{N}_{nk}}{\sum_{l=1}^{K}\pi_l\mathcal{N}_{nl}}
\end{equation}
其中$\mathcal{N}_{nk}$表示$\mathcal{N}(\boldsymbol{t}_n|\boldsymbol{\mu}_k(\boldsymbol{x}_n),\sigma_k^2(\boldsymbol{x}_n)$

关于控制混合系数的网络输出激活的导数为
\begin{equation}
	\frac{\partial E_n}{\partial a_k^{\pi}}=\pi_k-\gamma_{nk}
\end{equation}
类似地,关于控制分量均值的网络输出激活的导数为
\begin{equation}
	\frac{\partial E_n}{\partial a_k^{\pi}}=\gamma_{k}\left\{\frac{\mu_{kl}-t_{nl}}{\sigma_k^2} \right\}
\end{equation}
最后,关于控制分量方差的网络激活函数为
\begin{equation}
	\frac{\partial E_n}{\partial a_k^{\sigma}}=\gamma_{nk}\left\{L-\frac{\lVert \boldsymbol{t}_n - \boldsymbol{\mu}_k\rVert^2}{\sigma_k^2} \right\}
\end{equation}

\section{贝叶斯神经网络}
目前为止,我们对于神经网络的讨论集中于使用最大似然方法来确定网络的参数(权值和偏置)。正则化的最大似然方法可以看成MAP(maximum posterior)方法,其中正则化项可以被看成先验参数分布的对数。然而,在贝叶斯方法中,为了进行预测,我们需要对参数的概率分布进行积分或求和。

在多层神经网络的情况下,网络函数对于参数值的高度非线性的性质意味着精确的贝叶斯方法不再可行,事实上,后验概率分布的对数是非凸的,对应于误差函数中的多个局部极小值。

变分推断方法已经被用在了贝叶斯神经网络中,这种方法使用了对后验概率的分解的高斯近似,也使用了一个具有完成协方差矩阵的高斯分布。但是,最完整的贝叶斯方法是基于拉普拉斯的方法,这种方法构成了本节讨论的基础。我们会使用一个以真实后验概率的众数为中心的高斯分布来近似后验概率分布。此外,我们会假设这个高斯分布的协方差很小,从而网络函数关于参数空间的区域中的参数近似是线性关系。在参数空间中,后验概率距离概率为零的状态相当远。使用这两个近似,我们会得到与之前讨论的线性回归和线性分布的模型相类似的模型,从而我们就可以利用之前得到了结果了。这样,我们可以使用模型证据的框架来对参数进行点估计,并且比较不同的模型。
\subsection*{后验参数分布}
考虑从输入向量$\boldsymbol{x}$预测单一连续目标变量$t$的问题。我们假设条件概率分布$p(t|\boldsymbol{x})$是一个高斯分布,均值与$\boldsymbol{x}$有关,由神经网络模型的输出$y(\boldsymbol{x},\boldsymbol{w})$确定,精度$\beta$为
\begin{equation}
	p(t|\boldsymbol{x},\boldsymbol{w},\beta)=\mathcal{N}(t|y(\boldsymbol{x},\boldsymbol{w}),\beta^{-1})
\end{equation}
类似地,我们将权值$\boldsymbol{w}$的先验概率分布选为高斯分布,形式为
\begin{equation}
	p(\boldsymbol{w}|\alpha)=\mathcal{N}(\boldsymbol{w}|0,\alpha^{-1}\boldsymbol{I})
\end{equation}
对于N次独立同分布的观测$\boldsymbol{x}_1,\dots,\boldsymbol{x}_N$,对应的目标值集合$\mathcal{D}=\{t_1,\dots,t_N \}$,似然函数为
\begin{equation}
	p(\mathcal{D}|\boldsymbol{w},\beta)=\prod_{n=1}^{N}\mathcal{N}(t_n|y(\boldsymbol{x}_n,\boldsymbol{w}),\beta^{-1})
\end{equation}
因此最终的后验概率为
\begin{equation}
	p(\boldsymbol{w}|\mathcal{D},\alpha,\beta)\propto p(\boldsymbol{w}|\alpha)p(\mathcal{D}|\boldsymbol{w},\beta)
\end{equation}
由于$y(\boldsymbol{w},\boldsymbol{x})$与$\boldsymbol{w}$的关系是非线性的,因此后验概率不是高斯分布。

使用拉普拉斯近似,我们可以找到对于后验概率分布的一个高斯近似。为了完成这一点,我们必须首先找到后验概率分布的一个(局部)最大值,这必须使用迭代的数值最优化算法才能找到。比较方便的做法是最大化后验概率分布的对数,它可以写成下面的形式
\begin{equation}
	\ln p(\boldsymbol{w}|\mathcal{D})=-\frac{\alpha}{2}\boldsymbol{w}^T\boldsymbol{w}-\frac{\beta}{2}\sum_{n=1}^{N}\{y(\boldsymbol{x}_n,\boldsymbol{w})-t_n \}^2+\text{常数}
\end{equation}
这对应于一个正则化的平方和误差函数。假设$\alpha$和$\beta$都是定值,那么我们可以通过标准的非线性最优化算法,使用误差反向传播计算所需的导数,找到后验概率的最大值。我们将最大值的位置记作$\boldsymbol{w}_{MAP}$

找到了众数,我们就可以通过计算后验概率分布的负对数的二阶导数,建立一个局部的高斯近似。负对数后验概率的二阶导数为
\begin{equation}
	\boldsymbol{A}=-\triangledown\triangledown \ln p(\boldsymbol{w}|\mathcal{D},\alpha,\beta)=\alpha\boldsymbol{I}+\beta\boldsymbol{H}
\end{equation}
这里,$\boldsymbol{H}$是一个Hessian矩阵,由平方和误差函数关于$\boldsymbol{w}$的分量组成。这样,后验概率对应的高斯近似形式为
\begin{equation}
\label{5167}
	q(\boldsymbol{w}|\mathcal{D})=\mathcal{N}(\boldsymbol{w}_{MAP},\boldsymbol{A}^{-1})
\end{equation}
类似地,预测分布可以通过将后验概率分布求积分的方式获得
\begin{equation}
\label{5168}
	p(t|\boldsymbol{x},\mathcal{D})=\int p(t|\boldsymbol{x},\boldsymbol{w})q(\boldsymbol{w}|\mathcal{D})d\boldsymbol{w}
\end{equation}
然而,即使对于后验分布的高斯近似,这个积分仍然无法得到解析解,因为网络函数$y(\boldsymbol{x},\boldsymbol{w})$与$\boldsymbol{w}$的关系是非线性的。为了将计算过程进行下去,我们现在假设,与$y(\boldsymbol{x},\boldsymbol{w})$发生变化造成的$\boldsymbol{w}$幅度相比,后验概率分布的方差较小。这使得我们可以在$\boldsymbol{w}_{MAP}$附近对网络函数进行泰勒展开。只保留展开式的现行项,可得
\begin{equation}
	y(\boldsymbol{x},\boldsymbol{w})\simeq y(\boldsymbol{x},\boldsymbol{w}_{MAP})+\boldsymbol{g}^T(\boldsymbol{w}-\boldsymbol{w}_{MAP})
\end{equation}
其中,我们定义了
\begin{equation}
	\boldsymbol{g}=\triangledown_{\boldsymbol{w}}y(\boldsymbol{w},\boldsymbol{x})|_{\boldsymbol{w}=\boldsymbol{w}_{MAP}}
\end{equation}
使用这个近似,我们现在得到了一个线性高斯模型,$p(\boldsymbol{w})$为高斯分布。并且,$p(t|\boldsymbol{w})$也是高斯分布,它的均值是$\boldsymbol{w}$的线性函数,分布的形式为
\begin{equation}
	p(t|\boldsymbol{x},\boldsymbol{w},\beta)\simeq \mathcal{N}(t|y(\boldsymbol{x},\boldsymbol{w}_{MAP})+\boldsymbol{g}^T\underbrace{(\boldsymbol{w}-\boldsymbol{w}_{MAP})}_{\text{只有这里含有$\boldsymbol{w}$}},\beta^{-1} )
\end{equation}
于是,我们我们可以求出边缘分布$p(t)$
\begin{equation}
	p(t|\boldsymbol{x},\mathcal{D},\alpha,\beta)=\mathcal{N}(t|y(\boldsymbol{x},\boldsymbol{w}_{MAP}),\sigma^2(\boldsymbol{x})
\end{equation}
其中,与输入相关的方差为
\begin{equation}
	\sigma^2(\boldsymbol{x})=\beta^{-1}+\boldsymbol{g}^T\boldsymbol{A}^{-1}\boldsymbol{g}
\end{equation}
我们看到预测分布$p(t|\boldsymbol{x},\mathcal{D})$是一个高斯分布,它的均值由网络函数$y(\boldsymbol{x},\boldsymbol{w}_{MAP})$给出,参数设置为了MAP值。方差由两项组成。第一项来自目标变量的固有噪声,第二项是一个与$\boldsymbol{x}$相关的项,表示由于模型参数$\boldsymbol{w}$的不确定性造成的内插的不确定性。
\subsection*{超参数最优化}
目前为止,我们已经假定了超参数$\alpha$和$\beta$是固定的、已知的。我们可以使用模型证据框架,结合使用拉普拉斯近似得到的后验概率的高斯近似,得到确定这些超参数的值的步骤。

超参数的边缘似然函数,或者模型证据,可以通过对网络权值进行积分的方法得到,即
\begin{equation}
	p(D|\alpha,\beta)=\int p(D|\boldsymbol{w},\beta)p(\boldsymbol{w}|\alpha)\mathrm{d}\boldsymbol{w}
\end{equation}
通过使用拉普拉斯近似的结果$\ref{4135}$,这个积分很容易计算,取对数,可得
\begin{equation}
\label{5175}
	\ln p(D|\alpha,\beta)\simeq -E(\boldsymbol{w}_{MAP})-\frac{1}{2}\ln |\boldsymbol{A}|+\frac{W}{2}\ln \alpha +\frac{N}{2}\ln \beta -\frac{N}{2}\ln (2\pi)
\end{equation}
其中$W$是$\boldsymbol{w}$中参数的总数。正则化误差函数的定义为
\begin{equation}
	E(\boldsymbol{w}_{MAP})=\frac{\beta}{2}\sum_{n=1}^{N}\{y(\boldsymbol{x}_n,\boldsymbol{w}_{MAP})-t_n \}^2 +\frac{\alpha}{2}\boldsymbol{w}_{MAP}^T\boldsymbol{w}_{MAP}
\end{equation}
我们看到这与线性回归模型的对应的结果的函数形式相同。

在模型证据框架中,我们通过最大化$\ln p(D|\alpha,\beta)$对$\alpha$和$\beta$进行点估计。
\begin{enumerate}
	\item 关于$\alpha$进行最大化。首先,我们定义特征值方程
	\begin{equation}
		\beta \boldsymbol{Hu}_{i}=\lambda_i\boldsymbol{u}_i
	\end{equation}
	其中$\boldsymbol{H}$是在$\boldsymbol{w}=\boldsymbol{w}_{MAP}$处计算的Hessian矩阵,由平方和误差函数的二阶导数组成。通过类比公式$\ref{392}$,我们有
	\begin{equation}
		\alpha=\frac{\gamma}{\boldsymbol{w}_{MAP}^T\boldsymbol{w}_{MAP} }
	\end{equation}
	其中$\gamma$表示参数的有效数量,定义为
	\begin{equation}
	\label{5178}
		\gamma =\sum_{i=1}^{W}\frac{\lambda_i}{\alpha+\lambda_i}
	\end{equation}
	注意,这个结果与线性回归的情形完全相同。然而,对于非线性神经网络,它忽略了下面的事实:$\alpha$的改变会引起Hessian矩阵$\boldsymbol{H}$的改变,进而改变特征值。于是,我们隐式地忽略了涉及到$\lambda_i$关于$\alpha$的导数的项。
	
	\item 关于$\beta$最大化模型证据。可以得到下面的重估计公式 
	\begin{equation}
		\frac{1}{\beta}=\frac{1}{N-\gamma}\sum_{n=1}^{N}\{y(\boldsymbol{x}_n,\boldsymbol{w}_{MAP}-t_n) \}^2
	\end{equation}
	与线性模型一样,我们需要交替地进行超参数$\alpha$和$\beta$的重新估计以及后验概率分布的更新。然而,对于神经网络来说,由于后验概率分布的多峰性质,情况更复杂。
	
	为了比较不同的模型,例如具有不同隐含单元数量的神经网络,我们需要计算模型证据$p(D)$。将使用迭代最优化过程得到的超参数$\alpha$和$\beta$代入公式$\ref{5175}$,我们可以得到模型证据的近似。
\end{enumerate}
\subsection*{用于分类的贝叶斯神经网络}
目前,我们已经使用了拉普拉斯近似,推导出了神经网络回归模型的贝叶斯方法。我们现在要讨论的是,当应用于分类问题时,这个框架应该如何修改。这里,我们要考虑的网络有一个logistic sigmoid输出,对应于一个二分类问题。将网络扩展到多类softmax输出是很直接的。

模型的对数似然函数为
\begin{equation}
	\ln p(D|\boldsymbol{w})=\sum_{n=1}^{N}\{t_n\ln y_n +(1+t_n)\ln (1-y_n) \}
\end{equation}
其中$t_n\in \{0,1 \}$是目标值,且$y_n\equiv y(\boldsymbol{x}_n,\boldsymbol{w})$。注意,这里没有超参数$\beta$,因为我们假定数据点被正确标记了。

将拉普拉斯框架用在这个模型中的第一个阶段是初始化超参数$\alpha$,然后通过最大化对数后验概率分布的方法确定参数向量$\boldsymbol{w}$。这等价于最小化正则化误差函数。
\begin{equation}
	E(\boldsymbol{w})=-\ln p(D|\boldsymbol{w})+\frac{\alpha}{2}\boldsymbol{w}^T\boldsymbol{w}
\end{equation}
最小化的过程可以通过使用误差反向传播方法结合标准的最优化算法得到。

找到权向量的解$\boldsymbol{w}_{MAP}$之后,下一步是计算由负对数似然函数的二阶导数组成的Hessian矩阵$\boldsymbol{H}$。这样,后验概率的高斯近似就由公式$\ref{5167}$给出。

为了最优化超参数$\alpha$,我们再次最大化边缘似然函数。边缘似然函数的形式为
\begin{equation}
	\ln p(D|\alpha)=\simeq -E(\boldsymbol{w}_{MAP})-\frac{1}{2}\ln |\boldsymbol{A}|+\frac{W}{2}\ln \alpha
\end{equation}
其中,正则化的误差函数为
\begin{equation}
	E(\boldsymbol{w}_{MAP})=-\sum_{n=1}^{N}\{t_n\ln y_n+(1-t_n)\ln (1-y_n) \}+\frac{\alpha}{2}\boldsymbol{w}_{MAP}^T\boldsymbol{w}_{MAP}
\end{equation}
其中$y_n\equiv y(\boldsymbol{x}_n,\boldsymbol{w}_{MAP})$。关于$\alpha$,最大化这个模型证据函数,可以得到公式$\ref{5178}$给出的重估计方程。

最后,我们需要找到公式$\ref{5168}$定义的预测分布。与之前一样,由于网络函数的非线性的性质,积分是无法直接计算的。最简单的近似方法是假设后验概率非常窄,因此可以进行下面的近似
\begin{equation}
	p(t|\boldsymbol{x},D)\simeq p(t|\boldsymbol{x},\boldsymbol{w}_{MAP})
\end{equation}
然而,我们可以放宽这个假设,通过考虑后验概率分布的方差。在这种情况下,与回归问题的情形相同,对网络输出进行线性近似是不合适的,因为输出激活函数是logistic sigmoid函数,将输出限制在了区间$(0,1)$。相反,我们对输出激活函数进行线性近似,形式为
\begin{equation}
	a(\boldsymbol{x},\boldsymbol{w})\simeq a_{MAP}(\boldsymbol{x})+\boldsymbol{b}^T(\boldsymbol{w}-\boldsymbol{w_{MAP}})
\end{equation}
其中,$a_{MAP}(\boldsymbol{x})=a(\boldsymbol{x},\boldsymbol{w}_{MAP})$以及向量$\boldsymbol{b}\equiv \triangledown a(\boldsymbol{w}-\boldsymbol{w}_{MAP})$都可以通过反向传播方法求出。

由神经网络的权值的分布引出的输出单元激活的值的分布为
\begin{equation}
\label{5187}
	p(a|\boldsymbol{x},D)=\int \delta (a-a_{MAP}(\boldsymbol{x})-\boldsymbol{b}^T(\boldsymbol{x})(\boldsymbol{w}-\boldsymbol{w}_{MAP}))q(\boldsymbol{w}|D)\mathrm{d}\boldsymbol{w}
\end{equation}
其中$q(\boldsymbol{w}|D)$是公式$\ref{5187}$给出的对后验概率分布的高斯近似。由前面章节,我们看到这个分布是一个高斯分布,均值为$a_{MAP}\equiv a(\boldsymbol{x},\boldsymbol{w}_{MAP})$,方差为 
\begin{equation}
	\sigma_a^2(\boldsymbol{x})=\boldsymbol{b}^T(\boldsymbol{x})\boldsymbol{A}^{-1}\boldsymbol{b}(\boldsymbol{x})
\end{equation}
最后,为了得到预测分布,我们必须对$a$进行积分
\begin{equation}
	p(t=1|\boldsymbol{x},D)=\int \sigma(a)p(a|\boldsymbol{x},D)\mathrm{d}a
\end{equation}
高斯分布与logistic sigmoid函数的卷积是无法计算的。于是我们将公式$\ref{4153}$,可得
\begin{equation}
	p(t=1|\boldsymbol{x},D)=\sigma(k(\sigma_a^2)a_{MAP})
\end{equation}
其中,$k(\cdot)$由公式$\ref{4154}$定义。$\sigma_a^2$和$\boldsymbol{b}$都是$\boldsymbol{x}$的函数。

\chapter{混合模型和EM算法}
如果我们定义观测变量和潜在变量的一个联合概率分布,那么对应的观测变量本身的概率分布可以通过求边缘概率的方法得到。这使得观测变量上的复杂的边缘概率分布可以通过观测变量与潜在变量组成的扩展空间上的更加便于计算的联合概率分布来表示。因此,潜在变量的引入使得复杂的概率分布可以由简单的分量组成。本章中,我们会看到混合概率分布(例如高斯混合模型)可以用离散潜在变量来表示。连续潜在变量是后面章节的主题。

除了提供了一个构建更复杂的概率分布的框架之外,混合模型也可以用于数据聚类。因此,在开始讨论混合概率分布时,我们会考虑寻找数据点集合中的聚类的问题。我们首先使用一个非概率的方法解决这个问题,这个方法被称为K均值算法。之后,我们引入混合概率分布的潜在变量观点,其中离散潜在变量可以被看成数据点分配到了混合概率分布的具体成分当中。潜在变量模型中寻找最大似然估计的一个一般的方法是期望最大化(EM)算法。我们首先使用高斯混合分布,以一种相当非形式化的方法介绍EM算法,然后我们会基于潜在变量的观点,给出一个更加仔细的处理方法。我们会看到,K均值算法对应于用于高斯混合模型的EM算法的一个特定的非概率极限。最后,我们会以一种一般的方法讨论EM算法。

高斯混合模型广泛应用于数据挖掘、机器学习和统计分析中。在许多应用中,参数由最大似然方法确定,通常会使用EM算法。然而,正如我们将看到的那样,最大似然方法有一些巨大的局限性。在近似推断章节中,我们会看到,使用变分推断的方法,可以得到一个优雅的贝叶斯处理方式。与EM相比,这种方法几乎不需要额外的计算量,并且它解决了最大似然方法中的主要困难,也使得混合模型的分量的数量可以自动从数据中推断。
\section{一般形式的EM算法}
EM算法是一种迭代算法,1977年由Dempster等人总结提出的,用于含有隐变量(hidden variable)的概率模型参数的极大似然估计,或极大后验概率估计。EM算法的每次迭代由两步组成:E步,求期望(expectation);M步,求极大(maximization)。所以这一算法称为期望极大算法(expection maximization),简称EM算法。

概率模型有时既含有观测变量(observable variable),又含有隐变量或潜在变量(latent variable)。如果概率模型的变量都是观测变量,那么给定数据,可以直接用极大似然估计法,或贝叶斯估计法估计模型参数。但是,当模型含有隐变量时,就不能简单地使用这些估计方法。EM算法就是含有隐变量的概率模型参数的极大似然估计法,或极大后验概率估计法。仅讨论极大似然估计,极大后验概率估计与其类似。

\subsection*{EM算法的导出}
为什么EM算法能近似实验对观测数据的极大似然估计呢?下面通过近似求解观测数据的对数似然函数的极大化问题来导出EM算法。

我们面对一个含有隐变量的概率模型,目标是极大化观测数据(不完全数据)$Y$关于参数$\theta$的对数似然函数,即极大化
\begin{equation}
\label{极大}
	\begin{aligned}
		L(\theta)&=log P(Y|\theta) =log \sum_{Z}P(Y,Z|\theta) \\
		&=log\left(\sum_{Z}P(Y|Z,\theta)P(Z|\theta)\right)
	\end{aligned}
\end{equation}
注意到这一极大化的主要困难是式$\ref{极大}$中有未观测数据并有包含和(或积分)的对数。

事实上,EM算法是通过迭代逐步极大化$L(\theta)$的。假设在第$i$次迭代后$\theta$的估计值是$\theta^{(i)}$。我们希望新估计值$\theta$能使$L(\theta)$增加,即$L(\theta)>L(\theta^{(i)})$,并逐步达到极大值。为此,考虑两者的差:
\begin{equation}
	L(\theta)-L(\theta^{(i)})=log \left(\sum_{Z}P(Y|Z,\theta)P(Z|\theta) \right)-log P(Y|\theta^{(i)})
\end{equation}
利用Jensen不等式\footnote{这里用到的是$log\sum_{j}\lambda_jy_j\geqslant \sum_{j}\lambda_jlogy_j$}(Jensen inequality)得到其下界:
\begin{equation}
	\begin{aligned}
		L(\theta)-L(\theta^{(i)})&=log \left(\sum_{Z}{\color{red}{P(Y|Z,\theta^{(i)})}} \frac{P(Y|Z,\theta)P(Z|\theta)}{{\color{red}{P(Y|Z,\theta^{(i)})}}}  \right)-log P(Y|\theta^{(i)})\\
		&\geqslant \sum_{Z} P(Y|Z,\theta^{(i)})log\frac{P(Y|Z,\theta)P(Z|\theta)}{P(Y|Z,\theta^{(i)})}-log P(Y|\theta^{(i)})\\
		&=\sum_{Z}P(Y|Z,\theta^{(i)})log\frac{P(Y|Z,\theta)P(Z|\theta)}{P(Y|Z,\theta^{(i)})P(Y|\theta^{(i)})}
	\end{aligned}
\end{equation}
令
\begin{equation}
	B(\theta,\theta^{(i)})=L(\theta^{(i)})+\sum_{Z}P(Y|Z,\theta^{(i)})log\frac{P(Y|Z,\theta)P(Z|\theta)}{P(Y|Z,\theta^{(i)})P(Y|\theta^{(i)})}
\end{equation}
则
\begin{equation}
	L(\theta)\geqslant B(\theta,\theta^{(i)})
\end{equation}
即函数$B(\theta,\theta^{(i)})$是$L(\theta)$的一个下界,因此\textbf{任何可以使$B(\theta,\theta^{(i)})$增大的$\theta$,也可以使$L(\theta)$增大}。为了使$L(\theta)$尽可能大的增大,选择$\theta^{(i+1)}$使$B(\theta,\theta^{(i)})$达到极大,即
\begin{equation}
	\theta^{(i+1)}=arg \mathop
	{max}\limits_{\theta}B(\theta,\theta^{(i)})
\end{equation}
现在求$\theta^{(i+1)}$的表达式。省去对$\theta$的极大化而言是常数的项,有
\begin{equation}
\label{目标}
	\begin{aligned}
		\theta^{(i+1)}&=arg \mathop
		{max}\limits_{\theta}\left( L(\theta^{(i)})+\sum_{Z}P(Y|Z,\theta^{(i)})log\frac{P(Y|Z,\theta)P(Z|\theta)}{P(Y|Z,\theta^{(i)})P(Y|\theta^{(i)})} \right)\\
		&=arg \mathop
		{max}\limits_{\theta}\left( \sum_{Z}P(Y|Z,\theta^{(i)})log( P(Y|Z,\theta)P(Z|\theta)) \right)\\
		&=arg \mathop
		{max}\limits_{\theta}\left( {\color{red}{\sum_{Z}P(Y|Z,\theta^{(i)})log( P(Y,Z|\theta))}} \right)\\
		&=arg \mathop
		{max}\limits_{\theta}\underline{Q(\theta,\theta^{(i)})}
	\end{aligned}
\end{equation}
式$\ref{目标}$等价于EM算法的一次迭代,即求$Q$函数及其极大化。\textbf{EM算法是通过不断求解下界的极大化逼近求解对数似然函数极大化的算法}。
\begin{definition}{Q函数}{Q函数}
	完全数据的对数似然函数$logP(Y,Z|\theta)$关于在给定观测数据$Y$和当前参数$\theta^{(i)}$下对未观测数据$Z$的条件概率分布$P(Z|Y,\theta^{(i)})$的期望称为Q函数。即
	\begin{equation}
		Q(\theta,\theta^{(i)})=E_Z[log P(Y,Z|\theta)|Y,\theta^{(i)}]
	\end{equation}
\end{definition}
\subsection*{EM算法}
输入:观测变量数据$Y$,隐变量数据$Z$,联合分布$P(Y,Z|\theta)$,条件分布$P(Z|Y,\theta)$;

输出:模型参数$\theta$

\begin{enumerate}
	\item 选择参数的初值$\theta^{(0)}$,开始迭代;
	
	\item E步:记$\theta^{(i)}$为第i次迭代参数$\theta$的估计值,在第i+1次迭代的E步,计算
	\begin{equation}
		\begin{aligned}
			Q(\theta,\theta^{(i)}) &= E_Z[log P(Y,Z|\theta) |Y,\theta^{(i)}] \\
			&= \sum_Zlog P(Y,Z|\theta)P(Z|Y,\theta^{(i)})
		\end{aligned}	
	\end{equation}
	这里,$P(Z|Y,\theta^{(i)})$是在给定观测数据Y和当前的参数估计$\theta^{(i)}$下隐变量数据Z的条件概率分布;
	
	\item M步:求使$Q(\theta,\theta^{(i)})$极大化的$\theta$,确定第i+1次迭代的参数的估计值$\theta^{(i+1)}$
	\begin{equation}
		\theta^{(i+1)} = arg \mathop{max}\limits_\theta Q(\theta, \theta^{(i)})
	\end{equation}
	
	\item 重复第(2)、(3)步,直到收敛。
\end{enumerate}
\subsection*{EM算法的收敛性}
EM算法提供一种近似计算含有隐变量概率模型的极大似然估计的方法。EM算法的最大优点是简单性和普适性。我们自然地要问:EM算法得到的估计序列是否收敛?如果收敛,是否收敛到全局最大值或局部极大值?
\begin{theorem}{}{}
	设$P(Y|\theta)$为观测数据的似然函数,$\theta^{(i)}(i=1,2,\dots)$为EM算法得到的参数估计序列,$P(Y|\theta^{(i)})(i=1,2,\dots)$为对应的似然函数序列,则$P(Y|\theta^{(i)})$是单调递增的,即
	\begin{equation}
		P(Y|\theta^{(i+1)})\geqslant P(Y|\theta^{(i)})
	\end{equation}
	设$L(\theta)=logP(Y|\theta)$为观测数据的对数似然函数,$L(\theta^{(i)})(i=1,2,\dots)$为对应的对数似然函数序列。
	\begin{enumerate}
		\item 如果$P(Y|\theta)$有上界,则$L(\theta^{(i)})=logP(Y|\theta^{(i)})$收敛到某一值$L^*$
		\item 在函数$Q(\theta,\theta^{'})$与$L(\theta)$满足一定条件下,由EM算法得到的参数估计序列$\theta^{(i)}$的收敛值$\theta^*$是$L(\theta)$的稳定点
	\end{enumerate}
\end{theorem}

\section{EM的另一种观点}
\section{K均值聚类}
\section{混合高斯}
EM算法的一个重要应用是高斯混合模型的参数估计。高斯混合模型应用广泛,在许多情况下,EM算法是学习高斯混合模型(Gaussian misture model)的有效方法。
\begin{definition}{高斯混合模型}{}
	高斯混合模型是指具有如下形式的概率分布模型:
	\begin{equation}
		P(y|\theta)=\sum_{k=1}^{K}\alpha_k\phi(y|\theta_k)
	\end{equation}
	其中,$\alpha_k$是系数,$\alpha_k\geqslant 0,\sum\limits_{k=1}^{K}\alpha_k=1;\phi(Y|\theta_k)$是高斯分布密度,$\theta_k=(\mu_k,\sigma_k^2)$,
	\begin{equation}
		\phi(y|\theta_k)=\frac{1}{\sqrt{2\pi}\sigma_k}exp\left(-\frac{(y-\mu_k)^2}{2\sigma_k^2}\right)
	\end{equation}
	称为第k个模型。
\end{definition}
\subsection*{高斯混合模型参数估计的EM算法}
假设观测数据$y_1,y_2,\dots,y_N$由高斯混合模型生成,
\begin{equation}
	P(y|\theta)=\sum_{k=1}^{K}\alpha_k\phi(y|\theta_k)
\end{equation}
其中,$\theta=(\alpha_1,\alpha_2,\dots,\alpha_K;\theta_1,\theta_2,\dots,\theta_K)$。我们用EM算法估计高斯混合模型的参数$\theta$
\begin{enumerate}
	\item 明确隐变量,写出完全数据的对数似然函数
	
	隐变量由$\gamma_{jk}$表示,其定义如下:
	\begin{equation}
		\gamma_{jk}=\begin{cases}
		1,\qquad \text{第j个观测来自第k个模型}\\0,\qquad \text{否则}
		\end{cases}
	\end{equation}
	$\gamma_{jk}$是0-1随机变量。
	那么完全数据是
	\begin{equation*}
		(y_j,\gamma_{j1},\gamma_{j2},\dots,\gamma_{jK}),\qquad j=1,2,\dots,K
	\end{equation*}
	于是可以写出完全数据的似然函数
	\begin{equation*}
		\begin{aligned}
			P(y,\gamma|\theta)&=\prod_{j=1}^{N}P(y_j,\gamma_{j1},\gamma_{j2},\dots,\gamma_{jK}|\theta)\\
			&=\prod_{k=1}^{K}\prod_{j=1}^{N}\left[\alpha_k\phi(y|\theta_k)\right]^{\gamma_{jk}}\\
			&=\prod_{k=1}^{K}\alpha_k^{n_k}\prod_{j=1}^{N}\left[\phi(y|\theta_k)\right]^{\gamma_{jk}}\\
			&=\prod_{k=1}^{K}\alpha_k^{n_k}\prod_{j=1}^{N}\left[ \frac{1}{\sqrt{2\pi}\sigma_k}exp\left(-\frac{(y-\mu_k)^2}{2\sigma_k^2}\right) \right]^{\gamma_{jk}}\\
		\end{aligned}
	\end{equation*}
	式中,$n_k=\sum\limits_{j=1}^{N}\gamma_{jk},\sum\limits_{k=1}^{K}n_{k}=N$。(意思是选择第k个高斯模型的次数)
	
	那么,完全数据的对数似然函数为
	\begin{equation*}
		log P(y,\gamma|\theta)=\sum_{k=1}^{K}\left\{n_klog\alpha_k+\sum_{j=1}^{N}\gamma_{jk}\left[log\left(\frac{1}{\sqrt{2\pi}}\right)-log\sigma_k-\frac{1}{2\sigma_k^2}(y_j-\mu_k)^2\right] \right\}
	\end{equation*}
	\item EM算法的E步:确定Q函数
	
	\begin{equation*}
		\begin{aligned}
		Q(\theta,\theta^{(i)})&=E[log P(y,\gamma|\theta)|y,\theta^{(i)}]\\
		&=E\left\{\sum_{k=1}^{K}\left\{n_klog\alpha_k+\sum_{j=1}^{N}\gamma_{jk}\left[log\left(\frac{1}{\sqrt{2\pi}}\right)-log\sigma_k-\frac{1}{2\sigma_k^2}(y_j-\mu_k)^2\right] \right\}  \right\}\\
		&=\sum_{k=1}^{K}\left\{\overbrace{n_klog\alpha_k}^{{\color{red}{\text{常数}}}}+\sum_{j=1}^{N}{\color{red}{E(\gamma_{jk})}}\left[log\left(\frac{1}{\sqrt{2\pi}}\right)-log\sigma_k-\frac{1}{2\sigma_k^2}(y_j-\mu_k)^2\right] \right\} 
		\end{aligned}
	\end{equation*}
	这里需要计算$E(\gamma_{jk}|y,\theta)$,记为$\hat{\gamma}_{jk}$
	\begin{equation*}
		\begin{aligned}
		\hat{\gamma}_{jk}&=E(\gamma_{jk}|y,\theta)=P(\gamma_{jk}=1|y,\theta)(\text{二项分布})\\
		&=\frac{P(\gamma_{jk}=1|y,\theta)}{\sum\limits_{k=1}^{K}P(\gamma_{jk}=1|y,\theta)}\\
		&=\frac{P(y_j|\gamma_{jk}=1,\theta)P(\gamma_{jk}=1|\theta)}{\sum\limits_{k=1}^{K}P(y_j|\gamma_{jk}=1,\theta)P(\gamma_{jk}=1|\theta)}\\
		&=\frac{\alpha_k\phi(y_j|\theta_k)}{\sum\limits_{k=1}^{K}\alpha_k\phi(y_j|\theta_k)},\quad j=1,2,\dots,N;k=1,2,\dots,K
		\end{aligned}
	\end{equation*}
	$\hat{\gamma}_{jk}$是在当前模型参数下第j个观测数据来自第k个分模型的概率,称为分模型k对观测数据$y_j$的响应度。代入式中得
	\begin{equation}
		Q(\theta,\theta^{(i)})=\sum_{k=1}^{K}\left\{n_klog\alpha_k+\sum_{j=1}^{N}{\color{red}{\hat{\gamma}_{jk}}}\left[log\left(\frac{1}{\sqrt{2\pi}}\right)-log\sigma_k-\frac{1}{2\sigma_k^2}(y_j-\mu_k)^2\right] \right\} 
	\end{equation}
	\item 确定EM算法的M步
	迭代的M步是求函数$Q(\theta,\theta^{(i)})$对$\theta$的极大值,即求新一轮迭代的模型参数:
	\begin{equation*}
		\theta^{(i+1)} = arg \mathop{max}\limits_\theta Q(\theta, \theta^{(i)})
	\end{equation*}
	分别对$\mu_k,\sigma_k^2$求偏导并令其为0,即可得到$\hat{\mu}_k,\hat{\sigma}_k$。求$\hat{a}_k$是在$\sum\limits_{k=1}^K=1$条件下求偏导并令其为0得到。
	\begin{flalign}
		\hat{\mu}_k&=\frac{\sum\limits_{j=1}^N\hat{\gamma}_{jk}y_j}{\sum\limits_{j=1}^N\hat{\gamma}_{jk}},\quad k=1,2,\dots,K\\
		\hat{\sigma}_k^2&=\frac{\sum\limits_{j=1}^N\hat{\gamma}_{jk}(y_j-\mu_k)^2}{\sum\limits_{j=1}^N\hat{\gamma}_{jk}},\quad k=1,2,\dots,K\\
		\hat{\alpha}_k&=\frac{n_k}{N}=\frac{\sum\limits_{j=1}^N\hat{\gamma}_{jk}}{N},\quad k=1,2,\dots,K\\
	\end{flalign}
	\item 重复以上计算,直到对数似然函数值不再有明显的变化为止。
\end{enumerate}

\chapter{组合模型}
\input{./chapter/统计机器学习/集成学习/集成学习}
\section{贝叶斯模型平均}

\section{委员会}

\section{提升方法}

\section{基于树的模型}

\section{条件混合模型}

\section{logistic模型的混合}

\chapter{支持向量机}
支持向量机(support vector machines,SVM)是一种二类分类模型。它的基本模型是定义在特征空间上的间隔最大的线性分类器,间隔最大使它有别于感知机;支持向量机还包括核技巧,这使它成为实质上的非线性分类器。支持向量机的学习策略就是间隔最大化,可形式化为一个求解凸二次规划(convex quadratic programming)的问题,也等价于正则化的合页损失函数的最小化问题。支持向量机的学习算法是求解凸二次规划的最优化算法。
 
支持向量机学习方法包含构建由简至繁的模型:线性可分支持向量机(linear support vector machine in linearly separable case)、线性支持向量机(linear support vector machine)及非线性支持向量机(non-linear support vector machine)。简单模型是复杂模型的基础,也是复杂模型的特殊情况。当训练数据线性可分时,通过软件间隔最大化(hard margin maximization),学习一个线性的分类器,即线性可分支持向量机,又称为硬间隔支持向量机;当训练数据近似线性可分时,通过软件间隔最大化(soft margin maximization),也学习一个线性的分类器,即线性支持向量机,又称谓软间隔支持向量机;当训练数据线性不可分时,通过使用核技巧(kernel trick)及软间隔最大化,学习非线性支持向量机。 

当输入空间为欧氏空间或离散集合、特征空间为希尔伯特空间时,核函数(kernel function)表示将输入从输入空间映射到特征空间得到的特征向量之间的内积。通过使用核函数可以学习非线性支持微量机,等价于隐式地在高维的特征空间中学习线性支持向量机。这样的方法称为核技巧。核方法(kernel method)是比支持向量机更为一般的机器学习方法。 

Cortes与Vapnik提出线性支持向量机,Boser、Guyon与Vapnik又引入核技巧,提出非线性支持向量机。
\section{间隔与支持向量}

\section{对偶问题}
注意到式$\ref{eq:3}$本身是一个凸二次规划问题,能直接用现成的优化计算包求解,但我们可以有更高效的办法。由于这个问题的特殊结构,还可以通过拉格朗日对偶性变换到对偶变量的优化问题,即通过求解与原问题等价的对偶问题得到原始问题的最优解,这就是线性可分条件下支持向量机的对偶算法,这样做的优点在于:
\begin{enumerate}
	\item 对偶问题往往更容易求解;
	\item 可以自然的引入核函数,进而推广到非线性分类问题。	
\end{enumerate}
该问题的拉格朗日函数可写为
\begin{equation}
	L(w,b,a) = \frac{1}{2}\lVert w \rVert^2 - \sum_{i=1}^na_i (y_i(w^T x_i + b) - 1)
\end{equation}
然后令
\begin{equation}
	\theta (\boldsymbol{w}) = \mathop{max} \limits_{a_i \geq 0}\ L(\boldsymbol{w},b,a)
\end{equation}
具体写出来,目标函数变成了
\begin{equation}
	\mathop{min}\limits_{\boldsymbol{w},b}\ \theta(\boldsymbol{w}) = \mathop{min}\limits_{\boldsymbol{w},b}\mathop{ max}\limits_{a_i \geq 0}\ L(\boldsymbol{w},b,a) = p^*
\end{equation}
这里用$p^*$表示这个问题的最优值,且和最初的问题是等价的。如果直接求解,那么一上来便得面对$w$和$b$两个参数,而$a_i$以是不等式约束,这个求解过程不好做。考虑对偶问题
\begin{equation}
		\mathop{min}\limits_{\boldsymbol{w},b}\ \theta(\boldsymbol{w}) = \mathop{ max}\limits_{a_i \geq 0}\ \mathop{min}\limits_{\boldsymbol{w},b}\ L(\boldsymbol{w},b,a) = d^*
\end{equation}
原始问题通过满足KKT条件,已经转化成了对偶问题。而求解这个对偶问题,分为3个步骤
\begin{enumerate}
	\item \textbf{让$L(\boldsymbol{w},b,a)$关于$\boldsymbol{w}$和$b$最小化}
	
	首先固定$a$,要让$L$关于$w$和$b$最小化,分别对$w$和$b$求偏导,令其等于0;
	\begin{equation}
		\begin{aligned}
		\frac{\partial L}{\partial \boldsymbol{w}} &= \lVert \boldsymbol{w}\rVert - \sum_{i = 1}^n a_iy_ix_i \boldsymbol{w}^T = 0 \Rightarrow  \boldsymbol{w} = \sum_{i = 1}^n a_iy_ix_i  \\
		\frac{\partial L}{\partial b} &= \sum_{i = 1}^n a_iy_i = 0\Rightarrow  \sum_{i = 1}^n a_iy_i  = 0
		\end{aligned}
	\end{equation}
	将以上结果代入之前的L,得到
	\begin{equation}
		\begin{aligned}
		L(\boldsymbol{w},b,a) &= \frac{1}{2}\sum_{i,j=1}^na_ia_jy_iy_jx_i^Tx_j - \sum_{i,j=1}^na_ia_jy_iy_jx_i^Tx_j - b\sum_{i = 1}^n a_iy_i + \sum_{i = 1}^na_i \\
		&= \sum _{i = 1}^n a_i - \frac{1}{2}\sum_{i,j=1}^na_ia_jy_iy_jx_i^Tx_j
		\end{aligned}
	\end{equation}
	\item \textbf{求对$a$的极大}
	
	求对$a$的极大,即是关于对偶问题的最优化问题。经过上一个步骤的求解,得到的拉格朗日函数式子已经没有了变量$w$和$b$,只有$a$。从上面的式子得到
	\begin{equation}
		\begin{aligned}
		\mathop{max}\limits_a \quad &\sum _{i = 1}^n a_i - \frac{1}{2}\sum_{i,j=1}^na_ia_jy_iy_jx_i^Tx_j \\
		s.t. \quad &a_i \geq 0,i = 1,\dots, n \\
		&\sum_{i = 1}^n a_iy_i  = 0
		\end{aligned}
	\end{equation}
	这样,求出了$a_i$,从而根据
	\begin{equation}
		\begin{aligned}
		\boldsymbol{w}^* &= \sum_{i = 1}^n a_iy_ix_i \\
		b^* &= -\frac{\mathop{max}\ {\boldsymbol{w}^*}^T x_i + \mathop{min}\ {\boldsymbol{w}^*}^T x_i}{2} 
		\end{aligned}
	\end{equation}
	即可求出$w,b$,最终得出分离超平面和分类决策函数。
	\item \textbf{利用SMO算法求解对偶问题中的拉格朗日乘子}
	
	在求得$L(w,b,a)$关于$w$和$b$最小化和对$a$的极大之后,最后一步便是利用SMO算法求解对偶问题中的拉格朗日乘子	
\end{enumerate}

\section{序列最小最优算法}
接上一小节,讨论支持向量机学习的实现问题。讲述其中的序列最小最优化(sequential minimal optimization,SMO)算法。

SMO算法是一种启发式算法,其基本思路是:\textbf{如果所有变量的解都满足此最优化问题的KKT条件,那么这个最优化问题的解就得到了}。因为KKT条件是该最优化问题的充分必要条件。否则,选择两个变量,固定其他变量,针对这两个变量构建一个二次规划问题。这个二次规划问题关于这两个变量的解应该更接近原始二次规划问题的解,因为这会使得原始二次规划问题的目标函数值变得更小。重要的是,这时子问题可以通过解析方法求解,这样就可以大大提高整个算法的计算计算速度。子问题有两个变量,一个是违反KKT条件最严重的那一个,另一个由约束条件自动确定。如此,SMO算法将原问题不断分解为子问题并对子问题求解,进而达到求解原问题的目的。

SMO算法要解如下凸二次规划的对偶问题:
\begin{flalign}
	\mathop{min}\limits_a \quad &\frac{1}{2}\sum_{i=1}^N\sum_{j=1}^{N}a_ia_jy_iy_jK(x_i, x_j) - \sum _{i = 1}^N a_i\label{eq:最优化问题}\\
	s.t. \quad &C\geq a_i \geq 0,\quad i = 1,\dots, n \\
	&\sum_{i = 1}^N a_iy_i  = 0\label{eq:等式约束}
\end{flalign}
注意,子问题的两个变量中只有一个是自由变量。假设$a_1,a_2$为两个变量,$a_3,a_4,\dots,a_N$固定,那么由等式约束$\ref{eq:等式约束}$可知
\begin{equation}
	a_1 = - y_1 \sum_{i=2}^{N}a_iy_i
\end{equation}
如果$a_2$确定,那么$a_1$也随之确定。所以子问题中同时更新两个变量。

整个SMO算法包括两个部分:\textbf{求解两个变量二次规划的解析方法和选择变量的启发式方法}。

于是SMO的最优化问题$\ref{eq:最优化问题}$的子问题可以写成
\begin{flalign}
	\mathop{min}\limits_{a_1,a_2}\ W(a_1,a_2)&=\frac{1}{2}K_{11}a_1^2+\frac{1}{2}K_{22}a_2^2+y_1y_2K_{12}a_1a_2 \nonumber\\
	&-(a_1+a_2)+y_1a_1\sum_{i=3}^{N}y_ia_iK_{i1}+y_2a_2\sum_{i=3}^{N}y_ia_iK_{i2}\label{拆解后的公式}\\
	s.t\ &a_1y_1+a_2y_2=-\sum_{i=3}^{N}y_ia_i=\varsigma\\
	&0\leq a_i \leq C,\ i=1,2
\end{flalign}
其中,$K_{ij}=K(x_i,x_j),i,j=1,2,\dots,N,\varsigma$是常数。

为了求解两个变量的二次规划问题$\ref{拆解后的公式}$,首先分析约束条件,然后在此约束条件下求极小。由于只有两个变量$(a_1,a_2)$,约束可以用二维空间中的图形表示,如图所示

\vspace{4pt}
\begin{center}
\begin{tikzpicture}
	\draw (0,0) rectangle (4,4);
	\node (a) at (2,-0.5) {$a_2=0$};
	\node at (2,4.5) {$a_2=C$};
	\node at (-0.5,2) {$a_1=0$};
	\node at (4.5,2) {$a_1=C$};
	\node [below of=a]{$y_1\ne y_2\Rightarrow a_1-a_2=k$};
	\draw[domain=1:4] plot (\x,{\x -1});
	\draw[domain=0:3] plot (\x,{\x +1});
	
	\draw[domain=6:9] plot (\x,{-\x +9});
	\draw[domain=7:10] plot (\x,{-\x +11});  
	\draw (6,0) rectangle (10,4);
	\node (b) at (8,-0.5) {};
	\node [below of=b]{$y_1 = y_2\Rightarrow a_1+a_2=k$};
\end{tikzpicture}
\end{center}
不等式约束使得$(a_1,a_2)$在盒子$[0,C]\times[0,C]$内,等式约束使$(a_1,a_2)$在平行盒子的对角线的直线上。因此要求的是目标函数在一条平行于对角线线的线段上的最优值。这使得两个变量的最优化问题成为实质上的单变量的最优化问题,不妨考虑为变量$a_2$的最优化问题。

假设问题$\ref{拆解后的公式}$的初始可行解为$a_1^{old},a_2^{old}$,最优解为$a_1^{new},a_2^{new}$,并且假设在沿着约束方向未经剪辑时$a_2$的最优解为$a_2^{new,unc}$。

引进记号
\begin{flalign}
	g(x)&=\sum_{i=1}^{N}a_iy_iK(x_i,x)+b\\
	v_i&=\sum_{j=3}^{N}y_ia_iK(x_i,x_j)=g(x_i)-\sum_{j=1}^{2}a_jy_jK(x_i,x_j)-b,\ i=1,2
\end{flalign}
目标函数可写成
\begin{equation}
\label{目标函数}
	W(a_1,a_2)=\frac{1}{2}K_{11}a_1^2+\frac{1}{2}K_{22}a_2^2+y_1y_2K_{12}a_1a_2 -(a_1+a_2)+y_1a_1v_1+y_2a_2v_2
\end{equation}
令
\begin{equation}
	E_i=g(x_i)-y_i=\left(\sum_{j=1}^{N}a_jy_jK(x_j,x_i)+b\right)-y_i,\quad i=1,2
\end{equation}
当$i=1,2$时,$E_i$为函数$g(x)$对输入$x_i$的预测值与真实输出$y_i$之差。
由$a_1y_1=\varsigma -a_2y_2$及$y_i^2=1$,可将$a_1$表示为
\begin{equation}
	a_1=(\varsigma-y_2a_2)y_1
\end{equation}
代入式$\ref{目标函数}$,得到只是$a_2$的函数的目标函数,对$a_2$求导数
\begin{equation}
	\frac{\partial W}{\partial a_2}=K_{11}a_2+K_{22}a_2-2K_{12}a_2-K_{11}\varsigma y_2 + K_{12}\varsigma y_2+y_1y_2-1-v_1v_2+y_2v_2
\end{equation}
令其为0,得到 
\begin{equation}
\begin{split}
	(K_{11}+K_{22}-2K_{12})a_2=&y_2(y_2-y_1+\varsigma K_{11}-\varsigma K_{12}+v_1-v2)\\
	=&y2\left[ y_2-y_1+\varsigma K_{11}-\varsigma K_{12} +\left( g(x_1) - \sum_{j=1}^{2}y_ja_jK_{1j}-b \right)\right. \\
	&\left. \left( g(x_2) - \sum_{j=1}^{2}y_ja_jK_{2j}-b \right) \right]
\end{split}
\end{equation}
将$\varsigma=a_1^{old}y_1+a_2^{old}y_2$代入,得到
\begin{equation}
	\begin{split}
	(K_{11}+K_{22}-2K_{12})a_2^{new,unc}&=y_2((K_{11}+K_{22}-2K_{12})a_2^{old}y_2+y_2-y_1+g(x_1)-g(x_2))\\
	&=(K_{11}+K_{22}-2K_{12})a_2^{old}+y_2(E_1-E_2)
	\end{split}
\end{equation}
将$\eta = K_{11}+K_{22}-2K_{12}$代入,于是得到
\begin{equation}
\label{依赖}
	a_2^{new,unc}=a_2^{old}+\frac{y_2(E_1-E_2)}{\eta}
\end{equation}
要使其满足不等式的约束必须将其限制在区间$[L,H]$内,从而得到$a_2^{new}$的表达式
\begin{equation}
	a_2^{new}=
	\begin{cases}
	H,\qquad a_2^{new,unc}>H\\
	a_2^{new,unc},\qquad L\leq a_2^{new,unc}\leq H\\
	L,\qquad a_2^{new,unc}<L
	\end{cases}
\end{equation}
如果$y_1\ne y_2$
\begin{equation*}
	L=\mathop{max}(0,a_2^{old}-a_1^{old}),\quad H=\mathop{min}(C,C+a_2^{old}-a_1^{old})
\end{equation*}
如果$y_1 = y_2$
\begin{equation*}
L=\mathop{max}(0,a_2^{old}+a_1^{old}-C),\quad H=\mathop{min}(C,a_2^{old}+a_1^{old})
\end{equation*}
\textbf{变量的选择方法}

SMO算法在每个子问题中选择两个变量优化,其中至少一个变量是违反KKT条件的。
\begin{enumerate}
	\item 第1个变量的选择
	
	SMO称选择第1个变量的过程为外层循环。外层循环在训练样本中选取违反KKT条件最严重的样本点,并将其对应的变量作为第1个变量。具体地,检验训练样本点$(x_i,y_i)$是否满足KKT条件,即
	\begin{flalign}
		a_i=0&\Leftrightarrow y_ig(x_i)\geq 1\\
		C>a_i>0&\Leftrightarrow y_ig(x_i)= 1\\
		a_i=C&\Leftrightarrow y_ig(x_i)\leq 1
	\end{flalign}
	\item 第2个变量的选择
	
	SMO称选择第2个变量的过程为内层循环,假设在外层循环中已经找到第1个变量$a_1$,现在要在内层循环中找到第2个变量$a_2$。第2个变量选择的标准是希望能使$a_2$有足够大的变化。由式$\ref{依赖}$知,$a_2^{new}$是依赖于$|E_1-E_2|$的,加了加快计算速度,一种简单的做法是选择$a_2$,使其对应的$|E_1-E_2|$最大。
	\item 计算阈值$b$和差值$E_i$
\end{enumerate}

\section{非线性支持向量机与核函数}
SVM处理线性可分的情况,而对于非线性的情况,SVM的处理方法是选择一个核函数$K<\cdot,\cdot>$,通过将数据映射到高维空间,来解决在原始空间中线性不可分的问题。

此外,用对偶形式表示学习器的优势在于该表示中可调参数的个数不依赖输入属性的个数,通过使用恰当的核函数来替代内积,可以隐式得将非线性的训练数据映射到高维空间,而不增加可调参数的个数。

在线性不可分的情况下,支持向量机首先在低维空间中完成计算,然后通过核函数将输入空间映射到高维牲空间,最终在高维特征空间中构造出最优分离超平面,从而把平面上本身不好分的非线性数据分开。

而在我们遇到核函数之前,如果用原始的方法,那么在用线性学习器学习一个非线性关系,需要选择一个非线性特征集,并且将数据写成新的表达形式,这等价于应用一个固定的非线性映射,将数据映射到特征空间,在特征空间中使用线性学习器,因此考虑的假设集是这种类型的函数:
\begin{equation}
	f(x) = \sum_{i = 1}^Nw_i\phi_i(x) + b
\end{equation}
这里$\phi:X\to F$是从输入空间到某个特征空间的映射,这意味着建立非线性学习器分为两步:
\begin{enumerate}
	\item 首先使用一个非线性映射将数据变换到一个特征空间F
	\item 然后在特征空间使用线性学习器分类 
\end{enumerate}
而由于对偶形式就是线性学习器的一个重要性质,这意味着假设可以表达为训练点的线性组合,因此决策规则可以用测试点和训练点的内积来表示:
\begin{equation}
	f(x) = \sum_{i = 1}^la_iy_i<\phi(x_i),\phi(x)> + b	
\end{equation}
如果有一种方式可以在特征空间中直接计算内积$<\phi(x_i),\phi(x)>$,就像在原始输入点的函数中一样,就有可能将两个步骤融合到一起建立一个非线性的学习器,这样直接计算法的方法称为\textbf{核函数方法}
\begin{definition}{Kernel}{}
	核是一个函数K,对所有$x,z\in X$,满足$K(x,z) = <\phi(x),\phi(z)>$,这里$\phi$是从X到内积特征空间F的映射。
\end{definition}
下面举一个核函数把低维空间映射到高维空间的例子:

我们考虑核函数$K(v_1,v_2) = <v_i,v_2>^2$,即“内积平方”,这里$v_1 = (x_1,y_1),v_2 = (x_2,y_2)$是二维空间中的两个点。
这个核函数对应着一个二维空间到三维空间的映射,它的表达式是:
$$P(x,y) = (x^2,\sqrt{2}xy, y^2)$$
可以验证,
\begin{equation*}
	\begin{aligned}
		<P(v_1), p(v_2)> &= <(x_1^2,\sqrt{2}x_1y_1, y_1^2),(x_2^2,\sqrt{2}x_2y_2, y_2^2)> \\
		&= x_1^2x_2^2 + 2x_1x_2y_1y_2 + y_1^2y_2^2 \\
		&= (x_1x_2 + y_1y_2)^2 \\
		&= <v_1,v_2>^2 \\
		&= K(v_1,v_2)
	\end{aligned}
\end{equation*}
上面的例子所说,核函数的作用就是隐含着一个从低维空间到高维空间的映射,而这个映射可以把低维空间中线性不可分的两类点变成线性可分的。

\textbf{核函数的本质}
\begin{enumerate}
	\item 实际中,我们会经常遇到线性不可分的样例,此时,我们的常用做法是把特征映射到高维空间中去
	\item 但进一步,如果凡是遇到线性不可分的样例,一律映射到高维空间,那么这个维度大小会高到可怕的。那咋办呢?
	\item 此时,核函数就隆重登场了,核函数的价值在于它虽然也是讲特征进行从低维到高维的转换,但核函数绝就绝在它事先在低维上进行计算,而将实质上的分类效果表现在了高维上,也就避免了直接在高维空间中的复杂计算。
\end{enumerate}

几个核函数:
\begin{itemize}
	\item 多项式核$K(x_1,x_2) = (<x_1, x_2> + R)^d $
	\item 高斯核$K(x_1,x_2) = exp(-\lVert x_1 - x_2 \rVert ^2 / 2\sigma ^2)$
	\item 线性核$K(x_1,x_2) = <x_1, x_2> $
\end{itemize}

已知映射函数$\phi$,可以通过$\phi(x)$和$\phi(z)$的内积求得核函数$K(x,z)$。不用构造映射$\phi(x)$能否直接判断一个给定的函数$K(x,z)$是不是核函数?或者说,函数$K(x,z)$满足什么条件才能成为核函数?

已知,假设$K(x,z)$是定义在$\chi \times \chi$上的对称函数,并且对任意的$x_1,\dots,x_m\in \chi,K(x,z)$关于$x_1,\dots,x_m$的Gram矩阵是半正定的。可以依据函数$K(x,z)$构成一个希尔伯特空间(Hilbert space),其步骤是:首先定义映射$\phi$并构成向量空间$S$;然后在S上定义内积构成内积空间;最后将S完备化构成希尔伯特空间。
\begin{theorem}{正定核的充要条件}{}
	设$K:\chi \times \chi\to \boldsymbol{R}$是对称函数,则$K(x,z)$为正定核的充要条件是对任意$x_i\in \chi,i=1,2,\dots,m,K(x,z)$ 对应的Gram矩阵:
	\begin{equation}
		K=[K(x_i,x_j)]_{m\times m}
	\end{equation}
	是半正定矩阵。
\end{theorem}
\begin{proof}
	\begin{enumerate}
	\item 必要性。由于$K(x,z)$是$\chi\times\chi$上的正定核,所以存在从$\chi$到希尔伯特空间$\mathcal{H}$的映射$\phi$,使得
	\begin{equation}
		K(x,z)=\phi(x)\cdot\phi(z)
	\end{equation}
	于是,对任意$x_1,\dots,x_m$,构造$K(x,z)$关于$x_1,\dots,x_m$的Gram矩阵
	\begin{equation}
		[K_{ij}]_{m\times m}=[K(x_i,x_j)]_{m\times m}
	\end{equation}
	对任意$c_1,\dots,c_m\in \boldsymbol{R}$,有
	\begin{equation}
		\begin{aligned}
		\sum_{i,j=1}^{m}c_ic_jK(x_i,x_j)&=\sum_{i,j=1}^{m}c_ic_j(\phi(x_i)\cdot\phi(x_j))\\
		&=\left(\sum_{i,j=1}^{m}c_i\phi(x_i) \right)\cdot \left(\sum_{i,j=1}^{m}c_j\phi(x_j) \right)\\
		&=\Bigg\| \sum_ic_i\phi(x_i)\Bigg\|^2\geqslant 0
		\end{aligned}
	\end{equation}
	表明$K(x,z)$关于$x_1,\dots,x_m$的Gram矩阵是半正定的。
	\item 充分性
	根据已知,可以构造从$\chi$到某个希尔伯特空间$\mathcal{H}$的映射,从而表明$K(x,z)$是$\chi\times \chi$上的核函数。
	\end{enumerate}
\end{proof}

\section{线性支持向量机与软间隔最大化}
线性可分问题的支持向量机学习方法,对线性不可分训练数据是不适用的。因为这时上述方法中的不等式约束并不能都成立。怎么才能将它扩展到线性不可分问题呢?这就需要修改硬间隔最大化,使其成为软间隔最大化。

假设给定一个特征空间上的训练数据集
\begin{equation}
	T=\{(x_1,y_1),\dots,(x_N,y_N) \}
\end{equation}
其中$x_i\in \mathcal{X}=R^n,y_i\in \mathcal{Y}=\{+1,-1\},i=1,\dots,N$,$x_i$为第$i$个特征向量,$y_i$为$x_i$的类标记。再假设训练数据集不是线性可分的。通常情况是,训练数据中有一些特异点(outlier),将这些特异点除去后,剩下大部分的样本点组成的集合是线性可分的。

为了解决这个问题,引入松弛变量$\xi_i\geqslant 0$,使函数间隔加上松弛变量大于等于1。这样,约束条件为
\begin{equation}
	y_i(w\cdot x_i+b)\geqslant 1-\xi_i
\end{equation}
同时,对每个松弛变量$\xi_i$,支付一个代价$\xi_i$。目标函数由原来的$\frac{1}{2}\lVert w\rVert^2$变成
\begin{equation}
\label{731}
	\frac{1}{2}\lVert w\rVert^2+C\sum_{i=1}^{N}\xi_i
\end{equation}
这里,$C>0$称为惩罚参数,一般由应用问题决定,$C$值大时对误分类的惩罚增大,$C$值小时对误分类的惩罚减小。最小化目标函数$\ref{731}$包含两层含义:使$\frac{1}{2}\lVert w\rVert^2$尽量小即间隔尽量大,同时使误分类点的个数尽量小,$C$是调和二者的系数。

有了上面的思路,可以和训练数据集线性可分时一样来考虑训练数据集线性不可分时的线性支持向量机学习问题。相应于硬间隔最大化,它被称为软间隔最大化。

线性不可分的线性支持向量机的学习问题变成如下凸二次规划(convex quadratic programming)问题(原始问题):
\begin{flalign}
	\mathop{\mathrm{min}}\limits_{w,b,\xi}\quad & \frac{1}{2}\lVert w\rVert^2+C\sum_{i=1}^{N}\xi_i\\
	\mathrm{s.t.}\quad &y_i(w\cdot x_i+b)\geqslant 1-\xi,\quad i=1,2,\dots,N\\
	&\xi_i\geqslant 0,\quad i=1,2,\dots,N
\end{flalign}
可以证明$w$的解是唯一的,但$b$的解可能不唯一,而是存在于一个区间。

\subsection*{学习的对偶算法}
原始问题的对偶问题是
\begin{flalign}
\label{737}
	\mathop{\mathrm{min}}\limits_{a}\quad & \frac{1}{2}\sum_{i=1}^{N}\sum_{j=1}^{N}a_ia_jy_iy_j(x_i\cdot x_j)-\sum_{i=1}^{N}a_i\\
	\mathrm{s.t.}\quad &\sum_{i=1}^{N}a_iy_i=0\\
	\label{739}
	&0\leqslant a_i\leqslant C,\quad i=1,2,\dots,N
\end{flalign}
原始最优化问题的拉格朗日函数是
\begin{equation}
\label{740}
	L(w,b,\xi,a,\mu)\equiv \frac{1}{2}\lVert w\rVert^2+C\sum_{i=1}^{N}\xi_i-\sum_{i=1}^{N}a_i(y_i(w\cdot  x_i+b)-1+\xi_i)-\sum_{i=1}^{N}\mu_i\xi_i
\end{equation}
其中,$a_i\geqslant 0,\mu_i\geqslant 0$。

对偶问题是拉格朗日函数的极大极小问题。首先求$L(w,b,\xi,a,\mu)$对$w,b,\xi$的极小,由
\begin{flalign}
	\triangledown_wL(w,b,\xi,a,\mu)&=w-\sum_{i=1}^{N}a_iy_ix_i=0\\
	\triangledown_bL(w,b,\xi,a,\mu)&=-\sum_{i=1}^{N}a_iy_i=0\\
	\triangledown_{\mu}L(w,b,\xi,a,\mu)&=C-a_i-\mu_i=0\\
\end{flalign}
得
\begin{flalign}
	w&=\sum_{i=1}^{N}a_iy_ix_i\\
	\sum_{i=1}^{N}a_iy_i&=0\\
	C-a_i-\mu_i&=0
\end{flalign}
代入$\ref{740}$中,得
\begin{equation}
	\mathop{\mathrm{min}}\limits_{w,b,\xi}L(w,b,\xi,a,\mu)=- \frac{1}{2}\sum_{i=1}^{N}\sum_{j=1}^{N}a_ia_jy_iy_j(x_i\cdot x_j)-\sum_{i=1}^{N}a_i
\end{equation}
再对$\mathop{\mathrm{min}}\limits_{w,b,\xi}L(w,b,\xi,a,\mu)$求$a$的极大,即得对偶问题:
\begin{flalign}
	\mathop{\mathrm{max}}\limits_{a}\quad & -\frac{1}{2}\sum_{i=1}^{N}\sum_{j=1}^{N}a_ia_jy_iy_j(x_i\cdot x_j)-\sum_{i=1}^{N}a_i\\
	\mathrm{s.t.}\quad &\sum_{i=1}^{N}a_iy_i=0\\
	\label{746}
	&C-a_i-\mu_i=0\\
	&a_i\geqslant 0\\
	\label{748}
	&\mu_i \geqslant 0,\quad i=1,2,\dots,N
\end{flalign}
将对偶最优化问题进行变换:利用等式约束$\ref{746}$消去$\mu_i$,从而只留下变量$a_i$,并将约束$\ref{746}\sim \ref{748}$写成
\begin{equation}
	0\leqslant a_i\leqslant C
\end{equation}
再将目标函数求极大转换为求极小,于是得到对偶问题$\ref{737}\sim \ref{739}$。

可以通过求解对偶问题而得到原始问题的解,进而确定分离超平面和决策函数。为此,就可以定理的形式叙述原始问题的最优解和对偶问题的最优解的关系。
\subsection*{支持向量}
在线性不可分的情况下,将对偶问题的解$a^*=(a_1^*,a_2^*,\dots,a_N^*)^T$中对应于$a_i^*>0$的样本点$(x_i,y_i)$称为支持向量。软间隔的支持向量$x_i$或者在间隔边界上,或者在间隔边界与分离超平面之间,或者在分离超平面误分类一侧。
\subsection*{合页损失函数}
对于线性支持向量机学习来说,其模型为分离超平面$w^*\cdot x+b^*=0$及决策函数$f(x)=\mathrm{sign}(w^*\cdot x+b^*)$,其学习策略为软间隔最大化,学习算法为凸二次规划。

线性支持向量机学习还有另外一种解释,就是最小化以下目标函数:
\begin{equation}
	\sum_{i=1}^{N}[1-y_i(w\cdot x+b)]_+ +\lambda\lVert w\rVert ^2
\end{equation}
目标函数的第1项是经验损失或经验风险,函数
\begin{equation}
	L(y(w\cdot x+b))=[1-y_i(w\cdot x+b)]_+
\end{equation}
称为合页损失函数(hinge loss function)。下标“+”表示以下取正值的函数。
\begin{equation}
	[z]_+=\begin{cases}
		z,\quad z>0\\
		0,\quad z\leqslant 0
	\end{cases}
\end{equation}
这就是说,当样本点$(x_i,y_i)$被正确分类且函数间隔(确信度)$y_i(w\cdot x_i+b)$大于1时,损失是0,否则损失是$1-y_i(w\cdot x_i+b)$。目标函数的第2项是系数为$\lambda$的$w$的$L_2$范数,是正则化项。
\begin{theorem}{}{}
	线性支持向量机原始最优化问题
	\begin{flalign}
	\label{760}
		\mathop{\mathrm{min}}\limits_{w,b,\xi}\quad & \frac{1}{2}\lVert w\rVert^2+C\sum_{i=1}^{N}\xi_i\\
		\label{761}
		\mathrm{s.t.}\quad &y_i(w\cdot x_i+b)\geqslant 1-\xi,\quad i=1,2,\dots,N\\
		\label{762}
		&\xi_i\geqslant 0,\quad i=1,2,\dots,N
	\end{flalign}
	等价于最优化问题
	\begin{equation}
	\label{763}
		\mathop{\mathrm{min}}\limits_{w,b}\quad 	\sum_{i=1}^{N}[1-y_i(w\cdot x+b)]_+ +\lambda\lVert w\rVert ^2
	\end{equation}
\end{theorem}
\begin{proof}{}{}
	可将最优化问题$\ref{763}$写成问题$\ref{760}\sim \ref{762}$。令
	\begin{equation}
	\label{764}
		[1-y_i(w\cdot x+b)]_+=\xi_i
	\end{equation}
	则$\xi_i\geqslant 0$式$\ref{762}$成立。由式$\ref{764}$,当$1-y_i(w\cdot x_i+b)>0$时,有$1-y_i(w\cdot x_i+b)=1-\xi_i$;当$1-y_i(w\cdot x_i+b)\leqslant 0$时,有$1-y_i(w\cdot x_i+b)\geqslant 1-\xi_i$。故式$\ref{761}$成立。于是$w,b,\xi_i$满足约束条件$\ref{761}\sim \ref{762}$。所以最优化问题$\ref{763}$可写成
	\begin{equation}
		\mathop{\mathrm{min}}\limits_{w,b}\quad 	\sum_{i=1}^{N}\xi_i +\lambda\lVert w\rVert ^2
	\end{equation}
	若取$\lambda=\frac{1}{2C}$,则
	\begin{equation}
		\mathop{\mathrm{min}}\limits_{w,b}\quad 	\frac{1}{C}\left(\frac{1}{2}\lVert w\rVert ^2+C\sum_{i=1}^{N}\xi_i \right)
	\end{equation}
\end{proof}
\section{支持向量回归}

\section{核方法}

\chapter{概率图模型}
\begin{center}
	\begin{tikzpicture}[yscale=1.2]
		\node(1) at (-4,0) {$\text{概率图模型}$};
		\node(2) at (1,2) {$\text{贝叶斯网络}$};
		\node(3) at (1,0) {$\text{高斯网络}$};
		\node(4) at (1,-2) {$\text{马尔可夫网络}$};
		\node(5) at (5,1) {$\text{高斯贝叶斯网络}$};
		\node(6) at (5,-1) {$\text{高斯马尔可夫网络}$};
		
		\draw[->] (1) node[above,xshift=2.5cm,yshift=1.3cm] {$\text{离散/有向}$}-- (2);
		
		\draw[->] (1) node[above,xshift=2.5cm,yshift=0cm] {$\text{连续}$}-- (3);
		
		\draw[->] (1) node[above,xshift=2.5cm,yshift=-1.8cm] {$\text{离散/无向}$}-- (4);
		
		\draw[->] (3) node[above,xshift=1.5cm,yshift=0.5cm] {$\text{有向}$}-- (5);
		
		\draw[->] (3) node[above,xshift=1.5cm,yshift=-1.2cm] {$\text{无向}$}-- (6);
	\end{tikzpicture}
\end{center}

\section{贝叶斯网络}

\section{条件独立}
多变量概率分布的一个重要概念是条件独立(conditional independence)。如果一组变量的联合概率分布的表达式是根据条件概率分布的乘积表示的(即有向图的数学表达形式),那么原则上我们可以通过重复使用概率的加和规则和乘积规则测试是否具有潜在的条件独立性。在实际应用中,这种方法非常耗时。图模型的一个重要的优雅的特征是,联合概率分布的条件独立性可以直接从图中读出来,不用进行任何计算。完成这一件事的一般框架被称为“d-划分”(d-separation),其中“d”表示有向(directed)。
\subsection*{图的三个例子}
如果我们考虑以c为条件下的a,b的联合分布,我们可以用一种稍微不同的方式表示,即
\begin{equation}
\begin{aligned}
p(a,b|c)&=p(a|b,c)p(b|c)\\
&=p(a|c)p(b|c)
\end{aligned}
\end{equation}
因此,以c为条件,a和b的联合概率分布分解为了a的边缘概率分布和b的边缘概率分布的乘积(全部以c为条件)。我们有时会使用条件独立的一种简洁记号,即
\begin{equation}
	a\indep b\ |\ c
\end{equation}
表示给定c条件下a与b条件独立,等价于
\begin{equation}
	p(a|b,c)=p(a|c)
\end{equation}
我们开始讨论有向图的条件独立性质。考虑三个简单的例子。
\begin{enumerate}
	\item 三个例子中的第一个。如图所示。
	\begin{center}
		\begin{tikzpicture}[node distance=2cm]
			\tikzstyle{every node} = [color=red]
			\node[state](A) {$c$}; 
			\node[state](B) [below left of=A] {$a$};
			\node[state](C) [below right of=A] {$b$}; 
			
			\path (A) edge (B) edge (C);
		\end{tikzpicture}
	\end{center}
	使用公式给出的一般结果,对应于这个图的联合概率分布很容易写出来,即
	\begin{equation}
	\label{du}
		p(a,b,c)=p(a|c)p(b|c)p(c)
	\end{equation}
	如果没有变量是观测变量,那么我们可以通过对公式$\ref{du}$两边进行积分或求和的方式,考察a和b是否相互独立,即
	\begin{equation}
		p(a,b)=\sum_cp(a|c)p(b|c)p(c)
	\end{equation}
	一般地,这不能分解为乘积$p(a)p(b)$,因此
	\begin{equation}
		a\nindep b\ |\ \emptyset
	\end{equation}
	其中,$\emptyset$表示空集,符号$\nindep$表示条件独立性质不总是成立。
	现在假设我们以变量c为条件,如图
	\begin{center}
		\begin{tikzpicture}[node distance=2cm,->,semithick]
			\tikzstyle{every node} = [color=red]
			\node[state,fill=gray,fill opacity=0.5](A) {$c$}; 
			\node[state](B) [below left of=A] {$a$};
			\node[state](C) [below right of=A] {$b$}; 
			
			\path (A) edge (B) edge (C);
		\end{tikzpicture}
	\end{center}
	根据公式,给定c的条件下,a和b的条件概率分布,形式为
	\begin{equation}
		\begin{aligned}
			p(a,b|c)&=\frac{p(a,b,c)}{p(c)}\\
			&=p(a|c)p(b|c)
		\end{aligned}
	\end{equation}
	因此,我们可以得到条件独立性质
	\begin{equation}
		a\indep b\ |\ c
	\end{equation}
	通过考虑从结点a经过结点c到结点b。结点c被称为关于这个路径“尾到尾”(tail-to-tail),因为结点与两个箭头的尾部相连。这样的一个连接结点a和结点b的路径的存在使得结点相互依赖。然而,当我们以结点c为条件时,被用作条件的结点“阻隔”了从a到b的路径,使得a和b变得(条件)独立了。
	\item 三个例子中的第二个。如图所示。
	\begin{center}
		\begin{tikzpicture}[node distance=2cm,->,semithick]
			\tikzstyle{every node} = [color=red]
			\node[state](A) {$a$}; 
			\node[state](B) [right of=A] {$c$};
			\node[state](C) [right of=B] {$b$}; 
			
			\path (A) edge (B) 
				  (B) edge (C);
		\end{tikzpicture}
	\end{center}
	对应于这幅图的联合概率分布可以通过一般形式的公式得到,形式为\
	\begin{equation}
		p(a,b,c)=p(a)p(c|a)p(b|c)
	\end{equation}
	首先,假设所有的变量都不是观测变量。与之前一样,我们可以考察a和b是否是相互独立的,方法是对c积分或求和,结果为
	\begin{equation}
		p(a,b)=p(a)\sum_cp(c|a)p(b|c)=p(a)p(b|a)
	\end{equation}
	这通常不能够分解为$p(a)p(b)$,因此
	\begin{equation}
		a\nindep b\ | \emptyset
	\end{equation}
	现在假设我们以c为条件,如图所示
	\begin{center}
		\begin{tikzpicture}[node distance=2cm,->,semithick]
		\tikzstyle{every node} = [color=red]
			\node[state](A) {$a$}; 
			\node[state,fill=gray,fill opacity=0.5](B) [right of=A] {$c$};
			\node[state](C) [right of=B] {$b$}; 
			
			\path (A) edge (B) 
				  (B) edge (C);
		\end{tikzpicture}
	\end{center}
	使用贝叶斯定理,我们有
	\begin{equation}
		\begin{aligned}
			p(a,b|c)&=\frac{p(a,b,c)}{p(c)}\\
			&=\frac{p(a)p(c|a)p(b|c)}{p(c)}\\
			&=p(a|c)p(b|c)
		\end{aligned}
	\end{equation}
	从而我们又一次得到了条件独立性质
	\begin{equation}
		a\indep b\ |\ c
	\end{equation}
	结点c被称为关于从结点a到结点b的路径“头到尾”(head-to-tail)。这样的一个路径连接了结点a和结点b,并且使它们互相之间存在依赖关系。如果我们现在观测结点c,那么这样观测“阻隔”了从a到b的路径,因此我们得到了条件独立性质$a\indep b\ |\ c$。
	\item 三个例子中的第三个。如图所示。
	\begin{center}
		\begin{tikzpicture}[node distance=2cm,->,semithick]
			\tikzstyle{every node} = [color=red]
			\node[state](A) {$c$}; 
			\node[state](B) [above left of=A] {$a$};
			\node[state](C) [above right of=A] {$b$}; 
			
			\path (B) edge (A) 
				  (C) edge (A);
		\end{tikzpicture}
	\end{center}
	联合概率分布可以使用我们的一般结果得到 
	\begin{equation}
		p(a,b,c)=p(a)p(b)p(c|a,b)
	\end{equation}
	首先考虑没有变量是观测变量时的情形。对公式两侧关于c积分或求和,我们有
	\begin{equation}
		p(a,b)=p(a)p(b)
	\end{equation}
	因此当没有变量被观测时,a和b是独立的,这与前两个例子相反。我们可以把这个结果写成
	\begin{equation}
		a\indep b\ |\ \emptyset
	\end{equation}
	现在假设我们以c为条件,如图所示
	\begin{center}
		\begin{tikzpicture}[node distance=2cm,->,semithick]
			\tikzstyle{every node} = [color=red]
			\node[state,fill=gray,fill opacity=0.5](A) {$c$}; 
			\node[state](B) [above left of=A] {$a$};
			\node[state](C) [above right of=A] {$b$}; 
			
			\path (B) edge (A) 
				  (C) edge (A);
		\end{tikzpicture}
	\end{center}
	a和b的条件概率分布为
	\begin{equation}
		\begin{aligned}
			p(a,b|c)&=\frac{p(a,b,c)}{p(c)}\\
			&=\frac{p(a|c)p(b|c)p(c|a,b)}{p(c)}
		\end{aligned}
	\end{equation}
	因此
	\begin{equation}
		a\nindep b\ |\ c
	\end{equation}
	图形上,我们说结点c关于从a到b的路径是“头到头”(head-to-head),因为它连接了两个箭头的头。当结点c没有被观测到的时候,它“阻隔”了路径,从而变量a和b是独立的。然而,以c为条件时,路径被“解除阻隔”,使得a和b相互依赖了。
\end{enumerate}
\subsection*{d-划分}
现在给出有向图d-划分性质的一个一般的叙述。考虑一个一般的有向图,其中A,B,C是任意无交集的结点集合(它们的并集可能比图中结点的完整集合要小)。我们希望弄清楚,一个有向无环图是否暗示了一个特定的条件依赖表述$A\indep B\ |\ C$。为了解决这个问题,我们考虑从A中任意结点到B中任意结点的所有可能的路径。我们说这样的路径被“阻隔”,如果它包含一个结点满足下面两个性质中的任何一个
\begin{enumerate}
	\item 路径上的箭头以头到尾或者尾到尾的方式交汇于这个结点,且这个结点在集合C中。
	\item 箭头以头到头的方式交汇于这个结点,且这个结点和它的所有后继都不在集合C中。
\end{enumerate}
如果所有的路径都被“阻隔”,那么我们说C把A从B中d-划分开,且图中所有变量上的联合概率分布将会满足$A\indep B\ |\ C$。下图说明了d-划分的概念。
\begin{center}
	\begin{tikzpicture}[node distance=2cm,->,semithick]
		\tikzstyle{every node} = [color=red]
		
		\node[state] (E1) {$e$};
		\node[state] (A1) [above left of=E1]{$a$};
		\node[state] (F1) [above right of=E1]{$f$};
		\node[state,fill=gray,fill opacity=0.5] (C1) [below of =E1]{$c$};
		\node[state] (B1) [below right of=F1]{$b$};
		
		\node (N1) [right of=B1]{};
		
		\path (A1) edge (E1)
			  (F1) edge (E1)
			  (E1) edge (C1)
			  (F1) edge (B1);
		
		\node[state] (E2) [right of=N1]{$e$};
		\node[state] (A2) [above left of=E2]{$a$};
		\node[state,fill=gray,fill opacity=0.5] (F2) [above right of=E2]{$f$};
		\node[state] (C2) [below of =E2]{$c$};
		\node[state] (B2) [below right of=F2]{$b$};
		
		\path (A2) edge (E2)
			(F2) edge (E2)
			(E2) edge (C2)
			(F2) edge (B2);
	\end{tikzpicture}
\end{center}
在图(a)中,从a到b的路径没有被结点f阻隔,因为对于这个路径来说,它是一个尾到尾结点,并且没有被观测到。这个路径也没有被结点e阻隔,因为虽然后者是一个头到头的结点,但是它有一个后继c在集合中。因此条件独立关系$a\indep b\ |\ c$在这个图中不成立。在图(b)中,从a到b的路径被结点f阻隔,因为它是一个尾到尾的结点,并且被观测到。因此使用这幅图进行分解的任何概率分布都满足条件独立性质$a\indep b\ |\ f$。注意,这个路径也被结点e阻隔,因为e是一个头到头的结点,并且它和它的后继都没在条件集合中。

对于d-划分的目的来说,用小实心圆表示的参数与观测结点的行为相同。然而,这些结点没有边缘概率分布。结果,参数结点本身没有父结点,因此所有通过这些结点的路径总是尾到尾的,因此是阻隔的。从而它们在d-划分中没有作用。

我们已经看到一个特定的有向图表示将联合概率分布分解为条件概率分布乘积形式的一个具体的分解方式。图也表示一组条件独立的性质,这些性质通常d-划分的方式得到,并且d-划分定理实际上是一个等价于这两个性质的表示。为了让这一点更明显,将有向图想象成滤波器。假设我们考虑$\boldsymbol{x}$上的一个特定的联合概率分布$p(\boldsymbol{x})$,其中$\boldsymbol{x}$对应于图中的(未观测)结点。一个概率分布能够通过滤波器当且仅当它能够用与图对应的公式给出的分解方式进行分解。如果我们将变量$\boldsymbol{x}$的集合上的所有可能的概率分布$p(\boldsymbol{x})$输入到滤波器中,那么通过滤波器的概率分布的子集被记作$\mathcal{DF}$,表示有向分解(directed factorization)。

考虑一个联合概率分布$p(\boldsymbol{x}_1,\dots,\boldsymbol{x}_D)$,它由一个具有D个结点的有向图表示。考虑变量$\boldsymbol{x}_i$对应的结点上的条件概率分布,其中条件为所有剩余的变量$\boldsymbol{x}_{j\ne i}$。使用分解性质,我们可以将条件概率分布表示为下面的形式
\begin{equation}
	\begin{aligned}
		p(\boldsymbol{x}_i|\boldsymbol{x}_{\{j\ne i \}})&=\frac{p(\boldsymbol{x}_1,\dots,\boldsymbol{x}_D)}{\int p(\boldsymbol{x}_1,\dots,\boldsymbol{x}_D)d\boldsymbol{x}_i}\\
		&=\frac{\prod_{k}p(\boldsymbol{x}_k|\mathrm{pa}_k)}{\int \prod_{k}p(\boldsymbol{x}_k|\mathrm{pa}_k)d\boldsymbol{x}_i}
	\end{aligned}
\end{equation}
观察到任何与$\boldsymbol{x}_i$没有函数依赖关系的因子都可以提到$\boldsymbol{x}_i$的积分外面,从而在分子和分母之间消去。唯一剩余的因子是结点$\boldsymbol{x}_i$本身的条件概率分布$p(\boldsymbol{x}_i|\mathrm{pa}_i)$,以及满足下面性质的结点$\boldsymbol{x}_k$的条件概率分布:结点$\boldsymbol{x}_i$在$p(\boldsymbol{x}_k|\mathrm{pa}_k)$的条件集合中,即$\boldsymbol{x}_i$是$\boldsymbol{x}_k$的父结点。由父结点、子结点、同父结点组成的结点集合被称为马尔可夫毯(Markov blanket)或者马尔可夫边界(Markov boundary)。如图所示
\begin{center}
	\begin{tikzpicture}[node distance=2cm,->,semithick]
		\tikzstyle{every node} = [color=red]
		
		\node[state] (A) {$x_i$};
		\node[state,fill=gray,fill opacity=0.5,label=left:$\text{父结点}$] (B) [above left of=A] {};
		\node[state,fill=gray,fill opacity=0.5] (C) [above right of=A] {};
		\node[state,fill=gray,fill opacity=0.5,label=left:$\text{子结点}$] (D) [below left of=A] {};
		\node[state,fill=gray,fill opacity=0.5] (E) [below right of=A] {};
		\node[state,fill=gray,fill opacity=0.5,label=left:$\text{同父结点}$] (F) [above left of=D] {};
		\node[state,fill=gray,fill opacity=0.5] (G) [above right of=E] {};
		
		\path (B) edge (A)
			  (C) edge (A)
			  (A) edge (D) edge (E) 
			  (F) edge (D)
			  (G) edge (E);
	\end{tikzpicture}
\end{center}
我们可以将结点$\boldsymbol{x}_i$的马尔可夫毯想象成将$\boldsymbol{x}_i$与图的剩余部分隔离开的最小结点集合。
\section{马尔科夫随机场}
我们现在考虑图模型的第二大类,使用无向图描述的图模型,与之前一样,它表示一个分解方式,也表示一组条件独立关系。一个马尔可夫随机场(Markov random field),也被称为马尔可夫网络(Markov network)或者无向图模型(undirected graphical model),包含一组结点,每个结点都对应着一个变量或一组变量。链接是无向的,即不含有箭头。
\subsection*{条件独立性}
在有向图的情形下,我们看到可以通过使用被称为d-划分的图检测方法判断一个特定的条件独立性质是否成立。这涉及到判断链接两个结点集合的路径是否被“阻隔”。然而,由于头到头结点的存在,阻隔的定义多少有些微妙。对应于无向图模型,通过移除图中链接的方向性,父结点和子结点的非对称性也被移除了,因此头到头的微妙性也就不再存在了。

假设在一个无向图中,我们有三个结点集合,记作A,B,C。我们考虑条件独立性质 
\begin{equation}
	A\indep B\ |\ C
\end{equation}
为了判定由图定义的概率分布是否满足这个性质,我们考虑连接集合A的结点和集合B的结点的所有可能路径。如果所有这些路径都通过了集合C中的一个或多个结点,那么所有这样的路径都被“阻隔”,因此条件独立质成立。然而,如果存在至少一条未被阻隔的路径,那么条件独立的性质未必成立,或者更精确地说,存在至少某些对应于图的概率分布不满足条件独立性质。如下图中的例子
\begin{center}
	\begin{tikzpicture}[node distance=2cm,semithick]
		\tikzstyle{every node} = [color=red]
		
		\node[state,label=below:$A$] (A) {};
		\node[state,fill=gray!50] (B) [above right of=A] {};
		\node[state,fill=gray!50,label=below:$C$] (C) [below right of=A] {};
		\node[state,label=below:$B$] (D) [below right of=B] {};
		
		\path (A) edge (B) edge (C)
			  (B) edge (C)
			  (C) edge (D)
			  (B) edge (D);
			  
		\draw[color=green] (1.425,0) ellipse [x radius=0.8cm,y radius=2.2cm];
	\end{tikzpicture}
\end{center}
与d-划分的准则完全相同,唯一的差别在于没有头到头的现象。因此,无向图的条件独立性的检测比有向图简单。

另一种条件独立性的检测的方法是假设从图中把集合C中的结点及与这些结点相连的链接全部删除。然后,我们考虑是否存在一条从A中任意结点到B中任意结点的路径。如果没有这样的路径,那么条件独立的性质一定成立。

无向图的马尔可夫毯的形式相当简单,因为结点只条件依赖于相信结点,而条件独立于任何其他的结点,如图所示
\begin{center}
	\begin{tikzpicture}[node distance=2cm,semithick]
	\tikzstyle{every node} = [color=red]
	
	\node[state] (A) {};
	\node[state,fill=gray,fill opacity=0.5] (B) [above left of=A] {};
	\node[state,fill=gray,fill opacity=0.5] (C) [above right of=A] {};
	\node[state,fill=gray,fill opacity=0.5] (D) [below left of=A] {};
	\node[state,fill=gray,fill opacity=0.5] (E) [below right of=A] {};

	
	\path (B) edge (A)
		  (C) edge (A)
		  (A) edge (D) edge (E) ;
	\end{tikzpicture}
\end{center}

\subsection*{分解性质}
我们现在寻找无向图的一个分解规则,对应于上述条件独立性检测。与之前一样,这涉及到将联合概率分布$p(\boldsymbol{x})$表示为在图的局部范围内的变量集合上定义的函数的乘积。于是,我们需要给出这种情形下,局部性的一个合适定义。

如果我们考虑两个结点$x_i$和$x_j$,它们不存在链接,那么给定图中的所有其他结点,这两个结点一定是条件独立的。这是因为两个结点之间没有直接的路径,并且所有其他的路径都通过了观测的结点,因此这些路径都是被阻隔的。这个条件独立性可以表示为
\begin{equation}
	p(x_i,x_j|\boldsymbol{x}_{\backslash \{i,j\}})=p(x_i|\boldsymbol{x}_{\backslash\{i,j\}})p(x_j|\boldsymbol{x}_{\backslash\{i,j\}})
\end{equation}
其中$\boldsymbol{x}_{\backslash\{i,j\}}$表示所有变量$\boldsymbol{x}$去掉$x_i$和$x_j$的集合。于是,联合概率分布的分解一定要让$x_i$和$x_j$不出现在同一个因子中,从而让属于这个图的所有可能的概率分布都满足条件独立性质。

这将我们引向一个图形的概念,团块(clique)。它被定义为图中结点的一个子集,使得在这个子集中的每对结点之间都存在链接。换句话说,团块中的结点集合是全连接的。

于是,我们可以将联合概率分布分解的因子定义为团块中变量的函数。事实上,我们可以考虑最大团块的函数而不失一般性,因为其他团块一定是最大团块的子集。因此,如果$\{x_1,x_2,x_3 \}$是一个最大团块,并我们在这个团块上定义了任意的一个函数,那么定义在这些变量的一个子集上的其他因子都是冗余的。

让我们将团块记作C,将团块中的变量的集合记作$\boldsymbol{x}_C$。这样,联合概率分布可以写成图的最大团块的势函数(potential function)$\psi_C(\boldsymbol{x}_C)$的乘积的形式
\begin{equation}
\label{gongshi}
	p(\boldsymbol{x})=\frac{1}{Z}\prod_{C}\psi_C(\boldsymbol{x}_C)
\end{equation}
这里,Z有时被称为划分函数(partition function),是一个归一化常数,等于
\begin{equation}
	Z=\sum_{\boldsymbol{x}} \prod_{C}\psi_C(\boldsymbol{x}_C)
\end{equation}
它确保了公式给出的概率分布被正确地归一化。注意,我们不把势函数的选择限制为具有具体的概率含义(例如边缘概率分布或者条件概率分布)的函数。势函数$\psi_C(\boldsymbol{x}_C)$的这一通用性产生的一个结果是它们的乘积通常没有被正确地归一化。于是,我们必须引入一个显式的归一化因子。回忆一下,对于有向图的情形,由于分解后的每个作为因子的条件概率分布都被归一化了,因此联合概率分布会自动地被归一化。

归一化常数的存在是无向图的一个主要缺点。但是,对于局部条件概率分布的计算,划分函数是不需要的,因为条件概率是两个边缘概率的比值,当计算这个比值时,划分函数在分子和分母之间被消去了。类似地,对于计算局部边缘概率,我们可以计算未归一化的联合概率分布,然后在计算的最后阶段显式地归一化边缘概率。

目前为止,我们基于简单的图划分,讨论了条件独立性的概念,并且我们提出了对联合概率分布的分解,来尝试对应条件独立的图结构。然而,我们并没有将条件独立性和无向图的分解形式化地联系起来。

为了给出精确的关系,我们再次回到作为滤波器的图模型的概念中。考虑定义在固定变量集合上的所有可能的概率分布,其中这些变量对应于一个具体的无向图的节点。我们可以将$\mathcal{UI}$定义为满足下面条件的概率分布的集合:从使用图划分的方法得到的图中可以读出条件独立性质,这个概率分布应该与这些条件独立性质相容。类似地,我们可以将$\mathcal{UF}$定义为满足下面条件的概率分布的集合:可以表示为关于图中最大团块的分解的形式的概率分布,其中分解方式由公式$\ref{gongshi}$给出。Hammersley-Clifford定理表明,集合$\mathcal{UI}$和$\mathcal{UF}$是完全相同的。

由于我们的势函数被限制为严格大于零,因此将势函数表示为指数的形式更方便,即
\begin{equation}
	\psi_C(\boldsymbol{x}_C)=\exp\{-E(\boldsymbol{x}_C)\}
\end{equation}
其中$E(\boldsymbol{x}_C)$被称为能量函数(energy function),指数表示被称为玻尔兹曼分布(Boltzmann distribution)。联合概率分布被定义为势函数的乘积,因此总的能量可以通过将每个最大团块的能量相加的方法得到。

与有向图的联合分布的因子不同,无向图中的势函数没有一个具体的概率意义。虽然这使得选择势函数具有更大的灵活性,因为没有归一化的限制,但是这确实产生了一个问题,即对于一个具体的应用来说,如何选择势函数。可以这样做:势函数看成一种度量,它表示了局部变量的哪种配置优于其他的配置。具有相对高概率的全局配置对于各个团块的势函数的影响进行了很好的平衡。
\subsection*{与有向图的关系}
通常为了将有向图转化为无向图,我们首先在图中每个结点的所有父结点之间添加额外的无向链接,然后去掉原始链接的箭头,得到道德图。之后,我们将道德图的所有的团块势函数初始化为1.接下来,我们拿出原始有向图中所有的条件概率分布因子,将它乘到一个团块势函数中去。由于“伦理”步骤的存在,总会存在至少一个最大的团块,包含因子中的所有变量。注意,在所有情形下,划分函数都分Z=1。

如图所示(a)一个简单的有向图的例。(b)对应的道德图
\begin{center}
	\begin{tikzpicture}[node distance=2cm,semithick]
	\tikzstyle{every node} = [color=red]
	
	\node[state] (A1){$x_2$};
	\node[state] (B1) [above left of=A1] {$x_1$};
	\node[state] (C1) [above right of=A1]{$x_3$};
	\node[state] (D1) [below of=A1]{$x_4$};
	
	\node (N1) [right of=A1]{};
	\node (N2) [right of=N1]{};
	
	\node[state] (A2) [right of=N2] {$x_2$};
	\node[state] (B2) [above left of=A2] {$x_1$};
	\node[state] (C2) [above right of=A2]{$x_3$};
	\node[state] (D2) [below of=A2]{$x_4$};	
	
	\path (A1) edge[->] (D1)
		  (B1) edge[->] (D1)
		  (C1) edge[->] (D1);
		  
	\path (B2) edge (A2) edge (C2) edge (D2)
		  (C2) edge (A2) edge (D2)
		  (A2) edge (D2);
	\end{tikzpicture}
\end{center}
\section{图模型中的推断}
我们现在考虑图模型中的推断问题,图中的一些结点被限制为观测值,我们想要计算其他结点中的一个或多个子集的后验概率分布。正如我们看到的那样,我们可以利用图结构找到高效的推断算法,也可以让这些算法的结构变得透明。具体地说,我们会看到许多算法可以用图中局部信息传播的方式表示。本节中,我们会把注意力主要集中于精确推断的方法。后面的章节中,我们会考虑许多近似推断的算法。

首先,考虑贝叶斯定理的图表示。假设我们将两个变量$x$和$y$上的联合概率分布$p(x,y)$分解为因子的乘积的形式$p(x,y)=p(x)p(y|x)$。这可以用图表示。
\begin{center}
	\begin{tikzpicture}[node distance=2cm]
		\node[state,red] (A) {$x$};
		\node[state,red] (B) [below of=A]{$y$};
		\path (A) edge[->,semithick] (B);
	\end{tikzpicture}
\end{center}
现在假设我们观测到了$y$的值,如图所示
\begin{center}
	\begin{tikzpicture}[node distance=2cm]
		\node[state,red] (A) {$x$};
		\node[state,red,fill=gray!50] (B) [below of=A] {$y$};
		
		\path (A) edge[->] (B);
	\end{tikzpicture}
\end{center}
我们可以将边缘概率分布$p(x)$看成潜在变量$x$上的先验概率分布,我们的目标是推断$x$上对应的后验概率分布。使用概率的加和规则和乘积规则,我们可以计算 
\begin{equation}
	p(y) =\sum_{x^{'}}p(y|x^{'})p(x^{'})
\end{equation}
这个式子然后被用于贝叶斯定理中,计算 
\begin{equation}
	p(x|y)=\frac{p(y|x)p(x)}{p(y)}
\end{equation}
因此现在联合概率分布可以通过$p(y)$和$p(x|y)$。从图的角度看,联合概率分布$p(x,y)$现在可以表示为下图,其中箭头的方向翻转了。
\begin{center}
	\begin{tikzpicture}[node distance=2cm]
	\node[state,red] (A) {$x$};
	\node[state,red,fill=gray!50] (B) [below of=A] {$y$};
	
	\path (B) edge[->] (A);
	\end{tikzpicture}
\end{center}
这是图模型中推断问题的最简单的例子。
\subsection*{链推断}
现在考虑一个更加复杂的问题,涉及到图示中的结点链。这个例子是本节中对更一般的图的精确推断的讨论的基础。
\begin{center}
	\begin{tikzpicture}[node distance=2cm]
		\tikzstyle{every node} = [color=red]
		
		\node[state] (A) {};
		\node[state] (B) [right of=A]{};
		\node (C) [right of=B]{$\dots$};
		\node[state] (D) [right of=C]{};
		\node[state] (E) [right of=D]{};
		
		\path (A) edge (B)
			  (B) edge (C)
			  (C) edge (D)
			  (D) edge (E);
	\end{tikzpicture}
\end{center}
这个图的联合概率分布形式为
\begin{equation}
\label{ruf}
	p(\boldsymbol{x})=\frac{1}{Z}\psi_{1,2}(x_1,x_2)\psi_{2,3}(x_2,x_3)\dots\psi_{N-1,N}(x_{N-1},x_N)
\end{equation}
我们考虑一个具体的情形,即N个结点表示N个离散变量,每个变量都有K个状态。这种情况下的势函数$\psi_{n-1,n}(x_{n-1},x_n)$由一个$K\times K$的表组成,因此联合概率分布有$(N-1)K^2$个参数。

让我们考虑寻找边缘概率分布$p(x_n)$这一推断问题,其中$x_n$是链上的一个具体的结点。注意,现阶段,没有观测结点。根据定义,这个边缘概率分布可以通过对联合概率分布在除$x_n$以外的所有变量上进行求和的方式得到。
\begin{equation}
\label{daif}
	p(x_n)=\sum_{x_1}\dots\sum_{x_{n-1}}\sum_{x_{n+1}}\dots\sum_{x_N}p(\boldsymbol{x})
\end{equation}
在一个朴素的实现中,我们首先计算联合概率分布,然后显式地进行求和。联合概率分布可以表示为一组数,对应于$\boldsymbol{x}$的每个可能的值。因为有N个变量,每个变量有K个可能的状态,因此$\boldsymbol{x}$有$K^N$个可能的值,从而联合概率的计算和存储以及得到$p(x_n)$所需的求和过程,涉及到的存储量和计算量都会随着链的长度N而指数增长。

然而,通过利用图模型的条件独立性质,我们可以得到一个更加高效的算法。如果我们将联合概率分布的分解表达式$\ref{ruf}$代入到公式$\ref{daif}$中,那么我们可以重新整理加和与乘积的顺序,使得需要求解的边缘分布可以更加高效地计算。例如,考虑对$x_N$的求和。势函数$\psi_{N-1,N}(x_{N-1},x_N)$是唯一与$x_N$有关系的势函数,因此我们可以进行下面的求和
\begin{equation}
	\sum_{x_N}\psi_{N-1,N}(x_{N-1},x_N)
\end{equation}
得到一个关于$x_{N-1}$的函数。$x_1$上的求和式只涉及到势函数$\psi_{1,2}(x_1,x_2)$,因此可以单独进行,得到$x_2$的函数,以此类推。因为每个求和式都移除了概率分布中的一个变量,因此这可以被看成从图中移除一个结点。

如果我们使用这种方式对势函数和求和式进行分组,那么我们可以将需要求解的边缘概率密度写成下面的形式
\begin{equation}
\label{www}
	\begin{aligned}
	p(x_n)=&\frac{1}{Z}\\
	&\underbrace{\left[\sum_{x_{n-1}}\psi_{n-1,n}(x_{n-1},x_n)\dots\left[\sum_{x_2}\psi_{2,3}(x_2,x_3)\left[\sum_{x_1}\psi_{1,2}(x_1,x_2) \right] \right]\dots \right]}_{\mu_\alpha(x_n)}\\
	&\underbrace{\left[\sum_{x_{n+1}}\psi_{n,n+1}(x_{n},x_{n+1})\dots \left[\sum_{x_N}\psi_{N-1,N}(x_{N-1},x_N) \right]\dots \right]}_{\mu_\beta(x_n)}
	\end{aligned}
\end{equation}
这个重排列的方式,背后的思想组成了后续对于一般的加和-乘积算法的讨论的基础,这里,我们利用的关键概率是乘法对加法的分配率,即
\begin{equation}
	ab+ac=a(b+c)
\end{equation}
使用这种重排序的表达式之后,计算边缘概率分布所需的计算总代价是$O(NK^2)$。这是链长度的一个线性函数,与朴素方法的指数代价不同。

现在使用图中局部信息传递的思想,给出这种计算的一个强大的直观意义。根据公式$\ref{www}$,我们看到边缘概率分布$p(x_n)$的表达式分解成了两个因子的乘积乘以归一化常数
\begin{equation}
	p(x_n)=\frac{1}{Z}\mu_\alpha(x_n)\mu_\beta(x_n)
\end{equation}
我们把$\mu_\alpha(x_n)$看成从结点$x_{n-1}$到结点$x_n$的沿着链向前传递的信息。类似地,$\mu_\beta(x_n)$可以看成从结点$x_{n+1}$到结点$x_n$的沿着链向后传递的信息。

信息$\mu_\alpha(x_n)$可以递归地计算,因为 
\begin{equation}
\label{rang}
	\begin{aligned}
	\mu_\alpha(x_n)&=\sum_{x_{n-1}}\psi_{n-1,n}(x_{n-1},x_n)\left[\sum_{x_{n-2}}\dots \right]\\
	&=\sum_{x_{n-1}}\psi_{n-1,n}(x_{n-1},x_n)\mu_\alpha(x_{n-1})
	\end{aligned}
\end{equation}
因此我们首先计算 
\begin{equation}
	\mu_\alpha(x_2)=\sum_{x_1}\psi_{1,2}(x_1,x_2)
\end{equation}
然后重复应用公式$\ref{rang}$直到我们到达需要求解的结点。

类似地,信息$\mu_\beta(x_n)$可以递归的计算。计算方法为:从结点$x_N$开始,使用
\begin{equation}
	\begin{aligned}
	\mu_\beta(x_n)&=\sum_{x_{n+1}}\psi_{n,n+1}(x_n,x_{n+1})\left[\sum_{x_{n+2}}\dots \right]\\
	&=\sum_{x_{n+1}}\psi_{n,n+1}(x_n,x_{n+1})\mu_\beta(x_{n+1})
	\end{aligned}
\end{equation}
这种递归的信息传递如图所示。
\begin{center}
	\begin{tikzpicture}[node distance=2cm]
		\tikzstyle{every node} = [color=red]
		\node[state] (A) {$x_1$};
		\node (B) [right of=A]{$\dots$};
		\node[state] (C) [right of=B]{$x_{n-1}$};
		\node[state] (D) [right of=C]{$x_n$};
		\node[state] (E) [right of=D]{$x_{n+1}$};
		\node (F) [right of=E]{$\dots$};
		\node[state] (G) [right of=F]{$x_N$};
		
		\path (A) edge (B)
			(B) edge (C)
			(C) edge node[yshift=0.5cm,label=above:$\underrightarrow{\mu_\alpha(x_{n-1})}$]{} (D)
			(D) edge node[yshift=0.5cm,label=above:$\underleftarrow{\mu_\beta(x_n)}$]{}(E)
			(E) edge (F)
			(F) edge (G);
	\end{tikzpicture}
\end{center}
上图被称为马尔可夫链。对应的信息传递方程是马尔可夫过程的Chapman-Kolmogorov方程的一个例子。

现在我们想计算结点链中两个相邻结点的联合概率分布$p(x_{n-1},x_n)$。这类似于计算单一结点的边缘概率分布,区别在于现在有两个变量没有被求和出来。需要求解的边缘概率分布可以写成下面的形式
\begin{equation}
	p(x_{n-1},x_n)=\frac{1}{Z}\mu_\alpha(x_{n-1})\psi_{n-1,n}(x_{n-1},x_n)\mu_\beta(x_n)
\end{equation}
因此一旦我们完成了计算边缘概率分布所需的信息传递,我们就可以直接得到每个势函数中的所有变量上的联合概率分布。
\subsection*{树}
我们已经看到,一个由结点链组成的图的精确推断可以在关于结点数量的线性时间内完成,方法是使用通过链中信息传递表示的算法。更一般地,通过局部信息在更大的一类图中的传递,我们可以高效地进行推断。这类图被称为树(tree)。特别地,我们会对之前在结点链的情形中得到的信息传递公式进行简单的推广,得到加和-乘积算法(sum-product algorithm),它为树结构图的精确推断提供了一个高效的框架。

三个树结构的例子。(a)一个无向树,(b)一个有向树,(c)一个有向多树
\begin{center}
	\begin{tikzpicture}[node distance=1.5cm]
	\tikzstyle{every node} = [color=red]
	
	\node[state] (A1) {};
	\node[state] (B1) [above left of=A1] {};
	\node[state] (C1) [above right of=A1]{};
	\node[state] (D1) [below left of=A1]{};
	\node[state] (E1) [below right of=A1]{};
	
	\path (A1) edge (B1) 
			   edge (C1) 
			   edge (D1) 
			   edge (E1);
	
	\node (N1) [right of=C1]{};
	\node (M1) [right of=N1]{};
	
	\node[state] (A2) [right of=M1]{};
	\node[state] (B2) [below left of=A2]{};
	\node[state] (C2) [below right of=A2]{};
	\node[state] (D2) [below left of=B2]{};
	\node[state,xshift=-0.5cm] (E2) [below right of=B2]{};
	\node[state,xshift=0.5cm] (F2) [below left of=C2]{};
	\node[state] (G2) [below right of=C2]{};
	
	
	\path (A2) edge[->] (B2)
		  	   edge[->] (C2)
		  (B2) edge[->] (D2)
		  	   edge[->] (E2)
		  (C2) edge[->] (F2)
		  	   edge[->] (G2);
	
	\node (N2) [right of=A2]{};
	\node (M2) [right of=N2]{};
	
	\node[state] (A3) [right of=M2]{};
	\node[state] (B3) [below right of=A3]{};
	\node[state] (C3) [above right of=B3]{};
	\node[state] (D3) [below left of=B3]{};
	\node[state] (E3) [below right of=B3]{};
	
	\path (A3) edge[->] (B3)
		  (C3) edge[->] (B3)
		  (B3) edge[->] (D3) edge[->] (E3);
	\end{tikzpicture}
\end{center}
\subsection*{因子图}
在推导加和-乘积算法之前,引入一个新的图结构,被称为因子图(factor graph),那么算法的形式会变得特别简单并且具有一般性。

有向图和无向图都使得若干个变量的一个全局函数能够表示为这些变量的子集上的因子图的乘积。因子图显式地表示出了这个分解,方法是:在表示变量的结点的基础上,引入额外的结点表示因子本身。因子图也使我们能够更加清晰地了解分解的细节。

让我们将一组变量上的联合概率分布写成因子的乘积形式
\begin{equation}
\label{fasheng}
	p(\boldsymbol{x})=\prod_{s}f_s(\boldsymbol{x}_s)
\end{equation}
其中$\boldsymbol{x}_s$表示变量的一个子集。每个因子$f_s$是对应的变量集合$\boldsymbol{x}_s$的函数。

在因子图中,概率分布中的每个变量都有一个结点(与之前一样,用圆圈表示),这与有向图和无向图的情形相同。还存在其他的结点(用小正方形表示),表示联合概率分布中的每个因子$f_s(\boldsymbol{x}_s)$。最后,在每个因子结点和因子所依赖的变量结点之间,存在无向链接。例如,考虑一个表示为因子图形式的概率分布
\begin{equation}
	p(\boldsymbol{x})=f_a(x_1,x_2)f_b(x1,x_2)f_c(x_2,x_3)f_d(x_3) 
\end{equation}
这可表示为下图所示的因子图。
\begin{center}
	\begin{tikzpicture}[node distance=2cm]
	
		\node[state,color=red] (A) {$x_1$};
		\node[state,color=red] (B) [right of=A] {$x_2$};
		\node[state,color=red] (C) [right of=B] {$x_3$};
		
		\node[rectangle,color=red,fill=red,minimum size=0.5cm,label=below:$f_a$] (D) at (-1,-2) {};
		\node[rectangle,color=red,fill=red,minimum size=0.5cm,label=below:$f_b$] (E) at (1,-2) {};
		\node[rectangle,color=red,fill=red,minimum size=0.5cm,label=below:$f_c$] (F) at (3,-2) {};
		\node[rectangle,color=red,fill=red,minimum size=0.5cm,label=below:$f_d$] (G) at (5,-2) {};	
		
		\path (A) edge (D) edge (E)
			  (B) edge (D) edge (E) edge (F)
			  (C) edge (F) edge (G); 	
	\end{tikzpicture}
\end{center}
注意有两个因子$f_a(x_1,x_2)$和$f_b(x_1,x_2)$定义在同一个变量集合上。在一个无向图中,两个这样的因子的乘积被简单地合并到同一个团块势函数中。类似地,$f_c(x_2,x_3)$和$f_d(x_3)$可以结合到$x_2$和$x_3$上的一个单一势函数中。然而,因子图显示地写出这些因子,因此能够表达出关于分解本身的更加细节的信息。

如果我们有一个通过无向图表示的概率分布,那么我们可以将其转化为因子图。为了完成这一点,我们构造变量结点,对应于原始无向图,然后构造额外的因子结点,对应于最大团块$\boldsymbol{x}_s$。因子$f_s(\boldsymbol{x}_s)$被设置为与团块势函数相等。注意,对于同一个无向图,可能存在几个不同的因子图。下图说明了这些概念。
\begin{center}
	\begin{tikzpicture}[node distance=2cm]
		\tikzstyle{every node} = [color=red]
		
		\node[state] (A1) {$x_1$};
		\node[state,yshift=-0.5cm] (B1) [below right of=A1]{$x_2$};
		\node[state,yshift=0.5cm] (C1) [above right of=B1]{$x_3$};
		
		\path (A1) edge (B1) edge (C1)
			  (B1) edge (C1);
			
		\node[state] (A2) [right of=C1]{$x_1$};
		\node[state,yshift=-0.5cm] (B2) [below right of=A2]{$x_2$};
		\node[state,yshift=0.5cm] (C2) [above right of=B2]{$x_3$};
		
		\node[rectangle,draw,fill=red,minimum size=0.5cm,above of=B2,yshift=-0.7cm,label=above:$f$](R1){}; 
			
		\path (R1) edge (A2) edge (B2) edge (C2);	
			
		\node[state] (A3) [right of=C2]{$x_1$};
		\node[state,yshift=-0.5cm] (B3) [below right of=A3]{$x_2$};
		\node[state,yshift=0.5cm] (C3) [above right of=B3]{$x_3$};
		
		\node[rectangle,draw,fill=red,minimum size=0.5cm,above of=B3,yshift=-0.7cm,label=above:$f_a$](R2){}; 
		\node[rectangle,draw,fill=red,minimum size=0.5cm,below of=C3,yshift=0.7cm,label=right:$f_b$](R3){}; 
		
		\path (R2) edge (A3) edge (B3) edge (C3)
			  (R3) edge (B3) edge (C3);
	\end{tikzpicture}
\end{center}
(a)一个无向图,有一个单一的团块势函数$\psi(x_1,x_2,x_3)$。(b)一个因子图,因子$f(x_1,x_2,x_3)=\psi(x_1,x_2,x_3)$。(c)一个不同的因子图,表示相同的概率分布,它的因子满足$f_a(x_1,x_2,x_3)f_b(x_2,x_3)=\psi(x_1,x_2,x_3)$。

类似地,为了将有向图转化为因子图,我们构造变量结点对应于有向图中的结点,然后构造因子结点,对应于条件概率分布,最后添加上合适的链接。与之前一样,同一个有向图可能对应于多个因子图。有向图到因子图的转化如图所示。
\begin{center}
	\begin{tikzpicture}[node distance=2cm]
	\tikzstyle{every node} = [color=red]
	
	\node[state] (A1) {$x_1$};
	\node[state,yshift=-0.5cm] (B1) [below right of=A1]{$x_2$};
	\node[state,yshift=0.5cm] (C1) [above right of=B1]{$x_3$};
	
	\path (A1) edge[->] (B1) 
		  (C1) edge[->] (B1);
	
	\node[state] (A2) [right of=C1]{$x_1$};
	\node[state,yshift=-0.5cm] (B2) [below right of=A2]{$x_2$};
	\node[state,yshift=0.5cm] (C2) [above right of=B2]{$x_3$};
	
	\node[rectangle,draw,fill=red,minimum size=0.5cm,above of=B2,yshift=-0.7cm,label=above:$f$](R1){}; 
	
	\path (R1) edge (A2) edge (B2) edge (C2);	
	
	\node[state] (A3) [right of=C2]{$x_1$};
	\node[state,yshift=-0.5cm] (B3) [below right of=A3]{$x_2$};
	\node[state,yshift=0.5cm] (C3) [above right of=B3]{$x_3$};
	
	\node[rectangle,draw,fill=red,minimum size=0.5cm,above of=B3,yshift=-0.7cm,label=above:$f_c$](R2){}; 
	\node[rectangle,draw,fill=red,minimum size=0.5cm,below of=C3,label=right:$f_b$](R3){}; 
	\node[rectangle,draw,fill=red,minimum size=0.5cm,below of=A3,label=left:$f_a$](R4){}; 
		
	\path (R2) edge (A3) edge (B3) edge (C3)
		  (R3) edge (C3) 
		  (R4) edge (A3);
	\end{tikzpicture}
\end{center}
(a)一个有向图,可以分解为$p(x_1)p(x_2)p(x_3|x_1,x_2)$。(b)一个因子图,表示与有向图相同的概率分布,它的因子满足$f(x_1,x_2,x_3)=p(x_1)p(x_2)p(x_3|x_1,x_2)$。(c)一个不同的因子图,表示同样的概率分布,因子为$f_a(x_1)=p(x_1),f_b(x_2)=p(x_2),f_c(x_1,x_2,x_3)=p(x_3|x_1,x_2)$。

我们已经看到了树结构图对于进行高维推断的重要性。如果我们将一个有向树或者无向树转化为因子图,那么生成的因子图也是树(即,因子图没有环,且任意两个结点之间有且只有一条路径)。在有向多树的情形中,由于“伦理”的步骤的存在,转化为无向图会引入环,而转化后的因子图仍然是树。事实上,有向图中由于链接父结点和子结点产生的局部环可以转换到因子图时被移除,只需定义合适的因子函数即可。
\subsection*{加和-乘积算法}
我们会使用因子图框架推导一类强大的、高效的精确推断算法,这些算法适用于树结构的图。这里,我们把注意力集中于计算结点或者结点子集上的局部边缘概率分布,这会引出加和-乘积算法(sum-product algorithm)。稍后,我们会修改这个方法,使得概率最大的状态被找到,这就引出了最大加和算法(max-sum algorithm)。

我们假设原始的图是一个无向树或者有向树或者多树,从而对应的因子图有一个树结构。首先,我们将原始的图转化为因子图,使得我们可以使用同样的框架处理有向模型和无向模型。我们的目标是利用图的结构完成两件事:
\begin{enumerate}
	\item 得到一个高效的精确推断算法来寻找边缘概率;
	\item 在需要求解多个边缘概率的情形,计算可以高效地共享。
\end{enumerate}
首先,对于特定的变量结点$x$,我们寻找边缘概率$p(x)$。现阶段,我们假设所有的变量都是隐含变量。稍后我们会看到如何修改这个算法,使得观测变量被整合到算法中。根据定义,边缘概率分布通过对所有$x$之外的变量上的联合概率分布进行求和的方式得到,即
\begin{equation}
\label{jiao}
	p(x)=\sum_{\boldsymbol{x}\backslash x}p(\boldsymbol{x})
\end{equation}
其中$\boldsymbol{x}\backslash x$表示变量$\boldsymbol{x}$的集合去掉变量$x$。算法的思想是使用因子图的表达式$\ref{fasheng}$替换$p(\boldsymbol{x})$,然后交换加和与乘积的顺序,得到一个高效的算法。

考虑下图
\begin{center}
	\begin{tikzpicture}[node distance=1.5cm]
		\tikzstyle{every node} = [color=red]
		
		\node[rectangle,minimum size=0.5cm,fill=red,draw,label=below:$f_s$](A) {};
		
		\node[state,above left of=A](C) {};
		\node[state,below left of=A](D) {};
		
		\path (A) edge (C) edge (D)
			  (C) edge[dashed] (D);
		
		\begin{pgfonlayer}{background}
			\node[fill=gray!25,ellipse,draw,fit=(A)(C)(D)](P){};
		\end{pgfonlayer}
		\node[color=black,left of=A,xshift=-0.2cm]{\rotatebox{90}{$F_s(x,X_s)$}};
		
		\node[state,right of=A](E) {$x$};
		\path (A) edge node[color=black,above,yshift=0.3cm] {$\underrightarrow{\mu_{f_{s\to x}}(x)}$} (E);
		
		\node[rectangle,fill=red,draw,minimum size=0.5cm,above right of=E] (F) {};
		\node[rectangle,fill=red,draw,minimum size=0.5cm,below right of=E] (G) {};
		
		\path (E) edge (F) edge (G)
			  (F) edge[dashed] (G);		
	\end{tikzpicture}
\end{center}
我们看到图的树结构使得我们可以将联合概率分布中的因子划分为若干组,每组对应于变量结点$x$的相邻结点组成的因子结点集合。我们看到联合概率分布可以写成乘积的形式
\begin{equation}
\label{huan}
	p(\boldsymbol{x})=\prod_{s\in \mathrm{ne}(x)}F_s(x,X_s)
\end{equation}
其中$\mathrm{ne}(x)$表示与$x$相邻的因子结点的集合,$X_s$表示子树中通过因子结点$f_s$与变量结点$x$相连的所有变量的集合,$F_s(x,X_s)$表示分组中与因子$f_s$相关联的所有因子的乘积。

将公式$\ref{huan}$代入$\ref{jiao}$,交换加和与乘积的顺序,我们有
\begin{equation}
\label{zui}
	\begin{aligned}
		p(x)&=\prod_{s\in \mathrm{ne}(x)}\left[\sum_{\boldsymbol{X}_s}F_s(x,X_s) \right]\\
		&=\prod_{s\in \mathrm{ne}(x)}\mu_{f_{s\to x}}(x)
	\end{aligned}
\end{equation}
这里我们引入了一组新的函数$\mu_{f_{s\to x}}(x)$,定义为
\begin{equation}
\label{juli}
	\mu_{f_{s\to x}}(x)\equiv \sum_{\boldsymbol{X}_s}F_s(x,X_s)
\end{equation}
这可以被看做从因子结点$f_s$到变量结点$x$的信息(message)。我们看到,需要求解的边缘概率分布$p(x)$等于所有到达结点$x$的输入信息的乘积。

为了计算这些信息,我们再次回到上图。我们注意到每个因子$F_s(x,X_s)$有一个因子图(因子子图),因此本身可以被分解。特别地,我们有
\begin{equation}
\label{MMa}
	F_s(x,X_s)=f_s(x,x_1,\dots,x_M)G_1(x_1,X_{s1})\dots,G_M(x_M,X_{sM})
\end{equation}
其中,为了方便,我们将$x$之外的与因子$f_s$相关的变量记作$x_1,\dots,x_M$。因此也可以记作$\boldsymbol{x}_s$。

将公式$\ref{MMa}$代入公式$\ref{juli}$,我们有
\begin{equation}
\label{lin}
	\begin{aligned}
	\mu_{f_{s\to x}}(x)&=\sum_{x_1}\dots\sum_{x_M}f_s(x,x_1,\dots,x_M)\prod_{m\in \mathrm{ne}(f_s)\backslash x}\left[\sum_{X_{sm}}G_m(x_m,X_{sm}) \right]\\
	&=\sum_{x_1}\dots\sum_{x_M}f_s(x,x_1,\dots,x_M)\prod_{m\in \mathrm{ne}(f_s)\backslash x}\mu_{x_{m}\to f_s}(x_m)
	\end{aligned}
\end{equation}
其中$\mathrm{ne}(f_s)$表示因子结点$f_s$的相邻变量结点的集合,$\mathrm{ne}(f_s)\backslash x$表示同样的集合,但是移除了结点$x$。这里,我们定义了下面的从变量结点到因子结点的信息
\begin{equation}
\label{de}
	\mu_{x_{m}\to f_s}(x_m)\equiv \sum_{X_{sm}}G_m(x_m,X_{sm})
\end{equation}
于是,我们引入了两类不同的信息。一类信息是从因子结点到变量结点的信息,记作$\mu_{f\to x}(x)$,另一类信息是从变量结点到因子结点的信息,记作$\mu_{x\to f}(x)$。在任何一种情况下,我们看到沿着一条链接传递的信息总是一个函数,这个函数是与那个链接相连的变量结点相关的变量的函数。

公式$\ref{lin}$给出的结果表明,一个变量结点通过一个链接发送到一个因子结点的信息可以按照如下的方式计算:计算沿着所有进行因子结点的其他链接的输入信息的乘积,乘以与那个结点关联的因子,然后对所有与输入信息关联的变量进行求和。如图所示
\begin{center}
	\begin{tikzpicture}[node distance=2cm]
		\tikzstyle{every node} = [color=red]
		
		\node[rectangle,fill=red,draw,label=left:$f_s$,minimum size=0.5cm] (A){};
		\node[state,above left of=A,xshift=-0.5cm](B) {$x_M$};
		\node[state,below left of=A,xshift=-0.5cm](C) {$x_m$};
		\node[state,right of=A,xshift=0.5cm](D) {$x$};
		\begin{pgfonlayer}{background}
			\node[fill=gray!25,inner sep=0.6cm,ellipse,fit=(C)](E)[label=below:$G_m(x_m\text{,}X_{sm})$]{};
		\end{pgfonlayer}
		
		\path (A) edge (B) edge (C) edge (D)
			  (B) edge[dashed] (C);
	\end{tikzpicture}
\end{center}
值得注意的是,一旦一个因子结点从所有其他的相邻变量结点的输入信息,那么这个因子结点就可以向变量结点发送信息。

最后,我们推导变量结点到因子结点的信息的表达式,再次使用图分解(子图分解)。根据下图
\begin{center}
	\begin{tikzpicture}[node distance=2cm]
	\tikzstyle{every node} = [color=red]
	
	\node[state] (A){$x_m$};
	\node[rectangle,fill=red,draw,label=left:$f_L$,minimum size=0.5cm,above left of=A,xshift=-0.5cm](B) {$x_M$};
	\node[rectangle,fill=red,draw,label=left:$f_t$,minimum size=0.5cm,below left of=A,xshift=-0.5cm](C) {$x_m$};
	\node[rectangle,fill=red,draw,label=left:$f_s$,minimum size=0.5cm,right of=A,xshift=0.5cm](D) {$x$};
	\begin{pgfonlayer}{background}
	\node[fill=gray!25,inner sep=0.6cm,ellipse,fit=(C)](E)[label=below:$F_t(x_m\text{,}X_{sm})$]{};
	\end{pgfonlayer}
	
	\path (A) edge (B) edge (C) edge (D)
	(B) edge[dashed] (C);
	\end{tikzpicture}
\end{center}
我们看到与结点$x_m$关联的项$G_m(x_m,X_{sm})$由项$F_l(x_m,X_{lm})$的乘积组成,每一个这样的项都与连接到结点$x_m$的一个因子结点$f_l$相关联(不包括结点$f_s$),即
\begin{equation}
\label{ni}
	G_m(x_m,X_{sm})=\prod_{l\in \mathrm{ne}(x_m)\backslash f_s}F_l(x_m,X_{lm})
\end{equation}
其中求乘积的对象是结点$x_{m}$的所有相邻结点,排除结点$f_s$。注意,每个因子$F_l(x_m,X_{lm})$表示原始图的一个子树,这个原始图与公式$\ref{huan}$表示的图的形式完全相同。将公式$\ref{ni}$代入$\ref{de}$,我们可以得到 
\begin{equation}
\label{ai}
	\begin{aligned}
	\mu_{x_m\to f_s}(x_m)&=\prod_{l\in \mathrm{ne}(x_m)\backslash f_s}\left[\sum_{X_{lm}}F_l(x_m,X_{lm}) \right]\\
	&=\sum_{l\in \mathrm{ne}(x_m)\backslash f_s} \mu_{f_l\to x_m}(x_m)
	\end{aligned}
\end{equation}
其中我们使用了因子结点到变量结点的信息传递的表达式$\ref{juli}$。

因此,为了计算从一个变量结点到相邻因子结点沿着链接传递的信息,我们只需简单地在其他所有结点上对输入信息取乘积。注意,任何只有两个相邻结点的变量结点无需参与计算,只需将信息不变地传递过去即可。此外,我们注意到,一旦一个变量结点接收到了来自所有其他相邻因子结点的输入信息,那么这个变量结点就可以给因子结点发送信息。

我们的目标是计算变量结点$x$的边缘概率分布,这个边缘概率分布等于沿着所有到达这个结点的链接的输入信息的乘积。这些信息中的每一条信息都可以使用其他的信息递归地计算。为了开始这个递归计算的过程,我们可以将结点$x$看成树的根结点,然后从叶结点开始计算。根据公式$\ref{ai}$的定义,我们看到如果一个叶结点是一个变量结点,那么沿着与它唯一相连的链接发送的信息为
\begin{equation}
\label{70}
	\mu_{x\to f}(x)=1
\end{equation}
如图所示
\begin{center}
	\begin{tikzpicture}[node distance=2cm]
		\tikzstyle{every node} = [color=red]
		\node[state] (a) {$x$};
		\node[rectangle,fill=red,minimum size=0.5cm,right of=A,label=below:$f$](B){};
		\path (a) edge node[above,yshift=0.4cm,text =black]{$\underrightarrow{\mu_{x\to f}(x)=1}$} (B);
	\end{tikzpicture}
\end{center}
类似地,如果叶结点是一个因子结点,那么我们根据公式$\ref{lin}$可以看到,发送的信息的形式为
\begin{equation}
\label{71}
	\mu_{f\to x}(x)=f(x)
\end{equation}
如果所示

\begin{center}
\begin{tikzpicture}[node distance=2cm]
	\tikzstyle{every node} = [color=red]
	
	\node[state] (a) {$x$};
	\node[rectangle,fill=red,minimum size=0.5cm,left of=A,label=below:$f$](B){};
	\path (a) edge node[above,yshift=0.4cm,text =black]{$\underrightarrow{\mu_{f\to x}(x)=f(x)}$} (B);
\end{tikzpicture}
\end{center}
现在,我们停下来,总结一下计算边缘概率分布$p(x)$时得到的加和-乘积算法。首先,我们将变量结点$x$看成因子图的根结点,使用公式$\ref{70}$和公式$\ref{71}$,初始化图的叶结点的信息。之后,递归地应用信息传递步骤$\ref{lin}$和$\ref{ai}$,直到信息被沿着每一个链接传递完毕,并且根结点收到了所有相邻结点的信息。每个结点都可以向根结点发送信息。一旦结点收到了所有其他相邻结点的信息,那么它就可以向根结点发送信息。一旦根结点收到了所有相邻结点的信息,需要求解的边缘概率分布就可以使用公式$\ref{zui}$进行计算。

现在考虑一个简单的例子来说明加和-乘积算法。如图所示
\begin{center}
	\begin{tikzpicture}[node distance=1.5cm]
		\tikzstyle{every node} = [color=red]
		
		\node[state] (A) {$x_2$};
		\node[rectangle,text=white,fill=red,minimum size=0.5cm,left of=A](B){$f_a$};
		\node[state,left of=B](C) {$x_1$};
		\node[rectangle,text=white,fill=red,minimum size=0.5cm,below of=A](D){$f_c$};
		\node[state,below of=D](E) {$x_4$};
		\node[rectangle,text=white,fill=red,minimum size=0.5cm,right of=A](F){$f_b$};
		\node[state,right of=F](G) {$x_3$};
		
		\path (A) edge (B) edge (F) edge (D)
			  (B) edge (C)
			  (D) edge (E)
			  (F) edge (G);
	\end{tikzpicture}
\end{center}
图示给出了一个简单的4节点因子图,它的示归一化联合概率分布为
\begin{equation}
	\tilde{p}(\boldsymbol{x})=f_a(x_1,x_2)f_b(x_2,x_3)f_c(x_2,x_4)
\end{equation}
为了对这个图应用加和-乘积算法,让我们令结点$x_3$为根结点,此时有两个叶结点$x_1$和$x_4$。从叶结点开始,我们有下面六个信息组成的序列 
\begin{flalign}
	\mu_{x_1\to f_a}(x_1) &= 1\\
	\mu_{f_a\to x_2}(x_2) &= \sum_{x_1}f_a(x_1,x_2)\\
	\mu_{x_4\to f_c}(x_4) &= 1\\
	\mu_{f_c\to x_2}(x_2) &= \sum_{x_4}f_c(x_2,x_4)\\
	\mu_{x_2\to f_b}(x_2) &= \mu_{f_a\to x_2}(x_2)\mu_{f_c\to x_2}(x_2)\\
	\mu_{f_b\to x_3}(x_3) &= \sum_{x_2}f_b(x_2,x_3)\mu_{x_2\to f_b}(x_2)
\end{flalign}
信息流的方向如图所示
\begin{center}
	\begin{tikzpicture}[node distance=1.2cm]
	\tikzstyle{every node} = [color=red]
	
	\node[state] (A1) {$x_2$};
	\node[rectangle,text=white,fill=red,minimum size=0.5cm,left of=A1](B1){$f_a$};
	\node[state,left of=B1](C1) {$x_1$};
	\node[rectangle,text=white,fill=red,minimum size=0.5cm,below of=A1](D1){$f_c$};
	\node[state,below of=D1](E1) {$x_4$};
	\node[rectangle,text=white,fill=red,minimum size=0.5cm,right of=A1](F1){$f_b$};
	\node[state,right of=F1](G1) {$x_3$};
	
	\path (A1) edge node[above,blue]{$\leftarrow$} (B1) edge node[above,blue]{$\leftarrow$} (F1) edge node[left,blue]{$\downarrow$}(D1)
	(B1) edge node[above,blue]{$\leftarrow$}(C1)
	(D1) edge node[left,blue]{$\downarrow$}(E1)
	(F1) edge node[above,blue]{$\leftarrow$}(G1);

	%%%%%%%
	\node[state,right of=G1,xshift=3cm] (A) {$x_2$};
	\node[rectangle,text=white,fill=red,minimum size=0.5cm,left of=A](B){$f_a$};
	\node[state,left of=B](C) {$x_1$};
	\node[rectangle,text=white,fill=red,minimum size=0.5cm,below of=A](D){$f_c$};
	\node[state,below of=D](E) {$x_4$};
	\node[rectangle,text=white,fill=red,minimum size=0.5cm,right of=A](F){$f_b$};
	\node[state,right of=F](G) {$x_3$};
	
	\path (A) edge node[above,blue]{$\rightarrow$} (B) edge node[above,blue]{$\rightarrow$} (F) edge node[left,blue]{$\uparrow$}(D)
	(B) edge node[above,blue]{$\rightarrow$}(C)
	(D) edge node[left,blue]{$\uparrow$}(E)
	(F) edge node[above,blue]{$\rightarrow$}(G);
		\end{tikzpicture}
\end{center}
(a)从根结点向叶结点传递。(b)从叶结点向根结点传递。

现在一个信息已经在两个方向上通过了每个链接,因此我们现在可以计算边缘概率分布。作为一个简单的检验,让我们验证边缘概率分布$p(x_2)$由正确的表达式给出。使用公式$\ref{zui}$,使用上面的结果将信息替换掉,我们有
\begin{equation}
	\begin{aligned}
		\tilde{p}(x_2)&=\mu_{f_a\to x_2}(x_2)\mu_{f_b\to x_2}(x_2)\mu_{f_c\to x_2}(x_2)\\
		&=\left[\sum_{x_1}f_a(x_1,x_2) \right]\left[\sum_{x_3}f_b(x_2,x_3) \right]\left[\sum_{x_4}f_c(x_2,x_4) \right]\\
		&=\sum_{x_1}\sum_{x_3}\sum_{x_4}\tilde{p}(\boldsymbol{x})
	\end{aligned}
\end{equation}
这与我们预期的结果相同。

目前为止,我们已经假定图中所有的变量都是隐含变量。在大多数实际应用中,变量的一个子集会被观测到,我们希望计算以这些观测为条件的后验概率分布。观测结点在加和-乘积算法中很容易处理,如下所述。假设我们将$\boldsymbol{x}$划分为隐含变量$\boldsymbol{h}$和观测变量$\boldsymbol{v}$,且$\boldsymbol{v}$的观测值被记作$\hat{\boldsymbol{v}}$。然后,我们简单地将联合概率分布$p(\boldsymbol{x})$乘以$\prod_iI(v_i,\hat{v}_i)$,其中如果$v=\hat{v}$,则$I(v_i,\hat{v}_i)=1$,否则$I(v,\hat{v})=0$。这个乘积对应于$p(\boldsymbol{h},\boldsymbol{v}=\hat{\boldsymbol{v}})$,因此$p(\boldsymbol{h}|\boldsymbol{v}=\hat{\boldsymbol{v}})$的一个未归一化版本。通过运行加和-乘积算法,我们可以高效地计算后验边缘概率$p(h_i|\boldsymbol{v}=\hat{\boldsymbol{v}})$,忽略归一化系数。归一化系数的值可以使用一个局部的计算高效地计算出来。$\boldsymbol{v}$中变量上的任意求和式就退化成了单一的项。
\subsection*{最大加和算法}
加和-乘积算法使得我们能够将联合概率分布$p(\boldsymbol{x})$表示为一个因子图,并且高效地求出成分变量上的边缘概率分布。有两个其他的比较常见的任务,即找到变量的具有最大概率的一个设置,以及找到这个概率的值。这两个任务可以通过一个密切相关的算法完成,这个算法被称为最大加和(max-sum),可以被看成动态规划(dynamic programming)在图模型中的一个应用。

一个简单的寻找具有最大概率的潜在变量值的方法是,运行加和-乘积算法,得到每个变量的边缘概率分布$p(x_i)$,然后,反过来对于每个边缘概率分布,找到使边缘概率最大的$x_i^*$。然而,这会给出一组值,每个值都单独取得最大的概率。在实际应用中,我们通常希望找到联合起来具有最大概率的值的集合,换句话说,找到向量$\boldsymbol{x}^{\text{最大}}$,使得联合概率分布达到最大值,即
\begin{equation}
	\boldsymbol{x}^{\text{最大}}=\mathrm{arg}\mathop{\mathrm{max}}\limits_{\boldsymbol{x}}p(\boldsymbol{x})
\end{equation}
这样,联合概率分布的对应值为
\begin{equation}
p(\boldsymbol{x}^{\text{最大}})=\mathop{\mathrm{max}}\limits_{\boldsymbol{x}}p(\boldsymbol{x})
\end{equation}
通常,$\boldsymbol{x}^{\text{最大}}$与$x_i^*$的集合不同。于是,我们寻找一个高效的算法,来求出最大化联合概率分布$p(\boldsymbol{x})$的$\boldsymbol{x}$的值,这会使得我们得到在最大值处的联合概率分布的值。为了解决第二个问题,我们只需简单地写出分量的最大值算符,即
\begin{equation}
\label{zhengzha}
	\mathop{\mathrm{max}}\limits_{\boldsymbol{x}}p(\boldsymbol{x})=\mathop{\mathrm{max}}\limits_{x_1}\dots\mathop{\mathrm{max}}\limits_{x_M}p(\boldsymbol{x})
\end{equation}
其中M是变量的总数。之后,使用$p(\boldsymbol{x})$的用因子乘积形式表示的展开式替换$p(\boldsymbol{x})$即可。

在推导加和-乘积臭汗时,我们使用了乘法的分配律。这里,我们使用最大化算符的类似定律
\begin{equation}
	\mathrm{max}(ab,ac)=a\mathrm{max}(b,c)
\end{equation}
这对于$a\geqslant 0$的情形成立(这对于图模型的因子总成立)。这使得我们交换乘积与最大化的顺序。

首先考虑公式$\ref{ruf}$描述的结点链这一简单的例子。概率最大值的计算可以写成
\begin{equation}
	\begin{aligned}
	\mathop{\mathrm{max}}\limits_{\boldsymbol{x}}p(\boldsymbol{x})&=\frac{1}{Z}\mathop{\mathrm{max}}\limits_{x_1}\dots\mathop{\mathrm{max}}\limits_{x_N}[\psi_{1,2}(x_1,x_2)\dots\psi_{N-1,N}(x_{N-1},x_N)]\\
	&=\frac{1}{Z}\mathop{\mathrm{max}}\limits_{x_1}\left[\mathop{\mathrm{max}}\limits_{x_2}\left[\psi_{1,2}(x_1,x_2)\left[\dots\mathop{\mathrm{max}}\limits_{x_N}\psi_{N-1,N}(x_{N-1},x_N) \right]\dots \right] \right]
	\end{aligned}
\end{equation}
正如边缘概率的计算一样,我们看到交换最大值算符和乘积算法会产生一个更高效的计算,并且更容易表示为从结点$x_N$沿着结点链传递回结点$x_1$的信息。

我们可以将这个结点推广到任意树结构的因子图上,推广的方法为:将因子图表达式$\ref{fasheng}$代入公式$\ref{zhengzha}$中,然后交换乘积与最大化的计算顺序。这种计算的结构与加和-乘积算法完全相同,因此我们能够简单地将那些结果转化到当前的问题中。特别地,假设我们令图中的一个特定的变量结点为根结点。之后,我们计算起始的一组信息,然后从树的叶结点向内部传递到根结点。对于每个结点,一旦它接收到来自其他相邻结点的输入信息,那么它就向根结点发送信息。最后对所有到达根结点的信息的乘积进行最大化,得出$p(\boldsymbol{x})$的最大值。这可以被称为最大化乘积算法(max-produce algorithm),与加和-乘积算法完全相同,唯一的区别是求和被替换为了求最大值。注意,现阶段,信息被从叶结点发送到根结点,而没有相反的方向。

在实际应用中,许多小概率的乘积可以产生数值下溢的问题,因此更方便的做法是对联合概率分布的对数进行操作。对数函数是一个单调函数,因此求最大值的运算符可以与取对数的运算交换顺序,即
\begin{equation}
	\ln \left(\mathop{\mathrm{max}}\limits_{\boldsymbol{x}}p(\boldsymbol{x}) \right)=\mathop{\mathrm{max}}\limits_{\boldsymbol{x}}\ln p(\boldsymbol{x})
\end{equation}
分配性质仍然成立,因为
\begin{equation}
	\mathrm{max}(a+b,a+c)=a+\mathrm{max}(b,c)
\end{equation}
所以取对数的唯一效果是把最大化乘积算法中的乘积替换成了加和,因此我们得到了最大化加和算法(max-sum algorithm)。根据之前在加和-乘积算法中得到的公式给出的结果,我们可以基于信息传递写出最大化加和算法,只需把“加和”替换为“最大化”,把“乘积”替换为对数求和即可。结果为
\begin{flalign}
	\mu_{f_{s\to x}}(x)&=\mathop{\mathrm{max}}\limits_{x_1,\dots,x_M}\left[\ln f(x,x_1,\dots,x_M)+\sum_{m\in \mathrm{ne}(f)\backslash x} \mu_{x_m\to f}(x_m) \right]\\
	\mu_{x\to f}(x)&=\sum_{l\in \mathrm{ne}(x)\backslash f}\mu_{f_l\to x}(x)
\end{flalign}
最开始的由叶结点发送的信息可以通过类比公式得到,结果为
\begin{flalign}
	\mu_{f\to x}(x)&=\ln f(x)\\
	\mu_{x\to f}(x)&=0
\end{flalign}
根结点处的最大概率为
\begin{equation}
\label{zuida}
	p^{\text{最大}}=\mathop{\mathrm{max}}\limits_{\boldsymbol{x}}\left[\sum_{s\in \mathrm{ne}(x)}\mu_{f_s\to x}(x) \right]
\end{equation}
目前为止,我们已经看到了如何通过从叶结点到任意选择的根结点传递信息的方式找到联合概率分布的最大值。这个结果与根结点的选择无关。现在,我们转向第二个问题,即寻找联合概率达到最大值的变量的配置。目前,我们已经将信息从叶结点发送到了根结点。计算公式$\ref{zuida}$的过程也会得到根结点变量的概率最高的值$x^{\text{最大}}$,定义为
\begin{equation}
	x^{\text{最大}}=\mathop{\mathrm{arg\ max}}\limits_{x}\left[\sum_{s\in \mathrm{ne}(x)}\mu_{f_s\to x}(x) \right]
\end{equation}
使用动态规划的方法应用在图模型上就可以给出变量的精确最大化配置。这种方法的一个重要应用是寻找隐马尔可夫模型中隐含状态的最可能序列,这种情况下被称为Viterbi算法。
\subsection*{一般图的精确推断}
加和-乘积算法和最大化加和算法提供了树结构图中的推断问题的高效精确解法。然而,对于许多实际应用,我们必须处理带有环的图。

信息传递框架可以被推广到任意的图拓扑结构,从而得到一个精确的推断步骤,被称为联合树算法(junction tree algorithm)。联合树对于任意的图都是精确的、高效的。对于一个给定的图,通常不存在计算代价更低的算法。不幸的是,算法必须的计算代价由最大团块中的变量数量确定。在离散变量的情形中,计算代价会随着这个数量指数增长。
\subsection*{循环置信传播}
对于许多实际应用问题来说,使用精确推断是不可行的,因此我们需要研究有效的近似方法。这种近似方法中,一个重要的类别被称为变分法(variational)。作为这些确定性方法的补充,有一大类取样(sampling)方法,也被称为蒙特卡罗(Monte Carlo)方法。这些方法基于从概率分布中的随机数值取样,这两种方法将在后续章节详细讨论。

这里,我们考虑带有环的图中的近似推断的一个简单方法,它直接依赖于之前对树的精确推断的讨论。主要思想就是简单地应用加和-乘积算法,即使不保证能够产生好找结果。这种方法被称为循环置信传播(loopy belief propagation)。
\subsection*{学习图结构}
在我们关于图模型的推断的讨论中,我们假设图的结构已知且固定。然而,也有一些研究超出了推断问题的范围,关注于从数据推断图结构本身。这需要我们定义一个可能结构的空间,以及用于对每个结构评分的度量。
\chapter{采样方法}
对于大多数应用中的概率模型来说,精确推断是不可行的,因此我们不得不借助与某种形式的近似。上一章,我们讨论了基于确定性近似的推断方法,它包括诸如变分贝叶斯方法以及期望传播。这里,我们考虑基于数值采样的近似推断方法,也被称为蒙特卡罗(Monte Carlo)方法。

虽然对于一些应用来说,我们感兴趣的是非观测变量上的后验概率分布本身,但是在大部分情况下,后验概率分布的主要用途是计算期望,例如在做预测的情形下就是这样。因此,本章中,我们希望解决的基本的问题涉及到关于一个概率分布$p(\boldsymbol{z})$寻找某个函数$f(\boldsymbol{z})$的期望。这里,$\boldsymbol{z}$的元素可能是离散变量、连续变量或者二者的结合。因此,在连续变量的情形下,我们希望计算下面的期望
\begin{equation}
\label{zhongyao}
	\mathbb{E}[f]=\int f(\boldsymbol{z})p(\boldsymbol{z}d\boldsymbol{z})
\end{equation}
采样方法背后的一般思想是得到从概率分布$p(\boldsymbol{z})$中独立抽取的一组变量$\boldsymbol{z}^{(l)}$,其中$l=1,\dots,L$。这使得期望可以通过有限和的方式计算,即
\begin{equation}
\label{youxian}
	\hat{f}=\frac{1}{L}\sum_{l=1}^{L}f(\boldsymbol{z}^{(l)})
\end{equation}
只要样本$\boldsymbol{z}^{(l)}$是从概率分布$p(\boldsymbol{z})$中抽取的,那么$\mathbb{E}[\hat{f}]=\mathbb{E}[f]$,因此估计$\hat{f}$具有正确的均值。估计$f$的方差为
\begin{equation}
	\mathrm{var}[\hat{f}]=\frac{1}{L}\mathbb{E}[(f-\mathbb{E}[f])^2]
\end{equation}
它是函数$f(\boldsymbol{z})$在概率分布$p(\boldsymbol{z})$下的方差。因此,值得强调的一点是,估计的精度不依赖于$\boldsymbol{z}$的维度,并且原则上,对于数量相对较少的样本$\boldsymbol{z}^{(l)}$,可能会达到较高的精度。

\textbf{补充一个逻辑关系图}
\section{基本采样算法}

\section{马尔科夫链蒙特卡罗}
\subsection*{马尔可夫链}
\subsection*{Metropolis-Hastings算法}
\section{吉布斯采样}
吉布斯采样是一个简单的并且广泛应用的马尔可夫链蒙特卡罗算法,可以被看做Metropolis-Hastings算法的一个具体的情形。

考虑我们想采样的概率分布$p(z)=p(z_1,\dots,z_M)$,并且假设我们已经选择了马尔可夫链的某个初始的状态。吉布斯采样的每个步骤及到将一个变量的值替换为以剩余变量的值为条件,从这个概率分布中抽取的那个变量的值。因此我们将$z_i$替换为从概率分布$p(z_i|z_{\backslash i})$中抽取的值,其中$z_i$表示$z$的第$i$个元素,$z_{\backslash i}$表示$z_1,\dots,z_M$去掉$z_i$这一项。这个步骤要么按照某种特定的顺序在变量之间进行循环,要么每一步中按照某个概率分布随机地选择一个变量进行更新。

例如,假设我们有一个在三个变量上的概率分布$p(z_1,z_2,z_3)$,在算法的第$\tau$步,我们已经选择了$z_1^{(\tau)},z_2^{(\tau)},z_3^{(\tau)}$的值。首先,我们将$z_1^{(\tau)}$替换为新值$z_1^{(\tau+1)}$,这个新值是从条件概率分布
\begin{equation}
	p(z_1|z_2^{(\tau)},z_3^{(\tau)})
\end{equation}
中采样得到的。接下来,我们将$z_2^{\tau}$替换为新值$z_2^{\tau+1}$,这个新值是从条件概率分布
\begin{equation}
	p(z_2|z_1^{(\tau+1)},z_3^{(\tau)})
\end{equation}
中采样得到的,即$z_1$的新值可以在接下来的采样步骤中直接使用。然后,我们使用样本$z_3^{\tau+1}$更新$z_3$,其中$z_3^{(\tau+1)}$
是从
\begin{equation}
	p(z_3|z_1^{(\tau+1)},z_2^{(\tau+1)})
\end{equation}
中抽取的。以此类推,在这三个变量之间进行循环。
\begin{enumerate}
	\item 初始化$\{z_i:i=1,\dots,M \}$。
	\item 对于$\tau=1,\dots,T:$
	\begin{enumerate}[-]
		\item 采样$z_1^{(\tau+1)}\sim p(z_1|z_2^{(\tau)},z_3^{(\tau)},\dots,z_M^{(\tau)})$。
		\item 采样$z_2^{(\tau+1)}\sim p(z_2|z_1^{(\tau+1)},z_3^{(\tau)},\dots,z_M^{(\tau)})$。
		\item $\vdots$
		\item 采样$z_j^{(\tau+1)}\sim p(z_j|z_1^{(\tau+1)},\dots,z_{j-1}^{(\tau+1)},z_{j+1}^{(\tau)},\dots,z_M^{(\tau)})$。
		\item $\vdots$
		\item 采样$z_M^{(\tau+1)}\sim p(z_M|z_1^{(\tau+1)},z_2^{(\tau+1)},\dots,z_{M-1}^{(\tau+1)})$。
	\end{enumerate}
\end{enumerate}
为了证明这个步骤能够从所需的概率分布中采样,我们首先注意到对于吉布斯采样的每个步骤来说,概率分布$p(z)$是不变的,因此对于整个马尔可夫链来说也是不变的。为了让吉布斯采样能够从正确的概率分布中得到样本,第二个需要满足的要求为各态历经性。各态历经性的一个充分条件是没有条件概率分处处为零。为了完成算法,初始状态的概率分布也应该被指定,虽然在多轮迭代之后,样本与初始状态的分布无关。当然,马尔可夫链中的连续的样本是高度相关的,因此为了得到近似独立的样本,需要对序列进行下采样。

我们可以将吉布斯采样步骤看成Metropolis-Hastings算法的一个特定的情况,如下所述。考虑一个Metropolis-Hastings采样的步骤,它涉及到变量$z_k$,同时保持剩余的变量$z_{\backslash k}$不变,并且对于这种情形来说,从$z$到$z^*$的转移概率为$q_k(z^*|z)=p(z_k^*|z_{\backslash})$。我们注意到$z_{\backslash k}^*=z_{\backslash k}$,因为在采样的步骤中,向量的各个元素都不改变。并且,$p(z)=p(z_k|z_{\backslash k})p(z_{\backslash k})$。因此,确定Metropolis-Hastings算法中的接受概率的因子为
\begin{equation}
\begin{aligned}
	A(z^*,z)=\frac{p(z^*)q_k(z|z^*)}{p(z)q_k(z^*|z)}&=\frac{p(z_k^*|z_{\backslash k}^*)p(z_{\backslash k}^*)p(z_k|z_{\backslash k}^*)}{p(z_k|z_{\backslash k})p(z_{\backslash k})p(z_k^*|z_{\backslash k})}\\
	&=\frac{p(z_k^*|z_{\backslash k}^*)p(z_{\backslash k}^*)p(z_k|z_{\backslash k}^*)}{p(z_k|{\color{red}{z_{\backslash k}^*}})p({\color{red}{z_{\backslash k}^*}})p(z_k^*|{\color{red}{z_{\backslash k}^*}})}\\
	&=1
\end{aligned}
\end{equation}
推导时我们用到了$z_{\backslash k}^*=z_{\backslash k}$。因此Metropolis-Mastings步骤问题被接受的。

由于基本的吉布斯采样方法每次只考虑一个变量,因此它在连续样本之间具有很强的依赖性。在另一个极端情况下,如果我们直接从联合概率分布中采样(我们一直假定这种操作无法完成),那么连续地对一组变量进行采样,而不是对一个变量进行采样。这就是分块吉布斯(blocking Gibbs)采样算法。这种算法中,将变量集合分块(未必互斥),然后在每个块内部联合地采样,采样时以剩余的变量为条件。
\section{切片采样}
我们已经看到,Metropolis算法的一个困难之处是它对于步长的敏感性。如果步长过小,那么由于随机游走行为,算法会很慢。而如果步长过大,那么由于较高的拒绝率,算法会很低效。切片采样(slice sampling)方法提供了一个可以自动调节步长来匹配分布特征的方法。与之前一样,它需要我们能够计算未归一化的概率分布$\tilde{p}(z)$。

首先考虑一元变量的情形。切片采样涉及到使用额外的变量$u$对$z$进行增广,然后从联合的$(z,u)$空间中采样。目标是从下面的概率分布
\begin{equation}
	\hat{p}(z,u)=\begin{cases}
		\frac{1}{Z_p}\qquad \text{如果}0\leqslant u \leqslant \tilde{p}(z)\\
		0\qquad \text{其他情况}
	\end{cases}
\end{equation}
中均匀地进行采样,其中$Z_p=\int \tilde{p}(z)dz$。$z$上的边缘概率分布为
\begin{equation}
	\int \hat{p}(z,u)du=\int _0^{\tilde{p}(z)}\frac{1}{Z_p}du=p(z)
\end{equation}
因此,我们可能通过从$\hat{p}(z,u)$中采样,然后忽略$u$值的方式得到$p(z)$的样本。通过交替地对$z$和$u$进行采样即可完成这一点。给定$z$的值,我们可以计算$\tilde{p}(z)$的值,然后在$0\leqslant u \leqslant \tilde{p}(z)$上均匀地对$u$进行采样,这很容易。然后,我们固定$u$,在由$\{z:\tilde{p}(z)>u \}$定义的分布的“切片”上,对$z$进行均匀地采样。

在实际应用中,直接从穿过概率分布的切片中采样很困难,因此我们定义了一个采样方法,它保持$\hat{p}(z,u)$下的均匀分布具有不变性,这可以通过确保萍踪细节平衡的套件来实现。假设$z$的当前值记作$z^{(\tau)}$,并且我们已经得到了一个对应的样本$u$。$z$的下一个值可以通过考察包含$z^{(\tau)}$的区域$z_{\mathrm{min}}\leqslant z\leqslant z_{\mathrm{max}}$来获得。根据概率分布的特征长度标度来对步长进行调节就发生在这里。我们希望区域包含尽可能多的切片,从而使得$z$空间中能进行较大的移动,同时希望切片外的区域尽可能小,因为切片外的区域会使得采样变得低效。因为切片外的区域会使得采样变得低效。

一种选择区域的方法是,从一个包含$z^{(\tau)}$的具有某个宽度$w$的区域开始,然后测试每个端点,看它们是否位于切片内部。如果有端点没在切片内部,那么区域在增加$w$值的方向上进行扩展,直到端点位于区域外。然后,$z^{'}$的一个样本被从这个区域中均匀抽取。如果它位于切片内,那么它就构成了$z^{(\tau+1)}$。如果它位于切片外,那么区域收缩,使得$z^{'}$组成一个端点,并且区域仍然包含$z^{(\tau)}$。然后,另一个样本点从这个缩小的区域中均匀抽取,以此类推,直到找到位于切片内部的一个$z$值。

切片采样可以应用于多元分布中,方法是按照吉布斯采样的方式重复地对每个变量进行采样。这要求对于每个元素$z_i$,我们能够计算一个正比于$p(z_i|z_{\backslash i})$的函数。
\section{混合蒙特卡罗算法}

\section{估计划分函数}


\chapter{近似推断}
在概率模型的应用中,一个中心任务是在给定观测(可见)数据变量$\boldsymbol{X}$的条件下,计算潜在变量$\boldsymbol{Z}$的后验概率分布$p(\boldsymbol{Z}|\boldsymbol{X})$,以及计算关于这个概率分布的期望。模型可以也包含某些确定性参数,我们现在不考虑它。模型也可能是一个纯粹的贝叶斯模型,其中任何未知的参数都有一人先验概率分布,并且被整合到了潜在变量集合中,记作向量$\boldsymbol{Z}$。例如,在EM算法中,我们需要计算完整数据对数似然函数关于潜在变量后验概率分布的期望。对于实际应用中的许多模型来说,计算后验概率分布或者计算关于这个后验概率分布的期望是不可行的。这可能是由于潜在空间的维度太高,以至于无法直接计算,或者由于后验概率分布的形式特别复杂,从而期望无法解析地计算。在连续变量的情形中,需要求解的积分可能没有解析解,而空间的维度和被积函数的复杂度可能使得数值积分变得不可行。对于离散变量,求边缘概率的过程涉及到对隐含变量的所有可能的配置进行求和。这个过程虽然原则上总是可以计算的,但是我们在实际应用中经常发现,隐含状态的数量可能有指数多个,从而精确的计算所需的代价过高。

在这种情况下,我们需要借助近似方法。根据近似方法依赖于随机近似还是确定性近似,方法大体分为两大类。随机方法,例如后面章节介绍的马尔可夫链蒙特卡罗方法,使得贝叶斯方法能够在许多领域中广泛使用。这些方法通常具有这样的性质:给定无限多的计算资源,它们可以生成精确的结果,近似的来源是使用了有限的处理时间。在实际应用中,取样方法需要的计算量会相当大,经常将这些方法的应用限制在了小规模的问题中。并且,判断的一种取样方法是否生成了服从所需的概论分布的独立样本是很困难的。

本章中,我们介绍了一系列的确定性近似方法,有些方法对于大规模的数据很有适用。这些方法基于对后验概率分布的解析近似,例如通过假设后验概率分布可以通过一种特定的方式分解,或者假设后验概率分布有一个具体的参数形 式,例如高斯分布。对于这种情况,这些方法永远无法生成精确的解,因此这些方法的优点和缺点与取样方法是互补的。

前面的章节中我们讨论了拉普拉斯近似,它基于对概率分布的峰值(即,最大值)的局部高斯近似。这里,我们考虑一类近似方法,被称为变分推断(variational inference)或者变分贝叶斯(variational Bayes),它使用了更加全局的准则,并且被广泛应用于实际问题中。我们最后简要介绍另一种变分的框架,被称为期望传播(expectation propagation)。
\section{变分推断}
变分的方法起源于18世纪的欧拉、拉格朗日,以及其他的关于变分法(calculus of variations)的研究。虽然变分方法本质上没有任何近似的东西,但是它们通常会被用于寻找近似解。寻找近似解的过程可以这样完成:限制需要最优化算法搜索的函数的范围,例如只考虑二次函数,或者考虑由固定的基函数线性组合而成的函数,其中只有线性组合的系数可以发生变化。在概率推断的应用中,限制条件的形式可以是可分解的假设。

现在,让我们详细讨论变分最优化的概念如何应用于推断问题。假设我们有一个纯粹的贝叶斯模型,其中每个参数都有一个先验概率分布。这个模型也可以有潜在变量以及参数,我们会把所有潜在变量和参数组成的集合记作$\boldsymbol{Z}$。类似地,我们会把所有观测变量的集合记作$\boldsymbol{X}$。例如,我们可能有N个独立同分布的数据,其中$\boldsymbol{X}=\{\boldsymbol{x}_1,\dots,\boldsymbol{x}_N \}$且$\boldsymbol{Z}=\{\boldsymbol{z}_1,\dots,\boldsymbol{z}_N \}$。我们的概率模型确定了联合概率分布$p(\boldsymbol{X},\boldsymbol{Z})$,我们的目标是找到后验概率分布$p(\boldsymbol{Z}|\boldsymbol{X})$以及模型证据$p(\boldsymbol{X})$的近似。我们可以将对数边缘概率分解,即
\begin{equation}
\begin{aligned}
	\ln p(\boldsymbol{X})&=\ln p(\boldsymbol{X},\boldsymbol{Z}) - \ln p(\boldsymbol{Z}|\boldsymbol{X})\\
	&=\ln \frac{p(\boldsymbol{X},\boldsymbol{Z})}{q(\boldsymbol{Z})}-\ln \frac{p(\boldsymbol{Z}|\boldsymbol{X})}{q(\boldsymbol{Z})}
\end{aligned}
\end{equation}
等式左右两边乘以$q(\boldsymbol{Z})$并对z求积分有
\begin{equation}
	\begin{aligned}
		\text{左边}&=\int_{\boldsymbol{Z}}\ln p(\boldsymbol{X})q(\boldsymbol{Z})d\boldsymbol{Z}=\ln p(\boldsymbol{X})\\
		\label{fenjie}
		\text{右边}&=\underbrace{\int_{\boldsymbol{Z}}q(\boldsymbol{Z} )\cdot\ln \frac{p(\boldsymbol{X},\boldsymbol{Z})}{q(\boldsymbol{Z})}d\boldsymbol{Z}}_{\mathrm{ELBO}(evidence\ lower\ bound)}+\underbrace{\int_{\boldsymbol{Z}}-q(\boldsymbol{Z} )\cdot\ln \frac{p(\boldsymbol{Z}|\boldsymbol{X})}{q(\boldsymbol{Z})}d\boldsymbol{Z}}_{KL(q\| p)}
	\end{aligned}
\end{equation}
即
\begin{equation}\dfrac{num}{den}
	\ln p(\boldsymbol{X})=\zeta (q)+\mathrm{KL}(q\| p)
\end{equation}
与之前一样,我们可以通过关于概率分布$q(\boldsymbol{Z})$的最优化来使下界$\zeta(q)$达到最大值,这等价于最小化$\mathrm{KL}$散度。如果我们允许任意选择$q(\boldsymbol{Z})$的最优化来使下界$\zeta(q)$达到最大值出现在$\mathrm{KL}$散度等于零的时刻,此时$q(\boldsymbol{Z})$等于后验概率分布$p(\boldsymbol{Z}|\boldsymbol{X})$。即
\begin{equation}
	\begin{aligned}
		-\int_{\boldsymbol{Z}}q(\boldsymbol{Z} )\cdot\ln \frac{p(\boldsymbol{Z}|\boldsymbol{X})}{q(\boldsymbol{Z})}d\boldsymbol{Z}&=0\\
		&\Rightarrow \frac{p(\boldsymbol{Z}|\boldsymbol{X})}{q(\boldsymbol{Z})}=1\\
		&\Rightarrow q(\boldsymbol{Z})=p(\boldsymbol{Z}|\boldsymbol{X})
	\end{aligned}
\end{equation}
然而,我们假定在需要处理的模型中,对真实的概率分布进行操作是不可行的。

于是,我们转而考虑概率分布$q(\boldsymbol{Z})$的一个受限制的类别,然后寻找这个类别中使得$\mathrm{KL}$散度达到最小值的概率分布。我们的目标是充分限制$q(\boldsymbol{Z})$可以取得的概率分布的类别范围,使得这个范围中的所有概率分布都是可以处理的概率分布。同时,我们还要使得这个范围充分大、充分灵活,从而它能够提供对真实后验概率分布的一个足够好的近似。需要强调的是,施加限制条件的唯一目的是为了计算方便,并且在这个限制条件下,我们应该使用尽可能丰富的近似概率分布。特别地,对于高度灵活的概率分布来说,没有“过拟合”现象。使用灵活的近似仅仅使得我们更好好近似真实的后验概率分布。

限制近似概率分布的范围的一种方法是使用参数概率分布$q(\boldsymbol{Z}|\omega)$,它由参数集合$\omega$控制。这样,下界$\zeta(q)$变成了$\omega$的函数,我们可以利用标准的非线性最优化方法确定参数的最优值。

\subsection*{分解概率分布}
这里,我们考虑另一种方法,这种方法里,我们限制概率分布$q(\boldsymbol{Z})$的范围。假设我们将$\boldsymbol{Z}$的元素划分成若干个互不相交的组,记作$\boldsymbol{Z}_i$,其中$i=1,\dots,M$。然后,我们假定$q$分布关于这些分布可以进行分解,即
\begin{equation}
\label{bianfen}
	q(\boldsymbol{Z})=\prod_{i=1}^{M}q_i(\boldsymbol{Z}_i)
\end{equation}
需要强调的是,我们关于概率分布没有做更多的假设。特别地,我们没有限制各个因子$q_i(\boldsymbol{Z}_i)$的函数形式。变分推断的这个分解的形式对应于物理学中的一个近似框架,叫做平均场理论(mean field theory)。

在所有具有公式$\ref{bianfen}$的形式的概率分布$q(\boldsymbol{Z})$中,我们现在寻找下界$\zeta(q)$最大的概率分布。于是,我们希望对$\zeta(q)$关于所有的概率分布$q_i(\boldsymbol{Z}_i)$进行一个自由形式的(变分)最优化。我们通过关于每个因子进行最优化来完成整体的最优化过程。为了完成这一点,我们首先将公式$\ref{bianfen}$代入公式$\ref{fenjie}$,然后分离出依赖于一个因子$q_j(\boldsymbol{Z}_j)$的项。为了记号的简洁,我们简单地将$q_j(\boldsymbol{Z}_j)$记作$q_j$,这样我们有
\begin{equation}
\label{chang}
	\begin{aligned}
		\zeta(q)&=\int_{\boldsymbol{Z}}q(\boldsymbol{Z} )\cdot\ln \frac{p(\boldsymbol{X},\boldsymbol{Z})}{q(\boldsymbol{Z})}d\boldsymbol{Z}\\
		&=\underbrace{\int_{\boldsymbol{Z}}q(\boldsymbol{Z} )\cdot\ln p(\boldsymbol{X},\boldsymbol{Z})}_{\circled{1}}-\underbrace{\int_{\boldsymbol{Z}}q(\boldsymbol{Z} )\cdot\ln q(\boldsymbol{Z})d\boldsymbol{Z}}_{\circled{2}} \\
		\circled{1}&=\int_{\boldsymbol{z}_1\dots\boldsymbol{z}_M}\prod_{i=1}^{M}q_i(\boldsymbol{z}_i)\ln p(\boldsymbol{X},\boldsymbol{Z})d\boldsymbol{z}_1\dots d\boldsymbol{z}_M\\
		&=\int _{\boldsymbol{z}_j}q_j(\boldsymbol{z}_j)\left(\int_{\boldsymbol{Z}\backslash j}\prod_{i\ne j}^{M}q_i(\boldsymbol{z}_i)\ln p(\boldsymbol{X},\boldsymbol{Z})d\boldsymbol{Z}\backslash j \right)d\boldsymbol{z}_j\\
		&=\int _{\boldsymbol{z}_j}q_j(\boldsymbol{z}_j)\cdot \underbrace{\mathbb{E}_{\prod_{i\ne j}^{M}q_i(\boldsymbol{Z}_i)}[\ln p(\boldsymbol{X},\boldsymbol{Z})]}_{\text{近似作$\ln \tilde{p}(\boldsymbol{X},\boldsymbol{z}_j)$}}d\boldsymbol{z}_j\\
		\circled{2}&=\int_{\boldsymbol{z}_1\dots\boldsymbol{z}_M}\prod_{i=1}^{M}q_i(\boldsymbol{z}_i) \sum_{i=1}^{M}\ln q_i(\boldsymbol{z}_i)d\boldsymbol{z}_1\dots d\boldsymbol{z}_M\\
		&=\sum_{i=1}^{M}\int_{\boldsymbol{z}_i}q_i(\boldsymbol{z}_i)\ln q_i(\boldsymbol{z}_i)d\boldsymbol{z}_i\\
		&=\int_{\boldsymbol{z}_j}q_j(\boldsymbol{z}_j)\ln q_j(\boldsymbol{z}_j)d\boldsymbol{z}_j+\text{常数}\dots(\text{保持$\{q_{i\ne j}\}$}\text{固定})\\
		\zeta(q)=\circled{1}-\circled{2}&=\int q_j\ln \tilde{p}(\boldsymbol{X},\boldsymbol{Z}_j)d\boldsymbol{Z}_j-\int q_j\ln q_j d\boldsymbol{Z}_j+\text{常数}
	\end{aligned}
\end{equation}
其中我们定义了一个新的概率分布$\tilde{p}(\boldsymbol{X},\boldsymbol{Z}_j)$,形式为
\begin{equation}
	\ln \tilde{p}(\boldsymbol{X},\boldsymbol{Z}_j)=\mathbb{E}_{i\ne j}[\ln p(\boldsymbol{X},\boldsymbol{Z})]+\text{常数}
\end{equation}
这里,记号$\mathbb{E}_{i\ne j}[\dots]$表示关于定义在所有$\boldsymbol{z}_i(i\ne j)$上的q概率分布的期望,即
\begin{equation}
	\mathbb{E}_{i\ne j}[\ln p(\boldsymbol{X},\boldsymbol{Z})]=\int \ln p(\boldsymbol{X},\boldsymbol{Z})\prod_{i\ne j}q_i d\boldsymbol{Z}_i
\end{equation}

现在假设我们保持$\{q_{i\ne j}\}$固定,关于概率分布$q_j(\boldsymbol{Z}_j)$的所有可能的形式最大化公式$\ref{chang}$中的$\zeta(q)$。这很容易做,因为我们看到公式$\ref{chang}$是$q_j(\boldsymbol{Z}_j)$和$\tilde{p}(\boldsymbol{X},\boldsymbol{Z}_j)$之间的Kullback-Leibler散度的负值。因此,最大化$\ref{chang}$等价于最小化Kullback-Leibler散度,且最小值出现在$q_j(\boldsymbol{Z}_j)=\tilde{p}(\boldsymbol{X},\boldsymbol{Z}_j)$的位置。于是,我们得到了最优解$q_j^*(\boldsymbol{Z}_j)$的一般的表达式,形式为
\begin{equation}
\label{zuiyou}
	\ln q_j^*(\boldsymbol{Z}_j)=\mathbb{E}_{i\ne j}[\ln p(\boldsymbol{X},\boldsymbol{Z})] +\text{常数}
\end{equation}
这个解表明,为了得到因子$q_j$的最优解的对数,我们只需考虑所有隐含变量和可见变量上的联合概率分布的对数 ,然后关于所有其他的因子$\{q_i\}$取期望即可,其中$i\ne j$。

公式中的可加性常数通过对概率分布$q_j^*(\boldsymbol{Z}_j)$进行归一化的方式来设定。
\subsection*{分解近似的性质}
\subsection*{例子:一元高斯分布}
我们现在使用一元变量$x$上的高斯分布来说明分解变分近似。我们的目标是在给定$x$的观测值的数据集$D=\{x_1,\dots,x_N \}$的情况下,推断均值$\mu$和精度$\tau$的后验概率分布。其中,我们假设数据是独立地从高斯分布中抽取的。似然函数为
\begin{equation}
	p(D|\mu,\tau)=\left(\frac{\tau}{2\pi} \right)^{\frac{N}{2}}\exp \left\{-\frac{\tau}{2}\sum_{n=1}^{N}(x_n-\mu)^2 \right\}
\end{equation}
我们现在引入$\mu$和$\tau$的共轭先验分布 ,形式为
\begin{flalign}
	p(\mu|\tau)&=\mathcal{N}(\mu|\mu_0,(\lambda_0 \tau)^{-1})\\
	p(\tau)&=\mathrm{Gam}(\tau|a_0,b_0)
\end{flalign}
这些分布共同给出了一个高斯-Gamma共轭先验分布。

对于这个简单的问题,后验概率可以求出精确解,并且形式还是高斯-Gamma分布。然而,为了讲解的目的,我们会考虑对后验概率分布的一个分解变分近似,形式为
\begin{equation}
	q(\mu,\tau)=q_{\mu}(\mu)q_{\tau}(\tau)
\end{equation}
注意,真实的后验概率分布还可以按照这种形式进行分解。最优的因子$q_{\mu}(\mu)$和$q_{\tau}(\tau)$可以从一般的结果$\ref{zuiyou}$中得到,如下所述。对于$q_{\mu}(\mu)$,我们有
\begin{equation}
	\begin{aligned}
	\ln q_{\mu}^*(\mu) &= \mathbb{E}_{\tau}[\ln p(D|\mu,\tau)+\ln p(\mu|\tau)]+\text{常数}\\
	&=-\frac{\mathbb{E}[\tau]}{2}\left\{\lambda_0(\mu-\mu_0)^2+\sum_{n=1}^{N}(x_n-\mu)^2 \right\}+\text{常数}
	\end{aligned}
\end{equation}
对于$\mu$配平方,我们看到$q_{\mu}(\mu)$是一个高斯分布$\mathcal{N}(\mu|\mu_N,\lambda_N^{-1})$,其中,均值和方差为
\begin{flalign}
	\label{16}
	\mu_N&=\frac{\lambda_0\mu_0 +N\bar{x}}{\lambda_0+N}\\
	\label{17}
	\lambda_N&=(\lambda_0+N)\mathbb{E}[\tau]
\end{flalign}
类似地,因子$q_{\tau}(\tau)$的最优解为
\begin{equation}
	\begin{aligned}
	\ln q_{\tau}^*(\tau)&=\mathbb{E}_{\mu}[\ln p(D|\mu,\tau)+\ln p(\mu|\tau)]+\ln p(\tau)+\text{常数}\\
	&=(a_0-1)\ln \tau -b_0\tau +\frac{N+1}{2}\ln \tau \\
	&\quad -\frac{\tau}{2}\mathbb{E}_{\mu}\left[\sum_{n=1}^{N}(x_n-\mu)^2+\lambda_0(\mu-\mu_0)^2 \right]+\text{常数}
	\end{aligned}
\end{equation}
因此$q_{\tau}(\tau)$是一个Gamma分布$\mathrm{Gam}(\tau|a_N,b_N)$,参数为
\begin{flalign}
	a_N&=a_0+\frac{N+1}{2}\\
	b_N&=b_0+\frac{1}{2}\mathbb{E}_{\mu}\left[\sum_{n=1}^{N}(x_n-\mu)^2+\lambda_0(\mu-\mu_0)^2 \right]
\end{flalign}

应该强调的是,我们不假设最优概率分布$q_{\mu}(\mu)$和$q_{\tau}(\tau)$的具体的函数形式。它们的函数形式从似然函数和对应的共轭先验分布中自然地得到。

因此,我们得到了最优概率分布$q_{\mu}(\mu)$和$q_{\tau}(\tau)$的表达式,每个表达式依赖于关于其他概率分布计算得到的矩。因此,一种寻找解的方法是对$\mathbb{E}[\tau]$进行一个初始的猜测,然后使用这个猜测来重新计算概率分布$q_{\mu}(\mu)$。给定这个修正的概率分布之后,我们接下来可以计算所需的矩$\mathbb{E}[\mu]$和$\mathbb{E}[\mu^2]$,并且使用这些矩来重新计算概率分布$q_{\tau}(\tau)$,以此类推。

通常,我们需要使用一种迭代的方法来得到最优分解后验概率分布的解。然而,对于我们这里讨论的非常简单的例子来说,我们可以通过求解最优因子$q_{\mu}(\mu)$和$q_{\tau}(\tau)$的方程。得到一个显式的解。在做这件事之前,我们可以通过考虑无信息先验来简化表达式。无信息先验分布中,$\mu_0=a_0=b_0=\lambda_0=0$。虽然这些参数设置对应于一个反常先验,但是我们看到后验概率分布仍然具有良好的定义。使用Gamma分布的均值的标准结果$\mathbb{E}[\tau]=\frac{a_N}{b_N}$。我们有
\begin{equation}
	\frac{1}{\mathbb{E}[\tau]}=\frac{b_N}{a_N}=\mathbb{E}\left[\frac{1}{N+1}\sum_{n=1}^{N}(x_n-\mu)^2 \right]=\frac{N}{N+1}(\bar{x^2}-2\bar{x}\mathbb{E}[\mu]+\mathbb{E}[\mu^2])
\end{equation}
之后,使用公式$\ref{16}$和公式$\ref{17}$,我们得到了$q_{\mu}(\mu)$的一阶矩和二阶矩,形式为
\begin{flalign}
	\mathbb{E}[\mu]&=\mathbb{E}\left[\frac{\lambda_0\mu_0 +N\bar{x}}{\lambda_0+N} \right]=\bar{x}\\
	\mathbb{E}[\mu^2]&=\lambda_0+\mathbb{E}[\mu]^2=\bar{x}^2+\frac{1}{N\mathbb{E}[\tau]}
\end{flalign}
现在,我们可以将这些矩代入公式中,然后解出$\mathbb{E}[\tau]$,可得
\begin{equation}
	\frac{1}{\mathbb{E}[\tau]}=(\bar{x^2}-\bar{x}^2)=\frac{1}{N}\sum_{n=1}^{N}(x_n-\bar{x})^2
\end{equation}
\subsection*{模型比较}

\section{变分线性回归}
\subsection*{变分分布}
\subsection*{预测分布}
\subsection*{下界}

\section{指数族分布}
\section{局部变分方法}
\section{变分logistic回归}
\subsection*{变分后验概率分布}
\subsection*{最优化变分参数}
\subsection*{超参数的推断}

\section{期望传播}

\chapter{连续潜在变量}
我们讨论了具有离散潜在变量的概率模型,例如高斯混合模型。我们现在研究某些潜在变量或者全部潜在变量为连续变量的模型。研究这种模型的一个重要的动机是许多数据集具有下面的性质:数据点几乎全部位于比原始数据空间的维度低得多的流形中。

在实际应用中,数据点不会被精确限制在一个光滑的低维流形中,我们可以将数据点关于流形的偏移看做噪声。这就自然地引出了这种模型的生成式观点,其中我们首先根据某种潜在变量的概率分布在流形中选择一个点,然后通过添加噪声的方式生成观测数据点。噪声服从给定潜在变量下的数据变量的某个条件概率分布。

最简单的连续潜在变量模型对潜在变量和观测变量都作出了高斯分布的假设,并且使用了观测变量对于潜在变量状态的线性高斯依赖关系。这就引出了一个著名的技术——主成分分析(PCA)的概率表示形式,也引出了一个相关的模型,被称为因子分析。

本章中,我们首先介绍标准的、非概率的PCA方法,然后我们会说明,当求解线性高斯潜在变量模型的一种特别形式的最大似然解时,PCA如何自然地产生。这种概率形式的表示方法会带来很多好处,例如在参数估计时可以使用EM算法,对混合PCA模型的推广,以及主成分的数量可以从数据中自动确定的贝叶斯公式。最后,我们简短地讨论潜在变量概念的几个推广,使得潜在变量的概念不局限于线性高斯假设。这种推广包括非高斯潜在变量,它引出了独立成分分析(independent conponent analysis)的框架。这种推广还包括潜在变量与观测变量的关系是非线性关系的模型。
\section{主成分分析}
主成分分析,或者称为PCA,是一种被广泛使用的技术,应用的领域包括维度降低、有损数据压缩、特征抽取「数据可视化。它也被称为Karhunen-Loeve变换。

有两种经常使用的PCA的定义,它们会给出同样的算法。PCA可以被定义为数据在低维线性空间上的正交投影,这个线性空间被称为主空间(principal subspace),使得投影数据的方差被最大化。等价地,它也可以被定义为使得平均投影代价最小的线性投影。平均投影代价是指数据点和它们的投影之间的平均平方距离。

考察一组观测数据集$\{x_n\}$,其中$n=1,\dots,N$,因此$x_n$是一个D维欧几里德空间中的变量。我们的目标是将数据投影到维度$M<D$的空间中,同时最大化投影数据的方差。\\
样本集合
\begin{equation}
X=(x_1\ x_2\ \dots x_N)^T_{N\times p}=
\begin{pmatrix}
x_1^T\\x_2^T\\\vdots\\x_N^T
\end{pmatrix}=
\begin{pmatrix}
x_{11}&x_{12}&\dots&x_{1p}\\	x_{21}&x_{22}&\dots&x_{2p}\\
\dots&\dots&\dots&\dots\\
x_{11}&x_{12}&\dots&x_{np}
\end{pmatrix}_{N\times p}
\end{equation}
样本均值
\begin{equation}
\bar{x}=\frac{1}{N}\sum_{n=1}^{N}x_n=\frac{1}{N}(x_1\ x_2\ \dots\ x_n)
\begin{pmatrix}
1\\1\\\vdots \\1
\end{pmatrix}_{N\times 1}=\frac{1}{N}X^TI_N
\end{equation}
样本方差
\begin{equation}
\begin{aligned}
S&=\frac{1}{N}\sum_{n=1}^{N}(x_n-\bar{x})(x_n-\bar{x})^T\\
&=\frac{1}{N}(x_1-\bar{x}\ x_2-\bar{x}\ \dots \ x_N-\bar{x})
\begin{pmatrix}
(x_1-\bar{x})^T\\ (x_2-\bar{x})^T\\ \vdots \\ (x_N-\bar{x})^T
\end{pmatrix}\\
&=\frac{1}{N}X^T(\overbrace{I_N-\frac{1}{N}I_NI_N^T}^{H_N-\text{中心矩阵}})\cdot (I_N-\frac{1}{N}I_NI_N^T)^T\cdot X\\
&=\frac{1}{N}X^THX
\end{aligned}	
\end{equation}
中心矩阵$H$有以下良好的性质 
\begin{flalign}
H&=I_N-\frac{1}{N}I_NI_N^T\\
H^T&=H\\
H^n&=H
\end{flalign}
\subsection*{最大方差形式}
首先,考虑在一维空间$(M=1)$上的投影。我们可心使用D维向量$u_1$定义这个空间的方向。不失一般性,我们假定选择一个单位向量,从而$u_1^Tu_1=1$(注意,我们只对$u_1$的方向感兴趣,而对$u_1$本身的大小不感兴趣)。这样,每个数据点$x_n$被投影到一个标题值$x_1^Tx_n$上。投影数据的均值是$u_1^T\bar{x}$。投影数据的方差为
\begin{equation}
\begin{aligned}
	J&=\frac{1}{N}\sum_{n=1}^{N}\{u_1^Tx_n-u_1^T\bar{x}\}^2\\
	&=u_1^T\underbrace{\sum_{n=1}^{N}\frac{1}{N}(x_n-\bar{x})\cdot(x_n-\bar{x})^T}_Su_1\\
	&=u_1^TSu_1
\end{aligned}	
\end{equation}
S是数据的协方差矩阵,优化问题变成
\begin{equation}
	\begin{aligned}
		\hat{u_1}&=\arg \max u_1^TSu_1\\
		s.t\ &\ u_1^Tu_1=1
	\end{aligned}
\end{equation}
利用拉格朗日乘子法,求偏导并令其为零,我们看到驻点满足
\begin{equation}
	Su_1=\lambda_1u_1
\end{equation}
表明$u_1$一定是S的一个特征向量,这个特征向量被称为第一主成分。

我们可以用一种增量的方式定义额外的主成分,方法为:在所有与那些已经考虑过的方向正交的所有可能的方向中,将新的方向选择为最大化投影方差的方向。

\textbf{总结一下},主成分分析涉及到计算数据集的均值$\bar{x}$和协方差矩阵$S$,然后寻找$S$的对应于M个最大特征值的M个特征向量。
\subsection*{最小误差形式}
引入$D$维基向量的一个完整的单位正交集合$\{u_i\}$,其中$i=1,\dots,D$,且满足 
\begin{equation}
	u_i^Tu_j=\delta_{ij}
\end{equation}
由于基是完整的,因此每个数据点可以精确地表示为基向量的一个纯性组合,即
\begin{equation}
	x_n=\sum_{i=1}^{D}a_{ni}u_i
\end{equation}
其中,系数$a_{ni}$对于不同的数据点来说是不同的。这对应于将坐标系旋转到了一个由$\{u_i\}$定义的新坐标系,原始的D个分量$\{x_{n1},\dots,x_{nD}\}$被替换为一个等价的集合$\{a_{n1},\dots,a_{nD}\}$。我们有$a_{nj}=x_n^Tu_j$,因此不失一般性
\begin{equation}
	x_n=\sum_{i=1}^{D}(x_n^Tu_i)u_i
\end{equation}
然而,我们的目标是使用限定数量$M<D$个变量的一种表示方法来近似数据点,这对应于在低维子空间上的一个投影。不失一般性,M维线性子空间可以用前M个基向量表示,因此我们可以用下式来近似每个数据点$x_n$
\begin{equation}
	\widetilde{x}_n=\sum_{i=1}^{M}z_{ni}u_i+\sum_{i=M+1}^{D}b_iu_i
\end{equation}
其中$\{z_{ni}\}$依赖于特定的数据点,而$\{b_i\}$是常数,对于所有数据点都相同。我们可以任意选择$\{u_i\},\{z_{ni}\},\{b_i\}$,从而最小化由维度降低所引入的失真。作为失真的度量,我们使用原始数据点与它的近似点$\widetilde{x}_n$之间的平方距离,在数据集上取平均。因此我们的目标是最小化
\begin{equation}
	J=\frac{1}{N}\sum_{n=1}^{N}\Vert x_n -\widetilde{x}_n \Vert^2=\frac{1}{N}\sum_{n=1}^{N}\sum_{i=M+1}^{D}(x_n^Tu_i-\bar{x}^Tu_i)^2=\sum_{i=M+1}^{D}u_i^TSu_i
\end{equation}
剩下的任务是关于$\{u_i\}$对J进行最小化。同样利用拉格朗日乘子法能够得到 
\begin{equation}
	Su_i=\lambda_iu_i
\end{equation}
\subsection*{PCA的应用}
首先考虑PCA对于数据压缩的应用。我们可以写出对数据向量$\boldsymbol{x}_n$的PCA近似,形式为
\begin{equation}
\begin{aligned}
	\tilde{\boldsymbol{x}}_n&=\sum_{i=1}^{M}(\boldsymbol{x}_n^T\boldsymbol{u}_i)\boldsymbol{u}_i+\sum_{i=M+1}^{D}(\bar{\boldsymbol{x}}^T\boldsymbol{u}_i)\boldsymbol{u}_i\\
	&=\bar{\boldsymbol{x}}+\sum_{i=1}^{M}(\boldsymbol{x}_n^T\boldsymbol{u}_i-\bar{\boldsymbol{x}}^T\boldsymbol{u}_i)\boldsymbol{u}_i
\end{aligned}
\end{equation}
其中我们使用了关系
\begin{equation}
	\bar{\boldsymbol{x}}=\sum_{i=1}^{D}(\bar{\boldsymbol{x}}^T\boldsymbol{u}_i)\boldsymbol{u}_i
\end{equation}
这个关系来自于$\{\boldsymbol{u}_i\}$的完整性。这种方法表示了对数据集的一个压缩,因为对于每个数据点,我们将D维向量$\boldsymbol{x}_n$替换为M维向量,元素为$(\boldsymbol{x}_n^T\boldsymbol{u}_i-\bar{\boldsymbol{x}}^T\boldsymbol{u}_i)$。M的值越小,压缩的程度越大。

主成分分析的另一个应用是数据预处理。在这种情况下,目标不是维度降低,而是对数据集进行变换,使得数据集的某些属性得到标准化。这对于后续将模式识别算法成功应用于数据集来说很重要。

使用PCA,我们可以对数据进行更显著的归一化,得到零均值和单位方差的数据,从而不同的变量之间的相关性关系被消除。为了完成这一点,我们首先将特征向量方程写成下面的形式
\begin{equation}
	\boldsymbol{SU}=\boldsymbol{UL}
\end{equation}
其中,$\boldsymbol{L}$是一个$D\times D$的对角矩阵,元素为$\lambda_i$,$\boldsymbol{U}$是一个$D\times D$的正交矩阵,列为$\boldsymbol{u}_i$。然后对于每个数据点$\boldsymbol{x}_n$,我们定义一个变换,值为
\begin{equation}
	\boldsymbol{y}_n=\boldsymbol{L}^{-\frac{1}{2}}\boldsymbol{U}^T(\boldsymbol{x}_n-\bar{\boldsymbol{x}})
\end{equation}
其中$\bar{\boldsymbol{x}}$是样本均值。很明显,集合$\{\boldsymbol{y}_n\}$的均值为零,协方差是单位矩阵,因为
\begin{equation}
	\begin{aligned}
		\frac{1}{N}\sum_{n=1}^{N}\boldsymbol{y}_n\boldsymbol{y}_n^T&=\frac{1}{N}\sum_{n=1}^{N}\boldsymbol{L}^{-\frac{1}{2}}\boldsymbol{U}^T(\boldsymbol{x}_n-\bar{\boldsymbol{x}})(\boldsymbol{x}_n-\bar{\boldsymbol{x}})^T\boldsymbol{UL}^{-\frac{1}{2}}\\
		&=\boldsymbol{L}^{-\frac{1}{2}}\boldsymbol{U}^T\boldsymbol{SUL}^{-\frac{1}{2}}\\
		&=\boldsymbol{L}^{-\frac{1}{2}}\boldsymbol{LL}^{-\frac{1}{2}}=\boldsymbol{I}
	\end{aligned}
\end{equation}
这个操作被称为对数据的白化(whitening)或者球人形化(sphereing)。

主成分分析的另一个常见应用是数据可视化。例如,每个数据点被投影到二维(M=2)的主子空间中,从而数据点$\boldsymbol{x}_n$被画在了一个笛卡尔坐标第中,坐标系由$\boldsymbol{x}_n^T\boldsymbol{u}_1$和$\boldsymbol{x}_n^T\boldsymbol{u}_2$定义,其中$\boldsymbol{u}_1$和$\boldsymbol{u}_2$是特征向量,对应于最大的和第二大的特征值。
\subsection*{高维数据的PCA}
在主成分分析的一些应用中,数据点的数量小于数据空间的维度。在一个D维空间中,N个数据点(N<D)定义了一个线性子空间,它的维度最多为$N-1$,因此在使用PCA时,几乎没有M大于$N-1$的数据点。实际上,如果我们运行PCA,我们会发现至少$D-N+1$个特征值为零,对应于沿着数据集的方差为零的方向的特征向量。此外,通常的寻找$D\times D$矩阵的特征向量的算法的计算代价为$O(D^3)$。因此对于诸如图像这种应用来说,直接应用PCA在计算上是不可行的。

我们可以这样解这个问题。首先,我们将$\boldsymbol{X}$定义为$(N\times D)$维中心数据矩阵,它的第n行为$(\boldsymbol{x}_n-\bar{\boldsymbol{x}})^T$。这样,协方差矩阵可以写成$\boldsymbol{S}=N^{-1}\boldsymbol{X}^T\boldsymbol{X}$,对应的特征向量方程变成了
\begin{equation}
	\frac{1}{N}\boldsymbol{X}^T\boldsymbol{X}\boldsymbol{u}_i=\lambda_i\boldsymbol{u}_i
\end{equation}
现在,将两侧左乘$\boldsymbol{X}$,可得
\begin{equation}
	\frac{1}{N}\boldsymbol{X}\boldsymbol{X}^T(\boldsymbol{X}\boldsymbol{u}_i)=\lambda_i(\boldsymbol{Xu}_i)
\end{equation}
如果我们现在定义$\boldsymbol{v}_i=\boldsymbol{Xu}_i$,那么我们有
\begin{equation}
\label{lian}
	\frac{1}{N}\boldsymbol{XX}^T\boldsymbol{v}_i=\lambda_i\boldsymbol{v}_i
\end{equation}
它是$N\times N$矩阵$N^{-1}\boldsymbol{XX}^T$的一个特征向量方程。我们看到这个矩阵与原始的协方差矩阵具有相同的$N-1$个特征值,原始的协方差矩阵本身有额外的$D-N+1$个值为零的特征值。因此我们可以在低维空间中解决特征向量问题,计算代价为$O(N^3)$而不是$O(D^3)$。为了确定特征向量,我们将公式$\ref{lian}$两侧乘以$\boldsymbol{X}^T$,可得
\begin{equation}
	\left(\frac{1}{N}\boldsymbol{X}^T\boldsymbol{X}\right)(\boldsymbol{X}^T\boldsymbol{v}_i)=\lambda_i(\boldsymbol{X}^T\boldsymbol{v}_i)
\end{equation}
从中我们可以看到$(\boldsymbol{X}^T\boldsymbol{v}_i)$是$\boldsymbol{S}$的一个特征向量,对应的特征值为$\lambda_i$。但是,需要注意,这些特征向量的长度未必等于1。为了确定合适的归一化,我们使用一个常数来对$\boldsymbol{u}_i\propto \boldsymbol{X}^T\boldsymbol{v}_i$进行重新标度,使得$\lVert \boldsymbol{u}_i\rVert=1$。假设$\boldsymbol{v}_i$的长度已经被归一化,那么我们有
\begin{equation}
\label{huaxiang}
	\boldsymbol{u}_i=\frac{1}{(N\lambda_i)^{\frac{1}{2}}}\boldsymbol{X}^T\boldsymbol{v}_i
\end{equation}
总结一下,为了应用这种方法,我们首先计算$\boldsymbol{XX}^T$,然后找到它的特征向量和特征值,之后使用公式$\ref{huaxiang}$计算原始数据空间的特征向量。
\section{概率PCA}
前一节讨论的PCA的形式所基于的是将数据线性投影到比原始数据空间维度更低的子空间内。PCA也可以被视为概率潜在变量模型的最大似然解。PCA的这种形式,被称为概率PCA(probabilistic PCA),与传统的PCA相比,会带来如下几个优势。
\begin{itemize}
	\item 概率PCA表示高斯分布的一个限制形式,其中自由参数的数量可以受到限制,同时仍然使得模型能够描述数据集的主要的相关关系。
	\item 我们可以为PCA推导一个EM算法,这个算法在只有几个主要的特征向量需要求出的情况下,计算效率比较高,并且避免了计算数据协方差的中间步骤。
	\item 概率模型与EM的结合使得我们能够处理数据集里缺失值的问题。
	\item 概率PCA混合模型可以用一种有理有据的方式进行形式化,并且可以使用EM算法进行训练。
	\item 概率PCA构成了PCA的贝叶斯方法的基础,其中主子空间的维度可以自动从数据中找到。
	\item 似然函数的存在使得直接与其他的概率密度模型进行对比成为可能。相反,传统的PCA会给接近主子空间的数据点分配一个较低的重建代价,即使这些数据点的位置距离训练数据任意远。
	\item 概率PCA可以被用来对类条件概率密度建模,因此可以应用于分类问题。
	\item 概率PCA模型可以用一种生成式的方式运行,从而可以按照某个概率分布生成样本。
\end{itemize}
\subsection*{最大似然PCA}
\subsection*{用于PCA的EM算法}

\section{核PCA}
\section{非线性隐变量模型}
