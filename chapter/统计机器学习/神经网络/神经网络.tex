在上两章,我们考虑了由固定基函数的线性组合构成的回归模型和分类模型。我们看到,这些模型具有一些有用的分析性质和计算性质,但是它们的实际应用被维数灾难问题限制了。为了将这些模型应用于大规模的问题,有必要根据数据调节基函数。

支持向量机是这样解决这个问题的:首先定义以训练数据点为中心的基函数,然后在训练过程中选择一个子集。支持向量机的一个优点是,虽然训练阶段涉及到非线性优化,但是目标函数是凸函数,因此最优化问的解相对很直接,并且通常随着数据规模的增加而增多。相关向量机也选择固定基函数集合的一个子集,通常会生成一个相当稀疏的模型。与支持向量机不同,相关向量机也产生概率形式的输出,嘎然这种输出的产生会以训练阶段的非凸优化为代价。

另一种方法是事先固定基函数的数量,但是允许基函数可调节。换名话,就是使用参数形式的基函数,这些参数可以在训练阶段调节 。在模式识别中,这种类型的最成功的模型是有前馈神经网络,也被称为多层感知器(multilayer perceptron)。与具有同样泛化能力的支持向量机相比,最终的模型会相当简洁,因此计算的速度更快。这种简洁性带来的代价就是,与相关向量机一样,构成了网络训练根基的似然函数不再是模型参数的凸函数。然而,在实际应用中,考察模型在训练阶段消耗的计算资源是很有价值的,这样做会得到一个简洁的模型,它可以快速地处理新数据。

首先,我们考虑神经网络的函数形式,包括基函数的具体参数,然后我们讨论使用最大似然框架确定神经网络参数的问题,这涉及到非线性最优化问题的解。这种方法需要计算对数似然函数关于神经网络参数的导数,我们会看到这些导数可以使用误差反向传播(error backpropagation)的方法高效地获得。我们还会说明误差反向传播的框架如何推广到计算其他的导数,例如Jacobian矩阵和Hessian矩阵。接下来,我们讨论神经网络训练的正则化和各种方法,以及方法之间的关系。我们还会考虑神经网络模型的一些扩展。特别地,我们会描述一个通用的框架,用来对条件概率密度建模。这个框架被称为混合密度网络(mixture density network)。最后,我们讨论神经网络的贝叶斯观点。