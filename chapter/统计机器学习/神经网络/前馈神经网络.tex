\section{前馈神经网络}
深度前馈网络(deep feedforward network)也叫做前馈神经网络(feedforward neural network)或者多层感知机(multilayer perceptron,MLP),是典型的深度学习模型。前馈神经网络的目标是近似某个函数$f^*$。前馈网络定义了一个映射$y=f(x;\theta)$,并且学习参数$\theta$的值,使它能够得到最佳的函数近似。

前馈神经网络之所以被称作网络,是因为它们通常用许多不同函数复合在一起来表示。该模型与一个有向无环图相关联,而图描述了函数是如何复合在一起的。例如,我们有三个函数$f^{(1)},f^{(2)},f^{(3)}$连接在一个链上以形成$f^{(3)}(f^{(2)}(f^{(1)}(x)))$。

在神经网络训练过程中,我们让$f(x)$去匹配$f^*(x)$的值。训练数据为我们提供了在不同训练点上取值的、含有噪声的$f^*(x)$近似实例。每个样本$x$都伴随着一个标签$y\approx f^*(x)$。训练样本直接指明了输出层在每一点$x$上必须做什么;它必须产生一个接近$y$的值。但是训练数据并没有直接指明其他层应该怎么做。学习算法必须决定如何使用这些层来诞生想要的输出,但是训练数据并没有说每个单独的层应该做什么。相反,学习算法必须决定如何使用这些层来最好地实现$f^*$的近似。因为训练数据并没有给出这些层中的每一层所需的输出,所以这些层被称为隐藏层(hidden layer)。