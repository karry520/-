\section{矩阵的基本性质}
矩阵$A$的第$i$行和第$j$列的元素为$A_{ij}$。我们用$I_N$表示$N\times N$的单位矩阵。在没有歧义的情形下,我们简单地记作$I$。转置矩阵$A^T$的元素为$(A^T)_{ij}=A_{ji}$。根据转置的定义,我们有
\begin{equation}
	(AB)^T=B^TA^T
\end{equation}
$A$的逆矩阵,记作$A^{-1}$,满足
\begin{equation}
	AA^{-1}=A^{-1}A=I
\end{equation}
由于$ABB^{-1}A^{-1}=I$,我们有
\begin{equation}
\label{c3}
	(AB)^{-1}=B^{-1}A^{-1}
\end{equation}
我们还有
\begin{equation}
	(A^T)^{-1}=(A^{-1})^T
\end{equation}
关于矩阵的逆矩阵,下面这个恒等式很有用
\begin{equation}
\label{guji}
	(P^{-1}+B^TR^{-1}B)^{-1}B^TR^{-1}=PB^T(BPB^T+R)^{-1}
\end{equation}
两侧同时右乘$(BPB^T+R)$,很容易证明上式的正确性。假设$P$的维度为$N\times N$,而$R$的维度为$M\times M$,从而$B$的维度为$M\times N$。这样,如果$M\ll N$,那么估计公式$\ref{guji}$的右侧所花费的低价就远远小于估计左侧的代价。经常出现的一种情况是
\begin{equation}
	(I+AB)^{-1}A=A(I+BA)^{-1}
\end{equation}
另一个与矩阵的逆矩阵相背的有用的恒等式为
\begin{equation}
	(A+BD^{-1}C)^{-1}=A^{-1}-A^{-1}B(D+CA^{-1}B)^{-1}CA^{-1}
\end{equation}
这被称为Woodbury恒等式。将两侧同时乘以$(A+BD^{-1}C)$即可证明。例如,假设$A$是一个很大的对角矩阵(因此很容易求逆矩阵),$B$的行数很多列数很少($C$恰好相反),此时计算右侧的代价就远远小于计算左侧的代价。

一组向量$\{a_1,\dots,a_N\}$被称为线性相关(linearly independent)如果关系$\sum_n a_n \boldsymbol{a}_n=0$只在所有$a_n=0$时成立。这表明,没有任何一个向量能够表示为其余向量的线性组合。矩阵的秩是线性无关的行的最大数量(或者等价地,线性无关的列的最大数量)。
\section{迹和行列式}
迹和行列式适用于方阵。矩阵$A$的迹$\mathrm{Tr}(A)$被定义为主对角线上元素的和。我们可以看到
\begin{equation}
	\mathrm{Tr}(AB)=\mathrm{Tr}(BA)
\end{equation}
通过多次把这个公式应用到三个矩阵的乘积上,我们看到
\begin{equation}
	\mathrm{Tr}(ABC)=\mathrm{Tr}(CAB)=\mathrm{Tr}(BCA)
\end{equation}
这被称为迹操作符的循环(cyclic)性质。很明显这个性质可以扩展到任意数量矩阵的乘积。一个$N\times N$矩阵的行列式$|A|$定义为
\begin{equation}
	|A|=\sum (\pm 1)A_{1i_1}A_{2i_2}\dots A_{Ni_N}
\end{equation}
这个式子对所有满足下面性质的乘积进行求和:乘积包含每行的恰好一个元素和每列的恰好一个元素。系数$+1$或者$-1$取决于排列$i_1\dots i_N$是大奇排列还是偶排列。注意$|I|=1$,因此对于一个$2\times 2$矩阵,行列式的形式为
\begin{equation}
	|A|=\begin{vmatrix}
		a_{11}&a_{12}\\
		a_{21}&a_{22}
	\end{vmatrix}
	=a_{11}a_{22}-a_{12}a_{21}
\end{equation}
两个矩阵乘积的行列式为
\begin{equation}
\label{c12}
	|AB|=|A||B|
\end{equation}
此外,矩阵的逆矩阵的行列式为
\begin{equation}
	|A^{-1}|=\frac{1}{|A|}
\end{equation}
如果$A$和$B$是$N\times M$的矩阵,那么 
\begin{equation}
	|I_N+AB^T|=|I_M+A^TB|
\end{equation}
一种特殊情况是
\begin{equation}
	|I_N+ab^T|=1+a^Tb
\end{equation}
其中$a$和$b$是N维列向量。
\section{矩阵的导数}
有时,我们需要考虑向量和矩阵关于标量的导数。向量$\boldsymbol{a}$关于标量$x$的导数本身是一个向量,它的分量为
\begin{equation}
	\left(\frac{\partial \boldsymbol{a}}{\partial x} \right)_i=\frac{\partial a_i}{\partial x}
\end{equation}
矩阵的导数的定义与些类似。关于向量和矩阵的导数也可以被定义。例如
\begin{equation}
	\left(\frac{\partial x}{\partial \boldsymbol{a}} \right)_i=\frac{\partial x}{\partial a_i}
\end{equation}
类似地
\begin{equation}
	\left(\frac{\partial a}{\partial b} \right)_{ij}=\frac{\partial a_i}{\partial b_j}
\end{equation}
写出矩阵的各个元素,下面的性质很容易证明
\begin{equation}
	\frac{\partial }{\partial \boldsymbol{x}}(\boldsymbol{x}^T\boldsymbol{a})=\frac{\partial }{\partial \boldsymbol{x}}(\boldsymbol{a}^T\boldsymbol{x})=\boldsymbol{a}
\end{equation}
类似地
\begin{equation}
\label{weifen}
	\frac{\partial }{\partial \boldsymbol{x}}(AB)=\frac{\partial A}{\partial \boldsymbol{x}}B+A\frac{\partial B}{\partial \boldsymbol{x}}
\end{equation}
矩阵的逆矩阵的导数可以表示为
\begin{equation}
	\frac{\partial }{\partial x}(A^{-1})=-A^{-1}\frac{\partial A}{\partial x}A^{-1}
\end{equation}
使用公式$\ref{weifen}$对方程$A^{-1}A=I$求微分,然后右乘$A^{-1}$即可证明。并且 
\begin{equation}
	\frac{\partial }{\partial x}\ln |A|=\mathrm{Tr}\left(A^{-1}\frac{\partial A}{\partial x} \right)
\end{equation}
如果我们把$x$选成$A$中的元素,那么我们有
\begin{equation}
	\frac{\partial }{\partial A_{ij}}\mathrm{Tr}(AB)=B_{ji}
\end{equation}
写出矩阵的下标即可证明这个等式。我们可以把这个结论写成更加简洁的形式
\begin{equation}
	\frac{\partial }{\partial A}\mathrm{Tr}(AB)=B^T
\end{equation}
使用这种记号,我们有下列性质 
\begin{flalign}
	\frac{\partial }{\partial A}\mathrm{Tr}(A^TB)&=B\\
	\frac{\partial }{\partial A}\mathrm{Tr}(A)&=I\\
	\frac{\partial }{\partial A}\mathrm{Tr}(ABA^T)&=A(B+B^T)
\end{flalign}
这些也可以通过写出矩阵下标的方式证明出。我们也有
\begin{equation}
	\frac{\partial }{\partial A}\ln |A|=(A^{-1})^T
\end{equation}

\section{特征向量方程}
对于一个$M\times M$的方阵$A$,特征向量方程的定义为
\begin{equation}
\label{luqu}
	\boldsymbol{A}\boldsymbol{\mu}_i=\lambda_i\boldsymbol{\mu}_i
\end{equation}
其中$i=1,\dots,M$,$\boldsymbol{\mu}_i$被称为特征向量(eigenvector),$\lambda_i$被称为对应的特征值(eigenvalue)。这可以看成M个齐次线性方程组,角存在的条件为
\begin{equation}
	|A-\lambda_i I|=0
\end{equation}
这被称为特征方程(characteristic equation)。由于这是$\lambda_i$的M阶多项式,因此它一定有M个解(虽然这些解未必不同)。$A$的秩等于非零特征值的个数。

我们特别感兴趣的是对称矩阵。协方差矩阵、核矩阵、Hessian矩阵都是对称矩阵。对称矩阵的性质为$A_{ij}=A_{ji}$或者等价地,$A=A^T$。对称矩阵的逆矩阵也是对称的。将$A^TA=I$取转置,然后使用$AA^{-1}=I$以及$I$的对称性即可证明这一点。

通常情况下,矩阵的特征值是复数。但是对于对称矩阵,特征值$\lambda_i$为实数。这点可以用下面的方式证明。首先将公式$\ref{luqu}$左乘$(\boldsymbol{\mu}_i^*)^T$,其中$*$表示复共轭,我们可以得到 
\begin{equation}
	(\boldsymbol{\mu}_i^*)^TA\boldsymbol{\mu}_i=\lambda_i(\boldsymbol{\mu}_i^*)^T\boldsymbol{\mu}_i
\end{equation}
之后,我们对公式$\ref{luqu}$取复共轭,然后左乘$\boldsymbol{\mu}_i^T$,可得
\begin{equation}
	\boldsymbol{\mu}_i^TA\boldsymbol{\mu}_i^*=\lambda_i^*\boldsymbol{\mu}_i^T\boldsymbol{\mu}_i^*
\end{equation}
推导过程中,我们使用了$A^*=A$,因为我们只考虑实对称矩阵$A$。将第二个方程取转置,使用$A^T=A$,我们看到两个方程在左侧相同,从而$\lambda_i^*=\lambda_i$,因此$\lambda_i$一定是实数。

实对称矩阵的特征向量$\boldsymbol{\mu}_i$可以被选成单位正交的(即正交的并且长度为单位长度),使得
\begin{equation}
	\boldsymbol{\mu}_i^T\boldsymbol{\mu}_j=I_{ij}
\end{equation}
其中$I_{ij}$是单位矩阵$I$的元素。为了证明这一点,我们首先将公式$\ref{luqu}$左乘$\boldsymbol{\mu}_j^T$,得到 
\begin{equation}
	\boldsymbol{\mu}_j^T\boldsymbol{A}\boldsymbol{\mu}_i=\lambda_i\boldsymbol{\mu}_j^T\boldsymbol{\mu}_i
\end{equation}
因此,通过交换下标,我们有
\begin{equation}
	\boldsymbol{\mu}_i^T\boldsymbol{A}\boldsymbol{\mu}_j=\lambda_j\boldsymbol{\mu}_i^T\boldsymbol{\mu}_j
\end{equation}
我们现在对第二个方程取转置,使用对称性质$\boldsymbol{A}^T=\boldsymbol{A}$,然后将两个方程相减,可得
\begin{equation}
	(\lambda_i-\lambda_j)\boldsymbol{\mu}_i^T\boldsymbol{\mu}_j=0
\end{equation}
因此,对于$\lambda_i\ne \lambda_j$,我们有$\boldsymbol{\mu}_i^T\boldsymbol{\mu}_j=0$,因此$\boldsymbol{\mu}_i$和$\boldsymbol{\mu}_j$是正交的。如果两个特征值是相等的,那么任意线性组合$\alpha \boldsymbol{\mu}_i+\beta \boldsymbol{\mu}_j$也是一个有着相同特征值的特征向量,因此我们可以任意选择一个线性组合,然后选择第二个特征向量正交于第一个(可以证明这种退化的特征向量永远不会线性相关)。因此特征向量可以选择为正交的,然后归一化为单位长度。由于有M个特征值,对应的M个特征向量组成了一个完备集,因此任意一个M维的向量都可以表示为特征向量的线性组合。

我们可以令特征向量$\boldsymbol{\mu}_i$是$M\times M$的矩阵$\boldsymbol{U}$,根据单位正交性,我们有
\begin{equation}
\label{zhengming}
	\boldsymbol{U}^T\boldsymbol{U}=\boldsymbol{I}
\end{equation}
这样的矩阵被称为正交的(orthogonal)。有趣的是,矩阵的行也是正交的,即$\boldsymbol{U}\boldsymbol{U}^T=\boldsymbol{I}$。为了证明这一点,我们注意到,公式$\ref{zhengming}$表明$\boldsymbol{U}^T\boldsymbol{U}\boldsymbol{U}^{-1}=\boldsymbol{U}^{-1}=\boldsymbol{U}^T$,因此$\boldsymbol{U}\boldsymbol{U}^{-1}=\boldsymbol{U}\boldsymbol{U}^T=\boldsymbol{I}$。使用公式$\ref{c12}$,也可以看出$|U|=1$。

特征向量方程$\ref{luqu}$可以使用$\boldsymbol{U}$表示成下面的形式
\begin{equation}
\label{c38}
	\boldsymbol{AU}=\boldsymbol{U\Lambda}
\end{equation}
其中$\boldsymbol{\Lambda}$是一个$M\times M$的对角矩阵,对角线上的元素为特征值$\lambda_i$。

如果我们考虑一个列向量$\boldsymbol{x}$,它经过正交矩阵$\boldsymbol{U}$进行变换,得到新向量 
\begin{equation}
	\tilde{\boldsymbol{x}}=\boldsymbol{Ux}
\end{equation}
变换前后向量的长度不变,因为 
\begin{equation}
	\tilde{\boldsymbol{x}}^T\tilde{\boldsymbol{x}}=\boldsymbol{x}^T\boldsymbol{U}^T\boldsymbol{U}\boldsymbol{x}=\boldsymbol{x}^T\boldsymbol{x}
\end{equation}
类似地,任意两个向量的角度在变换前后也不变,因为
\begin{equation}
	\tilde{\boldsymbol{x}}^T\tilde{\boldsymbol{y}}=\boldsymbol{x}^T\boldsymbol{U}^T\boldsymbol{U}\boldsymbol{y}=\boldsymbol{x}^T\boldsymbol{y}
\end{equation}
因此,乘以$\boldsymbol{U}$可以表示为坐标第的刚性旋转.

根据公式$\ref{c38}$可得
\begin{equation}
	\boldsymbol{U}^T\boldsymbol{AU}=\boldsymbol{\Lambda}
\end{equation}
因为$\boldsymbol{\Lambda}$是对角矩阵,我们说矩阵$\boldsymbol{A}$被矩阵$\boldsymbol{U}$对角化(diagonalised)。如果我们左乘$\boldsymbol{U}$然后右乘$\boldsymbol{U}^T$,我们有
\begin{equation}
\label{c43}
	\boldsymbol{A}=\boldsymbol{U\Lambda U}^T
\end{equation}
取这个方程的逆,然后使用公式$\ref{c3}$以及$\boldsymbol{U}^{-1}=\boldsymbol{U}^T$,我们有
\begin{equation}
	\boldsymbol{A}^{-1}=\boldsymbol{U}\boldsymbol{\Lambda}^{-1}\boldsymbol{U}^T
\end{equation}
最后两个方程也可以写成 
\begin{flalign}
	\boldsymbol{A}&=\sum_{i=1}^{M}\lambda_i\boldsymbol{\mu}_i\boldsymbol{\mu}_i^T\\
	\boldsymbol{A}^{-1}&=\sum_{i=1}^{M}\frac{1}{\lambda_i}\boldsymbol{\mu}_i\boldsymbol{\mu}_i^T
\end{flalign}
如果我们取公式$\ref{c43}$的行列式,然后使用公式$\ref{c12}$,我们有
\begin{equation}
	|\boldsymbol{A}|=\prod_{i=1}^{M}\lambda_i
\end{equation}
类似地,取公式$\ref{c43}$的迹,使用迹运算的循环性以及$\boldsymbol{U}^T\boldsymbol{U}^T=\boldsymbol{I}$,我们有
\begin{equation}
	\mathrm{Tr}(\boldsymbol{A})=\sum_{i=1}^{M}\lambda_i
\end{equation}

一个矩阵$\boldsymbol{A}$被称为正定的(positive definite),记作$\boldsymbol{A}\succ 0$,如果对于向量$\boldsymbol{w}$的所有非零值都有$\boldsymbol{w}^T\boldsymbol{Aw}>0$。等价地,一个正定矩阵的所有特征值都有$\lambda_i>0$。令$\boldsymbol{w}$为每一个特征向量,然后注意到任意的向量都可以展开为特征向量的组合,我们即可以证明这一点。注意,正定不同于所有元素都为正。例如,矩阵
\begin{equation}
	\begin{pmatrix}
		1&2\\
		3&4
	\end{pmatrix}
\end{equation}
的特征值$\lambda_1 \simeq 5.37$且$\lambda_2\simeq -0.37$。一个矩阵被称为半正定的(positive semidefinite),如果对于$\boldsymbol{w}$的所有值都有$\boldsymbol{w}^T\boldsymbol{Aw}\geqslant 0$,记作$\boldsymbol{A}\succeq 0$。它等价于$\lambda_i \geqslant 0$。