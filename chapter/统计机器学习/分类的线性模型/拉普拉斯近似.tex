\section{拉普拉斯近似}
特别地,我们不能够精确地关于参数向量$\boldsymbol{x}$求积分,因为后验概率分布不再是高斯分布。因此,有必要介绍某种形式的近似。后面章节中,我们会介绍一系列分析估计和数值采样的技术。

拉普拉斯近似的目标是找到定义在一组连续变量上的概率密度的高斯近似。首先考虑单一连续变量$z$的情形,假设分布$p(z)$的定义为
\begin{equation}
	p(z)=\frac{1}{Z}f(z)
\end{equation}
其中$Z=\int f(z)dz$是归一化系数。我们假定Z的值是未知的。在拉普拉斯方法中,目标是寻找一个高斯近似$q(z)$,它的中心位于$p(z)$的众数的位置。第一步是寻找$p(z)$的众数,即寻找一个点$z_0$使得$p^{'}(z_0)=0$,或者等价地
\begin{equation}
	\frac{df(z)}{dz}\Bigg|_{z=z_0}=0
\end{equation}
高斯分布有一个性质,即它的对数是变量的二次函数。于是我们考虑$\ln f(z)$以众数$z_0$为中心的泰勒展开,即
\begin{equation}
	\ln f(z)\simeq \ln f(z_0) -\frac{1}{2}A(z-z_0)^2
\end{equation}
其中
\begin{equation}
	A=-\frac{d^2}{dz^2}\ln f(z)\Bigg|_{z=z_0}
\end{equation}
注意,泰勒展开式中的一阶项没有出现,因为$z_0$是概率分布的局部最大值。两侧同时取指数,我们有
\begin{equation}
	f(z)\simeq f(z_0)exp\left\{-\frac{A}{2}(z-z_0)^2 \right\}
\end{equation}
这样,使用归一化的高斯分布的标准形式,我们就可以得到归一化的概率分布$q(z)$,即
\begin{equation}
	q(z)=\left(\frac{A}{2\pi} \right)^{\frac{1}{2}}exp\left\{-\frac{A}{2}(z-z_0)^2 \right\}
\end{equation}
注意,高斯近似只在精度A>0时有良好的定义,换句话说,驻点$z_0$一定是一个局部最大值,使得$f(z)$在驻点$z_0$处的二阶导数为负。

我们可以将拉普拉斯方法推广,使其近似多元高斯分布。在应用拉普拉斯方法时,真实概率分布的归一化常数$Z$不必事先知道。根据中心极限定理,我们可以预见模型的后验概率会随着观测数据点的增多而越来越近似于高斯分布,因此我们可以预见在数据点相对较多的情况下,拉普拉斯近似会更有用。拉普拉斯近似的一个主要缺点是,由于它是以高斯分布为基础的,因此它只能直接应用于实值变量。在其他情况下,可以将拉普拉斯近似应用于变换之后的变量上。但是,拉普拉斯框架的最严重的局限性是,它完全依赖于真实概率分布在变量的某个具体位置上的性质,因此会无法描述一些重要的全局属性。
\subsection*{模型比较和BIC}
除了近似概率分布$p(z)$,我们也可以获得对归一化常数Z的一个近似。
\begin{equation}
\label{dada}
	\begin{aligned}
	   Z&=\int f(z)dz\\
		&\simeq f(z_0)\int exp\left\{-\frac{1}{2}(z-z_0)^TA(z-z_0) \right\}dz\\
		&=f(z_0)\frac{(2\pi)^{\frac{M}{2}}}{|A|^{\frac{1}{2}}}
	\end{aligned}
\end{equation}

我们可以使用公式$\ref{dada}$的结果来获得对于模型证据的一个近似。