\section{判别函数}
判别函数是一个以向量$x$为输入,把它分配到K个类别中的某一个类别(记作$C_k$)的函数。
\subsection*{二分类}
线性判别函数的最简单的形式是输入向量的线性函数,即
\begin{equation}
	y(\boldsymbol{x})=\boldsymbol{w}^T\boldsymbol{x}+w_0
\end{equation}
其中$\boldsymbol{w}$被称为权向量(weight vector),$w_0$被称为偏置(bias)。对应的决策边界由$y(\boldsymbol{x})=0$确定。对于一个输入向量$\boldsymbol{x}$,如果$y(\boldsymbol{x})\geqslant 0$,那么它被分到$C_1$中,否则被分到$C_2$中。向量$\boldsymbol{w}$与决策面内的任何向量都正交,从而$w$确定了决策面的方向。任意一点$x$到决策面的距离为
\begin{equation}
	r=\frac{y(\boldsymbol{x})}{\Vert \boldsymbol{w} \Vert}
\end{equation}
\subsection*{多分类}
\subsection*{用于分类的最小平方方法}
\subsection*{Fisher线性判别函数}
\subsection*{与最小平方的关系}
\subsection*{多分类的Fisher判别函数}
\subsection*{感知器算法}