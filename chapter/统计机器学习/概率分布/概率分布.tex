概率论在解决模式识别问题时起着重要作用。现在探究一下某些特殊的概率分布的例子以及它们的性质。概率分布的一个作用是在给定有限次观测$x_1,\dots,x_N$的前提下,对随机变量$\vec{x}$的概率分布$p(\vec{x})$建模。这个问题被称为密度估计(density estimation)。本章中,我们假设数据点是独立同分布的。

首先,我们考虑离散随机变量的二项分布和多项式分布,以及连续随机变量的高斯分布。这是参数分布(paramertric distribution)的具体例子。之所以被称为参数分布,是因为少量可调节的参数控制了整个概率分布。为了把这种模型应用到密度估计问题中,我们需要一个步骤,能够在给定观察数据集的条件下,确定参数的合适的值。在频率学家的观点中,我们通过最优化某些准则(例如似然函数)来确定参数的具体值。相反,在贝叶斯观点中,给定观察数据,我们引入参数的先验分布,然后使用贝叶斯定理来计算对应后验概率分布。

我们会看到,共轭先验(conjugate prior)有着很重要的作用,它使得后验概率分布的函数形式与先验概率相同,因此使得贝叶斯分析得到了极大的简化。指数族分布有很多重要的性质,将在本章详细讨论。

参数方法的一个限制是它假定分布有一个具体的函数形式,这对于一个具体应用来说是不合适的。另一种替代的方法是非参数(nonparametric)密度估计方法。这种方法中分布的形式通常依赖于数据集的规模。这些模型仍然具有参数,但是这些参数控制的是模型的复杂度而不是分布的形式。本章最后,我们会考虑三种非参数化方法,分布依赖于直方图、最近邻以及核函数。

