\section{混合高斯}
EM算法的一个重要应用是高斯混合模型的参数估计。高斯混合模型应用广泛,在许多情况下,EM算法是学习高斯混合模型(Gaussian misture model)的有效方法。
\begin{definition}{高斯混合模型}{}
	高斯混合模型是指具有如下形式的概率分布模型:
	\begin{equation}
		P(y|\theta)=\sum_{k=1}^{K}\alpha_k\phi(y|\theta_k)
	\end{equation}
	其中,$\alpha_k$是系数,$\alpha_k\geqslant 0,\sum\limits_{k=1}^{K}\alpha_k=1;\phi(Y|\theta_k)$是高斯分布密度,$\theta_k=(\mu_k,\sigma_k^2)$,
	\begin{equation}
		\phi(y|\theta_k)=\frac{1}{\sqrt{2\pi}\sigma_k}exp\left(-\frac{(y-\mu_k)^2}{2\sigma_k^2}\right)
	\end{equation}
	称为第k个模型。
\end{definition}
\subsection*{高斯混合模型参数估计的EM算法}
假设观测数据$y_1,y_2,\dots,y_N$由高斯混合模型生成,
\begin{equation}
	P(y|\theta)=\sum_{k=1}^{K}\alpha_k\phi(y|\theta_k)
\end{equation}
其中,$\theta=(\alpha_1,\alpha_2,\dots,\alpha_K;\theta_1,\theta_2,\dots,\theta_K)$。我们用EM算法估计高斯混合模型的参数$\theta$
\begin{enumerate}
	\item 明确隐变量,写出完全数据的对数似然函数
	
	隐变量由$\gamma_{jk}$表示,其定义如下:
	\begin{equation}
		\gamma_{jk}=\begin{cases}
		1,\qquad \text{第j个观测来自第k个模型}\\0,\qquad \text{否则}
		\end{cases}
	\end{equation}
	$\gamma_{jk}$是0-1随机变量。
	那么完全数据是
	\begin{equation*}
		(y_j,\gamma_{j1},\gamma_{j2},\dots,\gamma_{jK}),\qquad j=1,2,\dots,K
	\end{equation*}
	于是可以写出完全数据的似然函数
	\begin{equation*}
		\begin{aligned}
			P(y,\gamma|\theta)&=\prod_{j=1}^{N}P(y_j,\gamma_{j1},\gamma_{j2},\dots,\gamma_{jK}|\theta)\\
			&=\prod_{k=1}^{K}\prod_{j=1}^{N}\left[\alpha_k\phi(y|\theta_k)\right]^{\gamma_{jk}}\\
			&=\prod_{k=1}^{K}\alpha_k^{n_k}\prod_{j=1}^{N}\left[\phi(y|\theta_k)\right]^{\gamma_{jk}}\\
			&=\prod_{k=1}^{K}\alpha_k^{n_k}\prod_{j=1}^{N}\left[ \frac{1}{\sqrt{2\pi}\sigma_k}exp\left(-\frac{(y-\mu_k)^2}{2\sigma_k^2}\right) \right]^{\gamma_{jk}}\\
		\end{aligned}
	\end{equation*}
	式中,$n_k=\sum\limits_{j=1}^{N}\gamma_{jk},\sum\limits_{k=1}^{K}n_{k}=N$。(意思是选择第k个高斯模型的次数)
	
	那么,完全数据的对数似然函数为
	\begin{equation*}
		log P(y,\gamma|\theta)=\sum_{k=1}^{K}\left\{n_klog\alpha_k+\sum_{j=1}^{N}\gamma_{jk}\left[log\left(\frac{1}{\sqrt{2\pi}}\right)-log\sigma_k-\frac{1}{2\sigma_k^2}(y_j-\mu_k)^2\right] \right\}
	\end{equation*}
	\item EM算法的E步:确定Q函数
	
	\begin{equation*}
		\begin{aligned}
		Q(\theta,\theta^{(i)})&=E[log P(y,\gamma|\theta)|y,\theta^{(i)}]\\
		&=E\left\{\sum_{k=1}^{K}\left\{n_klog\alpha_k+\sum_{j=1}^{N}\gamma_{jk}\left[log\left(\frac{1}{\sqrt{2\pi}}\right)-log\sigma_k-\frac{1}{2\sigma_k^2}(y_j-\mu_k)^2\right] \right\}  \right\}\\
		&=\sum_{k=1}^{K}\left\{\overbrace{n_klog\alpha_k}^{{\color{red}{\text{常数}}}}+\sum_{j=1}^{N}{\color{red}{E(\gamma_{jk})}}\left[log\left(\frac{1}{\sqrt{2\pi}}\right)-log\sigma_k-\frac{1}{2\sigma_k^2}(y_j-\mu_k)^2\right] \right\} 
		\end{aligned}
	\end{equation*}
	这里需要计算$E(\gamma_{jk}|y,\theta)$,记为$\hat{\gamma}_{jk}$
	\begin{equation*}
		\begin{aligned}
		\hat{\gamma}_{jk}&=E(\gamma_{jk}|y,\theta)=P(\gamma_{jk}=1|y,\theta)(\text{二项分布})\\
		&=\frac{P(\gamma_{jk}=1|y,\theta)}{\sum\limits_{k=1}^{K}P(\gamma_{jk}=1|y,\theta)}\\
		&=\frac{P(y_j|\gamma_{jk}=1,\theta)P(\gamma_{jk}=1|\theta)}{\sum\limits_{k=1}^{K}P(y_j|\gamma_{jk}=1,\theta)P(\gamma_{jk}=1|\theta)}\\
		&=\frac{\alpha_k\phi(y_j|\theta_k)}{\sum\limits_{k=1}^{K}\alpha_k\phi(y_j|\theta_k)},\quad j=1,2,\dots,N;k=1,2,\dots,K
		\end{aligned}
	\end{equation*}
	$\hat{\gamma}_{jk}$是在当前模型参数下第j个观测数据来自第k个分模型的概率,称为分模型k对观测数据$y_j$的响应度。代入式中得
	\begin{equation}
		Q(\theta,\theta^{(i)})=\sum_{k=1}^{K}\left\{n_klog\alpha_k+\sum_{j=1}^{N}{\color{red}{\hat{\gamma}_{jk}}}\left[log\left(\frac{1}{\sqrt{2\pi}}\right)-log\sigma_k-\frac{1}{2\sigma_k^2}(y_j-\mu_k)^2\right] \right\} 
	\end{equation}
	\item 确定EM算法的M步
	迭代的M步是求函数$Q(\theta,\theta^{(i)})$对$\theta$的极大值,即求新一轮迭代的模型参数:
	\begin{equation*}
		\theta^{(i+1)} = arg \mathop{max}\limits_\theta Q(\theta, \theta^{(i)})
	\end{equation*}
	分别对$\mu_k,\sigma_k^2$求偏导并令其为0,即可得到$\hat{\mu}_k,\hat{\sigma}_k$。求$\hat{a}_k$是在$\sum\limits_{k=1}^K=1$条件下求偏导并令其为0得到。
	\begin{flalign}
		\hat{\mu}_k&=\frac{\sum\limits_{j=1}^N\hat{\gamma}_{jk}y_j}{\sum\limits_{j=1}^N\hat{\gamma}_{jk}},\quad k=1,2,\dots,K\\
		\hat{\sigma}_k^2&=\frac{\sum\limits_{j=1}^N\hat{\gamma}_{jk}(y_j-\mu_k)^2}{\sum\limits_{j=1}^N\hat{\gamma}_{jk}},\quad k=1,2,\dots,K\\
		\hat{\alpha}_k&=\frac{n_k}{N}=\frac{\sum\limits_{j=1}^N\hat{\gamma}_{jk}}{N},\quad k=1,2,\dots,K\\
	\end{flalign}
	\item 重复以上计算,直到对数似然函数值不再有明显的变化为止。
\end{enumerate}