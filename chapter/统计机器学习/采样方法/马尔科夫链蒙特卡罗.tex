\section{马尔科夫链蒙特卡罗}
前一节中,我们讨论了计算函数期望的拒绝采样方法和重要采样方法,我们看到在高维空间中,这两种方法具有很大的局限性。因此,我们在本节中讨论一个非常一般的并且强大的框架,被称为马尔可夫链蒙特卡罗(Markov chain Monte Carlo,MCMC),它使得我们可以从一大类概率分布中进行采样,并且可以很好地应对空间维度的增长。

与拒绝采样和重要采样相同,我们再一次从提议分布中采样。但是这次我们记录下当前状态$z^{(\tau)}$,以及依赖于这个当前状态的提议分布$q(z|z^{(\tau)})$,从而样本序列$z^{(1)},z^{(2)},\dots,$组成了一个马尔可夫链。与之前一样,如果我们有$p(z)=\frac{\tilde{p}(z)}{Z_p}$,那么我们会假定对于任意的z值都可以计算$\tilde{p}(z)$,虽然$Z_p$的值可能未知。提议分布本身被选择为足够简单,从而直接采样很容易。在算法的每次迭代中,我们从提议分布中生成一个候选样本$z^*$,然后根据一个恰当的准则接受这个样本。

在基本的Metropolis算法中,我们假定提议分布是对称的,即$q(z_A|z_B)=q(z_B|z_A)$对于所有的$z_A$和$z_B$成立。这样,候选的样本被接受的概率为
\begin{equation}
	A(z^*,z^{(\tau)})=\mathrm{min}\left(1,\frac{\tilde{p}(z^*)}{\tilde{p}(z^{(\tau)})}\right)
\end{equation}
可以这样实现:在单位区间$(0,1)$上的均匀分布中随机选择一个数$u$,然后如果$A(z^*,z^{(\tau)})>u$就接受这个样本。注意,如果从$z^{\tau}$到$z^*$引起了$p(z)$的值的增大,那么这个候选样本当然会被保留。

如果候选样本被接受,那么$z^{(\tau+1)}=z^*$,否则候选样本点$z^*$被丢弃,$z^{(\tau+1)}$被设置为$z^{(\tau)}$,然后从概率分布$q(z|z^{(\tau+1)})$中再次抽取一个候选样本。

通过考察一个具体的例子,即简单的随机游走的例子,我们可以对马尔可夫链蒙特卡罗算法的本质得到更深刻的认识。考虑一个由整数组成的状态空间$z$,概率为
\begin{flalign}
	p(z^{(\tau+1)}&=z^{(\tau)})=0.5\\
	p(z^{(\tau+1)}&=z^{(\tau)}+1)=0.25\\
	p(z^{(\tau+1)}&=z^{(\tau)}-1)=0.25
\end{flalign}
其中$z^{(\tau)}$表示在步骤$\tau$的状态。如果初始状态是$z^{(0)}=0$,那么根据对称性,在时刻$\tau$的期望状态也是零,即$\mathbb{E}[z^{(\tau)}]=0$,类似地很容易看到$\mathbb{E}[(z^{(\tau)})]=\frac{\tau}{2}$。因此,在$\tau$步骤之后,随机游走所经过的平均距离正比于$\tau$的平方根。这个平方根依赖关系是随机游走行为的一个典型性质,表明了随机游走在探索状态空间时是很低效的。设计马尔可夫链蒙特卡罗方法的一个中心目标就是避免随机游走行为。

MCMC方法是使用马尔可夫链的蒙特卡罗积分,其基本思想是:构造一条Markov链使其平稳分布为待估参数的后验分布,通过这条马尔可夫链产生后验分布的样本,并基于马尔可夫链达到平稳分布时的样本(有效样本)进行蒙特卡罗积分。设为某一空间n为产生的总样本数m为链条达到平稳时的样本数则MCMC方法的基本思路可概括为:
\begin{enumerate}
	\item 构造Markov链。构造一条Markov链,使其收敛到平稳分布;
	\item 产生样本。由其中的某一点出发,用(1)中的Markov链进行抽样模拟,产生点序列;
	\item 蒙特卡罗积分。任一函数的期望估计为
\end{enumerate}
在采用MCMC方法时马尔可夫链转移核的构造至关重要,不同的转移核构造方法将产生不同的MCMC方法,当前常用的MCMC方法主要有两种Gibbs抽样和Metropo-Lis-Hastings算法。
\subsection*{马尔可夫链}
在详细讨论MCMC方法之前,仔细研究马尔可夫链的一些一般的性质是很有用的。特别地,我们考察在什么情况下马尔可夫链会收敛到所求的概率分布上。 

如果对于所有的时刻$m$,转移概率都相同,那么这个马尔可夫链被称为同质的(homogeneous)。对于一个特定的变量,边缘概率可以根据前一个变量的边缘概率用链式乘积的方式表示出来,形式为
\begin{equation}
	p(z^{(m+1)})=\sum_{z^{(m)}}p(z^{(m+1)}|z^{(m)})p(z^{(m)})
\end{equation}
对于一个概率分布来说,如果马尔可夫链中的每一步都让这个概率分布保持不变,那么我们说这个概率分布关于这个马尔可夫链是不变的,或者静止的。因此,对于一个转移概率为$T(z^{'},z)$的同质的马尔可夫链来说,如果
\begin{equation}
	p^*(z)=\sum_{z^{'}}T(z^{'},z)p^*(z^{'})
\end{equation}
那么概率分布$p^*(z)$是不变的。注意,一个给定的马尔可夫链可能有多个不变的概率分布。例如,如果转换概率由恒等变换给出,那么任意的概率分布都是不变的。

确保所求的概率分布$p(z)$不变的一个充分(非必要)条件是令转移概率满足细节平衡(detailed balance)性质,定义为
\begin{equation}
	p^*(z)T(z,z^{'})=p^*(z^{'})T(z^{'},z)
\end{equation}
对特定的概率分布$p^*(z)$成立。很容易看到,满足关于特定概率分布的细节平衡性质的转移概率会使得那个概率分布具有不变性,因为 
\begin{equation}
	\sum_{z^{'}}p^*(z^{'})T(z^{'},z)=\sum_{z^{'}}p^*(z)T(z,z^{'})=p^*(z)\sum_{z^{'}}p(z^{'}|z)=p^*(z)
\end{equation}
满足细节平衡性质的马尔可夫链被称为可翻转的(reversible)。

我们的目标是使用马尔可夫链从一个给定的概率分布中采样。如果我们构造一个马尔可夫链使得所求的概率分布是不变的,那么我们就可以达到这个目标。然而,我们还要要求对于$m\to \infty$,概率分布$p(z^{(m)})$收敛到所求的不变的概率分布$p^*(z)$,与初始概率分布$p(z^{(0)})$无关。这种性质被称为各态历经性(ergodicity),这个不变的概率分布被称为均衡(equilibrium)分布。很明显,一个具有各态历经性的马尔可夫链只能有唯一的一个均衡分布。可以证明,同质的马尔可夫链具有各态历经性,只需对不变的概率分布和转移概率做出较弱的限制即可。


\subsection*{Metropolis-Hastings算法}
在算法的步骤$\tau$中,当前状态为$z^{(\tau)}$,我们从概率分布$q_k(z^{(\tau)})$中抽取一个样本$z^*$,然后以概率$A_k(z^*,z^{(\tau)})$接受它,其中
\begin{equation}
	A_k(z^*,z^{(\tau)})=\mathrm{min}\left(1,\frac{\tilde{p}(z^*)q_k(z^{(\tau)}|z^*)}{\tilde{p}(z^{(\tau)})q_k(z^{(\tau)}|z^*)} \right) 
\end{equation}
这里,k标记出可能的转移集合中的成员。与之前一样,接受准则的计算不需要知道概率分布$p(z)=\frac{\tilde{p}(z)}{Z_p}$中的归一化常数$Z_p$。

我们现在证明$p(z)$对于由Metropolis-Hastings算法定义的马尔可夫链是一个不变的概率分布,方法是证明满足细节平衡。我们有
\begin{equation}
	\begin{aligned}
		p(z)q_k(z^{'}|z)A_k(z^{'},z)&=\mathrm{min}(p(z)q_k(z^{'}|z),p(z^{'})q_k(z|z^{'})\\
		&=p(z^{'})q_k(z|z^{'})\mathrm{min}(\frac{p(z)q_k(z^{'}|z)}{p(z^{'})q_k(z|z^{'})},1)\\
		&=p(z^{'})q_k(z|z^{'})A_k(z,z^{'})
	\end{aligned}
\end{equation}
证明完毕。

