对于大多数应用中的概率模型来说,精确推断是不可行的,因此我们不得不借助与某种形式的近似。上一章,我们讨论了基于确定性近似的推断方法,它包括诸如变分贝叶斯方法以及期望传播。这里,我们考虑基于数值采样的近似推断方法,也被称为蒙特卡罗(Monte Carlo)方法。

虽然对于一些应用来说,我们感兴趣的是非观测变量上的后验概率分布本身,但是在大部分情况下,后验概率分布的主要用途是计算期望,例如在做预测的情形下就是这样。因此,本章中,我们希望解决的基本的问题涉及到关于一个概率分布$p(\boldsymbol{z})$寻找某个函数$f(\boldsymbol{z})$的期望。这里,$\boldsymbol{z}$的元素可能是离散变量、连续变量或者二者的结合。因此,在连续变量的情形下,我们希望计算下面的期望
\begin{equation}
\label{zhongyao}
	\mathbb{E}[f]=\int f(\boldsymbol{z})p(\boldsymbol{z}d\boldsymbol{z})
\end{equation}
采样方法背后的一般思想是得到从概率分布$p(\boldsymbol{z})$中独立抽取的一组变量$\boldsymbol{z}^{(l)}$,其中$l=1,\dots,L$。这使得期望可以通过有限和的方式计算,即
\begin{equation}
\label{youxian}
	\hat{f}=\frac{1}{L}\sum_{l=1}^{L}f(\boldsymbol{z}^{(l)})
\end{equation}
只要样本$\boldsymbol{z}^{(l)}$是从概率分布$p(\boldsymbol{z})$中抽取的,那么$\mathbb{E}[\hat{f}]=\mathbb{E}[f]$,因此估计$\hat{f}$具有正确的均值。估计$f$的方差为
\begin{equation}
	\mathrm{var}[\hat{f}]=\frac{1}{L}\mathbb{E}[(f-\mathbb{E}[f])^2]
\end{equation}
它是函数$f(\boldsymbol{z})$在概率分布$p(\boldsymbol{z})$下的方差。因此,值得强调的一点是,估计的精度不依赖于$\boldsymbol{z}$的维度,并且原则上,对于数量相对较少的样本$\boldsymbol{z}^{(l)}$,可能会达到较高的精度。

\textbf{补充一个逻辑关系图}