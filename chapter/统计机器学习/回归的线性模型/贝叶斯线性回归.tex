\section{贝叶斯线性回归}
使用最大似然方法设置线性回归模型的参数时,由基函数的数量控制的模型的复杂度需要根据数据集的规模进行调整。为对数似然函数增加一个正则化项意味着模型的复杂度可以通过正则化系数的值进行控制,虽然基函数的数量和形式的选择仍然对于确定模型的整体行为十分重要。

这就产生了对于特定的应用确定合适的模型复杂度的问题。这个问题不能简单地通过最大化似然函数来确定,因为这总会产生过于复杂的模型和过拟合现象。独立的额外数据能够用来确定模型的复杂度,但这需要较大的计算量,并且浪费了有价值的数据。因此我们转而考虑线性回归的贝叶斯方法,这会避免最大似然的过拟合问题,也会引出使用训练数据本身确定模型复杂度的自动化方法。
\subsection*{参数分布}
关于线性拟合的贝叶斯方法的讨论,我们首先引入模型参数$w$的先验概率分布。把噪声精度参数$\beta$当做已知常数。对应的共轭先验是高斯分布。
\begin{equation}
	p(w)\equiv \mathcal{N}(w|\boldsymbol{m}_0,\boldsymbol{S}_0)
\end{equation}
均值为$\boldsymbol{m}_0$协方差为$\boldsymbol{S}_0$
后验分布
\begin{equation}
\label{349}
	p(w|\boldsymbol{t})=\mathcal{N}(w|\boldsymbol{m}_N,\boldsymbol{S}_N)
\end{equation}
其中,
\begin{flalign}
	\label{fafa}
	\boldsymbol{m}_N&=\boldsymbol{S}_N(\boldsymbol{S}_0^{-1}+\beta\Phi^T\boldsymbol{t})\\
	\label{354}
	\boldsymbol{S}_N^{-1}&=\boldsymbol{S}_0^{-1}+\beta\Phi^T\Phi
\end{flalign}
为了简化起见,考虑高斯先验的一个特定的形式——零均值各向同性高斯分布。这个分布由一个精度参数$\alpha$控制,即
\begin{equation}
	p(\boldsymbol{w}|\alpha)=\mathcal{N}(\boldsymbol{w}|0,\alpha^{-1}I)
\end{equation}
后验概率分布的对数由对数似然函数与先验的对数求和的方式得到。它是$w$的函数,形式为
\begin{equation}
\label{355}
	\ln p(\boldsymbol{w}|\boldsymbol{t})-\frac{\beta}{2}\sum_{n=1}^{N}\{t_n-\boldsymbol{w}^T\phi(x_n) \}^2-\frac{\alpha}{2}\boldsymbol{w}^T\boldsymbol{w}+\text{常数}
\end{equation}
于是,后验分布关于$\boldsymbol{w}$的最大化等价于对平方和误差函数加上一个二次正则项进行最小化。

\textbf{补充一个直线拟合的例子图示}
\subsection*{预测分布}
在实际应用中,我们通常感兴趣的不是$\boldsymbol{w}$本身的值,而是对于新的$\boldsymbol{x}$值预测出$t$的值。这需要计算出预测分布(predictive distribution),定义为
\begin{equation}
	p(t|\boldsymbol{t},\alpha,\beta)=\int p(t|\boldsymbol{w},\beta)p(\boldsymbol{w}|\boldsymbol{t},\alpha,\beta)d\boldsymbol{w}
\end{equation}
其中$\boldsymbol{t}$是训练数据的目标变量的值组成的向量。并且,为了简化记号,我们在右侧省略了条件概率中出现的输入向量。预测分布的形式为
\begin{equation}
	p(t|\boldsymbol{x},\boldsymbol{t},\alpha,\beta)=\mathcal{N}(t|\boldsymbol{m}^T_N\phi(\boldsymbol{x}),\sigma^2_N(\boldsymbol{x}))
\end{equation}
其中预测分布的方差$\sigma_N^2(\boldsymbol{x})$为
\begin{equation}
\label{aaa}
	\sigma_N^2(\boldsymbol{x})=\frac{1}{\beta}+\phi(\boldsymbol{x})^T\boldsymbol{S}_N\phi(\boldsymbol{x})
\end{equation}
公式$\ref{aaa}$的第一项表示数据中的噪声,而第二项反映了与参数$\boldsymbol{w}$关联的不确定性。由于噪声和$\boldsymbol{w}$的分布是相互独立的高斯分布,因此它们的值是可以相加的。注意,当额外的数据点被观测到的时候,后验概率分布会变窄。从而可以证明出$\sigma^2_{N+1}(\boldsymbol{x})\leqslant \sigma_N^2(\boldsymbol{x})$在极限$N\to \infty$的情况下,公式$\ref{aaa}$的第二项趋于零,从而预测分布的方差只与参数$\beta$控制的具有可加性的噪声有关。

如果我们使用局部的基函数(例如高斯基函数),那么在距离基函数中心比较远的区域,公式$\ref{aaa}$给出的预测方差的第二项的贡献将会趋于零,只剩下噪声的贡献$\beta^{-1}$。因此,当对基函数所在的区域之外的区域进行外插的时候,模型对于它做出的预测会变得相当确定,这通常不是我们想要的结果,通过使用被称为高斯过程的另一种贝叶斯回归方法,这个问题可以被避免。

注意,如果$\boldsymbol{w}$和$\beta$都被当成未知的,我们可以引入一个由高斯-Gamma分布定义的共轭先验分布$p(\boldsymbol{w},\beta)$。在这种情况下,预测分布是一个学生$t$分布。
\subsection*{等价核}
公式$\ref{fafa}$给出的线性基函数模型的后验均值解有一个有趣的解释,这个解释为核方法(包括高斯过程)提供了舞台。如果把公式$\ref{fafa}$代入线性基函数模型中,可以写成下面的形式
\begin{equation}
\begin{aligned}
	y(\boldsymbol{x},\boldsymbol{m}_N)&=\boldsymbol{m}_N^T\phi(\boldsymbol{x})\\
	&=\beta\phi(\boldsymbol{x})^T\boldsymbol{S}_N\Phi^T\boldsymbol{t}\\
	&=\sum_{n=1}^{N}\underline{\beta\phi(\boldsymbol{x})^T\boldsymbol{S}_N\phi(\boldsymbol{x}_n)}t_n\\
	&=\sum_{n=1}^{N}k(\boldsymbol{x},\boldsymbol{x}_n)t_n
\end{aligned}
\end{equation}
可以看成在点$\boldsymbol{x}$处的预测均值由训练集目标变量$t_n$的线性组合给出。函数$k(x,x_n)$被称为平滑矩阵(smoother matrix)或者等价核(equivalent kernel)。像这样的回归函数,通过对训练集里目标值进行线性组合做预测,被称为线性平滑(linear smoother)。核函数$k(x,x^{'})$给出了$x$与$x^{'}$的函数关系。可以看到,$x$处的预测分布的均值$y(x,\boldsymbol{m}_N)$可以通过对目标值加权组合的方式获得。距离$x$较近的数据点可以赋一个较高的权值,而距离$x$较远的数据点可以赋一个较低的权值。这种局部性不仅对于局部的高斯基函数成立,对于非局部的多项式基函数和sigmoid基函数也成立。

考虑$y(\boldsymbol{x})$和$y(\boldsymbol{x^{'}})$的协方差
\begin{equation}
	\begin{aligned}
	cov[y(\boldsymbol{x}),y(\boldsymbol{x^{'}})]&=cov[\phi(\boldsymbol{x})^T\boldsymbol{w},\boldsymbol{w}^T\phi(\boldsymbol{x^{'}})]\\
	&=\phi(\boldsymbol{x})^T\boldsymbol{S}_N\phi(\boldsymbol{x^{'}})\\
	&=\beta^{-1}k(\boldsymbol{x},\boldsymbol{x^{'}})
	\end{aligned}
\end{equation}
根据等价核的形式,我们可以看到在附近的点处的预测均值相关性较高,而对于距离较远的点处,相关性就较低。

用核函数表示线性回归给出了解决回归问题的另一种方法。我们不引入一组基函数(它隐式地定义了一个等价的核),而是直接定义一个局部的核函数,然后在给定观测数据集的条件下,使用这个核函数对新的输入变量$\boldsymbol{x}$做预测。这就引入了用于回归问题(以及分类问题)的一个很实用的框架,被称为高斯过程(Gaussian process)。

我们已经看到,一个等价核定义了模型的权值。通过这个权值,训练数据集里的目标值被组合,然后对新的$\boldsymbol{x}$值做预测。可以证明这些权值的和等于1,即
\begin{equation}
	\sum_{n=1}^{N}k(\boldsymbol{x},\boldsymbol{x}^{'})=1
\end{equation}
对于所有的$\boldsymbol{x}$值都成立。同时等价核满足一般的核函数共有的一个重要性质,即它可以表示为非线性函数的向量$\psi(\boldsymbol{x})$和内积的形式,即
\begin{equation}
	k(\boldsymbol{x},\boldsymbol{z})=\phi(\boldsymbol{x})^T\psi(\boldsymbol{z})
\end{equation}